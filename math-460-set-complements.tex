\documentclass[fleqn]{beamer}
%\usetheme[height=7mm]{Rochester}
\usetheme{Boadilla} %{Rochester}

\setbeamertemplate{footline}[text line]{%
\parbox{\linewidth}{\vspace*{-8pt}\hfill\insertshortauthor\hfill\insertpagenumber}}
\setbeamertemplate{navigation symbols}{}
%\author[BW]{Dr.\ Barton Willis}
\usepackage{amsmath}\usepackage{amsthm}
\usepackage{isomath}
\usepackage{upgreek}
\usepackage{comment,enumerate,xcolor}

\usepackage[english]{babel}
\usepackage[final,babel]{microtype}%\usepackage[dvipsnames]{color}
\usefonttheme{professionalfonts}
%\usefonttheme{serif}

\newcommand{\reals}{\mathbf{R}}
\newcommand{\complex}{\mathbf{C}}
\newcommand{\integers}{\mathbf{Z}}
\DeclareMathOperator{\range}{range}
\DeclareMathOperator{\domain}{dom}
\DeclareMathOperator{\dom}{dom}
\DeclareMathOperator{\codomain}{codomain}
\DeclareMathOperator{\sspan}{span}
\DeclareMathOperator{\F}{F}
\DeclareMathOperator{\G}{G}
\DeclareMathOperator{\B}{B}
\DeclareMathOperator{\D}{D}
\DeclareMathOperator{\id}{id}
\DeclareMathOperator{\ball}{ball}

\newcommand{\true}{\mathrm{true}}
\newcommand{\false}{\mathrm{false}}

\usepackage{graphicx}
\usepackage{color}
\usepackage{amsmath}
\DeclareMathOperator{\nullspace}{nullity}
\theoremstyle{definition}
\newtheorem{mydef}{Definition}
\newtheorem{myqdef}{Quasi-definition}
\newtheorem{myex}{Example}
\newtheorem{myth}{Theorem}
\newtheorem{myfact}{Fact}
\newtheorem{metathm}{Meta Theorem}
\newtheorem{Question}{Question}
\newtheorem{Answer}{Answer}
\newtheorem{myproof}{Proof}[]
\newtheorem{hurestic}{Hurestic}

%\usepackage{array}   % for \newcolumntype macro
%\newcolumntype{L}{>{$}l<{$}} % math-mode version of "l" column type

\newenvironment{alphalist}{
  \vspace{-0.4in}
  \begin{enumerate}[(a)]
    \addtolength{\itemsep}{1.0\itemsep}}
  {\end{enumerate}}



\usepackage{pifont}

\newenvironment{checklist}{
  \begin{enumerate}[\ding{51}]
    \addtolength{\itemsep}{-0.0\itemsep}}
  {\end{enumerate}}

\newenvironment{numberlist}
   {\begin{enumerate}[(1)]
       \addtolength{\itemsep}{-0.5\itemsep}}
     {\end{enumerate}}
\usepackage{amsfonts}
\makeatletter
\def\amsbb{\use@mathgroup \M@U \symAMSb}
\makeatother
\usepackage{bbold}



\newcommand{\llnot}{\lnot \,} % is accepted


\subtitle{%Lesson 6 \\ \vspace{1.0in} 
The whole problem with the world is that fools and fanatics are always so certain of themselves, and wiser people so full of doubts. \\ \vspace{0.25in} Bertrand Russel}
\title{\textbf{Set complements}}
%\author[Barton Willis] % (optional, for multiple authors)
%{Barton~Willis}%
%\institute[UNK] % (optional)

%{
 % \inst{1}%
%  ``The secret of getting ahead is getting started.'' Mark Twain
%   }
  \date{}


%\usepackage{courier}
%\lstset{basicstyle=\ttfamily\footnotesize,breaklines=true}
%\lstset{framextopmargin=50pt,frame=bottomline}


%\begin{document}



%--------
%usepackage[usenames,dvipsnames,svgnames,table]{color}



\begin{document}



\frame{\titlepage}

\begin{frame}{The set complement}

\begin{mydef}  Let \( U \) be the universal set. For any set \(A\), we define its complement \(A^c\) by \(A^c = U  \setminus A\).

\end{mydef}


\begin{checklist}

\item We have
\[
     (x \in A^c)  \equiv   (x \in U)    \land  (x \notin A)  .
\]

\item Assisted by Mr.\  DeMorgan, we have
\[
     (x \notin A^c)  \equiv   (x \notin U)    \lor   (x  \in A)  .
\]


\end{checklist}
\end{frame}


\begin{frame}{Every definition deserves a theorem}


\begin{myth} Let \(A\) be a set.  We have
\begin{align*}
    \varnothing^c &= U, \\
    U^c  &= \varnothing, \\
    (A^c)^c & = A, \\
      A \cap A^c &= \varnothing, \\
       A \cup A^c &=  U.
\end{align*}

\end{myth}
\end{frame}

\begin{frame}

\begin{myproof}[ \(U  = \varnothing^c\)] We have
\[
     \varnothing^c = \{ u \in U \mid x \notin \varnothing \} = \{ u \in U \mid  \true \} = U.
\]
\end{myproof}


\begin{checklist}

\vspace{0.2in}

\item We used the fact that   \( (\forall x)(x \notin \varnothing) \equiv \true\).

\vspace{0.2in}
\item The proof is a string of equalities. The conclusion of the proof is to compare the far left to the far right of the string.

\vspace{0.2in}

\item Alternatively, we could show that \(U  \subset  \varnothing^c\) and \(\varnothing^c \subset U\).  But I think this proof is more clear.
\end{checklist}

\end{frame}
\begin{frame}

\begin{myproof}[\((A^c)^c  = A\)] Let \(A\) be a set. We have
\[
     (A^c)^c = \{ u \in U \mid u \notin A^c \} =  \{ u \in U \mid u \in A \} = A .
\]
\end{myproof}

\begin{checklist}

\item Arguably this proof makes too large a jump in logic from  \(u \notin A^c \) to \(u \in A\).

\item Here is a fix:
\begin{align*}
      \{ u \in U \mid u \notin A^c \} &=  \{ u \in U \mid  (u \notin U) \lor  (u  \in A)\}, \\
                                                               &=  \{ u \in U \mid  \false  \lor  (u  \in A)\} , \\
                                                              &=   \{ u \in U \mid  u  \in A\},\\
                                                              &    = A.
 \end{align*}

 \item We have
 \[
     x \in A \setminus B \equiv (x \in A) \land (x \notin B)
 \]


  \item So
 \[
     x \notin A \setminus B \equiv    \lnot ((x \in A) \land (x \notin B)) = (x \notin A) \lor (x \in B).
 \]
\end{checklist}

\end{frame}

\begin{frame}

\begin{myproof}[\(  \varnothing  = U^c\)]
We've already shown that \(\varnothing^c = U\).  Using this fact, we have
\[
  \varnothing = (\varnothing^c)^c  = (U)^c.
\]
\end{myproof}

\begin{checklist}

\item Our proof uses the fact that for all sets \(A\), we have \((A^c)^c = A\).
\vspace{0.2in}
\item With malice aforethought, we proved the statements in the proposition in a different order than they were presented.

\vspace{0.2in}
\item  We switched up the order to make use of \((A^c)^c = A\) in a later proof.
\end{checklist}
\end{frame}

\begin{frame}{Looks like homework}


\begin{myproof}[  \(A \cap A^c = \varnothing\)]

(This looks like it should be homework--it's all up to you. You are certainly allowed to use all the results we have proved so far.)
\end{myproof}

\vspace{0.2in}

\begin{myproof}[  \(A \cup A^c = U \)]
(This looks like it should be homework--it's all up to you. You are certainly allowed to use all the results we have proved so far.)
\end{myproof}
\end{frame}

\end{document}
