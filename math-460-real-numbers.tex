\documentclass[fleqn]{beamer}
%\usetheme[height=7mm]{Rochester}
\usetheme{Boadilla} %{Rochester}

\setbeamertemplate{footline}[text line]{%
\parbox{\linewidth}{\vspace*{-8pt}\hfill\insertshortauthor\hfill\insertpagenumber}}
\setbeamertemplate{navigation symbols}{}
%\author[BW]{Dr.\ Barton Willis}
\usepackage{amsmath}\usepackage{amsthm}
\usepackage{isomath}
\usepackage{upgreek}
\usepackage{comment,enumerate,xcolor}

\usepackage[english]{babel}
%\usepackage[final,babel]{microtype}%\usepackage[dvipsnames]{color}
%\usefonttheme{professionalfonts}
%\usefonttheme{serif}

\newcommand{\reals}{\mathbf{R}}
\newcommand{\complex}{\mathbf{C}}
\newcommand{\integers}{\mathbf{Z}}
\newcommand{\rational}{\mathbf{Q}}

\DeclareMathOperator{\range}{range}
\DeclareMathOperator{\domain}{dom}
\DeclareMathOperator{\dom}{dom}
\DeclareMathOperator{\codomain}{codomain}
\DeclareMathOperator{\sspan}{span}
\DeclareMathOperator{\F}{F}
\DeclareMathOperator{\G}{G}
\DeclareMathOperator{\B}{B}
\DeclareMathOperator{\D}{D}
\DeclareMathOperator{\id}{id}
\DeclareMathOperator{\ball}{ball}

\usepackage{graphicx}
\usepackage{color}
\usepackage{amsmath}
\DeclareMathOperator{\nullspace}{nullity}
\theoremstyle{definition}
\newtheorem{mydef}{Definition}
\newtheorem{myqdef}{Quasi-definition}
\newtheorem{myex}{Example}
\newtheorem{myth}{Theorem}
\newtheorem{myfact}{Fact}
\newtheorem{metathm}{Meta Theorem}
\newtheorem{Question}{Question}
\newtheorem{Answer}{Answer}
\newtheorem{myproof}{Proof}

\newtheorem{myfakeproof}{Fake Proof}
\newtheorem{mybadformproof}{Bad Form Proof}

\newtheorem{hurestic}{Hurestic}
\newtheorem{prop}{Proposition}
%\usepackage{array}   % for \newcolumntype macro
%\newcolumntype{L}{>{$}l<{$}} % math-mode version of "l" column type

\newenvironment{alphalist}{
  \vspace{-0.0in}
  \begin{enumerate}[(a)]
    \addtolength{\itemsep}{1.0\itemsep}}
  {\end{enumerate}}

\newenvironment{snowflakelist}{
  \vspace{-0.4in}
  \begin{enumerate}[\textleaf]
    \addtolength{\itemsep}{-1.2\itemsep}}
  {\end{enumerate}}





\newenvironment{checklist}{
  \begin{enumerate}[\ding{52}]
    \addtolength{\itemsep}{-1.0\itemsep}}
  {\end{enumerate}}

\newenvironment{numberlist}
   {\begin{enumerate}[(1)]
       \addtolength{\itemsep}{-0.5\itemsep}}
     {\end{enumerate}}
\usepackage{amsfonts}
\makeatletter
\def\amsbb{\use@mathgroup \M@U \symAMSb}
\makeatother
\usepackage{bbold}

\usepackage{array}
\newcolumntype{C}{>$c<$}

\newcommand{\llnot}{\lnot \,} % is accepted
\DeclareMathOperator{\3F2}{{}_3  F_2}
\newcommand\pochhammer[2]{\left[\genfrac..{0pt}{}{#1}{#2}\right]}
\newmuskip\pFqmuskip

\newcommand*\pFq[6][8]{%
  \begingroup % only local assignments
  \pFqmuskip=#1mu\relax
  % make the comma math active
  % \mathcode`\,=\string"8000
  % and define it to be \pFqcomma
  \begingroup\lccode`\~=`\,
  \lowercase{\endgroup\let~}\pFqcomma
  % typeset the formula
      {}_{#2}\!\F_{#3}{\left[\genfrac..{0pt}{}{#4}{#5};#6\right]}   %\F{\left[\genfrac..{0pt}{}{#4}{#5};#6\right]}  %alt: {}_{#2}\F_{#3}{\left[\genfrac..{0pt}{}{#4}{#5};#6\right]}
  \endgroup
}

\newcommand{\pFqcomma}{\mskip \pFqmuskip}
\newcommand{\mydash}{\text{--}}


%------------------

%\subtitle{Lesson 11}
\title{\textbf{Real numbers}}
%\author[Barton Willis] % (optional, for multiple authors)
%{Barton~Willis}%
%\institute[UNK] % (optional)

%{
 % \inst{1}%
 % ``The secret of getting ahead is getting started.'' Mark Twain
 %  }
 % \date{}


\usepackage{courier}
%\lstset{basicstyle=\ttfamily\footnotesize,breaklines=true}
%\lstset{framextopmargin=50pt,frame=bottomline}





%--------
%usepackage[usenames,dvipsnames,svgnames,table]{color}



\begin{document}

\frame{\titlepage}


\begin{frame}{Binary operator}

\begin{mydef}  A \emph{binary operator} on a set \(S\) is a function from \(S \times S\) to \(S\).  A binary operator \(F\) is commutative provided
\[
   (\forall a,b \in S)(F(a,b) = F(b,a)).
\]
It is associative provided
\[
   (\forall a,b,c \in S)(F(a,F(b,c)) = F(F(a,b),c)).
\]
It has a \emph{left identity element}  provided
\[
  (\exists\,\, \theta \in S)((\forall a \in S)(F(\theta,a) = a).
\]
And it has a \emph{right identity element}  provided
\[
  (\exists\,\, \theta \in S)((\forall a \in S)(F(a, \theta) = a).
\]
\end{mydef}

\end{frame}
\begin{frame}
\begin{enumerate}

\item Addition and multiplication of real numbers are examples of binary operators; 
these operators are commutative and associative.

\item In this context, binary means that the function takes \emph{two} members of 
the same set; the use of binary has nothing to do with base two representation of 
a number.

\item Usually binary operators are expressed in \emph{infix notation}; that is, 
the operator is in between its arguments.

\item For example, we write \(1 + 107 = 108\), not \(+(1,107) = 108\).

\item  For a commutative binary operator,  every right identity element is a 
left identity element; so we'll call them collectively an identity element.
\end{enumerate}
\end{frame}

\begin{frame}{Examples}

\begin{alphalist}

\item Addition \(+\) is a binary operator on \(\reals\).  Since \(x +0 = x\) for 
all real \(x\), the identity element for addition is zero.  Further we know that 
addition is commutative and associative.

\item Function composition \(\circ \) is a binary operator on the set of functions 
from \(\reals\) to \(\reals\).  The function \(x \in \reals \mapsto x\) is the 
identity element for function composition.  Function composition is associative, 
but not commutative.

\end{alphalist}

\end{frame}

\begin{frame}{Unique elements}
\begin{myth} Let \( S \) be a set and let \(F\) be a commutative binary operator 
  on \(S\).  Then \(F\) has at most one identity element.

 \end{myth}

 \begin{myproof} Let \(\theta\) and \(\theta^\prime\) be identity elements for \(F\). 
  We'll show that \(\theta = \theta^\prime\).  We have
 \begin{align*}
    \theta &= F(\theta^\prime, \theta),    &\mbox{(because } \theta \mbox{ is an identity element.}) \\
                &= F(\theta, \theta^\prime),    &\mbox{(because } F \mbox{ is commutative})\\
                &= \theta^\prime.   &\mbox{(because } \theta^\prime \mbox{ is an  identity element.})
 \end{align*}
 So \(\theta = \theta^\prime\).
 \end{myproof}

\end{frame}

\begin{frame}{Fields}

We would like to capture the important features of the real numbers and give all 
such structures a name. This object is a \emph{field}.

\begin{mydef}A field is an ordered triple  \(({\mathcal F},  +, \times)\) 
  where \({\mathcal F} \) is a set and both \(+\) and \(\times\) are commutative 
  and associative binary operators on \({\mathcal F}\) that have  identity elements; 
  the  identity element for \(+\) is 0 and the  identity element  for \(\times\) is 1.



\begin{enumerate}


\item For all \(a,b,c \in  {\mathcal F}\), we have  \(a \times (b + c) = a \times b  + a \times c\).

\item For all \(a \in {\mathcal F}\) there is \(-a \in {\mathcal F} \) such that \(a + -a = 0\).


\item For all \(a \in  {\mathcal F}_{\neq 0}\) there is \(a^{-1} \in {\mathcal F} \) such that \(a a^{-1} = 1\).
\end{enumerate}
\end{mydef}

\begin{enumerate}

\item We say that \(-a\) is an additive inverse of \(a\).

\item We say that \(a^{-1}\) is a multiplicative inverse of \(a\).

\end{enumerate}

\end{frame}
\begin{frame}{Unique inverses}

\begin{myth}  Let   \(({\mathcal F},  +, \times)\) be a field.  The additive and multiplicative inverses are unique. \end{myth}


\begin{myproof}  Let \(a \in {\mathcal F}\) and suppose \(a+b = 0\) and \(a + b^\prime = 0\). We'll show that \( b = b^\prime\). We have
\begin{align*}
   b &= b+ 0, \\
      &= b  + (a + b^\prime), \\
      &= (b + a) + b^\prime, \\
      &= (a+b) + b^\prime,\\
      &= 0  + b^\prime,\\
      &=  b^\prime.
\end{align*}
The proof for the multiplicative inverse is similar and left as an exercise for the willing.
  \end{myproof}




\end{frame}

\begin{frame}{Famous fields}


Let \(+\) and \(\times\) be ordinary number addition and multiplication, respectively. Then
\begin{alphalist}

\item \((\reals, +, \times)\) is the real field.

\item \((\rational, +, \times)\) is the rational field.
Certainly the sum and product of rational numbers is a rational number so indeed, \(+  : \rational \times \rational \to \rational\) and
similarly for \(\times\).  The other required conditions are ``inherited'' from the properties of the real field.

\item \((\integers, +, \times)\)  isn't a field because, for example, there is no \(x \in \integers\) such that \(2 x = 1\).

\end{alphalist}

\end{frame}


\begin{frame}{It's true: $x  \times \theta  = \theta$}

  \begin{prop} Let $(\mathcal{F},+,  \times)$ be a field and let $\theta$
    be its additive identity. Then
    \begin{equation}
      \left(\forall a \in \mathcal F\right) \left (a \times \theta = \theta \right)
    \end{equation}
    \end{prop}

\begin{proof} Let $x \in \mathcal{F}$. We have $x + \theta  = x$. Multiplying this by $x$ and using 
the distributive property, we have $x^2 + x  \theta = x^2$. Adding $-x^2$ to this yields $x  \times \theta =  \theta$.

\end{proof}
  
\end{frame}

\begin{frame}{It's true: $a  \times (-b)  = - (a \times b)$}

  \begin{prop} Let $(\mathcal{F},+,  \times)$ be a field and let $\mathcal O$
    be its additive identity. Then
    \begin{equation}
      \left(\forall a,b \in \mathcal F\right) \left (a  \times (-b)  = - (a \times b) \right).
    \end{equation}
    \end{prop}

\begin{proof} Let $a,b \in \mathcal{F}$.  We have 
\begin{equation*}
   \theta = a \times  \theta = a \times (b + -b)  = a \times b + a \times (-b).
\end{equation*}
So $-(a \times b) = a \times (-b)$
\end{proof}
\end{frame}

\begin{frame}{Something for nothing}

We know that $-(a \times b) = a \times (-b)$ is an identity.  Replacing the silent variable $a$ by $-a$ gives
an syntactically different but semantically equal identity
\begin{equation*}
   -((-a) \times b) = (-a) \times (-b)
\end{equation*}
But $((-a) \times b) = b \times (-a) = - (a \times b)$. That tells gives us the famous result that
\begin{equation*}
   (a \times b) = (-a) \times (-b).
\end{equation*}

Let's \textbf{not} confuse this mathematical fact with \textbf{rubbish} such as ``Morwenna deposits -\$10 a total of -5 times. What is Morwenna's account value?'' 

\end{frame}

\begin{frame}{Ordered Fields}

\begin{mydef}  A field    \(({\mathcal F},  +, \times)\) is ordered provided there is a subset \(P\) of  \({\mathcal F}\) such that

\begin{alphalist}

\item If \(a,b \in P\), we have \(a+b \in P\),

\item If \(a,b \in P\), we have \(a \times b \in P\),

\item For all \(a \in {\mathcal F}\) exactly one of the following is true:  (i)  \(a \in P\),  (ii) \(-a \in P\), (iii) \(a = 0\).

\end{alphalist}

\end{mydef}

\end{frame}
\end{document}

We will assume that there is a set \(\reals\) with members called \emph{real numbers} along with functions \(+ : \reals \times \reals \to \reals\) and \(\times : \reals \times \reals \to \reals\)
such that
