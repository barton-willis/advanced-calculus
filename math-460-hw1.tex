\documentclass[fleqn, 12pt]{exam}
\usepackage{pifont}
\usepackage{dingbat}
\usepackage{amsmath,amssymb}
\usepackage{fleqn}
\usepackage{epsfig}

\usepackage{geometry}
\geometry{letterpaper, margin=0.75in}
\addpoints
\boxedpoints
\pointsinmargin
\pointname{pts}

\usepackage[activate={true,nocompatibility},final,tracking=true,kerning=true,factor=1100,stretch=10,shrink=10]{microtype}
\usepackage[american]{babel}
%\usepackage[T1]{fontenc}
\usepackage{fourier}
\usepackage{isomath}
\usepackage{upgreek,amsmath}
\usepackage{amssymb}

\newcommand{\dotprod}{\, {\scriptzcriptztyle
    \stackrel{\bullet}{{}}}\,}

\newcommand{\reals}{\mathbf{R}}
\newcommand{\lub}{\mathrm{lub}} 
\newcommand{\glb}{\mathrm{glb}} 
\newcommand{\complex}{\mathbf{C}}
\newcommand{\dom}{\mbox{dom}}
\newcommand{\cover}{{\mathcal C}}
\newcommand{\integers}{\mathbf{Z}}
\newcommand{\vi}{\, \mathbf{i}}
\newcommand{\vj}{\, \mathbf{j}}
\newcommand{\vk}{\, \mathbf{k}}
\newcommand{\bi}{\, \mathbf{i}}
\newcommand{\bj}{\, \mathbf{j}}
\newcommand{\bk}{\, \mathbf{k}}
\DeclareMathOperator{\Arg}{\mathrm{Arg}}
\DeclareMathOperator{\Ln}{\mathrm{Ln}}
\newcommand{\imag}{\, \mathrm{i}}

\usepackage{graphicx}
\newcommand\AM{{\sc am}}
\newcommand\PM{{\sc pm}}
     
\newcommand{\quiz}{1}
\newcommand{\term}{Fall}
\newcommand{\due}{Saturday 27 August  at 11:59 \PM}
\begin{document}
\large
\vspace{0.1in}
\noindent\makebox[3.0truein][l]{{\bf MATH 460}}
{\bf Name:}  \\
\noindent \makebox[3.0truein][l]{\bf Homework   \quiz, \term \/ \the\year}
%{\bf Row:}\hrulefill\
\vspace{0.1in}

\noindent  Homework    \quiz\/  has questions 1 through  \numquestions \/ with a total of  \numpoints\/  points.   

\textbf{Very neatly} hand write your solutions, digitize them (pdf works the best--please no *.HEIC files. Canvas cannot display them), and submit the digitized copy gitized to Canvas.  This assignment is due \emph{\due}.

\vspace{0.1in}




\begin{questions} 

\question A function $F$ is increasing on its domain provided
\begin{equation*}
  \left(\forall x,y \in \dom(F) \right) \left(x < y \implies F(x) \leq F(y) \right).
  \end{equation*}
  
  \begin{parts}
  
  \part [5] Without using negation (the symbol \(\lnot\)), write the negation of 
\begin{equation*} 
  \left(\forall x,y \in \dom(F) \right) \left(x < y \implies F(x) \leq F(y) \right)
  \end{equation*}
   in symbolic form. For assistance with the logic, see the section ``Tautologies'' in our class Quick Reference.
  
  \part[5] Show that the function $x \in [-1,1] \mapsto |x|$ is not increasing on its domain.
  
   
  \end{parts} 
  
  \question A function $F$ is subadditive on its domain provided
  \begin{equation*}
  \left(\forall x,y \in \dom(F) \right) \left(  F(x+y) \leq F(x) + F(y) \right).
  \end{equation*}
  
  \begin{parts}
  
  \part [5] Without using negation (the symbol \(\lnot\)), write the negation of 
  \begin{equation*}
  \left(\forall x,y \in \dom(F) \right) \left(  F(x+y) \leq F(x) + F(y) \right)
  \end{equation*}
   in symbolic form.
  
  \part[5] Show that the function $x \in \reals \mapsto x^2 $ is not subadditive on its domain.
  
  \part[5]  Show that the function  $x \in \reals \mapsto |x| $ is  subadditive on its domain.
  To do this, you may use the triangle inequality without proving it.
  \end{parts}
  

\begin{solution}  
\end{solution}




\end{questions}



\end{document}