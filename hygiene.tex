\documentclass[fleqn]{beamer}
%\usetheme[height=7mm]{Rochester}
%\usetheme{Boadilla} %{Rochester}
\usetheme{metropolis}
\setbeamertemplate{footline}[text line]{%
\parbox{\linewidth}{\vspace*{-8pt}\hfill\insertshortauthor\hfill\insertpagenumber}}
\setbeamertemplate{navigation symbols}{}
%\author[BW]{Dr.\ Barton Willis}
\usepackage{amsmath}\usepackage{amsthm}
\usepackage{isomath}
\usepackage{upgreek}
\usepackage{comment,enumerate,xcolor}

\usepackage[english]{babel}
\usepackage[final,babel]{microtype}%\usepackage[dvipsnames]{color}
\usefonttheme{professionalfonts}
%\usefonttheme{serif}

\newcommand{\reals}{\mathbf{R}}
\newcommand{\complex}{\mathbf{C}}
\newcommand{\integers}{\mathbf{Z}}
\DeclareMathOperator{\range}{range}
\DeclareMathOperator{\domain}{dom}
\DeclareMathOperator{\dom}{dom}
\DeclareMathOperator{\codomain}{codomain}
\DeclareMathOperator{\sspan}{span}
\DeclareMathOperator{\F}{F}
\DeclareMathOperator{\G}{G}
\DeclareMathOperator{\B}{B}
\DeclareMathOperator{\D}{D}
\DeclareMathOperator{\id}{id}
\DeclareMathOperator{\ball}{ball}

\newcommand{\true}{\mathrm{true}}
\newcommand{\false}{\mathrm{false}}

\usepackage{graphicx}
\usepackage{color}
\usepackage{amsmath}
\DeclareMathOperator{\nullspace}{nullity}
\theoremstyle{definition}
\newtheorem{mydef}{Definition}
\newtheorem{myqdef}{Quasi-definition}
\newtheorem{myex}{Example}
\newtheorem{myth}{Theorem} 
\newtheorem{myfact}{Fact}
\newtheorem{metathm}{Meta Theorem}
\newtheorem{Question}{Question}
\newtheorem{Answer}{Answer}
\newtheorem{myproof}{Proof}
\newtheorem{hurestic}{Hurestic}

%\usepackage{array}   % for \newcolumntype macro
%\newcolumntype{L}{>{$}l<{$}} % math-mode version of "l" column type

\newenvironment{alphalist}{
  \vspace{-0.4in}
  \begin{enumerate}[(a)]
    \addtolength{\itemsep}{1.0\itemsep}}
  {\end{enumerate}}



\usepackage{pifont}

\newenvironment{checklist}{
  \begin{enumerate}[\ding{51}]
    \addtolength{\itemsep}{-0.0\itemsep}}
  {\end{enumerate}}

\newenvironment{numberlist}
   {\begin{enumerate}[(1)]
       \addtolength{\itemsep}{-0.5\itemsep}}
     {\end{enumerate}}
\usepackage{amsfonts}
\makeatletter
\def\amsbb{\use@mathgroup \M@U \symAMSb}
\makeatother
\usepackage{bbold}



\newcommand{\llnot}{\lnot \,} % is accepted


\subtitle{\emph{MATH 202} \\ \emph{Fall \the\year}  \\ $\phantom{xxx}$ \\ 
\emph{The whole problem with the world is that fools and fanatics are 
always so certain of themselves, and wiser people so full of doubts.} \\ \vspace{0.25in} Bertrand Russell}
\title{\textbf{Math hygiene}}
%\author[Barton Willis] % (optional, for multiple authors)
%{Barton~Willis}%
%\institute[UNK] % (optional)

%{
 % \inst{1}%
%  ``The secret of getting ahead is getting started.'' Mark Twain
%   }
  \date{}
 

%\usepackage{courier}
%\lstset{basicstyle=\ttfamily\footnotesize,breaklines=true}
%\lstset{framextopmargin=50pt,frame=bottomline}


%\begin{document}



%--------
%usepackage[usenames,dvipsnames,svgnames,table]{color}



\begin{document}

\frame{\titlepage}

\begin{frame}{Let's Play True or False}

    \textbf{True or False:} teddy bear, stinkbug, guacamole.

    \vspace{0.5in}
    \textbf{Answer} This is not a statement--it's a list of things; it doesn't 
    have a truth value. I'd say it's a trick question.

\vfill 
\end{frame}

\begin{frame}{True or False redact}

  \textbf{True or False:} In my house this morning, you'll 
  find either a teddy bear, a stinkbug, or guacamole.

  \vspace{0.5in}
  \textbf{Answer} This is true. This morning I opened my fridge. On the
  middle shelf there was a container that passed the guacamole color,
  texture, and taste test; therefore it is true that in my house this morning, you'll 
  find either a teddy bear, a stinkbug, or guacamole.

\vfill 
\end{frame}

\begin{frame}{Now with math}

  \textbf{Question} Is the following work correct?

  \begin{equation*}
    \int_3^4 x \, \mathrm{d}x \quad 
     \frac{1}{2} x^{2} \quad \frac{7}{2}
  \end{equation*}

  \vspace{0.5in}
  \textbf{Answer} Just like the list ``teddy bear, stinkbug, 
  guacamole,''
  this is a list of things. As such, it's not 
  a statement, and it doesn't have a truth value. It's another trick 
  question.
  \vfill 
\end{frame}

\begin{frame}{Math redact}

  \textbf{Question} Is the following work correct?

  \begin{equation*}
    \int_3^4 x \, \mathrm{d}x =
     \left. \frac{1}{2} \,\, x^{2} \right |_{3}^4 = \frac{7}{2}
  \end{equation*}

  \vspace{0.5in}
  \textbf{Answer} Yes, this work is correct. This time we're 
  given something that has a truth value. And its truth value is
  true.

  \vfill
\end{frame}

\begin{frame}{The Mad Gardener}

  \textbf{Question} Is this work correct? 
   (Lewis Carroll, From \emph{The Mad Gardener’s Song})

   \vspace{0.5in}
  \begin{quote}
    He thought he saw an Argument\\
  That proved he was the Pope:\\
  He looked again, and found it was\\
  A Bar of Mottled Soap.\\
  "A fact so dread," he faintly said,\\
  "Extinguishes all hope!  \\
   \end{quote}
 
  \vspace{0.25in}
  \textbf{Answer} It's amusing nonsense poetry--it's meaningless
  and neither correct nor incorrect. Another trick question.
  
\end{frame}

\begin{frame}{With math}
  \textbf{Question} Is this work correct? 

  \begin{align*}
    \int_2^3 \sqrt{x+1} \, \mathrm{d}x &=  \int\sqrt{z}\\
                                       &= \int \frac{3}{2} (1+ z)^{3/2}\\
                                       &=  \frac{3}{2} (4^{3/2} - 3^{3/2})
  \end{align*}

  \textbf{Answer} Unlike the \emph{The Mad Gardener’s Song}, this is 
  \textbf{nonamusing} nonsense. It's 
  meaningless and neither correct nor incorrect. It's yet another 
  trick question.

  %\vspace{0.5in}
  \textbf{Comment} Work like this (a) confuses me (b) should 
  confuse you, and (c) is the result of \textbf{abject sloth}.
\end{frame}

\begin{frame}{Math redact}

  Let $z = x+1$. Then $\mathrm{d}z = \mathrm{d}x$. Further
  $x=2$ implies $z=3$; and $x=3$ implies $z=4$. Now that we
  have gathered our four ingredients, we have

\begin{align*}
    \int_2^3 \sqrt{x+1} \, \mathrm{d}x &=  \int_3^4 \sqrt{z} \, \mathrm{d} z,\\
                                       &=  \left. \frac{2}{3} \, \,  z^{3/2} \right |_{3}^4, \\
                                       &= \frac{2}{3} \left(4^{3/2} - 3^{3/2}\right).
  \end{align*}
\end{frame}

\begin{frame}{First Amendment Rights}

\textbf{Question} Which answer is simplified:

\begin{equation*}
    \int_2^3 \sqrt{x+1} \, \mathrm{d}x =  \frac{2}{3} \left(4^{3/2} - 3^{3/2}\right)
\end{equation*}
or 
\begin{equation*}
    \int_2^3 \sqrt{x+1} \, \mathrm{d}x =  \frac{16}{3} - 2 \sqrt{3}
\end{equation*}

\textbf{Answer} Arguably $\frac{16}{3} - 2 \sqrt{3}$ is more simple 
than is $\frac{2}{3} \left(4^{3/2} - 3^{3/2}\right)$ because it 
has one, not two radicals. But I consider this a first Amendment 
Rights issue. Either answer is OK. If your answer is $\frac{2}{3} \left(4^{3/2} - 3^{3/2}\right)$
and you like it, I say let it be (LIB).
\end{frame}
    

\end{document}