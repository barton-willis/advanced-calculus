\documentclass[12pt, fleqn, answers]{exam}
\usepackage{pifont}
\usepackage{dingbat}
\usepackage{amsmath,amsthm}
\usepackage{fleqn}
\usepackage{epsfig}
\usepackage{fourier} %{mathptm}
\usepackage{amssymb, wasysym}

\addpoints
\boxedpoints
\pointsinmargin
\pointname{pts}

\usepackage{color}\shadedsolutions
\definecolor{SolutionColor}{rgb}{1,1,0.7}

\newcommand{\reals}{\mathbf{R}}
\newcommand{\integers}{\mathbf{Z}}
\newcommand{\bp}{\mathrm{bp}}
\newcommand{\lp}{\mathrm{lp}}
\newcommand{\range}{\mathrm{range}}
\newcommand{\True}{\mathrm{True}}
\newcommand{\ball}{\mathrm{ball}}
\newenvironment{handlist}{
  \begin{enumerate}[\leftthumbsup]
    \addtolength{\itemsep}{-1.0\itemsep}}
  {\end{enumerate}}


\newcommand{\dotprod}{\, {\scriptzcriptztyle \stackrel{\bullet}{{}}}\,}
\begin{document}
\large
\vspace{0.1in}
\noindent\makebox[3.0truein][l]{{\bf Advanced Calculus}}
{\bf Name:}\hrulefill\
\noindent \makebox[3.0truein][l]{\bf Exam II Review, \today}
{\bf Row and Seat:}\hrulefill\
%\normalsize

\vspace{0.1in}

\small
%\noindent \emph{``All students are enjoined in the strongest possible terms to eschew proofs by contradition!} \hfill \mbox{{\sc H.L Royden} \emph{Real Analysis}}}
\normalsize
\begin{questions}

\question Show that the sequence $F = k \in \integers_{\geq 1} \mapsto 8 - \frac{1}{k}$ is
bounded above.
\begin{solution}We'll show that $F$ is bounded above by 8. Let $k \in \integers_{\geq 1} $. We have
\begin{align*}
 \left[ 8 - \frac{1}{k} < 8 \right] &\equiv \left[  0 <  \frac{1}{k} \right], && \mbox{(add $ \frac{1}{k} -8$)} \\
                                               & \equiv \left[  0 < 1 \right], && \mbox{(multiply by $k$)} \\
  &\equiv \True.
\end{align*}

\end{solution}
\question Show that the sequence $F = k \in \integers_{\geq 1} \mapsto \frac{(-1)^k}{k^2}$ converges.
\begin{solution} We'll show that
\begin{equation}
 \left(\exists L \in \reals \right)
 \left(\forall \varepsilon \in \reals\right)
 \left(\exists N \in \integers\right)
 \left(\forall k \in \integers_{>N} \right)
 \left(|F_k - L | < \varepsilon\right).
\end{equation}
Choose $L = 0$. Let  $\varepsilon \in \reals$. Choose $N =  \lceil \sqrt{\frac{1}{\varepsilon}} \rceil $. 
Then $N \in \integers$ as required.  Let $k \in \integers_{> N}$. We have
\begin{align*}
   |F_k -L | &= |\frac{(-1)^k}{k^2}|, \\
                 &= \frac{1}{k^2}, \\
                 &< \frac{1}{N^2}, \\
                 &=  \frac{1}{\lceil \sqrt{\frac{1}{\varepsilon}} \rceil^2}, \\
                 &\leq \frac{1}{\sqrt{\frac{1}{\varepsilon}}^2}, 
                 &= \varepsilon.
\end{align*}
\end{solution}
\question Show that the sequence $F = k \in \integers_{\geq 1} \mapsto \begin{cases} k! & k < 10^9 \\
\frac{(-1)^k}{k^2} & k \geq 10^9 \end{cases}$ converges.

\begin{solution} This is just like the previous problem, but we need to choose $N = \max(10^9,  \lceil \sqrt{\frac{1}{\varepsilon}} \rceil)$.

\end{solution}

\question Show that the sequence $F = k \in \integers_{\geq 1} \mapsto \frac{3 k+ 1}{2 k + 8}$ converges.

\begin{solution}
\begin{proof} We'll show that $F$ converges to $\frac{3}{2}$. Specifically,
  we'll show that 
  \begin{equation*}
    \left(\varepsilon \in \reals_{> 0} \right)
    \left(\exists N \in \integers_{> 0} \right)
    \left(\forall k \in \integers_{> N} \right)
   \left( \left| \frac{ 3k+1}{2 k + 8} - \frac{3}{2} \right| < \varepsilon \right).
  \end{equation*}
Let $\varepsilon \in \reals_{> 0}$. Choose $N = \lceil \frac{11}{2 \varepsilon} \rceil$.
Then as required, $N \in \integers_{> 0}$. Let $k \in \integers_{>N}$. 
We have
\begin{align*}
 \left| \frac{ 3k+1}{2 k + 8} - \frac{3}{2} \right| &= \frac{11}{2 \left( k\operatorname{+}4\right) }, 
                &&\mbox{(\text{ algebra})},\\
                &< \frac{11}{2 N}, && (k > N) \\
                &=  \frac{11}{2 \lceil \frac{11}{2} \rceil}, && \mbox{(substitution)} \\
                &\leq \frac{11}{2 \frac{11}{2 \varepsilon}},  && \mbox{(ceiling function property)} \\
                &=  \varepsilon. &&\mbox{(\text{ algebra})},
\end{align*}

\end{proof}
\end{solution}
\question Show that the sequence $F = k \in \integers_{\geq 1} \mapsto k - 3 \lfloor \frac{k}{3} \rfloor $ does 
not converge to $1$.

\begin{solution}
\begin{proof}  We'll show that
\begin{equation*}
  \left(\exists \varepsilon \in \reals_{> 0} \right)
  \left(\forall  N \in \integers_{> 0} \right)
   \left(\exists  k \in \integers_{> N} \right)
   \left(  | F_k - 1 | \geq \varepsilon \right)
\end{equation*}
Choose $\varepsilon = \frac{1}{2}$. Let $N \in \integers_{>0}$. Choose $k = 3 N$. Then $k \in \in \integers_{> N}$
as required. We have $|F_k - 1| = |F_{3 N} -1 | = 1 > \varepsilon$
\end{proof}
\end{solution}

\question Using the definition from the QRS, show that the interval $(-\infty, 8)$ is open.

\begin{solution}
\begin{proof} We'll show that 
\begin{equation*}
   \left(\forall x \in (-\infty, 8)\right) \left(\exists r \in \reals_{>0}\right)(\ball(x,r)  \subset (-\infty, 8))
\end{equation*}
Let $x \in   (-\infty, 8)$. Choose $r = 4 - \frac{x}{2}$. Since $x < 8$, we have $r > 0$, as required.  
We have 
\begin{equation*}
x + r = x + 4  - \frac{x}{2} = \frac{x}{2} + 4 < \frac{8}{2} + 4 = 8.
\end{equation*}
So $\ball(x,r)  \subset (-\infty, 8)$.
\end{proof}
\end{solution}

\question Let $A \subset \reals$.  Using the definition of an open set in the QRS, write
the undefintion of an open set. That is, complete the statement:
\begin{equation*}
  A \text{ is not open} \equiv 
\end{equation*}

\begin{solution}
\begin{equation*}
  A \text{ is not open} \equiv \left(\exists a \in A\right) \left(\forall r \in \reals_{>0}\right) \left(\ball(a,r) \not \subset A \right).
 \end{equation*}
\end{solution}
\question Using the undefition from the previous question, show that the set \mbox{$(-\infty, 8) \cup \{9 \}$} is
not open.

\begin{solution}
\begin{proof}
Choose $a = 9$. Let $r \in \reals_{>0}$. Then $\ball(9,r) \not \subset (-\infty, 8) \cup \{9 \}$.
\end{proof}
\end{solution}
\question Let $A \subset \reals$.  Using the definition of a limit point in the QRS, write
the undefintion of limit point. That is, complete the statement:
\begin{equation*}
  x \not \in \lp(A) \equiv 
\end{equation*}
\question  Use your undefinition from the previous question to show that \(5 \notin  \lp (\integers)\).


\begin{solution}%[1.5in]

\end{solution}

\question  Use the QRS defintion of a \emph{boundary point} to show that $12 \in \bp((0,12))$.


\begin{solution}%[2.5in]
Let \(\delta\) be a positive number, and let \(x^\star = \begin{cases} 12 - \delta/2 & \delta < 24 \\
                                                                      6  & \delta \geq 24 
\end{cases}\). Then \(x^\star \in (12-\delta, 12+ \delta)\) and \(x^\star \in (0,12)\). Further
we have \(12 + \delta /2 \notin (0,12)\) and \(12 + \delta/2 \in (12-\delta, 12+ \delta)\).

\textbf{Alternative} For every positive number \(\delta\), we have
\( (0,12) \cap B(12, \delta)  = (\mbox{max} \{0, 12 - \delta \}, 12) \neq \varnothing\). 
Further \( (0,12)^C \cap B(12, \delta)  = (12, 12 + \delta) \neq \varnothing\). 

\end{solution}

\question  Use the result of the previous question to show that \((0,12)\) is not
closed.

\begin{solution}%[1.0in]
A closed set contains all of its boundary points. We showed that 12 is a boundary point \((0,12)\), but
\(12 \notin (0,12)\); therefore \((0,12)\) is not closed.
\end{solution}


\question  Show that the set \(\reals\) is not compact by 
showing that there is an open cover of \( \reals \) that has no
finite subcover.

\begin{solution} See classnotes for Monday 9 October.

\end{solution}

\question  Show that the set \(\mathbf{Z}\) is not compact by 
showing that there is an open cover of \( \mathbf{Z} \) that has no
finite subcover.

\begin{solution}%[3.5in]
  Define \(\mathcal{C} = \{(0,k) \subset \reals \, | \, k \in \mathbf{N}\). Since 
  \(\underset{x \in \mathcal{C}}{\cap} x = (0,\infty)\), the set \(\mathcal{C}\)
is a cover for  \(\mathbf{N}\). Let  \(\mathcal{C}^\prime\) be any finite subset
of  \(\mathcal{C}\). Then \(\underset{x \in \mathcal{C}^\prime}{\cap} x\) is
bounded because it's a finite union of bounded sets.  But \(\mathbf{N}\) isn't
bounded, so \(\mathbf{N} \not \subset \underset{x \in \mathcal{C}^\prime}{\cap} x\);
therefore \(\mathbf{N}\) isn't compact.
\end{solution}

%\newpage

\question Let \(F\) be a convergent sequence, and let \(\alpha \in
\reals\).  Show that \(\alpha F\) is a convergent sequence.



\begin{solution}%[3.0in]
Let \(\varepsilon > 0\). Since \(f\) converges, there is a number \(L\) and 
\(M \in \mathbf{N}\) such that for all \(k > M\), we have \(|f_k - L| < \frac{\varepsilon}{1 + |\alpha|}\).
For all \(k > M\), we have
\[
  |\alpha f_k - \alpha L| = |\alpha| |f_k - L|
                          < \frac{ |\alpha|}{1 + |\alpha|} \varepsilon
                          < \varepsilon.
\]

\end{solution}

\question Let $F$ be a convergent sequence and suppose $\range(F) \subset([0,\infty))$. Show that $\sqrt{F}$ converges.
You may use the fact that $\left(\forall x,y \in \reals_{\geq 0}\right)  (| \sqrt{x} - \sqrt{y} | \leq \sqrt{|x - y|} )$

\begin{solution}
Let \(\varepsilon > 0\). Define \(\delta = \mbox{min}\{1, \varepsilon^2 / 5\}\). 
For \(x \in B^\prime(2, \delta)\), we
have \(x - 2 \in [-1,1]\). Thus \(x + 2 \in [3,5]\); consequently
\(|x+2| \leq 5\). Again, for \(x \in B^\prime(2, \delta)\), we have
\begin{align*}
  |\sqrt{1+x^2} - \sqrt{5} | &\leq \sqrt{|x^2 - 4|}, \\ 
                             &= \sqrt{|x + 2|} \sqrt{|x - 2|}, \\
                             &< \sqrt{5 \delta}, \\
                             &< \varepsilon.
\end{align*}
\end{solution}

\question For the sequence $F  = k \in \integers_{\geq 0} \mapsto k - 3 \lfloor \frac{k}{3} \rfloor $, give three examples of
a convergent subsequence.

\question Give an example of a sequence $F$ and a real number $\alpha$ such that $\alpha F$ converges and $F$ diverges.
\begin{solution} Choose $F = k \in \integers_{\geq 0} \mapsto k$ and choose $\alpha = 0$. Then $F$ diverges but $0F$ 
converges.
\end{solution}
 \question Give an example of sequences $F$ and $G$ such that both $F$ and $G$ diverge, but $F+ G$ converges.
 \begin{solution} Choose $F = k \in \integers_{\geq 0} \mapsto k$ and $F = k \in \integers_{\geq 0} \mapsto -k$.
 Both $F$ and $G$ diverge, but $F+G$ converges.
 \end{solution}
 
  \question Give an example of sequences $F$ and $G$ such that both $F$ and $G$ diverge, but $F G$ converges.
 \begin{solution} Choose $F = k \in \integers_{\geq 0} \mapsto (-1)^k$ and $G = k \in \integers_{\geq 0} \mapsto (-1)^k$.
 Both $F$ and $G$ diverge, but $F G =  k \in \integers_{\geq 0} \mapsto 1$ converges.
 \end{solution}
\question Show that $\left(\forall x,y \in \reals_{\geq 0}\right)(\sqrt{x+y} \leq \sqrt{x} +\sqrt{y})$.

\begin{solution} We begin by proving that $\left(\forall x,y \in \reals_{\geq 0}\right)(\sqrt{x^2 + y^2} \leq x + y) $.
Let $x,y \in \reals_{\geq 0}$. We have
\begin{align*}
\sqrt{x^2 + y^2} &= \sqrt{x^2 + 2 x y  y^2 - 2 x y}, &&\mbox{(add and subtract)} \\
                           &= \sqrt{(x + y)^2 - 2 x y}, &&\mbox{(algebra)} \\
                           &\leq \sqrt{(x + y)^2}, &&\mbox{(squre root function is increasing} \\
                           &= x + y.  &&\mbox{(algebra)} 
\end{align*}
To finish the proof, we replace $x \to \sqrt{x}$ and  $y \to \sqrt{y}$ in the above. This yields
$\sqrt{x+y} \leq \sqrt{x} + \sqrt{y}$.

\end{solution}
\end{questions}
\end{document}