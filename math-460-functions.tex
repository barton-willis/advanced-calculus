\documentclass[fleqn]{beamer}
%\usetheme[height=7mm]{Rochester}
\usetheme{Boadilla} %{Rochester}

\setbeamertemplate{footline}[text line]{%
\parbox{\linewidth}{\vspace*{-8pt}\hfill\insertshortauthor\hfill\insertpagenumber}}
\setbeamertemplate{navigation symbols}{}
%\author[BW]{Dr.\ Barton Willis}
\usepackage{amsmath}\usepackage{amsthm}
\usepackage{isomath}
\usepackage{upgreek}
\usepackage{comment,enumerate,xcolor}

\usepackage[english]{babel}
\usepackage[final,babel]{microtype}%\usepackage[dvipsnames]{color}
\usefonttheme{professionalfonts}
%\usefonttheme{serif}

\newcommand{\reals}{\mathbf{R}}
\newcommand{\complex}{\mathbf{C}}
\newcommand{\integers}{\mathbf{Z}}
\newcommand{\rationals}{\mathbf{Q}}
\DeclareMathOperator{\range}{range}
\DeclareMathOperator{\domain}{dom}
\DeclareMathOperator{\dom}{dom}
\DeclareMathOperator{\codomain}{codomain}
\DeclareMathOperator{\sspan}{span}
\DeclareMathOperator{\F}{F}
\DeclareMathOperator{\G}{G}
\DeclareMathOperator{\B}{B}
\DeclareMathOperator{\D}{D}
\DeclareMathOperator{\id}{id}
\DeclareMathOperator{\ball}{ball}

\usepackage{graphicx}
\usepackage{color}
\usepackage{amsmath}
\DeclareMathOperator{\nullspace}{nullity}
\theoremstyle{definition}
\newtheorem{mydef}{Definition}
\newtheorem{myqdef}{Quasi-definition}
\newtheorem{myex}{Example}
\newtheorem{myth}{Theorem} 
\newtheorem{myfact}{Fact}
\newtheorem{metathm}{Meta Theorem}
\newtheorem{Question}{Question}
\newtheorem{Answer}{Answer}
\newtheorem{myproof}{Proof}

\newtheorem{mycounterexample}{Counterexample}
\newtheorem{hurestic}{Hurestic}

%\usepackage{array}   % for \newcolumntype macro
%\newcolumntype{L}{>{$}l<{$}} % math-mode version of "l" column type

\newenvironment{alphalist}{
  \vspace{-0.4in}
  \begin{enumerate}[(a)]
    \addtolength{\itemsep}{1.0\itemsep}}
  {\end{enumerate}}



\usepackage{pifont}

\newenvironment{checklist}{
  \begin{enumerate}[\ding{51}]
    \addtolength{\itemsep}{-0.0\itemsep}}
  {\end{enumerate}}

\newenvironment{numberlist}
   {\begin{enumerate}[(1)]
       \addtolength{\itemsep}{-0.5\itemsep}}
     {\end{enumerate}}
\usepackage{amsfonts}
\makeatletter
\def\amsbb{\use@mathgroup \M@U \symAMSb}
\makeatother
\usepackage{bbold}



\newcommand{\llnot}{\lnot \,} % is accepted


%\subtitle{Lesson 7}
\title{\textbf{Functions}}
%\author[Barton Willis] % (optional, for multiple authors)
%{Barton~Willis}%
%\institute[UNK] % (optional)

%{
 % \inst{1}%
%  ``The secret of getting ahead is getting started.'' Mark Twain
%   }
  \date{}
 

%\usepackage{courier}
%\lstset{basicstyle=\ttfamily\footnotesize,breaklines=true}
%\lstset{framextopmargin=50pt,frame=bottomline}


%\begin{document}



%--------
%usepackage[usenames,dvipsnames,svgnames,table]{color}



\begin{document}

\frame{\titlepage}


\begin{frame}{Functions}

\begin{checklist}

\item There is an elegant way of defining a function from at set \(A\) to a set \(B\)
  purely in terms of a subset of \(A \times B\).

\vspace{0.2in}

\item And as we have seen, members of \(A \times B\) can be defined purely in terms 
of sets.

\vspace{0.2in}

\item But we don't often think of a function this way.

\vspace{0.2in}

\item Functions usually have a formula (or recipe) for determining the output for 
every input. But sometimes there is no known recipe--for example
\[
    F(x) = \begin{cases}   1   & x \in \rationals \\   -1   & x \notin \rationals  \end{cases}
 \]
 Last I checked, nobody knows the value of \(F(\uppi - \mathrm{e}) \).    
 But certainly \(F(107) = 1 \) and  \(F(\sqrt{2}) = -1 \) .
     
%\item Rather, we'll think of a function has a machine-like thing that produces an output (a member of the \emph{range}) from an input (a member of the \emph{domain}).

\end{checklist}

\end{frame}

\begin{frame}{Functions}

To define a function \(F\) with domain \(A\) and formula \(\mbox{blob}\), we can 
write
\[
  F = x \in A \mapsto \mbox{blob}.
\]
In the rare cases that it's important to give the function a codomain, we can write
\[
  F = x \in A \mapsto \mbox{blob} \in B,
\]
where \(\codomain(F) = B\). Generically for a function \(F\) with domain \(A\) and 
codomain \(B\),  we say that \(F\) is a function from \(A\) to \(B\).

\begin{example} The notation
\[
   F = x \in [-1,1] \mapsto 2 x + 1
\]
is our compact way of writing: Define \(F(x) = 2x + 1 \) and \(\dom(F) = [-1,1] \).
\end{example}
\end{frame}

\begin{frame}{Function signature}

The notation \(F : A \to B\) means

\begin{enumerate}

\item \(F\) is a function.
\item \(\dom(F) = A \).
\item  \(\codomain(F) = B\).
\end{enumerate}

\vspace{0.1in}


We'll say that  \(A \to B\) is the \emph{signature} of a function.  The signature of a function doesn't tell us its formula. It does tell us the domain of a function and it indicates what the outputs of the function can be.


\end{frame}
\begin{frame}{Range}
\begin{definition} For any function, we define
\[
   \range(F) = \left \{F(x)  \mid  x \in \domain(F)  \right \}.
\]
Thus \(\range(F)\) is the set of all outputs.
\end{definition}

\begin{myfact} Let \(F\) be a function. Then
\[
     \left[ y \in \range(F)  \right] \equiv  \left(\exists x \in \domain(F) \right)(y = F(x)).
 \]
\end{myfact}

\begin{example} Define \(F = x \in [-1,1] \mapsto 2 x + 1\). Then \(\frac{3}{2} \in \range(F)\) because \(\frac{1}{4} \in \domain(F)\) and \(F(\frac{1}{4}) = \frac{3}{2}\).

\end{example}
\end{frame}
\begin{frame}{Ontoness}

The codomain of a function tells us something about its outputs, but remember that the range and the codomain of a function need not be the same. For all functions \(F\), we have
\[
   \range{F} \subset \codomain(F).
\]

\begin{mydef} A function is \emph{onto} if its range and codomain are equal. \end{mydef}

\begin{myex} \textbf{Question}: Is the sine function onto?  \textbf{Answer} It is if its codomain is \([-1,1]\).  But if its codomain is \(\reals\), then no it's not onto. There is no standard value for the codomain of the trigonometric functions, so the asking ``Is the sine function onto?'' is  rubbish.\end{myex}
\end{frame}
\begin{frame}{Equality}

\begin{mydef} Functions \(F\) and \(G\) are \emph{equal }  \(\dom(F) = \dom(G)\) and for all \(x \in \dom(F) \), we have \(F(x) = G(x)\).  Equivalently
\[
  (F = G) \equiv (\dom(F) = \dom(G)) \land (\forall  x \in \dom(F))(F(x) = G(x)).
\]
\end{mydef}

\begin{enumerate}
\item The definition of function equality does not involve the codomain of the function. Thus two functions can be equal, but have unequal codomains. 
\end{enumerate}

\begin{myex}  The functions \(F = x \in [-1,1] \mapsto x \in [-1,1]\) and  \(G = x \in [-1,1] \mapsto x \in \reals \)  are equal, but \(F\) is onto and \(G\) is not onto. Thus
ontoness isn't a  property of a function. \end{myex}


\end{frame}




\begin{frame}{Apply a function to a set}

\begin{mydef}  Let \(F : A \to B\).  For any subset \(A^\prime\) of \(A\) define
\[
    F(A^\prime) = \{ F(x) | x \in A^\prime \}.
\]
Equivalently, we have
\[
  y \in F(A^\prime)  \equiv (\exists x \in A^\prime)(y = F(x)).
\]
\end{mydef}  

\begin{myth} For all functions \(F\), we have \(F(\dom{F}) = \range(F) \).   Further \(F(\varnothing) = \varnothing \). \end{myth}


\end{frame}

\begin{frame}{Inverse image}

\begin{mydef}  Let \(F : A \to B\).  For any subset \(B^\prime\) of \(B\) define
\[
    F^{-1} (B^\prime) = \{ x \in A  | F(x) \in B \}.
\]
Equivalently, we have
\[
  x  \in F^{-1} (B^\prime)  \equiv F(x) \in B.
\]
\end{mydef}
\end{frame}  

\begin{frame}

\begin{myth}  Let \(F : X \to Y\) and let \(A\) and \(B\) be subsets of \(X\).  Then  \(F(A \cap B) = F(A) \cap F(B) \).
\end{myth}

\begin{myproof}  Suppose \(y \in (F(A \cap B)\); we'll show that \(y \in F(A) \cap F(B) \).   Since \(y \in (F(A \cap B)\), there is
\(x \in A \cap B\) such that \(y = F(x)\).  But \(x \in A \cap B\) implies either \(x \in A\) or \(x \in B\).  If \(x \in A\), we have \(y \in F(A\); similarly if \(x \in B\), we have \(y \in F(B)\). So
either \(y \in F(A)\) or \(y \in F(B)\) ; therefore  \(y \in F(A) \cap F(B) \).  

 \quad Suppose  \(y \in F(A) \cap F(B) \).  We'll show that \(y \in F(A \cap B)\).


\end{myproof}  



\end{frame}
\end{document}

