\documentclass[fleqn]{beamer}
%\usetheme[height=7mm]{Rochester}
\usetheme{Boadilla} %{Rochester}

\setbeamertemplate{footline}[text line]{%
\parbox{\linewidth}{\vspace*{-8pt}\hfill\insertshortauthor\hfill\insertpagenumber}}
\setbeamertemplate{navigation symbols}{}
%\author[BW]{Dr.\ Barton Willis}
\usepackage{amsmath}\usepackage{amsthm}
\usepackage{isomath}
\usepackage{upgreek}
\usepackage{comment,enumerate,xcolor}

\usepackage[english]{babel}
\usepackage[final,babel]{microtype}%\usepackage[dvipsnames]{color}
\usefonttheme{professionalfonts}
%\usefonttheme{serif}

\newcommand{\reals}{\mathbf{R}}
\newcommand{\complex}{\mathbf{C}}
\newcommand{\integers}{\mathbf{Z}}
\DeclareMathOperator{\range}{range}
\DeclareMathOperator{\domain}{dom}
\DeclareMathOperator{\dom}{dom}
\DeclareMathOperator{\codomain}{codomain}
\DeclareMathOperator{\sspan}{span}
\DeclareMathOperator{\F}{F}
\DeclareMathOperator{\G}{G}
\DeclareMathOperator{\B}{B}
\DeclareMathOperator{\D}{D}
\DeclareMathOperator{\id}{id}
\DeclareMathOperator{\ball}{ball}

\usepackage{graphicx}
\usepackage{color}
\usepackage{amsmath}
\DeclareMathOperator{\nullspace}{nullity}
\theoremstyle{definition}
\newtheorem{mydef}{Definition}
\newtheorem{myqdef}{Quasi-definition}
\newtheorem{myex}{Example}
\newtheorem{myth}{Proposition}
\newtheorem{myfact}{Fact}
\newtheorem{metathm}{Meta Theorem}
\newtheorem{Question}{Question}
\newtheorem{Answer}{Answer}
\newtheorem{myproof}{Proof}

\newtheorem{mycounterexample}{Counterexample}
\newtheorem{hurestic}{Hurestic}

%\usepackage{array}   % for \newcolumntype macro
%\newcolumntype{L}{>{$}l<{$}} % math-mode version of "l" column type

\newenvironment{alphalist}{
  %\vspace{-0.4in}
  \begin{enumerate}[(a)]
    \addtolength{\itemsep}{1.0\itemsep}}
  {\end{enumerate}}



\usepackage{pifont}

\newenvironment{checklist}{
  \begin{enumerate}[\ding{51}]
    \addtolength{\itemsep}{-0.0\itemsep}}
  {\end{enumerate}}

\newenvironment{numberlist}
   {\begin{enumerate}[(1)]
       \addtolength{\itemsep}{-0.5\itemsep}}
     {\end{enumerate}}
\usepackage{amsfonts}
\makeatletter
\def\amsbb{\use@mathgroup \M@U \symAMSb}
\makeatother
\usepackage{bbold}



\newcommand{\llnot}{\lnot \,} % is accepted


%\subtitle{Lesson 3}
\title{\textbf{Sets}}
%\author[Barton Willis] % (optional, for multiple authors)
%{Barton~Willis}%
%\institute[UNK] % (optional)

%{
 % \inst{1}%
%  ``The secret of getting ahead is getting started.'' Mark Twain
%   }
  \date{}


%\usepackage{courier}
%\lstset{basicstyle=\ttfamily\footnotesize,breaklines=true}
%\lstset{framextopmargin=50pt,frame=bottomline}


%\begin{document}



%--------
%usepackage[usenames,dvipsnames,svgnames,table]{color}



\begin{document}

\frame{\titlepage}

\begin{frame}{Sets}
\begin{myqdef} We don't attempt to define a set, but we describe a set as a collection of things, often called \emph{members}.   
The  members of a set can be numbers, ordered pairs,  functions, or sets themselves. In a bit, we'll learn that there are some things that might appear to be valid sets, but really are not sets. 
\end{myqdef}

\end{frame}

\begin{frame}
Here are some examples of sets:
\begin{alphalist}

\item  $\{46, 107\}$ is a set with two members, namely 46 and 107. We reserve the curly braces to delineate a set.

\item  $\{46, \{46, 107 \} \}$ is a set with two members, namely 46 and  \(\{46, 107 \}\).  One member of this set is an integer, but the other is a set with two members--that's OK.

\item $\{0,1,2,3, \dots \}$ is apparently the set of all nonnegative integers.  I say apparently because the ellipses (the $\dots$) isn't entirely clear.

\item $\{ \}$ is a set with no members. 
\end{alphalist}
\end{frame}
\begin{frame}{Named Sets}

We'll use the following names for subsets of real numbers:
\begin{align*}
&\reals = \mbox{the set of real numbers}, \\
& \amsbb{R} =  \mbox{the set of real numbers for handwritten text}, \\
&\reals_{> 0} = \{x \in \reals  \mid  x > 0\}, \\
&\reals_{\neq 0}  =  \{x \in \reals  \mid   x \neq 0\},  \mbox{(and similarly for other subscripts)} \\
&\integers = \mbox{the set of integers}, \\
& \amsbb{Z} = \mbox{ the set of integers for handwritten text}, \\
&\mathbf{Q} = \mbox{the set of rational numbers}, \\
& \amsbb{Q} = \mbox{ the set of rational numbers for handwritten text}, \\
&\varnothing = \mbox{A set with no members, that is the empty set}
\end{align*}

\end{frame}

\begin{frame}{Membership}

For a set $A$, we define a predicate (boolean valued function) as
\[
   x \in A = \begin{cases} \mathrm{T} & \mbox{ if }  x  \mbox{ is a member of } A \\
                                     \mathrm{F} &  \mbox{ if } x \mbox{ is not  member of } A   
                                     \end{cases}          
\]
For example:

\begin{alphalist}
\item $ 107 \in \{46, 107\} = \mathrm{T}$
\item $ \{107 \} \in \{46, \{107 \} \} = \mathrm{T}$
\item $ \{107 \} \in \{46, 107 \} = \mathrm{F}$
\end{alphalist}

\end{frame}

\begin{frame}{Set Operators}

\begin{mydef}
Let \(A\) and \(B\) be sets. Define the set \emph{union, intersection}, and \emph{difference}
\begin{align*}
   A \cap B &= \{x  \mid  (x \in A)\land (x \in B) \}, \\
   A \cup B &= \{x  \mid (x \in A) \lor (x \in B) \}, \\
   A \setminus B &= \{x  \mid  (x \in A) \land (x \notin B) \},
\end{align*}
respectively.
\end{mydef}
\end{frame}

\begin{frame}{Set (an) example}

\begin{myex}
We have
\begin{align*}
   \{6, 107\} \cap  \{28,107\} &= \{107\}, \\
      \{6, 107\} \cup  \{28,107\} &= \{6,28, 107\}, \\
         \{6, 107\} \setminus  \{28,107\} &= \{6\}, \\
            \{28, 107\} \setminus  \{6,107\} &= \{28\}.
\end{align*}
\end{myex}

\begin{checklist}

\item The last two examples show that in  general  \(A \setminus B \neq B \setminus A\).

\item The set difference is so much like real number subtraction, sometimes the symbol "-" is used instead of \(\setminus\).

\end{checklist}
\end{frame}


\begin{frame}{Set predicates}

\begin{mydef}
Let \(A\) and \(B\) be sets. Define
\begin{align*}
   A \subset B &\equiv (\forall x \in A)(x \in B),\\
   A = B &\equiv  (A \subset B) \land (B \subset A).
\end{align*}
\end{mydef}

Specializing \(A \subset B\) to \(A = \varnothing\) gives
\[
   \left[\varnothing \subset B \right] \equiv  (\forall x \in \varnothing)(x \in B) \equiv \mbox{true}.
\]
We've shown that:

\begin{myth} Thus for all sets \(A\) and for any empty set \(\varnothing\), we have \(\varnothing \subset A\).  \end{myth}


\end{frame}

\begin{frame}{Set equality}

To show that sets \(A\) and \(B\) are equal, we almost always prove that \(A \subset B\) and \(B \subset A\).  If a proposition has the form

\begin{myth} If \(H_1, H_2, \dots\), and \( H_n\), then \(A = B\). \end{myth}

where  \(H_1, H_2, \dots H_n\) is the hypothesis, a template for proving the theorem is
\begin{myproof} Suppose \(x \in A\).  We'll show that \(x \in B\).  Since  \(x \in A, H_1, H_2, \dots\) and \( H_n\),  we have \dots; therefore \(x \in B\).

\quad Suppose \(x \in B\).  We'll show that \(x \in A\).  Since  \(x \in B, H_1, H_2, \dots\) and  \(H_n\),  we have \dots; therefore \(x \in A\).

\end{myproof}

\begin{enumerate}

\item Notice how in the first case we append \(x \in A\) to the hypothesis; and in the second case, we append \(x \in B\).
\end{enumerate}

\end{frame}

\begin{frame}{Establish notation}

\begin{myth} The set union is associative. \end{myth}

\begin{myproof}  Let \(A,B\), and \(C\) be sets.  We'll show that \(A \cup (B \cup C) = (A \cup B) \cup C\).   Our proof uses the fact that
the disjunction is associative; we have
\begin{align*}
A \cup (B \cup C) &= \{x \mid (x \in A) \lor (x \in B \cup C) \}, \\
                                &= \{x \mid (x \in A) \lor (x \in B)  \lor  (x \in  C) \}, \\
                                            &= \{x \mid   ((x \in A) \lor  (x \in B))   \lor x \in C) \}, \\
                                &= (A \cup B) \cup C.
 \end{align*}
  \end{myproof}

  \begin{checklist}

  \item The statement of the proposition doesn't introduce notation, so the proof must do so.

  \item Alternatively, we can show that   \(  A \cup (B \cup C)  \subset  (A \cup B) \cup C) \) and  \( (A \cup B) \cup C) \subset A \cup (B \cup C)\).

  \end{checklist}

\end{frame}

\begin{frame}{Alternative proofs}


\begin{myproof}  Let \(A,B\), and \(C\) be sets.  We'll show that \(A \cup (B \cup C) =  (A \cup B) \cup C\).
We have
\begin{align*}
  x \in A \cup (B \cup C)  &\implies (x \in A) \lor (x \in B \cup C), \\
                                            &\implies  (x \in A) \lor ( (x \in  B) \lor (x \in C)), \\
                                           &  \implies  (  (x \in A) \lor  (x \in  B) ) \lor (x \in C), \\
                                           & \implies x \in  ( A \cup B)  \cup C.
\end{align*}
Similarly, we can show that \(   x \in (A \cup B) \cup C \implies x \in A \cup (B \cup C)\).
\end{myproof}

\end{frame}
\begin{frame}{The uniqueness of emptiness}

\begin{myth} There is at most one empty set. \end{myth}

\begin{myproof} Let \(O\) and \(O^\prime\) be empty sets.  Since  \(O\) is empty, we have \(O \subset O^\prime\). Similarly since   \(O^\prime \) is empty, we have \(O^\prime  \subset O\). We have shown that \(O \subset O^\prime\) and \(O^\prime  \subset O\); therefore \(O = O^\prime\). \end{myproof}

\begin{enumerate}

\item With impunity, we can now refer to  \textbf{the} empty set.

\item A clumsy  way to proof this is by contradiction. The proof assumes that there are empty sets \(O\) and \(O^\prime\), but \(O \neq O^\prime\).

\end{enumerate}
\end{frame}

\begin{frame}{Conflation}

\textbf{Question:}  True or false:  \(\varnothing = \{\varnothing \}\).


\vspace{0.2in}

\textbf{Answer:} It's false. We have \(\varnothing \in \{\varnothing \} \),
but \(\varnothing \not \in \varnothing\), so \(\varnothing \neq \{\varnothing \} \).




\end{frame}

\begin{frame}{A unique template}

If a proposition has the form
\begin{myth} If \(H_1, H_2, \dots\), and \( H_n\), there is at most one object \(X\). \end{myth}

A template for its proof is

\begin{myproof} Let \(X\) and \(X^\prime\) be such objects. Since \(H_1, H_2, \dots\), and \( H_n\), we have \dots.  ; therefore \(X = X^\prime\). \end{myproof}



\begin{enumerate}

\item When \(X\) and \(X^\prime \) are real numbers, we might prove \(X = X^\prime\) by showing that both \(X \leq X^\prime\) and  \(X^\prime  \leq X\) are true. Together,
these inequalities prove that  \(X = X^\prime\).
\end{enumerate}

\end{frame}
\begin{comment}
\begin{frame}{Generalized disjunctions}


Let \(I\) be a set. And suppose that every member of \(I\) is a statement. Define
\[
       \underset{x \in I}{\lor} x  \equiv (\exists x \in I)(x).
\]
In this context, the set \(I\) is called an \emph{index set}. When \(I\) is a finite set, say \(I = \{P_1, P_2, \dots P_n\}\), we have
\[
       \underset{x \in I}{\lor} x  \equiv  P_1 \lor  P_2 \lor \cdots \lor P_n.
\]
The disjunction is associative and commutative, so we don't need to parenthesize   \( P_1 \lor  P_2 \lor \cdots \lor P_n \).


\begin{myth}  Let \(I\) and \(I^\prime\) be sets, and suppose that every member of \(I\) is a statement and every member of \(I^\prime \) is a statement.
Then
\[
  \underset{x \in I \cup I^\prime}{\lor} x  \equiv    \left ( \underset{x \in I}{\lor}  x \right) \lor \left ( \underset{x \in I^\prime}{\lor} x \right).
\]
 \end{myth}


\end{frame}


\begin{frame}{Vacuous unions}

We have
\[
      \underset{x \in \varnothing}{\lor} x  \equiv (\exists x \in \varnothing)(x) \equiv \mbox{false}.
\]
Thus we have
\begin{align*}
     \underset{x \in I }{\lor} x \in I   &\equiv   \underset{x \in I  \cup \varnothing}{\lor} x \in I, \\
                                                               &\equiv  \left( \underset{x \in I }{\lor} x \in I \right)  \lor  \left( \underset{x \in \varnothing }{\lor} x \in I \right), \\
                                                               &\equiv  \left(  \underset{x \in I }{\lor} x \in I \right)  \lor  \mbox{ false }, \\
                                                               &\equiv  \underset{x \in I }{\lor} x \in I.
\end{align*}
Comparing the first and the last, we have the tautology
\[
    \underset{x \in I }{\lor} x \in I  \equiv \underset{x \in I }{\lor} x \in I .
    \]
\end{frame}
\end{comment}
\begin{frame}{Generalized unions}

Let \(I\) be a set. And suppose that every member of \(I\) is a set.  Since every member of \(I\) is a set, we can find the union of all of its members. We define
\[
    \underset{x \in I}{\cup} x  = \{ a  \, \,  | \, \, (\exists x \in I)(a \in x) \}.
\]
In this context, we say that the set \(I\) is an \emph{index set.}

\begin{myex}  Define \(I = \{ \{1,2\}, \{107\} \} \). Then \(I\) is a set and each member of \(I\) is a set. We have
\begin{align*}
 \underset{x \in I}{\cup} x  &= \{ a  \, \,  | \, \, (\exists x \in I)(a \in x) \}, \\
                                                   &= \{ a  \, \,  | \, \,   a \in   \{1,2\}  \lor    a \in   \{107\} \}, \\
                                                   &= \{1,2\}  \cup \{107\}, \\
                                                   &= \{1,2,107\}.
\end{align*}

\end{myex}
\end{frame}

\begin{frame}{Finite unions}
\begin{myth} Let \(A_1, A_2, \dots, A_n\) be sets.  Define an index set \(I\) by  \(I = \{A_1, A_2, \dots, A_n \} \). Then
\[
   \underset{x \in I}{\cup} x = A_1 \cup A_2 \cup \cdots \cup A_n.
\]
\end{myth}
\begin{enumerate}
  \item The set union is associative and commutative, so  the meaning of \(A_1 \cup A_2 \cup \cdots \cup A_n \) unambiguous.

  \item An index set neededn't be finite.
\end{enumerate}
\end{frame}

\begin{frame}{Nonfinite unions}

\begin{myex}  The index set needn't be finite--here is an example.  For \(x \in \reals\), define \( I = \{ (-\infty, x) \,\, | \,\, x \in \reals \} \). Our index set is a set of open intervals.
We claim that
\[
    \underset{x \in I}{\cup} x  = \reals.
\]
\end{myex}

\begin{myproof}  Suppose \(a \in   \underset{x \in I}{\cup} x  \).  We'll show that \(a \in \reals\).  Since  \(a \in   \underset{x \in I}{\cup} x  \), there is \(z \in I\) such that \(a \in z\).  But \(z \subset \reals\), so \(a \in \reals\); we've shown that   \( \underset{x \in I}{\cup} x  \subset \reals\).

\quad Suppose \(a \in   \reals\).   We'll show that \(a \in  \underset{x \in I}{\cup} x  \).  We have \(a \in (-\infty, a + 1)\). Further  \( (-\infty, a + 1) \in I\); therefore  \(a \in  \underset{x \in I}{\cup} x \).
\end{myproof}

\begin{enumerate}

\item Notice that \( a \not \in  (-\infty, a)\).  But it is true that \(a \in (-\infty, a + 1)\).

\item It's also true that  \(a \in (-\infty, a + 107 \, \uppi^2)\).
\end{enumerate}
\end{frame}

\begin{frame}{Generalized intersections}
Let \(I\) be a set. And suppose that every member of \(I\) is a set.  Since every member of \(I\) is a set, we can find the intersection of all of its members. We define
\[
    \underset{x \in I}{\cap} x  = \{ a  \, \,  | \, \, (\forall x \in I)(a \in x) \}.
\]


\begin{myex}  Define \(I = \{ \{1,2\}, \{107\} \} \). Then \(I\) is a set and each member of \(I\) is a set. We have
\begin{align*}
 \underset{x \in I}{\cap} x  &= \{ a  \, \,  | \, \, (\forall x \in I)(a \in x) \}, \\
                                                   &= \{ a  \, \,  | \, \,   a \in   \{1,2\}  \land    a \in   \{107\} \}, \\
                                                   &= \{1,2\}  \cap \{107\}, \\
                                                   &=  \varnothing.
\end{align*}

\end{myex}
\end{frame}

\begin{frame}{Finite intersections}
\begin{myth} Let \(A_1, A_2, \dots, A_n\) be sets.  Define an index set \(I\) by  \(I = \{A_1, A_2, \dots, A_n \} \). Then
\[
   \underset{x \in I}{\cap} x = A_1 \cap A_2 \cap  \cdots \cap A_n.
\]
\end{myth}
\begin{enumerate}
  \item The set intersection  is associative and commutative, so  the meaning of \(A_1 \cap A_2 \cap \cdots \cap A_n \) unambiguous.

  \item An index set needn't be finite.
\end{enumerate}
\end{frame}

\begin{frame}{Nonfinite intersections}

\begin{myex}  For \(x \in \reals\), define \( I = \{ (-\infty, x) \,\, | \,\, x \in \reals \} \). Our index set is a set of open intervals.
We claim that
\[
    \underset{x \in I}{\cap} x  = \varnothing.
\]
\end{myex}

\begin{myproof}   We'll prove this using contradiction.  Suppose  \( \underset{x \in I}{\cap} x \) has at least one member; say \(a \in  \underset{x \in I}{\cap} x \). We have
\[
    (\forall x \in \reals)(a \in (-\infty, x)).
\]
In particular, we have \(a \in (-\infty, a)\).  But \(a \in (-\infty, a)\) is false; therefore \( \underset{x \in I}{\cap} x \) cannot have a member, so \( \underset{x \in I}{\cap} x \)  is the empty set.
\end{myproof}



\end{frame}

\begin{frame}{Alternative notation}

Sometimes we take the index set to be a subset of \(\reals\) and we denote the sets members by subscripts. Say \(I \subset \reals\) and  \(A_x\) is a set for each \(x \in I\).  This notation is particlarly
popular when \(I = \integers_{> 0}\).  For example
\[
  \underset{k  \in \integers_{> 0}} {\cap} A_k   = \{ a \,\, | \,\, (\forall n \in  \integers_{> 0})(a \in A_n) \}
\]
And
\[
  \underset{k  \in \integers_{> 0}} {\cup} A_k   = \{ a \,\, | \,\, (\exists  n \in  \integers_{> 0})(a \in A_n) \}
\]
When the index set is uncountable, maybe it's just me, but definitions such as
\[
   A_x = (-\infty, x)   \mbox{  for all } x \in \reals
\]
are semi-bazaar looking. For such cases, I think it's more clear to define the index set to be a set of sets:
\[
   I  =  \{ (-\infty, x)   \,\, | \,\, x \in \reals\}.
\]


\end{frame}

\begin{comment}

\begin{frame}{Vacuous Truth}  Let \(T\) be a boolean valued function. Consider the statement
\[
   (\exists x \in \varnothing)(T(x)).
\]
To show that it's true, you would have to find a member of the empty set that makes \(T\) true. There are no members of the empty set, so for any predicate \(T\), we have
\[
     (\exists x \in \varnothing)(T(x)) \equiv \mbox{false}.
\]
Thus it's negation is true; that is
\[
    (\exists x \in \varnothing)(\lnot T(x)) \equiv \mbox{true}.
\]
\begin{myfact} For any boolean valued function \(T\), we have
 \begin{align*}
  (\exists x \in \varnothing)(T(x)) \equiv \mbox{false}, \\
   (\forall x \in \varnothing)(T(x)) \equiv \mbox{true}.
    \end{align*}
\end{myfact}

\end{frame}
\end{comment}

\begin{frame}{Functions}

To define a function \(F\) with domain \(A\) and formula \(\mbox{blob}\), we can write
\[
  F = x \in A \mapsto \mbox{blob}.
\]
In the rare cases that it's important to give the function a codomain, we can write
\[
  F = x \in A \mapsto \mbox{blob} \in B,
\]
where \(\codomain(F) = B\). Generically for a function \(F\) with domain \(A\) and codomain \(B\),  we say that \(F\) is a function from \(A\) to \(B\).

\begin{example} The notation
\[
   F = x \in [-1,1] \mapsto 2 x + 1
\]
is our compact way of writing: Define \(F(x) = 2x + 1 \), for \( -1 \leq x \leq 1\).
\end{example}
\end{frame}

\begin{frame}{Function signature}

The notation \(F : A \to B\) means

\begin{enumerate}

\item \(F\) is a function.
\item \(\dom(F) = A \).
\item  \(\codomain(F) = B\).
\end{enumerate}

\vspace{0.1in}


We'll say that  \(A \to B\) is the \emph{signature} of a function.  The signature of a function doesn't tell us its formula. It does tell us the domain of a function and it indicates what the outputs of the function can be.


\end{frame}
\begin{frame}{Range}
\begin{definition} For any function, we define
\[
   \range(F) = \left \{F(x) | x \in \domain(F)  \right \}.
\]
Thus \(\range(F)\) is the set of all outputs.
\end{definition}

\begin{myfact} Let \(F\) be a function. Then
\[
     \left[ y \in \range(F)  \right] \equiv  \left(\exists x \in \domain(F) \right)(y = F(x)).
 \]
\end{myfact}

\begin{example} Define \(F = x \in [-1,1] \mapsto 2 x + 1\). Then \(\frac{3}{2} \in \range(F)\) because \(\frac{1}{4} \in \domain(F)\) and \(F(\frac{1}{4}) = \frac{3}{2}\).

\end{example}
\end{frame}
\begin{frame}{Ontoness}

The codomain of a function tells us something about its outputs, but remember that the range and the codomain of a function need not be the same. For all functions \(F\), we have
\[
   \range{F} \subset \codomain(F).
\]

\begin{mydef} A function is \emph{onto} if its range and codomain are equal. \end{mydef}

\begin{myex} \textbf{Question}: Is the sine function onto?  \textbf{Answer} It is if its codomain is \([-1,1]\).  But if its codomain is \(\reals\), then no it's not onto. There is no standard value for the codomain of the trigonometric functions, so the asking ``Is the sine function onto?'' is  rubbish.\end{myex}
\end{frame}
\begin{frame}{Equality}

\begin{mydef} Functions \(F\) and \(G\) are \emph{equal }  \(\dom(F) = \dom(G)\) and for all \(x \in \dom(F) \), we have \(F(x) = G(x)\).  Equivalently
\[
  (F = G) \equiv (\dom(F) = \dom(G)) \land (\forall  x \in \dom(F))(F(x) = G(x)).
\]
\end{mydef}

\begin{enumerate}
\item The definition of function equality does not involve the codomain of the function. Thus two functions can be equal, but have unequal codomains.
\end{enumerate}

\begin{myex}  The functions \(F = x \in [-1,1] \mapsto x \in [-1,1]\) and  \(G = x \in [-1,1] \mapsto x \in \reals \)  are equal, but \(F\) is onto and \(G\) is not onto. Thus
ontoness isn't a  property of a function. \end{myex}


\end{frame}




\begin{frame}{Apply a function to a set}

\begin{mydef}  Let \(F : A \to B\).  For any subset \(A^\prime\) of \(A\) define
\[
    F(A^\prime) = \{ F(x) | x \in A^\prime \}.
\]
Equivalently, we have
\[
  y \in F(A^\prime)  \equiv (\exists x \in A^\prime)(y = F(x)).
\]
\end{mydef}

\begin{myth} For all functions \(F\), we have \(F(\dom{F}) = \range(F) \).   Further \(F(\varnothing) = \varnothing \). \end{myth}


\end{frame}

\begin{frame}{Inverse image}

\begin{mydef}  Let \(F : A \to B\).  For any subset \(B^\prime\) of \(B\) define
\[
    F^{-1} (B^\prime) = \{ x \in A  | F(x) \in B \}.
\]
Equivalently, we have
\[
  x  \in F^{-1} (B^\prime)  \equiv F(x) \in B.
\]
\end{mydef}
\end{frame}
\end{document}
