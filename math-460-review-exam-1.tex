\documentclass[12pt,fleqn,answers]{exam}
\usepackage{pifont}
\usepackage{dingbat}
\usepackage{amsmath,amssymb,amsthm}
\usepackage{epsfig}
\usepackage[]{hyperref}
\usepackage{geometry}

\geometry{letterpaper, margin=0.5in}
\addpoints
\boxedpoints
\pointsinmargin
\pointname{pts}

\newtheorem{prop}{Proposition}

\usepackage{enumerate}
\newenvironment{alphalist}{
  %\vspace{-0.4in}
  \begin{enumerate}[(a)]
    \addtolength{\itemsep}{0.0\itemsep}}
  {\end{enumerate}}
\frenchspacing
\usepackage[activate={true,nocompatibility},final,tracking=true,kerning=true,factor=1100,stretch=10,shrink=10]{microtype}
\usepackage[american]{babel}
%\usepackage[T1]{fontenc}
\usepackage{fourier}
\usepackage{isomath}
\usepackage{upgreek,amsmath}
\usepackage{amssymb}

\newcommand{\dotprod}{\, {\scriptzcriptztyle
    \stackrel{\bullet}{{}}}\,}

\newcommand{\reals}{\mathbf{R}}
\newcommand{\lub}{\mathrm{lub}} 
\newcommand{\glb}{\mathrm{glb}} 
\newcommand{\complex}{\mathbf{C}}
\newcommand{\dom}{\mbox{dom}}
\newcommand{\cover}{{\mathcal C}}
\newcommand{\integers}{\mathbf{Z}}
\newcommand{\vi}{\, \mathbf{i}}
\newcommand{\vj}{\, \mathbf{j}}
\newcommand{\vk}{\, \mathbf{k}}
\newcommand{\bi}{\, \mathbf{i}}
\newcommand{\bj}{\, \mathbf{j}}
\newcommand{\bk}{\, \mathbf{k}}
\DeclareMathOperator{\Arg}{\mathrm{Arg}}
\DeclareMathOperator{\Ln}{\mathrm{Ln}}
\newcommand{\imag}{\, \mathrm{i}}
\newcommand{\true}{\mbox{true}}
\usepackage{graphicx}
\newcommand\AM{{\sc am}}
\newcommand\PM{{\sc pm}}
     
\usepackage{color}
\shadedsolutions
\definecolor{SolutionColor}{rgb}{0.8,0.9,1}

\newcommand{\quiz}{Review for Exam I}
\newcommand{\term}{Fall}
\newcommand{\due}{Saturday 10 September  at 11:59 \PM}
\begin{document}
\large
\noindent{\textbf \quiz}
%\vspace{0.1in}
%\noindent\makebox[3.0truein][l]{{\bf MATH 460}}
%{\bf Name:}  \\
%\noindent \makebox[3.0truein][l]{\bf Homework   \quiz, \term \/ \the\year}
%{\bf Row:}\hrulefill\
%\vspace{0.1in}


\begin{questions} 

\question Let $A$ and $B$ be subsets of $\reals$. Show that if $A$ and $B$ are bounded above,
then $A \cup B$ is bounded above.  You may use the fact that for real numbers $a$ and $b$, we have
$a \leq \max(a,b)$ and $b \leq \max(a,b)$.

\begin{solution}
\begin{proof} Let  $A$ and $B$ be subsets of $\reals$ that are bounded above. We'll show that $A \cup B$ 
is bounded above. Since $A$ and $B$ are bounded above, there are $p,q \in \reals$ such that
for all $a \in A$, we have $a \leq p$ and for all $b \in A$, we have $b \leq q$.  Define $M = \max(p,q)$. 
We'll show that $M$ is an upper bound for $A \cup B$. Let $x \in A \cup B$. Either $x \in A$ or $x \in B$.
If $x \in A$, we have $x \leq p \leq M$. Similarly, if If $x \in B$, we have $x \leq q \leq M$. \qedhere

\end{proof}
\end{solution}

\question Give an example of a subset of $\reals$ that does not have a least upper bound.

\begin{solution} The set $\reals$ is a subset of $\reals$ that does not have a least upper bound.
\end{solution}

\question Give an example of a subset $A$ of $\reals$ such that $\lub(A) \in A$.
\begin{solution} We have $\lub((0,1]) = 1$ and $1 \in (0,1])$.
\end{solution}

\question Give an example of a subset $A$ of $\reals$ such that $\lub(A) \notin A$.
\begin{solution} We have $\lub((0,1)) = 1$ and $1 \notin (0,1])$.
\end{solution}

\question Show that $\lub((-\infty, 2)) = 2$.

\begin{solution} We'll show that 2 is an upper bound for $(-\infty, 2)$ and that for all $r \in \reals_{>0}$, 
there is $a \in (-\infty, 2)$ such that $2 - r < a$.

The fact that 2 is an upper bound for $(-\infty, 2)$ is apparent.  Let $r \in \reals_{>0}$. Choose
$a = 2 - r/2$. Then $a \in (-\infty, 2)$ as required.  Further since $r >0$, we have $2 - r < 2 - r/2$.
\end{solution}


\question Show that $\lub([0, 2)) = 2$.

\begin{solution} We'll show that 2 is an upper bound for $([0, 2)$ and that for all $r \in \reals_{>0}$, 
there is $a \in ([0, 2)$ such that $2 - r < a$.

The fact that 2 is an upper bound for $[0, 2)$ is apparent.  Let $r \in \reals_{>0}$. Choose
$a = \max(1, 2- r/2)$. Then $a \in [0, 2)$ as required.  For $r < 2$, we have $a = 2 - r/2$.
Since $r > 0$, we have $a < 2$, For $r \geq 2$, we have $a=1$. We have $1 < 2$ as required.

\end{solution}
\question Let $A$ be a subset of $\reals$. Show that $A$ has at most one least upper bound.

\begin{solution}  See class notes.

\end{solution}

\question Write a proof for

\begin{prop}
  For all $x,y \in \reals$, there is $a \in \reals$ such that
    $x < y$ implies $x<a<y$. \label{p1}   
\end{prop}

\begin{solution}  See class notes.

\end{solution}

\question Write a proof for
    \begin{prop}   For all $x \in \reals_{>0}$ there is 
      $y \in \reals_{> 0}$ such that $y < x$.  \label{p5}
    \end{prop}
    
    \begin{solution}  See class notes.

\end{solution}
    
\question Without explicitly using negation, write the negation of 
        \begin{prop} There are $x,y \in \reals$ such that  $\sin(x) = \sin(y) \implies x = y$. 
     \end{prop} 
     
\question Either write a proof of 
     \begin{prop} There are $x,y \in \reals$ such that  $\sin(x) = \sin(y) \implies x = y$. 
     \end{prop} 
or write a proof of its negation.

\question Let $(\mathcal{F},+, \times)$ be a field and let $O$ be the additive identity and $I$ be the multiplicative
identity. Given that $O = I$, show that $\mathcal{F} = \{O\}$.

\begin{solution}  See class notes.

\end{solution}

\question Let $(\mathcal{F},+, \times)$ be a field. Show that for all $a,b \in \mathcal{F}$, we have $a \times b = a \times (-b)$.

\begin{solution}  See class notes.

\end{solution}
\question Let $(\mathcal{F},+, \times)$ be an ordered  field. For all $a,b,c \in \mathcal{F}$, show that $a < b$ and $c < 0$
implies $a \times c > b \times c$.
\begin{solution}  See class notes.

\end{solution}

\question Show that 
\[
    \left(\forall k \in \integers_{>1} \right) 
      \left(\frac{1}{k^2} \leq \frac{1}{k-1} - \frac{1}{k} \right).
\]

\begin{solution}
  We'll write our solution as a sequence of logical equivalences. Let $k \in 
  \integers_{>1}$. We have
  \begin{align*}
    \left[\frac{1}{k^2} \leq \frac{1}{k-1} - \frac{1}{k} \right] &\equiv
    \left[\frac{1}{k^2} - \frac{1}{k-1} + \frac{1}{k}  \leq 0 \right], &\mbox{(algebra)} \\
    &\equiv \left[ -\frac{1}{(k-1) k^2}  \leq 0 \right], &\mbox{(factor)} \\
    &\equiv \true. &\mbox{($k - 1 > 0$ and $k^2 > 0$)}
  \end{align*}
  
\end{solution}
\question Show that
\[
    \left(\forall x \in (-\infty, 1) \right)\left( \exists r \in \reals_{>0} \right)
    \left((x-r,x+r) \subset (-\infty, 1) \right).
\]

\begin{solution}
  We need to choose a number $r$ such that $x+r < 1$ and $0 < r$. 
  Thus $0 < r < 1-x$. One choice is $r = \frac{1-x}{2}$. Since $x < 1$,
  this choice does satisfy the condition $r > 0$.

  \textbf{Proof} Let $x \in (-\infty, 1) $. Choose $r = \frac{1-x}{2}$. Since  $x < 1$,
  it follows that $r \in  \reals_{>0}$ as required. Since $ r > 0$, the condition
  \( (x-r,x+r) \subset (-\infty, 1)\) is equivalent to $x + r < 1$.
  We have
  \[
    \left[x + r < 1 \right] \equiv  \left[x +  \frac{1-x}{2} < 1 \right]
    \equiv \left[\frac{1+x}{2} < 1 \right] \equiv \left[1+x < 2 \right]
    \equiv \left[ x < 1 \right] \equiv \true.
  \]
\end{solution}

\question Let $A,B$ be subsets of $\reals$ and let $A$ be bounded above.
Show that $A \setminus B$ is bounded above.

\begin{solution}
  Since $A$ is bounded above, there is $M \in \reals$ such that
  $(\forall x \in A)(x \leq M)$.  We will show that
  \[
    (\exists M^\prime \in \reals)(\forall x \in A \setminus B)(x \leq M^\prime).
  \]
  Choose $M^\prime= M$. Let $x \in A \setminus B$. Then $x \in A$; thus we have
  \[
     [x \leq M^\prime]  \equiv [x \leq M] \equiv \true.
  \]


\end{solution}

\question Give an example of subsets $A,B$ of $\reals$
such that $A \setminus B$ is bounded above, but $A$ is not bounded
above.

\begin{solution}
  One (of many) example is $A = \reals$ and $B = \reals$. Then
  $A$ is not bounded above, but $A \setminus B = \varnothing$,
  so  $A \setminus B$ is bounded above (because the empty set is bounded above).
\end{solution}

\question Define $F = x \in \reals \mapsto x^2$.  Enumerate the members of the set
\begin{equation*}
  F(\{-4,-1,0,1,4 \}).
\end{equation*}
\begin{solution}
  \[
    F(\{-4,-1,0,1,4 \}) = \{F(-4),F(-1),F(0),F(1),F(4) \} =
     \{0,1,16\}.
  \]
\end{solution}


\question Show that
\[
    \left(\forall a \in \reals \right) \left(\exists m \in \reals \right)
    \left(\forall x \in \reals \right) \left(x^2 - a^2 \geq m (x-a) \right).
\]
\begin{solution}
  We will write our proof as a sequence of logical 
  equivalences.   Let $a \in \reals$. Choose $m = 2 a$. Let $x \in \reals$. We have
  \begin{align*}
    \left [x^2 - a^2 \geq m (x-a) \right ] &=
    \left [x^2 - a^2 \geq 2a  (x-a) \right ], &\mbox{(substitution for $m$)} \\
    &=
    \left [x^2 -2 a (x-a) - a^2 \geq 0 \right ],  &\mbox{(algebra)}\\
    &=
    \left [x^2 -2 a  + a^2 \geq 0 \right ], &\mbox{(algebra)} \\
    &=
    \left [(x - a)^2 \geq 0 \right ], &\mbox{(factor)} \\
    &=
    \true.
  \end{align*}
  
 
\end{solution}

\end{questions}
\end{document}