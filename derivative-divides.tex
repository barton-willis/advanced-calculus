\documentclass[fleqn]{beamer}
%\usetheme[height=7mm]{Rochester}
\usetheme{Boadilla} %{Rochester}

\setbeamertemplate{footline}[text line]{%
\parbox{\linewidth}{\vspace*{-8pt}\hfill\insertshortauthor\hfill\insertpagenumber}}
\setbeamertemplate{navigation symbols}{}
%\author[BW]{Dr.\ Barton Willis}
\usepackage{amsmath}\usepackage{amsthm}
\usepackage{isomath}
\usepackage{upgreek}
\usepackage{comment,enumerate,xcolor}

\usepackage[english]{babel}
\usepackage[final,babel]{microtype}%\usepackage[dvipsnames]{color}
\usefonttheme{professionalfonts}
%\usefonttheme{serif}

\newcommand{\reals}{\mathbf{R}}
\newcommand{\complex}{\mathbf{C}}
\newcommand{\integers}{\mathbf{Z}}
\newcommand{\rationals}{\mathbf{Q}}
\DeclareMathOperator{\range}{range}
\DeclareMathOperator{\domain}{dom}
\DeclareMathOperator{\dom}{dom}
\DeclareMathOperator{\codomain}{codomain}
\DeclareMathOperator{\sspan}{span}
\DeclareMathOperator{\F}{F}
\DeclareMathOperator{\G}{G}
\DeclareMathOperator{\B}{B}
\DeclareMathOperator{\D}{D}
\DeclareMathOperator{\id}{id}
\DeclareMathOperator{\ball}{ball}

\usepackage{graphicx}
\usepackage{color}
\usepackage{amsmath}
\DeclareMathOperator{\nullspace}{nullity}
\theoremstyle{definition}
\newtheorem{mydef}{Definition}
\newtheorem{myqdef}{Quasi-definition}
\newtheorem{myex}{Example}
\newtheorem{myth}{Theorem} 
\newtheorem{myfact}{Fact}

\newtheorem{heuristic}{Heuristic}

\newtheorem{metathm}{Meta Theorem}
\newtheorem{Question}{Question}
\newtheorem{Answer}{Answer}
\newtheorem{myproof}{Proof}

\newtheorem{mycounterexample}{Counterexample}
\newtheorem{hurestic}{Hurestic}

%\usepackage{array}   % for \newcolumntype macro
%\newcolumntype{L}{>{$}l<{$}} % math-mode version of "l" column type

\newenvironment{alphalist}{
  \vspace{-0.4in}
  \begin{enumerate}[(a)]
    \addtolength{\itemsep}{1.0\itemsep}}
  {\end{enumerate}}



\usepackage{pifont}

\newenvironment{checklist}{
  \begin{enumerate}[\ding{51}]
    \addtolength{\itemsep}{-0.0\itemsep}}
  {\end{enumerate}}

\newenvironment{numberlist}
   {\begin{enumerate}[(1)]
       \addtolength{\itemsep}{-0.5\itemsep}}
     {\end{enumerate}}
\usepackage{amsfonts}
\makeatletter
\def\amsbb{\use@mathgroup \M@U \symAMSb}
\makeatother
\usepackage{bbold}

\usepackage{mathabx}


\newcommand{\llnot}{\lnot \,} % is accepted


\subtitle{AKA the U substitution}
\title{\textbf{Derivative divides integration}}
%\author[Barton Willis] % (optional, for multiple authors)
%{Barton~Willis}%
%\institute[UNK] % (optional)

%{
 % \inst{1}%
%  ``The secret of getting ahead is getting started.'' Mark Twain
%   }
  \date{}
 

%\usepackage{courier}
%\lstset{basicstyle=\ttfamily\footnotesize,breaklines=true}
%\lstset{framextopmargin=50pt,frame=bottomline}


%\begin{document}



%--------
%usepackage[usenames,dvipsnames,svgnames,table]{color}



\begin{document}

\frame{\titlepage}


\begin{frame}{Chain rule swapperoo}
  
Recall the chain rule; for differentiable functions \(F\) and \(g\) in function notation, we have
\[
     (F \circ g)^\prime = g^\prime F^\prime \circ g.
\]  
Equivalently in formula notation, the chain rule is
\[
    \frac{\mathrm{d}}{\mathrm{d} x} (F (g(x)) = g^\prime(x) 
    F^\prime(g(x)).
\]  
\begin{myfact}
Up to an additive constant, the antiderivative ``undoes'' the derivative; thus
\[
\int g^\prime(x) F^\prime(g(x)) \, \mathrm{d} x  =  F (g(x))  + c,
\]   
where \(c \in \reals\).
\end{myfact}
\end{frame} 

\begin{frame}{Match this}

If we can match an integrand to  \(  g^\prime(x) F^\prime(g(x))  \) and we know an antiderivative of \(F\),  \textbf{we win.}  An example
\[
    \int 2 x  \sin(x^2) \, \mathrm{d} x.
\]
The match isn't particularly hidden:

\begin{checklist}
\item \(g(x) = x^2   \implies g^\prime(x) = 2 x\),
\item \(F^\prime = \sin \implies F = -\cos \).
\end{checklist}

Thus
\[
    \int 2 x  \sin(x^2) \, \mathrm{d} x =  F(g(x)) + c = -\cos(x^2) + c.
\]
\end{frame} 

\begin{frame}{Out with the old, in with the new}
Let's re-do the problem  \(\int 2 x  \sin(x^2) \, \mathrm{d} x \), but organize our work differently.

\begin{checklist}
\item  The argument of \(\sin\) is \(x^2\).  Let's define a new variable \(u = x^2\).   Then
\[
    \mathrm{d} u = \frac{\mathrm{d} u}{\mathrm{d} x} \mathrm{d} x = 2 x  \, \mathrm{d} x.
\]

\item  We now need to write \(  2 x  \sin(x^2) \, \mathrm{d} x  \)  \emph{entirely} in terms of the new variable \(u\).  

\item When I say entirely, I mean \emph{entirely}. This includes expressing \(\mathrm{d} x \) in terms of  \(\mathrm{d} u \) .

\item Grouping the factor of \(2 x\) together with \(\mathrm{d} x\), we have 
\[
   2 x  \sin(x^2) \, \mathrm{d} x   = \sin(x^2)     (2 x \,  \mathrm{d} x  ) = \sin(u) \, \mathrm{d} u.
\]
\item So
\[
   \int 2 x  \sin(x^2) \, \mathrm{d} x   = \int  \sin(u) \, \mathrm{d} u = -\cos(u) = -\cos(x^2).
\]
\item For the step   \(-\cos(u) = -\cos(x^2) \) we reverted to the ``original'' integration variable.

\end{checklist}

\end{frame}

\begin{frame}{Example redux}

Let's try  \(\int x  \exp(x^2) \, \mathrm{d} x \)

\begin{checklist}
\item  The argument of \(\exp\) is \(x^2\).  Let's define a new variable \(u = x^2\).   Then
\[
    \mathrm{d} u = \frac{\mathrm{d} u}{\mathrm{d} x} \mathrm{d} x = 2 x  \, \mathrm{d} x.
\]

%\item  We now need to write \(  x  \exp(x^2) \, \mathrm{d} x  \)  \emph{entirely} in terms of the new variable \(u\).  

\item  Unlike the previous problem, the integrand is missing a factor of \(2\) for a complete matching.  No big deal; we have
\[
  [\mathrm{d} u  =  2 x  \, \mathrm{d} x ]  =   \left [\frac{1}{2} \mathrm{d} u  =  x  \, \mathrm{d} x \right ] 
\]



\item So  
\[
    x  \exp(x^2) \, \mathrm{d} x   = \exp(x^2)     (x \,  \mathrm{d} x  ) = \frac{1}{2} \exp(u) \, \mathrm{d} u.
\]
\item So
\[
   \int x  \exp(x^2) \, \mathrm{d} x   = \int   \frac{1}{2} \exp(u) \, \mathrm{d} u =   \frac{1}{2} \exp(u)=  \frac{1}{2} \exp(x^2).
\]
\item In step   \(\exp(u) = \exp(x^2) \) we reverted to the ``original'' integration variable.

\end{checklist}

\end{frame}


\begin{frame}{Derivative divides}

\begin{checklist}

\item When we match \(u = x^2\) to find \( \int  x  \exp(x^2) \, \mathrm{d} x \),  the derivative of \(u\), that is  
\(\frac{\mathrm{d} u}{\mathrm{d} x}\), doesn't exactly match the remaining factor
of \(x\),   but it does match up to a multiplicative  factor of 2.

\item Since the quotient of  \(\frac{\mathrm{d} u}{\mathrm{d} x} \)  divided by the  the remaining factor of \(x\) is a constant, the method is called \emph{derivative divides}.

\item But calculus books call the method  \emph{U substitution}

\item If you don't like my explanations, no problem; watch  \url{https://www.youtube.com/watch?v=8B31SAk1nD8}.

\end{checklist}
\end{frame}


\begin{frame}{Finding You}

\begin{heuristic}

\begin{checklist}

\item The integrand should be a product.

\item Choose \(u\) to be an expression that is ``inside'' a function with a known antiderivative.

\item The derivative of \(u\) times a constant should match the remaining factors of the integrand.
\end{checklist}

\end{heuristic}

\begin{checklist}

\item A heuristic is a guiding principle that often works, but occasionally  fails. 
\end{checklist} 
\end{frame}

\begin{frame}{Matching examples}


 \( \int x^2 \cos(x^3) \, \mathrm{d} x \).   The cosine has a known antiderivative.  The expression inside cosine is \(x^3\).  So choose \(u = x^3\).   Then
\(\frac{\mathrm{d} u}{\mathrm{d} x} = 3 x^2\) matches the remaining factor  \(x^2\) in the integrand up to a multiplicative factor.  \mbox{\textbf{We win}}.

\[
    [u = x^3] = [\mathrm{d} u = 3 x^2 \mathrm{d} x] = \left[  x^2 \mathrm{d} x = \frac{1}{3}  \mathrm{d} u \right].
\]
So
\[
 \int x^2 \cos(x^3) \, \mathrm{d} x = \int \frac{1}{3} \cos(u) \, \mathrm{d} u = \frac{1}{3} \sin(u) =  \frac{1}{3} \sin(x^3).
\]

\end{frame}

\begin{frame}{Actually, we are losers}

Actually, we are \textbf{losers}. Why?  Because we didn't check our work:

\[
   \frac{\mathrm{d}}{\mathrm{d} x} \left(  \frac{1}{3} \sin(x^3)   \right) =   \frac{1}{3} 3 x^2 \cos(x^3)   = x^2 \cos(x^3)
\]
The integrand is  \( x^2 \cos(x^3)\).  So our answer is OK.


\begin{checklist}

\item Anytime we fail to check our work, we \textbf{are losers.}

\item \emph{When you check your work, you only cry once. But when we don't,  we cry many times.}


\end{checklist}



\end{frame}
\begin{frame}{Example}

 \( \int x  \exp(-x^2) \, \mathrm{d} x \).   The function \(\exp\)  has a known antiderivative.  The expression inside \(\exp\) is \(-x^2\).  So choose \(u =-x^2\).   Then
\(\frac{\mathrm{d} u}{\mathrm{d} x} = -2 x \) matches the remaining factor  \(x\) in the integrand up to a multiplicative factor. \textbf{We win}.


\[
    [u = -x^2] = [\mathrm{d} u = -2 x  \mathrm{d} x] = \left[  x \mathrm{d} x = -  \frac{1}{2}  \mathrm{d} u \right].
\]
So
\[
  \int x  \exp(-x^2) \, \mathrm{d} x   = \int  - \frac{1}{2} \exp(u)  \mathrm{d} u = -  \frac{1}{2}   \exp(u)  =  -  \frac{1}{2}  \exp(-x^2) .
\]

\end{frame}

\begin{frame}{Example}

\( \int  \sqrt{5 x + 7}  \, \mathrm{d} x \).    Oops! The integrand isn't a product. Are we losers?  No way.   The integrand is \( \int 1 \times  \sqrt{5 x + 7}  \, \mathrm{d} x \). 

The square root has a known antiderivative, so choose \(u = 5 x + 7\). Then \(\frac{\mathrm{d} u}{\mathrm{d} x} = 5 \) matches the remaining factor  \(1\) in the integrand 
 up to a multiplicative factor. \textbf{We win}.
 
 
\[
    [u = 5 x + 7] = [\mathrm{d} u = 5   \mathrm{d} x] = \left[  \mathrm{d} x =  \frac{1}{5}  \mathrm{d} u \right].
\]
So
\[
   \int  \sqrt{5 x + 7}  \, \mathrm{d} x    =  \int \frac{1}{5} \sqrt{u} \, \mathrm{d} u =  \frac{1}{5}    \times  \frac{2}{3} u^{3/2} =  \frac{2}{15}  (5 x + 7)^{3/2}.
\]



\end{frame}

\begin{frame}{Fake  matches}

Let's try the problem
\[
   \int  x \exp(x^4) \, \mathrm{d} x.
\]
Since \(\exp\) has a known antiderivative, for \(u\) choose what is inside \(\exp\). Thus
\[
   [u = x^4 ] = [\mathrm{d} u = 4 x^3 \mathrm{d} x] = \left[   \mathrm{d} x =  \frac{1}{4 x^3} \mathrm{d} u \right]
\]
So far, Okie-dokie.
\[
   \int  x \exp(x^4) \, \mathrm{d} x = \int x \exp(u)  \frac{1}{4 x^3} \mathrm{d} u = \int  \exp(u)  \frac{1}{4 x^2} \mathrm{d} u
\]

\begin{checklist}

\item Since   \(\int  \exp(u)  \frac{1}{4 x^2} \mathrm{d} u \) depends on both the old variable \(x\) and the new variable \(u\), we \textbf{have not completed our task of out with the old and in with the new}.

\item We could push on this a bit using \(x = u^{1/4} \) to eliminate the remaining terms involving \(x\). This gives
\[
    \int \frac{1}{4 \sqrt{u}} \exp(u) \, \mathrm{d} u.
\]

\item But we're stuck because this isn't an antiderivative that we know about.
\end{checklist}

\end{frame}


\begin{frame}{Occult  (that is hidden) matches}

\[
    \int  \frac{x}{\sqrt{1-x^4}} \, \mathrm{d} x
\]


\begin{checklist}

\item We know the antiderivative of reciprocial square root. That suggests
\[
    u  = 1-x^4
\]
But then
\[
    \mathrm{d} u = - 4 x^3 \, \mathrm{d} x.
\]
\item The substitution yields
\[
   \int  \frac{x}{\sqrt{1-x^4}} \, \mathrm{d} x = \int   -   \frac{x}{\sqrt{u}} \frac{\mathrm{d} u}{4 x^3}  =  \int   -   \frac{1}{\sqrt{u}} \frac{\mathrm{d} u}{4 x^2}  
\] 

\item As yet, we have a yucky mixture of old and new.  This doesn't look promising.  Should we bail out, or try some thing new? 
\end{checklist}
\end{frame} 

\begin{frame}{The rule of holes \footnote{When you are in a hole, digging faster isn't the answer.}} 



\begin{checklist}

\item Let's bailout.


\item We also know the antiderivative of \(\int \frac{1}{\sqrt{1-x^2}} \mathrm{d} x \). 

\item This suggests choosing  \(1- x^4 = 1-u^2 \); equivalently \(x^2  = u\). Then
\[
     \mathrm{d} u = 2 x \mathrm{d} x.
\]

\item So
\[
       \int  \frac{x}{\sqrt{1-x^4}} \, \mathrm{d} x = \int \frac{1}{2} \frac{1}{\sqrt{1-u^2}} \, \mathrm{d} u = \frac{1}{2} \sin^{-1}(u) =  \frac{1}{2} \sin^{-1}(x^2) 
\]

\end{checklist}

\end{frame} 
\end{document} 