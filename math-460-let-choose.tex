\documentclass[fleqn]{beamer}
%\usetheme[height=7mm]{Rochester}
\usetheme{Boadilla} %{Rochester}

\setbeamertemplate{footline}[text line]{%
\parbox{\linewidth}{\vspace*{-8pt}\hfill\insertshortauthor\hfill\insertpagenumber}}
\setbeamertemplate{navigation symbols}{}
%\author[BW]{Dr.\ Barton Willis}
\usepackage{amsmath}\usepackage{amsthm}
\usepackage{isomath}
\usepackage{upgreek}
\usepackage{comment,enumerate,xcolor}

\usepackage[english]{babel}
\usepackage[final,babel]{microtype}%\usepackage[dvipsnames]{color}
%\usefonttheme{professionalfonts}
%\usefonttheme{serif}

\newcommand{\reals}{\mathbf{R}}
\newcommand{\complex}{\mathbf{C}}
\newcommand{\integers}{\mathbf{Z}}
\DeclareMathOperator{\range}{range}
\DeclareMathOperator{\domain}{dom}
\DeclareMathOperator{\dom}{dom}
\DeclareMathOperator{\codomain}{codomain}
\DeclareMathOperator{\sspan}{span}
\DeclareMathOperator{\F}{F}
\DeclareMathOperator{\G}{G}
\DeclareMathOperator{\B}{B}
\DeclareMathOperator{\D}{D}
\DeclareMathOperator{\id}{id}
\DeclareMathOperator{\ball}{ball}

\usepackage{graphicx}
\usepackage{color}
\usepackage{amsmath}
\DeclareMathOperator{\nullspace}{nullity}
\theoremstyle{definition}
\newtheorem{mydef}{Definition}
\newtheorem{myqdef}{Quasi-definition}
\newtheorem{myex}{Example}
\newtheorem{myth}{Proposition}
\newtheorem{myfact}{Fact}
\newtheorem{metathm}{Meta Theorem}
\newtheorem{Question}{Question}
\newtheorem{Answer}{Answer}
\newtheorem{myproof}{Proof}
\newtheorem{hurestic}{Hurestic}

%\usepackage{array}   % for \newcolumntype macro
%\newcolumntype{L}{>{$}l<{$}} % math-mode version of "l" column type

\newenvironment{alphalist}{
  \vspace{-0.4in}
  \begin{enumerate}[(a)]
    \addtolength{\itemsep}{1.0\itemsep}}
  {\end{enumerate}}

\newenvironment{snowflakelist}{
  \vspace{-0.4in}
  \begin{enumerate}[\textleaf]
    \addtolength{\itemsep}{-1.2\itemsep}}
  {\end{enumerate}}





\newenvironment{checklist}{
  \begin{enumerate}[\ding{52}]
    \addtolength{\itemsep}{-1.0\itemsep}}
  {\end{enumerate}}

\newenvironment{numberlist}
   {\begin{enumerate}[(1)]
       \addtolength{\itemsep}{-0.5\itemsep}}
     {\end{enumerate}}
\usepackage{amsfonts}
\makeatletter
\def\amsbb{\use@mathgroup \M@U \symAMSb}
\makeatother
\usepackage{bbold}

\usepackage{array}
\newcolumntype{C}{>$c<$}

\newcommand{\llnot}{\lnot \,} % is accepted
\DeclareMathOperator{\3F2}{{}_3  F_2}
\newcommand\pochhammer[2]{\left[\genfrac..{0pt}{}{#1}{#2}\right]}
\newmuskip\pFqmuskip

\newcommand*\pFq[6][8]{%
  \begingroup % only local assignments
  \pFqmuskip=#1mu\relax
  % make the comma math active
  % \mathcode`\,=\string"8000
  % and define it to be \pFqcomma
  \begingroup\lccode`\~=`\,
  \lowercase{\endgroup\let~}\pFqcomma
  % typeset the formula
      {}_{#2}\!\F_{#3}{\left[\genfrac..{0pt}{}{#4}{#5};#6\right]}   %\F{\left[\genfrac..{0pt}{}{#4}{#5};#6\right]}  %alt: {}_{#2}\F_{#3}{\left[\genfrac..{0pt}{}{#4}{#5};#6\right]}
  \endgroup
}

\newcommand{\pFqcomma}{\mskip \pFqmuskip}
\newcommand{\mydash}{\text{--}}


%------------------

%\subtitle{Lesson $\alpha$}
\title{\textbf{Let / Choose Proofs}}
%\author[Barton Willis] % (optional, for multiple authors)
%{Barton~Willis}%
%\institute[UNK] % (optional)

%{
 % \inst{1}%
%  ``The secret of getting ahead is getting started.'' Mark Twain
%   }
  \date{}


\usepackage{courier}
%\lstset{basicstyle=\ttfamily\footnotesize,breaklines=true}
%\lstset{framextopmargin=50pt,frame=bottomline}


%\begin{document}



%--------
%usepackage[usenames,dvipsnames,svgnames,table]{color}



\begin{document}

\frame{\titlepage}


\begin{frame}{The Let/Choose Template}

Many propositions have the form
\vspace{0.1in}

 \quad \quad  (string of \(\forall\) \(\exists\) qualifiers  in involving \(x_1\) thru \(x_n\)) (\(P (x_1, \dots x_n)\)),

  \vspace{0.1in}

where \(P\) is a predicate.  It behooves us to have a template for proving such propositions.   Let's try the example

\begin{myth} For every \(x \in \reals\) there is \(y \in \reals\) such that \(x < y\). \end{myth}

\begin{enumerate}

\item[\# 0] Write the proposition in symbolic form:
\[
    (\forall x \in \reals)(\exists y \in \reals)(x < y).
\]

\item[\#1] Write the proposition in the form of a question:

\quad \emph{Given a real number \(x\), can I find a number \(y\) such that \(x < y\)?}

\item[\#2] Answer your question.

\quad \emph{Sure--a number that is greater than \(x\) is \(x +1\).}

\end{enumerate}

  \end{frame}
  \begin{frame}
  \begin{enumerate}

\item[\#3] Using the symbolic form of the proposition and  \emph{strictly moving from left to right}, replace \(\forall \) with ``Let,'' and \(\exists\) with ``Choose.''  After each ``choose'' make a box to fill in.
Finish with the predicate:

Let \(x \in \reals\).  Choose \(y = \fbox{\phantom{XX}}\).  We have
\[
      [x < y] =
\]

\item[\#4] Fill in the boxes with the answers you chose, and attempt to show that the predicate is true:
Let \(x \in \reals\).  Choose \(y = \fbox{x+1}\).  We have
\begin{align*}
      [x < y] &\equiv  [x < x + 1] , &(\mbox{substitute for } y)\\
                   &\equiv  [0 < 1],   &(\mbox{subtract } x \mbox{ from both sides})\\
                     &\equiv  \mbox{true}.
\end{align*}
\item[\#5]  Erase the boxes:

\begin{myproof}
Let \(x \in \reals\).  Choose \(y = x+1\).  We have
\begin{align*}
      [x < y] &\equiv  [x < x + 1] , &(\mbox{substitute for } y)\\
                   &\equiv  [0 < 1],   &(\mbox{subtract } x \mbox{ from both sides})\\
                     &\equiv  \mbox{true}.
\end{align*}
\end{myproof}

\item[\#6] Proofread your work.
\end{enumerate}
    \end{frame}

     \begin{frame}{\(\lnot\) Pedantic }

     \begin{enumerate}

     \item Proof construction is a creative activity--there is no step of steps that will always generate a proof.

     \item But having patterns to follow and knowing techniques is useful for all creative endeavors.
     \end{enumerate}
     \end{frame}

     \begin{frame}{Respecting order}

     The order of qualifiers matters.  To show this, let's reverse the order of qualifiers in the previous proposition:

     \begin{myth}  There is \(y \in \reals\)  such that for  every \(x \in \reals\) we have \(x < y\). \end{myth}


     \vspace{0.2in}


     \textbf{Question} Can I find a real number \(y\) such that for every real number \(x\), we have \(x < y \)?

          \vspace{0.2in}

     \textbf{Answer:}  No I don't think so--the number we choose has to be larger than \(10^{10}\), larger than  
     \(10^{{10}^{10}} \) and larger than every number. The statement requires that $y$ be a
     \emph{real number}, so choosing $y = \infty$ isn't an option.

            \vspace{0.2in}

      
            \textbf{Tip}  Proving things that are wrong take too much time. So try to avoid attempting.
    \end{frame}
    \begin{frame}

   Let's show that the proposition is false by showing that its negation is true; the negation of the proposition is

    \begin{myth}  For all \(y \in \reals\)  there is \(x \in \reals\)  such that \(x \geq  y\). \end{myth}

   \begin{myproof}   Let \(y \in \reals\).  Choose \(x = y\).  Then \([ x \geq  y] \equiv [x \geq  x] \equiv \mbox{true} \).
    \end{myproof}

    \begin{enumerate}

    \item We could choose \(x = y+1\), but we only need \(x \geq y\), so we can choose \(x = y\).

    \end{enumerate}
     \end{frame}

\begin{frame}{Later, rinse, repeat}

 \begin{myth}  For all \(x \in \reals_{> 0}\)  there is \(y \in \reals_{>0} \)  such that \(y < x\). \end{myth}

 \begin{enumerate}
 \item Write the proposition in the form of a question:

\quad \emph{Given a positive  real number \(x\), can I find a positive  \(y\) such that \(y\) is smaller than \(x\)?}

\item Answer your question.

\quad \emph{Sure--a positive number that is smaller than \(x\) is the average of zero and \(x\); that is \(x/2\).}

\item I'm ready, I think:

\begin{myproof}  Let \(x \in \reals_{> 0}\).  Choose \(y = x/2\).  Then \(y \in \reals_{> 0}\).  Further
\begin{align*}
     [y < x] &\equiv [x/2 < x]     &(\mbox{substitute for } y)\\
              &\equiv [1/2 < 1]        &(\mbox{divide inequality by positive number } x)\\
               &\equiv \mbox{true}.
     \end{align*}
\end{myproof}

\end{enumerate}

      \end{frame}

  \end{document}
