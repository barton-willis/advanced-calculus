\documentclass[professionalfonts]{beamer}%
\usetheme{Madrid}
\usecolortheme{whale}
\newcommand{\integers}{\mathbf{Z}}
\newcommand{\reals}{\mathbf{R}}
\newcommand{\dom}{\mathrm {dom}}
%\usepackage{alltt}
%\usepackage[]{algorithm2e}
%\usepackage{tikz}
%\usetikzlibrary{arrows}

\usepackage{upgreek}
\usepackage{amsmath,amssymb,enumerate,verbatim}
\let\checkmark\relax
\usepackage{dingbat}
\usepackage{epsfig}
\usepackage{pifont,bbding,enumerate,color}

\usepackage{epsfig}

\usepackage{enumerate}
%\def\changemargin#1#2{\list{}{\rightmargin#2\leftmargin#1}\item[]}
%\let\endchangemargin=\endlist 
\newenvironment{handlist}{
  \begin{enumerate}[\leftthumbsup]
    \addtolength{\itemsep}{-0.3\itemsep}}
  {\end{enumerate}}

\newenvironment{alphalist}{
  \begin{enumerate}[(a)]
    \addtolength{\itemsep}{-0.5\itemsep}}
  {\end{enumerate}}

\usepackage[euler-digits,euler-hat-accent,T1]{eulervm}

\usepackage{isomath}
\usepackage{xcolor}
\usepackage[]{algorithm2e}
%%
\title[Proofs for humans] % (optional, only for long titles)
{Grammar for mathematicians and other humans}
\author[Barton Willis] % (optional, for multiple authors)
{Barton~Willis}
\institute[UNK] % (optional)
{
 % \inst{1}%
  MATH 460 \\
   \medskip
  Department of Mathematics and Statistics\\
  University Nebraska at Kearney
  }
  \date{}


\begin{document}



\frame{\titlepage}


\begin{frame}[fragile]{Write proofs as (English) sentences}{}%

Write proofs as (English) sentences; specifically:

\begin{handlist}
\item Every sentence must start with a word, not  a mathematical expression.
\item Every sentence of a proof must  end with a \emph{period or a semicolon.}
\item Except for  enumeration, generally mathematical expressions should be separated by  a word or phrase. 
\end{handlist}

\vfill 
\end{frame}
\begin{frame}[fragile]{}{}%

\vspace{0.1in}

\begin{block}{Examples}
\textbf{Wrong:} m,n integers

\textbf{Correct:} Let \(m\) and \(n\) be integers.\\

\vspace{0.1in}

\textbf{Wrong:} \(x > 0\),  we have \(x + 1 > 0\).\\
\textbf{Correct:}  Since \(x > 0\), we have \(x + 1 > 0\).

\vspace{0.1in}

\textbf{Wrong:} If \(x \in A\), \(x \in B\).\\
\textbf{Correct:}  If \(x \in A\), then  \(x \in B\).
\end{block}

\vfill
\end{frame} 

\begin{frame}[fragile]{No poetry}{}%

\textbf{Ode To Tomatoes by Pablo Neruda}

The street\\
filled with tomatoes,\\
midday,\\
summer,\\
light is\\
halved\\
like\\
a\\
tomato \\

\begin{handlist}

\item Write proofs as regular text, not as poetry with wide margins.
\end{handlist}




\end{frame}
\begin{frame}[fragile]{}{}%


\begin{block}{Example}

\textbf{Wrong:}
\begin{centering}
Let \(\varepsilon > 0\). \\
Choose \(\delta = \varepsilon / 3 \). \\
For \(|x - 1| < \delta \) we have \(x < 1 + \delta\). \\
\end{centering}

\vspace{0.1in}

\textbf{Correct:} Let \(\varepsilon > 0\). Choose \(\delta = \varepsilon / 3 \).  For \(|x - 1| < \delta \) we have \(x < 1 + \delta\). 
\end{block}

\vfill
\end{frame}

\begin{frame}[fragile]{Say what you mean}{}%

Try reading your text out loud. Make sure it makes sense.

\begin{block}{Examples}

\textbf{Wrong:} Let \(x \in A \subset B\).



\textbf{Correct} Let \(x \in A\). Since \(A \subset B\), we have \(x \in B\).

\vspace{0.1in}

\textbf{Wrong:}   Let \(k > 0\) be an integer.

\textbf{Correct:} Let \(k\) be a positive integer.

\end{block}

\begin{handlist}
\item The sentence
\begin{quote} Let \(x \) be a member of \(A\) is a subset of \(B\). \end{quote}
is nonsence. So is
\begin{quote} Let \(k\) is  greater than zero be an integer. \end{quote}.

\end{handlist}
\vfill 
\end{frame}
\begin{frame}[fragile]{First waffle rule}{}%

The first waffle is never perfect; neither is the first attempt at a proof. Revise your work until it is 
as close to perfect as you can make it.

\begin{handlist}
\item But first be sure your work is logical--correcting the form of illogical work is a waste of time.

\item By all means, if it helps you construct a proof, draw pictures and
diagrams filled with lines and arrows.  

\item But do not include your scratch work in the final copy.

\item In a quest for perfection, mathematicians have been known to write math on resturant menus, unpaid bills, and on birth certificates.
\end{handlist}

\end{frame}

\begin{frame}[fragile]{Pick-and-show idiom}{}%

Anytime you need to show one set is a subset of another, you should use the
``pick-and-show'' idiom; it looks like this:

\begin{quote}

{\bf Proposition} Let \(A\) and \(B\) be sets and suppose \(H_1\), \(H_2\), \dots , and \(H_n\). Then  \(A \subset B\).

\vspace{0.1in}

{\bf Proof} If \(x \in A\), we have (deductions made using the 
facts \(H_1\) through \(H_n\)); therefore \(x \in B\).

\end{quote}
Here, the statements \(H_1\) through \(H_n\) are the hypothesis of the
proposition. To demonstrate set equality, use the pick-and-show idiom twice. 

\end{frame}

\begin{frame}[fragile]{Pick-and-show shown}{}%

\begin{block}{Pick-and-show example}
\textbf{Proposition} Let \(A\) and \(B\) be sets with \(A \subset B\). Then \(B^\mathrm{C} \subset A^\mathrm{C}\).

\vspace{0.1in}

\textbf{Proof} If \(x \in B^\mathrm{C} \), then \(x \notin B\). Since \(x \notin B\) and  \(A \subset B\), we have \(x \notin A\); therefore
\(x \in A^\mathrm{C}\).

\end{block}

\begin{handlist}

\item The \emph{conclusion} of the proposition is  \(B^\mathrm{C} \subset A^\mathrm{C}\). Thus pick-and-show starts with ``If \(x \in B^\mathrm{C}\).''

\item The hypothesis is  \(A \subset B\). Starting pick-and-show starting with `If \(x \in A^\mathrm{C}\)'' is the exit ramp to nowhere.

\end{handlist}

\end{frame}

\begin{frame}[fragile]{Jeep\footnote{there's only one} idiom}{}%

To show that there is only one thing of some object, assume \(\mbox{Thing}_1\) and \(\mbox{Thing}_2\) are these objects and show that  \(\mbox{Thing}_1 = \mbox{Thing}_2\)


\begin{block}{There's only one}

\textbf{Proposition} There is at most one empty set.

\vspace{0.1in}
\textbf{Proof} Suppose \(E\) and \(E^\prime\) are empty sets. Since \(E\) is empty, we have \(E \subset E^\prime\). Similarly since \(E^\prime\) is empty,
we have \(E^\prime \subset E\); therefore \(E = E^\prime\).

\end{block}

\end{frame}
\end{document}

