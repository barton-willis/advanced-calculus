\documentclass[fleqn]{beamer}
%\usetheme[height=7mm]{Rochester}
\usetheme{Boadilla} %{Rochester}

\setbeamertemplate{footline}[text line]{%
\parbox{\linewidth}{\vspace*{8pt}\hfill\insertshortauthor\hfill\insertpagenumber}}
\setbeamertemplate{navigation symbols}{}
%\author[BW]{Barton Willis}
\usepackage{amsmath}\usepackage{amsthm}
\usepackage{isomath}
\usepackage{upgreek}
\usepackage{comment,enumerate,xcolor}

\usepackage[english]{babel}
\usepackage[final,babel]{microtype}%\usepackage[dvipsnames]{color}
%\usefonttheme{professionalfonts}
%\usefonttheme{serif}

\newcommand{\reals}{\mathbf{R}}
\newcommand{\complex}{\mathbf{C}}
\newcommand{\integers}{\mathbf{Z}}
\DeclareMathOperator{\range}{range}
\DeclareMathOperator{\domain}{dom}
\DeclareMathOperator{\dom}{dom}
\DeclareMathOperator{\codomain}{codomain}
\DeclareMathOperator{\sspan}{span}
\DeclareMathOperator{\F}{F}
\DeclareMathOperator{\G}{G}
\DeclareMathOperator{\B}{B}
\DeclareMathOperator{\D}{D}
\DeclareMathOperator{\id}{id}
\DeclareMathOperator{\ball}{ball}

\usepackage{graphicx}
\usepackage{color}
\usepackage{amsmath}
\DeclareMathOperator{\nullspace}{nullity}
\theoremstyle{definition}
\newtheorem{mydef}{Definition}
\newtheorem{myqdef}{Quasi-definition}
\newtheorem{myex}{Example}
\newtheorem{myth}{Theorem}
\newtheorem{myfact}{Fact}
\newtheorem{metathm}{Meta Theorem}
\newtheorem{Question}{Question}
\newtheorem{Answer}{Answer}
\newtheorem{myproof}{Proof}

\newtheorem{myfakeproof}{Fake Proof}
\newtheorem{mybadformproof}{Bad Form Proof}

\newtheorem{hurestic}{Hurestic}

\newenvironment{alphalist}{
  \vspace{-0.4in}
  \begin{enumerate}[(a)]
    \addtolength{\itemsep}{1.0\itemsep}}
  {\end{enumerate}}

\newenvironment{snowflakelist}{
  \vspace{-0.4in}
  \begin{enumerate}[\textleaf]
    \addtolength{\itemsep}{-1.2\itemsep}}
  {\end{enumerate}}

\newenvironment{checklist}{
  \begin{enumerate}[\ding{52}]
    \addtolength{\itemsep}{-1.0\itemsep}}
  {\end{enumerate}}

\newenvironment{numberlist}
   {\begin{enumerate}[(1)]
       \addtolength{\itemsep}{-0.5\itemsep}}
     {\end{enumerate}}
\usepackage{amsfonts}
\makeatletter
\def\amsbb{\use@mathgroup \M@U \symAMSb}
\makeatother
\usepackage{bbold}

\usepackage{array}
\newcolumntype{C}{>$c<$}

\newcommand{\llnot}{\lnot \,} % is accepted
\newcommand{\mydash}{\text{--}}


%------------------


\title{\textbf{Fake Proofs}}
%\author[Barton Willis] % (optional, for multiple authors)
%{Barton~Willis}%
%\institute[UNK] % (optional)
\subtitle{%Lesson 10   \\ \vspace{0.5in}
  ``If you can't prove what you want to prove, demonstrate something else and pretend that they are the same.''  \\   \vspace{0.15in}{Darrell Huff} \\ \vspace{0.15in} \emph{How to Lie with Statistics} \\ 
 \vspace{1.0in}
  \tiny Barton Willis, Attribution 4.0 International (CC BY 4.0), 2020, 2023 \normalsize
   }
  \date{}

\begin{document}

\frame{\titlepage}

\begin{frame}{Prove a special case}

\begin{myth}   Every function  that is differentiable at zero is continuous at zero.  \end{myth}

\begin{myfakeproof}  Let \(F = x \in \reals \mapsto x^2\).  This function is 
  differentiable at zero. And it's continuous at zero. We've shown that  a 
  function that is differentiable at zero is continuous at zero.
\end{myfakeproof}

\begin{enumerate}
\item We can't prove an ``every'' statement by checking just one out of many cases.
\item Occasionally, students \textbf{mistakenly} believe that if the instructions are 
to \emph{show} instead of \emph{prove} that something is true, it's OK to verify the 
statement is true for one particular case. But no, \emph{showing} that something is 
true and  \emph{proving} that is true are identical instructions.


\end{enumerate}

\end{frame}

\begin{frame}{Assume the conclusion}


\begin{myth}   Let \(A,B\),  and \(C\) be sets.  If \(A \subset B\) and \(B \subset C\), we have \(A \subset C\). \end{myth}

\begin{myfakeproof}  Suppose \(A \subset C\).  [Rubbish Deleted]; therefore  \(A \subset C\). \end{myfakeproof}

\begin{enumerate}
\item Terrific--we just proved the tautology \(   (A \subset C) \implies  (A \subset C) \).

\item A proof needs to flow from the hypothesis (in this case  \(A \subset B\) and \(B \subset C\)) to the conclusion (\(A \subset C\)).
\end{enumerate}

\end{frame}

\begin{frame}{Assuming facts not in evidence}

\begin{myth}  Every continuous function has an antiderivative.  \end{myth}


\begin{myfakeproof}      Define  \(F(x) = a_0 + a_1 x + \cdots + a_n x^n \).  An antiderivative of \(F\) is
\[
    a_0 x + \frac{1}{2} a_1 x^2 + \cdots + \frac{1}{1+n} a_n x^{n+1}.
\]
Therefore, \(F\) has an antiderivative.
\end{myfakeproof}

\begin{enumerate}

\item Assuming that \(F\) is a polynomial is assuming facts not  in evidence.

\item Arguably, this fake proof falls under the ``Prove a special case'' category.
\end{enumerate}
\end{frame}


\begin{frame}{Proof by obfuscation}



\begin{myth}  For every \(a,b \in \reals \) with \(a \neq b\) there is  \(x \in \reals\) such that \( \min(a,b) < x < \max(a,b) \).  \end{myth}

\begin{mybadformproof}
We the People of the United States, in Order to form a more perfect Union, establish Justice, insure domestic Tranquility, provide for the common defense, promote the general Welfare, and secure the
Blessings of Liberty to ourselves and our Posterity, do ordain and establish this Constitution for the United States of America.  Choose \(x = \frac{a+b}{2} \).

\quad Four score and seven years ago our fathers brought forth on this continent, a new nation, conceived in Liberty, and dedicated to the proposition that all men are created equal.  We have
\( x = \frac{a+b}{2} < \frac{ \max(a,b)  +  \max(a,b) }{2} =  \max(a,b)\).

\quad We hold these truths to be self-evident, that all men are created equal, that they are endowed by their Creator with certain unalienable Rights, that among these are Life, Liberty and the pursuit of Happiness. Similarly \(\min(a,b) < x\). Therefore  \( \min(a,b) < x < \max(a,b) \).

\end{mybadformproof}
\end{frame}


\begin{frame}{The Twisties}

\begin{myth}  We have
\[
   \left (\exists m \in \reals_{> 0} \right) \left (\forall x \in  \reals_{> 0} \right) \left(\left| \frac{x}{x+1} \right| \leq m |x| \right).
\]
\end{myth}
\begin{myfakeproof}    Let \(x > 0\).  Choose \(m = \frac{1}{1+x}\).  Then \(m > 0\), as required. Further,
\[
   \left [ \left | \frac{x}{x+1} \right | \leq m |x| \right] \equiv  \left [\left| \frac{x}{x+1} \right| \leq \left| \frac{x}{x+1} \right| \right] \equiv \mbox{True}.
\]
\end{myfakeproof}
\begin{enumerate}

\item We proved that  \( \left (\forall x \in  \reals_{> 0} \right) \left (\exists m \in \reals_{> 0} \right)  \left (\left| \frac{x}{x+1} \right| \leq m |x| \right) \).

\item In the actual proposition, the first introduced variable is \(m\) and the second is \(x\).  Since \(m\) comes first, it \textbf{cannot depend} on \(x\).

\end{enumerate}

\end{frame}
\begin{frame}{Working backwards}

\begin{myth} We have \(3^{1/3} > 5^{1/5} \). \end{myth}

\begin{myfakeproof}
We have
\small
\[
   3^{1/3} > 5^{1/5} \implies (3^{1/3})^{15} > (5^{1/5})^{15} \implies 3^5> 5^3 \implies 243 > 125  =  \mathrm{true!}
\]
\end{myfakeproof}

\begin{enumerate}
\item Congratulations--we just proved that if  \(3^{1/3} > 5^{1/5}\) then \( 243 > 125\).

\item This fake proof can be salvaged by starting with \( 243 > 125\) and reversing each implication.

\item But not all such ``backward'' proofs can be salvaged this way.

\item The  suggestion to ``start with what you want and work backwards'' is sometimes OK, but it
is often terrible advice.
\end{enumerate}
\end{frame}


\begin{frame}{Backwards redux}

\begin{myproof}
We have
\begin{align*}
   \left[   3^{1/3} > 5^{1/5} \right]  &\equiv \left[   (3^{1/3})^{15} > (5^{1/5})^{15}  \right]    & ( x \in \reals \mapsto x^{15} \mbox{ is  increasing}) \\
                                                               &\equiv  \left[ 3^5> 5^3 \right]   &(\mbox{algebra}) \\
                                                                   &\equiv  \left[ 243 > 125 \right]   &(\mbox{algebra}) \\
                                                                   &\equiv  \mbox{true}
  \end{align*}
\end{myproof}
\begin{enumerate}

\item Isn't this proof also backwards? It starts with the conclusion!

\item Yes, the main part of the proof starts with the conclusion, but it shows that   \(    \left[   3^{1/3} > 5^{1/5} \right]    \equiv  \mbox{true} \).

\item And that's fine.
\end{enumerate}
\end{frame}

\begin{frame}{Nonreversable backwardness}

  Not all proofs that were constructed by ``working backwards''
  can be salvaged; an example.

  \begin{myth} We have $0 = 1$. \end{myth}

  \begin{myfakeproof}
   We have
   \begin{align*}
    \left[0 = 1\right] &\implies \left[0 \times 0 = 1 \times 0 \right], & \mbox{(multiply by zero)} \\
                       &\implies [0 = 0],   &\mbox{(algebra)} \\
                       &\equiv \mbox{True}.  &\mbox{(syntactic equality)}
   \end{align*}
    Thus $0 = 1$.
  \end{myfakeproof}

  \begin{enumerate}
    \item Great! We've just proved that if $0 = 1$, then $0=0$. 

    \item Reversing the implication in $\left[0 = 1\right] \implies \left[0 \times 0 = 1 \times 0 \right]$
    requires \emph{dividing} by zero. And that's rubbish.
  \end{enumerate}
    





 
\end{frame}


\begin{frame}{Proof by Haiku }

\begin{myth}  Let \(A,B\), and \(C\) be sets.  Then \( \left (A \setminus B \right ) \subset A\)  \end{myth}

\begin{mybadformproof}
   \begin{center}  \( A \setminus B \) \end{center}

     \begin{center}  \( A \setminus B \)  takes $B$ away from \(A\) \end{center}

     \begin{center}  So   \end{center}

     \begin{center}    \(\left(A \setminus B \right) \subset A\)  \end{center}

\end{mybadformproof}

\begin{enumerate}

\item Proofs aren't poems, so don't format them as poetry.

\item Using a phrase such as ``takes away from'' is poetic license.  Let's stick mathematical terminology.
\end{enumerate}
\end{frame}

\end{document}

https://www.quora.com/What-are-the-most-common-mistakes-people-make-when-writing-mathematical-proofs

"All other proofs contain numerous sentence fragments, formulas floating in outer 
space, non sequiturs, and the depressingly ubiquitous unquantified variables. 
Those are xs and vs and ϵs that make an appearance without the the crucial bits of 
"For any real number x", "we can find some ϵ such that" or "we can now define v to 
be the".
