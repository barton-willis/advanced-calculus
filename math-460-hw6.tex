\documentclass[12pt,fleqn]{exam}
\usepackage{pifont}
\usepackage{dingbat}
\usepackage{amsmath,amssymb}
\usepackage{fleqn}
\usepackage{epsfig}
%\usepackage{mathptm}
%\usepackage{euler}
\usepackage{bbding}

\addpoints
\boxedpoints
\pointsinmargin
\pointname{pts}

\usepackage[activate={true,nocompatibility},final,tracking=true,kerning=true,factor=1100,stretch=10,shrink=10]{microtype}
\usepackage[american]{babel}
%\usepackage[T1]{fontenc}
\usepackage{fourier}
\usepackage{isomath}
\usepackage{upgreek,amsmath}
\usepackage{amssymb}
%\usepackage[euler-digits,euler-hat-accent,T1]{eulervm}

\newcommand{\dotprod}{\, {\scriptzcriptztyle
    \stackrel{\bullet}{{}}}\,}

\newcommand{\reals}{\mathbf{R}}
\newcommand{\complex}{\mathbf{C}}
\newcommand{\dom}{\mbox{dom}}
\newcommand{\cover}{{\mathcal C}}
\newcommand{\integers}{\mathbf{Z}}
\newcommand{\vi}{\, \mathbf{i}}
\newcommand{\vj}{\, \mathbf{j}}
\newcommand{\vk}{\, \mathbf{k}}
\newcommand{\bi}{\, \mathbf{i}}
\newcommand{\bj}{\, \mathbf{j}}
\newcommand{\bk}{\, \mathbf{k}}
\DeclareMathOperator{\Arg}{\mathrm{Arg}}
\DeclareMathOperator{\Ln}{\mathrm{Ln}}
\newcommand{\imag}{\, \mathrm{i}}

\usepackage{graphicx}

\newcommand{\quiz}{1}
\newcommand{\term}{Spring}
\begin{document}
\large
\vspace{0.1in}
\noindent\makebox[3.0truein][l]{{\bf MATH 460}}
{\bf Name:} (append your name) \\
\noindent \makebox[3.0truein][l]{\bf Homework   \quiz, \term \/ \the\year}
%{\bf Row:}\hrulefill\
\vspace{0.1in}

\noindent  Homework    \quiz\/  has questions 1 through  \numquestions \/ with a total of  \numpoints\/  points.   Edit this file and append you answers using La\TeX.

\vspace{0.1in}


\noindent \textbf{Examples}

\textbf{Question} Write the statement  \emph{There is a positive real number \(x\) such that \(x^2 = 2\)} in symbolic form;

\textbf{Answer}   \( (\exists x \in \reals)(x^2 = 2)\).

\vspace{0.25in}

\textbf{Question} In symbolic form, write the negation of your answer to the previous question. Use the negation rules 
\( \lnot (\forall x \in A)(P(x)) \equiv (\exists x \in A)(\lnot P(x)) \) and \(  \lnot (\exists x \in A)(P(x)) \equiv (\forall x \in A)(\lnot P(x)) \) to transform the your symbolic form (replace the left side by the right side.)

\textbf{Answer}  \( (\forall  x \in \reals)(x^2 \neq  2)\).


\begin{questions}

\question [3] Write the statement  \emph{For every positive real number \(x\), there is a positive real number \(y\) such that \( y < x\) } in symbolic form.

\question [3]  Write the negation statement  \emph{For every positive real number \(x\), there is a positive real number \(y\) such that \( y < x\) } in symbolic form.  Use the negation rules 
\( \lnot (\forall x \in A)(P(x)) \equiv (\exists x \in A)(\lnot P(x)) \) and \(  \lnot (\exists x \in A)(P(x)) \equiv (\forall x \in A)(\lnot P(x)) \) to transform the your symbolic form (replace the left side by the right side.)

\question [3] Write the statement \( (\forall x,y \in \reals) (x^2 = y^2 \implies x = y) \) as an English sentence that doesn't use logical symbols.

\question [3] Write the statement \( \lnot (\forall x,y \in \reals) (x^2 = y^2 \implies x = y\) as an English sentence that doesn't use logical symbols. Again,  use the negation rules 
\( \lnot (\forall x \in A)(P(x)) \equiv (\exists x \in A)(\lnot P(x)) \) and \(  \lnot (\exists x \in A)(P(x)) \equiv (\forall x \in A)(\lnot P(x)) \) to transform the your symbolic form (replace the left side by the right side.)

\question [3] Write the \emph{contrapositive} of the statement \emph{If \(F\) is a real-valued function that is continuous at zero, then \(F\) is differentiable at zero.} as an English sentence.

\question [3] Write the \emph{converse} of the statement \emph{If \(F\) is a real-valued function that is continuous at zero, then \(F\) is differentiable at zero.} as an English sentence.


\end{questions}

\end{document}