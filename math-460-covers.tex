\documentclass[fleqn]{beamer}
%\usetheme[height=7mm]{Rochester}
\usetheme{Boadilla} %{Rochester}

\setbeamertemplate{footline}[text line]{%
\parbox{\linewidth}{\vspace*{8pt}\hfill\insertshortauthor\hfill\insertpagenumber}}
\setbeamertemplate{navigation symbols}{}
%\author[BW]{Barton Willis}
\usepackage{amsmath}\usepackage{amsthm}
\usepackage{isomath}
\usepackage{upgreek}
\usepackage{comment,enumerate,xcolor}

\usepackage[english]{babel}
\usepackage[final,babel]{microtype}%\usepackage[dvipsnames]{color}
\usefonttheme{professionalfonts}
%\usefonttheme{serif}

\newcommand{\reals}{\mathbf{R}}
\newcommand{\complex}{\mathbf{C}}
\newcommand{\integers}{\mathbf{Z}}
\newcommand{\cover}{\mathcal{C}}

\DeclareMathOperator{\range}{range}
\DeclareMathOperator{\domain}{dom}
\DeclareMathOperator{\dom}{dom}
\DeclareMathOperator{\codomain}{codomain}
\DeclareMathOperator{\sspan}{span}
\DeclareMathOperator{\F}{F}
\DeclareMathOperator{\G}{G}
\DeclareMathOperator{\B}{B}
\DeclareMathOperator{\D}{D}
\DeclareMathOperator{\id}{id}
\DeclareMathOperator{\ball}{B}

\usepackage{graphicx}
\usepackage{color}
\usepackage{amsmath}
\DeclareMathOperator{\nullspace}{nullity}
\theoremstyle{definition}
\newtheorem{mydef}{Definition}
\newtheorem{myqdef}{Quasi-definition}
\newtheorem{myex}{Example}
\newtheorem{myth}{Theorem}
\newtheorem{myfact}{Fact}
\newtheorem{metathm}{Meta Theorem}
\newtheorem{Question}{Question}
\newtheorem{Answer}{Answer}
\newtheorem{myproof}{Proof}

\newtheorem{myfakeproof}{Fake Proof}
\newtheorem{mybadformproof}{Bad Form Proof}

\newtheorem{hurestic}{Hurestic}

\newenvironment{alphalist}{
  \vspace{-0.4in}
  \begin{enumerate}[(a)]
    \addtolength{\itemsep}{1.0\itemsep}}
  {\end{enumerate}}

\newenvironment{snowflakelist}{
  %\vspace{-0.4in}
  \begin{enumerate}[\textleaf]
    %\addtolength{\itemsep}{1.2\itemsep}
 }
  {\end{enumerate}}

\newenvironment{checklist}{
  \begin{enumerate}[\ding{52}]
    \addtolength{\itemsep}{-1.0\itemsep}}
  {\end{enumerate}}

\newenvironment{numberlist}
   {\begin{enumerate}[(1)]
       \addtolength{\itemsep}{-0.5\itemsep}}
     {\end{enumerate}}
\usepackage{amsfonts}
\makeatletter
\def\amsbb{\use@mathgroup \M@U \symAMSb}
\makeatother
\usepackage{bbold}

\usepackage{array}
\newcolumntype{C}{>$c<$}

\newcommand{\llnot}{\lnot \,} % is accepted
\newcommand{\mydash}{\text{--}}


%------------------


\title{\textbf{Covers}}
%\author[Barton Willis] % (optional, for multiple authors)
%{Barton~Willis}%
%\institute[UNK] % (optional)

\begin{document}

\frame{\titlepage}

\begin{frame}{Covers}

  \begin{mydef} Let \(A\) be a subset of \(\reals\).  We say \(\cover\) is a \emph{cover of the set} \(A\) provided
    \begin{enumerate}
      \item \(\cover\) is a set,
      \item every member of \(\cover\) is an open set,
      \item we have \(A \subset \underset{x \in \cover}{\cup} x\).
    \end{enumerate}


  \end{mydef}

  \begin{snowflakelist}
    \item A cover is a bit like a quilt--each square of the quilt is too small to cover the bed, but collectively (that is their union) covers the bed.
    \item When a set has a cover, it is \emph{not} unique.
  \end{snowflakelist}

\end{frame}


\begin{frame}{Examples and nonexamples}

  \begin{myex}
    \begin{enumerate}
     \item The set \(\reals\) is not a cover of \(\reals\). Why? The members of \(\reals\) are real numbers, not open sets.

     \item The set \(\{\reals \}\) is a cover of \(\reals\).  Why? 

   \begin{enumerate}
    \item   \(\{\reals \}\) is a set--sure, it's a set with one member that is a set.

   \item every member of \(\{\reals \}\) is an open set--sure, the only member is \(\reals\) and we know that \(\reals\) is open.

    \item We have
     \[
      \underset{x \in \{\reals \} }{\cup} x = \reals.
     \]

\end{enumerate}

  \item The set \(\varnothing\) is a cover of itself.

\begin{enumerate}
    \item   \(\varnothing \) is a set--sure, it's a set.

   \item every member of \(\varnothing \) is an open set--sure,  it's vacuously true.

    \item We have
     \[
      \underset{x \in  \varnothing }{\cup} x = \varnothing.
     \]
\end{enumerate}
\end{enumerate}
  \end{myex}
\end{frame}

\begin{frame}{More Examples, less  nonexamples}

  \begin{myex}

    \begin{enumerate}
\item Define \(\cover = \{ \ball(0,k) \mid k \in \integers_{> 0} \}\). Then \(\cover\) is a cover of \(\reals\).
\begin{enumerate}
    \item   \(\cover \) is a set--sure, it's a set.

   \item every member of \(\cover \) is an open set--sure,  every member is an open ball.

    \item We have
     \[
      \underset{x \in  \cover }{\cup} x = \reals. 
     \]
   
\textbf{Claim:}   \(\reals \subset   \underset{x \in  \cover }{\cup} x \).

\textbf{ Proof:}  Let \(z \in \reals\).  Then \(z \in \ball(0, \lceil |z| \rceil + 1) \).  But \(  \ball(0, \lceil |z| \rceil + 1) \in \cover \); therefore \(z \in   \underset{x \in  \cover }{\cup} x \).
\end{enumerate}
\end{enumerate}
\end{myex}
\end{frame}

\begin{frame}{Subcovers}

\begin{mydef}  Let \(\cover\) be a cover of a set \(A\).  Any subset \(\cover^\prime\) of \(\cover\) is a \emph{subcover}  of \(\cover\) provided \(\cover^\prime\) is a cover of \(A\).
If \(\cover^\prime\) is a finite set, it's called a \emph{finite subcover of} \(A\). 
\end{mydef}

\begin{enumerate}

\item A set is finite if either it is empty or  its members can be uniquely labeled using the integers 1 to \(n\), for some integer \(n\).

\item The set \(\{ \reals \} \) is  finite; the set \(\reals\) is not finite.


\item The set \(\{ \infty \} \) is  finite. 
\end{enumerate}

\end{frame}


\begin{frame}{Examples of subcovers}

\begin{myex}

\begin{enumerate}  


\item The set \(\{ \ball(0,x) \mid x \in \reals_{>0}\} \) is a cover of \([0,1]\).      The set \(\{ \ball(0,2)  \} \) is a finite subcover. 

\item  The set \(\{ \ball(0,x) \mid x \in \reals_{>0}\} \) is a cover of \( \reals\).  This cover has no finite subcover. 

\textbf{Why} Every member of t \(\{ \ball(0,x) \mid x \in \reals_{>0}\} \) is bounded. The finite union of bounded sets is bounded. Thus regardless of what finite subset of \(\{ \ball(0,x) \mid x \in \reals_{>0}\} \) we choose,
its union will be bounded.  But \(\reals\) is unbounded, so is is not contained in any bounded set.
\end{enumerate}
\end{myex}

\end{frame}
\end{document}
