\documentclass[fleqn]{beamer}
%\usetheme[height=7mm]{Rochester}
\usetheme{Boadilla} %{Rochester}

\setbeamertemplate{footline}[text line]{%
\parbox{\linewidth}{\vspace*{-8pt}\hfill\insertshortauthor\hfill\insertpagenumber}}
\setbeamertemplate{navigation symbols}{}
%\author[BW]{Dr.\ Barton Willis}
\usepackage{amsmath}\usepackage{amsthm}
\usepackage{isomath}
\usepackage{upgreek}
\usepackage{comment,enumerate,verbatim}

\usepackage[english]{babel}
\usepackage[final,babel]{microtype}%\usepackage[dvipsnames]{color}
%\usefonttheme{professionalfonts}
%\usefonttheme{serif}

\newcommand{\reals}{\mathbf{R}}
\newcommand{\complex}{\mathbf{C}}
\newcommand{\integers}{\mathbf{Z}}
\DeclareMathOperator{\range}{range}
\DeclareMathOperator{\domain}{dom}
\DeclareMathOperator{\codomain}{codomain}
\DeclareMathOperator{\sspan}{span}
\DeclareMathOperator{\F}{F}
\DeclareMathOperator{\G}{G}
\DeclareMathOperator{\B}{B}
\DeclareMathOperator{\D}{D}
\DeclareMathOperator{\id}{id}
\DeclareMathOperator{\ball}{ball}

\usepackage{graphicx}
\usepackage{color}
\usepackage{amsmath}
\DeclareMathOperator{\nullspace}{nullity}
\theoremstyle{definition}
\newtheorem{mydef}{Definition}
\newtheorem{myqdef}{Quasi-definition}
\newtheorem{myex}{Example}
\newtheorem{myth}{Theorem} 
\newtheorem{myfact}{Fact}
\newtheorem{metathm}{Meta Theorem}
\newtheorem{Question}{Question}
\newtheorem{Answer}{Answer}
\newtheorem{myproof}{Proof}
\newtheorem{hurestic}{Hurestic}

%\usepackage{array}   % for \newcolumntype macro
%\newcolumntype{L}{>{$}l<{$}} % math-mode version of "l" column type

\newenvironment{alphalist}{
  \vspace{-0.4in}
  \begin{enumerate}[(a)]
    \addtolength{\itemsep}{1.0\itemsep}}
  {\end{enumerate}}

\newenvironment{snowflakelist}{
  \vspace{-0.4in}
  \begin{enumerate}[{\color{green} \textleaf}]
    \addtolength{\itemsep}{-1.2\itemsep}}
  {\end{enumerate}}



\newenvironment{handlist}{
 \vspace{-0.2in}
  \begin{enumerate}[{\color{green} \leftthumbsup}]
    \addtolength{\itemsep}{-1.0\itemsep}}
  {\end{enumerate}}

\newenvironment{checklist}{
  \begin{enumerate}[\ding{52}]
    \addtolength{\itemsep}{-1.0\itemsep}}
  {\end{enumerate}}

\newenvironment{numberlist}
   {\begin{enumerate}[(1)]
       \addtolength{\itemsep}{-0.5\itemsep}}
     {\end{enumerate}}
\usepackage{amsfonts}
\makeatletter
\def\amsbb{\use@mathgroup \M@U \symAMSb}
\makeatother
\usepackage{bbold}

\usepackage{array}
\newcolumntype{C}{>$c<$}

\newcommand{\llnot}{\lnot \,} % is accepted
\DeclareMathOperator{\3F2}{{}_3  F_2}
\newcommand\pochhammer[2]{\left[\genfrac..{0pt}{}{#1}{#2}\right]}
\newmuskip\pFqmuskip

\newcommand*\pFq[6][8]{%
  \begingroup % only local assignments
  \pFqmuskip=#1mu\relax
  % make the comma math active
  % \mathcode`\,=\string"8000
  % and define it to be \pFqcomma
  \begingroup\lccode`\~=`\,
  \lowercase{\endgroup\let~}\pFqcomma
  % typeset the formula
      {}_{#2}\!\F_{#3}{\left[\genfrac..{0pt}{}{#4}{#5};#6\right]}   %\F{\left[\genfrac..{0pt}{}{#4}{#5};#6\right]}  %alt: {}_{#2}\F_{#3}{\left[\genfrac..{0pt}{}{#4}{#5};#6\right]}
  \endgroup
}

\newcommand{\pFqcomma}{\mskip \pFqmuskip}
\newcommand{\mydash}{\text{--}}


%------------------

\subtitle{Lesson 2}
\title{\textbf{Using \TeX\/  and \LaTeX}}


%\author[Barton Willis] % (optional, for multiple authors)
%{Barton~Willis}%
%\institute[UNK] % (optional)

%{
 % \inst{1}%
%  ``The secret of getting ahead is getting started.'' Mark Twain
%   }
  \date{}
 

\usepackage{courier}
%\lstset{basicstyle=\ttfamily\footnotesize,breaklines=true}
%\lstset{framextopmargin=50pt,frame=bottomline}


%\begin{document}



%--------
%usepackage[usenames,dvipsnames,svgnames,table]{color}



\begin{document}


\maketitle

\begin{frame}{\TeX and \LaTeX}

\begin{alphalist}

\item \TeX\/ is a system for \emph{typesetting} documents, especially documents that use mathematical notation.

\item Typing mathematics using a word processor is clumsy.

\item Leslie Lamport created \LaTeX,  an add-on to \TeX\/.  

\item  \TeX\/  allows the author to focus on \emph{content} and less on \emph{appearance}.

\item This document was typeset using  \LaTeX\/.


\end{alphalist}

\vspace{0.2in}

\begin{quote}
\textcolor{blue}{ “Thinking doesn't guarantee that we won't make mistakes. But not thinking  guarantees that we will.''  \hfill  {\sc  Leslie Lamport}}
\end{quote}
\end{frame}

\begin{frame}{Preamble}

\begin{alphalist}


\item The start of a \LaTeX\/ file has commands that control the typeface, font, spacing, and more.

\vspace{0.15in}

\item This part of a \LaTeX\/  file is  called the \emph{preamble}.

\vspace{0.15in}

\item For the most part,  you do not need to fiddle  with the preamble--just use the preamble of the problem set.

\vspace{0.15in}

\item In the preamble, you can define your own commands.

\end{alphalist}

\end{frame}

\begin{frame}[fragile]{Text}


\vspace{0.25in}

\begin{alphalist}

\item Following the preamble, the text goes in between  \begin{verbatim} \begin{document}  \end{verbatim}  and
\begin{verbatim} \end{document}  \end{verbatim}. This is an example of an \emph{environment}.


\item   Words spacing is handled for you:

\begin{example}
 Once processed, the text
\begin{verbatim} Every    function that is a derivative has             
       the  intermediate value property. \end{verbatim}

\vspace{0.12in}
typesets as 

\vspace{0.2in}

Every     function that is a derivative has            the  intermediate value property. 
\end{example}
 
 \item  To start a new paragraph, leave a blank line.
 
 \vspace{0.2in}
\end{alphalist}



\end{frame}

\begin{frame}[fragile]{Problem sets}


For a problem set, type your answer following the question, surrounded by a \emph{solution environment}

\begin{example}
\begin{verbatim}
\question [3] Write the statement  \emph{For 
every positive real  number \(x\), there is a 
positive real  number \(y\) such that \( y < x\) } 
in symbolic form.

\begin{solution}
 This is my answer, and I'm sticking to it.
\end{solution}

\end{verbatim}

\end{example}


\end{frame}

\begin{frame}[fragile]{Finish what you start}

If you start an environment, such as  

\begin{verbatim}   \begin{solution}  \end{verbatim}, 

be sure to terminate it with     


\begin{verbatim} \end{solution}  \end{verbatim}.

\

If you don't properly end an environment, you'll get errors that might be difficult to understand.


\end{frame}


\begin{frame}[fragile]{The math environment}


Within text, put mathematics between  \(\backslash (  \) and  \(\backslash ) \).   For example

\begin{example}
\begin{verbatim}Define  \( F = x  \in \reals} \mapsto x^2 \cos(x)\)  \end{verbatim}


Typeset, this is: Define $ F = x  \in \reals \mapsto x^2 \cos(x)    $ 
\end{example}

\vspace{0.5in}

\begin{alphalist}

\item The command \( \backslash \mathrm{reals}   \) is  defined in the preamble of our problem sets.  We use it to typeset \(\reals\).

\item Function names that have two or more characters should be in a non-italic font. 

\item To  typeset the cosine function, use the command \(\backslash \mathrm{cos}\), not \(\mathrm{cos}\).

\end{alphalist}
\end{frame}

\begin{frame}[fragile]{The math environment}
To put mathematics on a separate line, use the \(\backslash [    \phantom{XXX}  \backslash] \) environment; for example

\begin{example}
\begin{verbatim}   
We have shown that
\[
  E = m c^2 .
 \]
\end{verbatim}

Typeset: We have shown that
\[  E = m c^2. \]    
\end{example}

\end{frame}

\begin{frame}[fragile]{Alignments}

\begin{example}

\begin{align*}
    0 < a <  1  &\implies 0 < a^2 < 1, \\
                      & \implies 0 < 1 - a^2, \\
                       &\equiv  1-a^2 > 0.
\end{align*}

enter


\begin{verbatim}
\begin{align*}
    0 < a <  1  &\implies 0 < a^2 < 1, \\
                      & \implies 0 < 1 - a^2, \\
                      &\equiv  1-a^2 > 0.
\end{align*}
\end{verbatim}
\end{example}


To start a new line, terminate with a double slash; to align on a symbol, put an ampersand before the symbol.
\end{frame}

\begin{frame}[fragile]{Mistakes? Me, never}


\begin{alphalist}

\item If an environment isn't closed, your file will not process and you will get an error message.

\item The error message might indicate where the error is located.

\item If you have trouble finding the location of an error,  try placing hunks of text inside a  \(\backslash  \mathrm{begin}  \{ \mathrm{comment} \} \) environment.  \LaTeX\/ ignores text in a comment environment.

\item To use the comment environment, the preamble needs the command   \(\backslash\mathrm{usepackage} \{ \mathrm{comment}\} \).


\item When the error vanishes, you know it is inside a comment environment.


\end{alphalist}
\end{frame}

\begin{frame}[fragile]{Help}

\begin{alphalist}


\item I sometimes forget how to do something using \LaTeX,  but 

\item almost surely, the answer can be found with a web search.

\item \textbf{Beware:}  A web search that includes the word ``latex'' might result in some  not particularly safe for class items.



\end{alphalist}
\end{frame}




\begin{frame}[fragile]{Further study}

\tiny
\begin{thebibliography}{10}

\bibitem{Overleaf}
\alert{Overleaf}
\newblock{\url{https://www.overleaf.com/learn}}

\bibitem{Overleaf}
\alert{Overleaf}
\newblock{\url{https://www.overleaf.com/learn/latex/Learn_LaTeX_in_30_minutes}}

\bibitem{YouTube}
\alert{YouTube}
\newblock{\url{https://www.youtube.com/watch?v=g8Ejj0T0yG4}}

\bibitem{YouTube}
\alert{YouTube}
\newblock{\url{https://www.youtube.com/watch?v=P5EWoPOnZTU}}

\bibitem{Other}
\alert{Latex Cheat Sheet}
\newblock{\url{http://joshua.smcvt.edu/undergradmath/undergradmath_0.png}}

\bibitem{Other}
\alert{Latex Cheat Sheet}
\newblock{\url{http://joshua.smcvt.edu/undergradmath/undergradmath_1.png}}

\bibitem{Other}
\alert{Symbol Cheat Sheet}
\newblock{\url{https://oeis.org/wiki/List\_of\_LaTeX\_mathematical_symbols}}

\end{thebibliography}


\end{frame}

\end{document}
