\documentclass[fleqn]{beamer}
%\usetheme[height=7mm]{Rochester}
\usetheme{Boadilla} %{Rochester}

\setbeamertemplate{footline}[text line]{%
\parbox{\linewidth}{\vspace*{-8pt}\hfill\insertshortauthor\hfill\insertpagenumber}}
\setbeamertemplate{navigation symbols}{}
%\author[BW]{Dr.\ Barton Willis}
\usepackage{amsmath}\usepackage{amsthm}
\usepackage{isomath}
\usepackage{upgreek}
\usepackage{comment,enumerate,xcolor}

\usepackage[english]{babel}
\usepackage[final,babel]{microtype}%\usepackage[dvipsnames]{color}
\usefonttheme{professionalfonts}
%\usefonttheme{serif}

\newcommand{\reals}{\mathbf{R}}
\newcommand{\complex}{\mathbf{C}}
\newcommand{\integers}{\mathbf{Z}}
\DeclareMathOperator{\range}{range}
\DeclareMathOperator{\domain}{dom}
\DeclareMathOperator{\dom}{dom}
\DeclareMathOperator{\codomain}{codomain}
\DeclareMathOperator{\sspan}{span}
\DeclareMathOperator{\F}{F}
\DeclareMathOperator{\G}{G}
\DeclareMathOperator{\B}{B}
\DeclareMathOperator{\D}{D}
\DeclareMathOperator{\id}{id}
\DeclareMathOperator{\ball}{ball}

\usepackage{graphicx}
\usepackage{color}
\usepackage{amsmath}
\DeclareMathOperator{\nullspace}{nullity}
\theoremstyle{definition}
\newtheorem{mydef}{Definition}
\newtheorem{myqdef}{Quasi-definition}
\newtheorem{myex}{Example}
\newtheorem{myth}{Theorem}
\newtheorem{myfact}{Fact}
\newtheorem{metathm}{Meta Theorem}
\newtheorem{Question}{Question}
\newtheorem{Answer}{Answer}
\newtheorem{myproof}{Proof}
\newtheorem{hurestic}{Hurestic}

%\usepackage{array}   % for \newcolumntype macro
%\newcolumntype{L}{>{$}l<{$}} % math-mode version of "l" column type

\newenvironment{alphalist}{
  \vspace{-0.4in}
  \begin{enumerate}[(a)]
    \addtolength{\itemsep}{1.0\itemsep}}
  {\end{enumerate}}



\usepackage{pifont}

\newenvironment{checklist}{
  \begin{enumerate}[\ding{51}]
    \addtolength{\itemsep}{-0.0\itemsep}}
  {\end{enumerate}}

\newenvironment{numberlist}
   {\begin{enumerate}[(1)]
       \addtolength{\itemsep}{-0.5\itemsep}}
     {\end{enumerate}}
\usepackage{amsfonts}
\makeatletter
\def\amsbb{\use@mathgroup \M@U \symAMSb}
\makeatother
\usepackage{bbold}



\newcommand{\llnot}{\lnot \,} % is accepted


%\subtitle{Lesson 4 \\
%\vspace{1.0in}
 %\tiny Barton Willis, Attribution 4.0 International (CC BY 4.0), 2020 \normalsize}
\title{\textbf{Sets as ordered pairs}}
%\author[Barton Willis] % (optional, for multiple authors)
%{Barton~Willis}%
%\institute[UNK] % (optional)

%{
 % \inst{1}%
%  ``The secret of getting ahead is getting started.'' Mark Twain
%   }
  \date{}


%\usepackage{courier}
%\lstset{basicstyle=\ttfamily\footnotesize,breaklines=true}
%\lstset{framextopmargin=50pt,frame=bottomline}


%\begin{document}



%--------
%usepackage[usenames,dvipsnames,svgnames,table]{color}



\begin{document}

\frame{\titlepage}

\begin{frame}{Ordered pairs}

An ordered pair is a familiar object--the Cartesian coordinates of a point in a plane, 
for example, is an ordered pair of real numbers.

\begin{myex}
Examples of ordered pairs of real numbers:
\[
    (0,0), \quad (2,-6),  \quad  (2, \sqrt{42}), \quad   (107, 28).
\]
\end{myex}

\begin{checklist}

\item We say that \(a\) is the \emph{first coordinate} of the ordered pair \((a,b)\), and \(b\) is its \emph{second coordinate}.

\item  We efine equality of ordered pairs using   \([ (a,b) = (a^\prime, b^\prime)]   \equiv  [a = a^\prime] \land  [b = b^\prime] \).

\item Of course, the symbols \(a\) through \(b^\prime\) need to be objects for which 
equality is defined.
\end{checklist}

\end{frame}
\begin{frame}{As nice this may be}

\textbf{Question} Ordered pairs are somewhat like sets, but the order matters.  
Can we define an ordered pair as a set?

\textbf{Answer}   Sure.  To an ordered pair \((a,b)\) we associate it with the 
set  \(\{\{a\}, \{a,b\}\} \).

\vspace{0.2in}

\begin{checklist}

\item Since \(\{\{a\}, \{a,b\}\} \) is  a set of sets, we can form the intersection 
of its members. Define  \(I = \{\{a\}, \{a,b\}\} \). Then
\[
   \underset{x \in I}{\cap} x =  \{a\} \cap \{a,b\} = \{a\}.
\]
So the intersection of the member of \(I\) gives the first coordinate of the 
ordered pair \((a,b)\).

\item How do we extract the second coordinate?
\[
    \underset{x \in I}{\cup} x   \setminus \underset{x \in I}{\cap} x  = \{a,b\} \setminus \{a\} = \{b\}.
\]



\end{checklist}


\end{frame}
\begin{frame}

\begin{myth}  If  \(  \{\{a\}, \{a,b\} \} =   \{\{a^\prime\}, \{a^\prime,b^\prime\} \}\), 
  then \(a = a^\prime\) and  \(b = b^\prime\).
\end{myth}

\begin{checklist}

\item A proof uses the ingredients:  If \( \{a \} = \{a^\prime \}\), 
then \(a = a^\prime\). It also uses
\[
   \underset{x \in I}{\cap} x =  \{a\} \cap \{a,b\} = \{a\},
\]
and
\[
    \underset{x \in I}{\cup} x \setminus \underset{x \in I}{\cap} x  = \{a,b\} \setminus \{a\} = \{b\}.
\]
where \(I = \{\{a\}, \{a,b\}\} \).

\end{checklist}
\end{frame}
\end{document}
