\documentclass[12pt,fleqn,answers]{exam}
\usepackage{pifont}
\usepackage{dingbat}
\usepackage{amsmath,amssymb}
\usepackage{epsfig}
\usepackage[]{hyperref}
\usepackage{geometry}
\geometry{letterpaper, margin=0.5in}
\addpoints
\boxedpoints
\pointsinmargin
\pointname{pts}

\usepackage{enumerate}
\newenvironment{alphalist}{
  %\vspace{-0.4in}
  \begin{enumerate}[(a)]
    \addtolength{\itemsep}{0.0\itemsep}}
  {\end{enumerate}}

\usepackage[activate={true,nocompatibility},final,tracking=true,kerning=true,factor=1100,stretch=10,shrink=10]{microtype}
\usepackage[american]{babel}
%\usepackage[T1]{fontenc}
\usepackage{fourier}
\usepackage{isomath}
\usepackage{upgreek,amsmath}
\usepackage{amssymb}

\newcommand{\dotprod}{\, {\scriptzcriptztyle
    \stackrel{\bullet}{{}}}\,}

\newcommand{\reals}{\mathbf{R}}
\newcommand{\lub}{\mathrm{lub}} 
\newcommand{\glb}{\mathrm{glb}} 
\newcommand{\complex}{\mathbf{C}}
\newcommand{\dom}{\mbox{dom}}
\newcommand{\cover}{{\mathcal C}}
\newcommand{\integers}{\mathbf{Z}}
\newcommand{\vi}{\, \mathbf{i}}
\newcommand{\vj}{\, \mathbf{j}}
\newcommand{\vk}{\, \mathbf{k}}
\newcommand{\bi}{\, \mathbf{i}}
\newcommand{\bj}{\, \mathbf{j}}
\newcommand{\bk}{\, \mathbf{k}}
\DeclareMathOperator{\Arg}{\mathrm{Arg}}
\DeclareMathOperator{\Ln}{\mathrm{Ln}}
\newcommand{\imag}{\, \mathrm{i}}

\usepackage{graphicx}
\newcommand\AM{{\sc am}}
\newcommand\PM{{\sc pm}}
     
\usepackage{color}
\shadedsolutions
\definecolor{SolutionColor}{rgb}{0.8,0.9,1}

\newcommand{\quiz}{3}
\newcommand{\term}{Fall}
\newcommand{\due}{Saturday 10 September  at 11:59 \PM}
\begin{document}
\large
\vspace{0.1in}
\noindent\makebox[3.0truein][l]{{\bf MATH 460}}
{\bf Name:}  \\
\noindent \makebox[3.0truein][l]{\bf Homework   \quiz, \term \/ \the\year}
%{\bf Row:}\hrulefill\
\vspace{0.1in}

\begin{quote}
    \fbox{I have neither given nor received unauthorized assistance on this assignment.}
    \end{quote}
\noindent  Homework    \quiz\/  has questions 1 through  \numquestions \/ with a total of  \numpoints\/  points.   Edit this file and append you answers using La\TeX. Be sure to fill in your name. Upload the converted pdf of your work to Canvas.   This assignment is due \emph{\due}.

\vspace{0.1in}

\noindent{\textbf{Link to your Overleaf work: }}\url{XXX}

\begin{questions} 

\question[5] Show that
\(
 \left(\forall x \in (-1,1) \right) 
  \left(\exists r \in \reals_{>0} \right)
    \left( (x-r,x+r) \subset (-1,1) \right)
\).

\begin{solution}  
\end{solution}

\question[5] Define $S = \{(-k,k) | k \in \integers_{>0} \}$. Show that
 $\underset{q \in S}{\cup} q = \reals$.
 \begin{solution}  
 \end{solution}

 \question[5] On $\reals^2$ define the binary operators
 $+$ and $\times$ by
\begin{align*}
  (a,b) + (c,d) &= (a+c,b+d),\\
  (a,b) \times (c,d) &= (ac+2bd, ad+bc).
\end{align*}
These operators are commutative and associative. Further, the
additive identity is $(0,0)$ and the multiplicative identity is
$(1,0)$. Given these facts, show that $(\reals^2, +, \times)$ is a 
field.
\begin{solution}  
\end{solution}

\question[5] Show that the complex field is not ordered. Hint: Suppose 
it is. Let $P$ be its positive set. Since $\mathrm{i} \neq 0$, either
$\mathrm{i} \in P$ or $-\mathrm{i} \in P$. Show that both 
$\mathrm{i} \in P$ or $-\mathrm{i} \in P$ are 
contradictions.
\begin{solution}  
  We will prove this by contradiction. Suppose the complex field
  is ordered, and let $P$ be its positive set. Since $\mathrm{i}  \neq 0$,
  either $\mathrm{i} \in P$ or $-\mathrm{i} \in P$. If
  $\mathrm{i} \in P$, closer of $P$ under multiplication implies
  that $\mathrm{i}^2 \in P$ and   $\mathrm{i}^4 \in P$. But this
  says that $-1 \in P$ and $1 \in P$. That violates tricotomy.
  
  \quad Similarly, the assumption $-\mathrm{i} \in P$ violates
  tricotomy;  therefore the assumption the complex field is ordered is false.


  \textbf{Fun Fact} It is possible to define $<$ on the complex field that
  has the properties
  \begin{alphalist}
    \item for all $a,b \in \complex$ exactly one of the following
    is true:  $a<b$ or $a=b$ or $b < a$.
    \item for all $a,b,c \in \complex$, we have $a<b$ and $b < c$ implies
     $a < c$.
  \end{alphalist}
But the set $\{z \in \complex | 0 < z\}$ does not have the properties 
required by an ordered field to be a positive set.

\end{solution}

\end{questions}



\end{document}