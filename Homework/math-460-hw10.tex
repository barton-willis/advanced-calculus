\documentclass[12pt,fleqn,answers]{exam}
\usepackage{pifont}
\usepackage{dingbat}
\usepackage{amsmath,amssymb}
\usepackage{fleqn}
\usepackage{epsfig}
%\usepackage{mathptm}
%\usepackage{euler}
\usepackage{bbding}
\usepackage{url}
\addpoints
\boxedpoints
\pointsinmargin
\pointname{pts}

\usepackage{xcolor}
\usepackage{framed}
\colorlet{shadecolor}{lightgray!15}
\newenvironment{myproof}
  {\begin{shaded}\begin{proof}}
  {\end{proof}\end{shaded}}

\usepackage{amsthm}
\newtheorem{prop}{Proposition}

\usepackage[activate={true,nocompatibility},final,tracking=true,kerning=true,factor=1100,stretch=10,shrink=10]{microtype}
\usepackage[american]{babel}
%\usepackage[T1]{fontenc}
\usepackage{fourier}
\usepackage{isomath}
\usepackage{upgreek,amsmath}
\usepackage{amssymb}
%\usepackage[euler-digits,euler-hat-accent,T1]{eulervm}

\newcommand{\dotprod}{\, {\scriptzcriptztyle
    \stackrel{\bullet}{{}}}\,}

\newcommand{\reals}{\mathbf{R}}
\newcommand{\complex}{\mathbf{C}}
\newcommand{\dom}{\mbox{dom}}
\newcommand{\cover}{{\mathcal C}}
\newcommand{\rat}{\mathbf{Q}}

\newcommand{\curly}[1]{\mathcal #1}
\newcommand{\integers}{\mathbf{Z}}
\newcommand{\vi}{\, \mathbf{i}}
\newcommand{\vj}{\, \mathbf{j}}
\newcommand{\vk}{\, \mathbf{k}}
\newcommand{\bi}{\, \mathbf{i}}
\newcommand{\bj}{\, \mathbf{j}}
\newcommand{\bk}{\, \mathbf{k}}
\DeclareMathOperator{\Arg}{\mathrm{Arg}}
\DeclareMathOperator{\Ln}{\mathrm{Ln}}
\newcommand{\imag}{\, \mathrm{i}}
\newcommand{\range}{\mathrm{range}}
\newcommand{\true}{\mathrm{True}}
\newcommand{\saw}{\mathrm{saw}}
\usepackage{enumerate}
\newenvironment{alphalist}{
  \begin{enumerate}[(a)]
    \addtolength{\itemsep}{-0.5\itemsep}}
  {\end{enumerate}}
 
\usepackage{amsthm}
\newtheorem{Rubbish}{Theorem}
\usepackage{graphicx}

%\usepackage{tgschola} %to look retro
\newenvironment{mypar}[2]
  {\begin{list}{}%
    {\setlength\leftmargin{#1}
    \setlength\rightmargin{#2}}
    \item[]}
  {\end{list}}
  
\newcommand{\quiz}{10}
\newcommand{\term}{Fall}

\usepackage{xspace}
\makeatletter
\DeclareRobustCommand{\maybefakesc}[1]{%
  \ifnum\pdfstrcmp{\f@series}{\bfdefault}=\z@
    {\fontsize{\dimexpr0.8\dimexpr\f@size pt\relax}{0}\selectfont\uppercase{#1}}%
  \else
    \textsc{#1}%
  \fi
}
\newcommand\AM{\,\maybefakesc{am}\xspace}
\newcommand\PM{\,\maybefakesc{pm}\xspace}

\begin{document}
\large
\vspace{0.1in}
\noindent\makebox[3.0truein][l]{ \textbf{MATH 460}}
{\bf Name:}  \\
\noindent \makebox[3.0truein][l]{\textbf{Homework \quiz, \term \/ \the\year}}
%{\bf Row:}\hrulefill\
\vspace{0.1in}

\noindent  Homework \quiz\/  has questions 1 through  \numquestions \/ with a total 
of  \numpoints\/  points. 
This work is due \textbf{Saturday 11 November} at 11:59 \PM.

\textbf{For this assignment, neatly handwrite your solutions and submit
a digitized version to Canvas.}

\vspace{0.1in}


\begin{questions}

 
    \question[10] Define a function $F = x \in \reals \mapsto \begin{cases}
        1 & x < 2 \\ 2 & x \geq 2 \end{cases}$.  Find $F^{-1} \left(\left(-1/2, 1/2\right) \right)$.
        Explain why this shows that $F$ is not continuous on $\reals$.

        You may use a convincing picture to find $F^{-1} \left(\left(-1/2, 1/2\right) \right)$.

        \question[10] Define $F = x \in [0,1] \mapsto x$. Find an open set $\mathcal{G}$
        such that $F^{-1}\left(\left(-1/2,1/2\right)\right) = \mathcal{G} \cap [0,1]$.
        This shows that $F^{-1}\left(\left(-1/2,1/2\right)\right)$ is open relative to 
        $\dom(F)$.

        \question[10] Define $F = (2x -1) \begin{cases}  \frac{1}{8} (2x+1) & x < 1/2 \\
        -\frac{1}{8} (2x - 3) & x \geq 1/2 \end{cases}$. Show that 
        $F$ is differentiable at $1/2$. To do this you can either
        use a limit of a Newton quotient or you can show that the
        function $x \in \reals \mapsto \begin{cases}  \frac{1}{8} (2x+1) & x < 1/2 \\
            -\frac{1}{8} (2x - 3) & x \geq 1/2 \end{cases}$ is
            continuous at $1/2$.
        
        Finally, show that 
        $
         F^\prime = x \in \reals \mapsto \min(x,1-x).
        $
        




\end{questions}
\end{document}