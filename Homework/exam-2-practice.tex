\documentclass[12pt,fleqn]{exam}
\usepackage{pifont}
\usepackage{dingbat}
\usepackage{amsmath}
\usepackage{fleqn}
\usepackage{epsfig}
\usepackage{mathptm}
\usepackage{amssymb}

\addpoints
\boxedpoints
\pointsinmargin
\pointname{pts}

\newcommand{\reals}{\mathbf{R}}
\newcommand{\ball}{\mathrm{ball}}
\newcommand{\integers}{\mathbf{Z}}
\newcommand{\LP}{\mathrm{LP}}
\newcommand{\dotprod}{\, {\scriptzcriptztyle \stackrel{\bullet}{{}}}\,}
\begin{document}
\large
\vspace{0.1in}
\noindent\makebox[3.0truein][l]{{\bf Advanced Calculus, Fall 2022}}
{\bf Name:}\hrulefill\
\noindent \makebox[3.0truein][l]{\bf Practice Exam II}
{\bf Row and Seat:}\hrulefill\
%\normalsize

\vspace{0.1in}


\begin{questions}

\question Show that the sequence $k \in \integers_{>0} \mapsto \frac{k+1}{k+5}$ converges.

\question Give an example of a convergent subsequence of $k \in \integers_{>0} \mapsto (-1)^k$.

\question Show that sequence $k \in \integers_{>0} \mapsto \begin{cases}  k!  & k < 1000 \\
\frac{k+1}{k+5} & k \geq 1000 \end{cases} $ converges.

\question  Use the QRS definition of an open set to show that interval \((0,1)\) is open.

\question  Use the QRS definition of a closed set to show that interval \([0,1]\) is closed.

\question   Use the QRS definitions of open and closed to show that the set \(\reals\) is open and closed.

\begin{solution}
Let \(x\) be a real number.  We have \(\ball(x,1) \subset
\reals\); therefore, \(\reals\) is open.  To show that \(\reals\) is
closed, we'll show that \(\reals^c = \varnothing\) is open. To
show that \(\varnothing\) is open, no proof is required (give
me an element of \(\varnothing\), and I'll surround it with an
open interval.  I'm waiting\dots).
 \end{solution}
 
\question   Use the QRS definition of a boundary point to show that \(\partial(0,1] = \{0,1\}\).  Use this
result to explain why \((0,1]\) is not closed.
\begin{solution}
 First, we'll show that \( 0 \in \partial(0,1]
\). Let \(\delta\) be a positive real number.  We have \(-\delta/2
\notin (0,1]\). Further define
\( x^\star = \displaystyle \begin{cases} \frac{\delta}{2} & \mbox{ if } \delta <
  1 \\ \frac{1}{2} & \mbox { if } \delta \geq 1 
   \end{cases} \).  Then \(x^\star \in (0,1]\) and \(x^\star \in
   \ball(0,\delta)\). I'll leave it to you to show that  \(1 \in (0,1]\). 


\quad The set  \((0,1]\) is not closed because \(0 \in
 \partial(0,1]\) and \(0 \notin (0,1] \). (A closed set
must contain all of its boundary points.)
\end{solution}

\question  Use the QRS definition to show that \(0 \not \in  \LP(\integers)\).

\begin{solution}
We have \(\ball(0,1) \cap \integers = \{0\}\). Since
\(\{0\}\) isn't an infinite set, 0 isn't a limit point of \(\integers\).
\end{solution}

%\vspace*{3.0in}

\question  Show that the function
\(\displaystyle
  F(x) = \begin{cases} -1, & \mbox{if } x < 5, \\
                        1, & \mbox{if } x \geq 5
         \end{cases}
\)
does not have a limit toward 5.

%\textbf{Proof}  This barely differs from the problem in the class notes.


\question   Show that the function \(F(x) = x^2\) has a limit
toward 2.
\begin{solution}
Let \(\varepsilon\) be a positive real number.  Choose
\(\delta = \mbox{min} \{1, \frac{\delta}{5} \}\).  For \(x \in
\ball(2,\delta)\), we have
\begin{align*}
  | x^2 - 4 | &= |x -2| |x + 2|, \\
              &< |x + 2| \delta, \\
              &< (|x| + 2) \delta, \\
              &< 5 \delta, \\
              &\leq \varepsilon.
\end{align*}
\end{solution}

\question   Show that the set \((0,\infty)\) is not compact by 
showing that there is an open cover of \((0,\infty)\) that has no
finite subcover.

\begin{solution}
 For \(k \in \integers_{<0}\), define \(I_k = (-k,k)\).
The set \(\mathcal{C} = \{I_1, I_2, \dots \}\) is a cover of 
\((0,\infty)\).  The union of every finite subset of \(\mathcal{C}\)
is bounded. Thus no finite subset of \(\mathcal{C}\) is a cover of
\((0,\infty)\); therefore, \((0,\infty)\) is not compact.
\end{solution}

\question   Show that if a subset of \(\reals\) is not bounded, it
is not compact. Do this using the definition of compact that involves
open covers.

\begin{solution}
\textbf{Proof} See your class notes.
\end{solution}






\question   Show that the union of two compact sets is compact.  Do
this using  the definition of compact that involves
open covers.

\begin{solution}
 Let \(F_1\) and \(F_2\) be compact, and let
\(\mathcal{C}\) be an open cover of \(F_1 \cup  F_2\). Then
\(\mathcal{C}\) is an open cover of \(F_1\) and \(\mathcal{C}\) is an
open cover of \(F_2\). Since \(F_1\) and \(F_2\) are compact,
there are sets \(\mathcal{C}_1 \subset \mathcal{C}\) and
\(\mathcal{C}_2 \subset \mathcal{C}\) such that 
\(
  F_1 \subset \underset{x \in \mathcal{C}_1}{\cup} x\) and
\(
  F_2 \subset \underset{x \in \mathcal{C}_2}{\cup} x.
\)
Thus
\(\displaystyle
   F_1 \cup  F_2  \subset \underset{x \in \mathcal{C}_1 \cup \mathcal{C}_2 }{\cup} x
\).
Since \(\mathcal{C}_1 \cup \mathcal{C}_2\) is finite, it follows that 
\( F_1 \cup  F_2 \) is compact.

\end{solution}

\question Show that if sets $A$ and $B$ are closed, so it $A \cup B$.

\question   Give an example of open sets \(\mathcal{G}_1,
\mathcal{G}_3, \mathcal{G}_3, \dots \) such that the intersection
\( \displaystyle
  \underset{k \in \mathbf{Z}_{> 0}}{\cap} \mathcal{G}_k
\)
is not open.

\begin{solution}
\textbf{Answer} For \(k \in \mathbf{Z}_{>0}\), define \(I_k = (1-1/k,
\infty)\). We have
\(\displaystyle
  \underset{k \in \mathbf{Z}_{>0}}{\cap} I_k = [1,\infty).
\)
The set \([1,\infty)\) isn't open.

\end{solution}

\question   Let \(F : \mathbf{Z} \to \mathbf{R}\) and let
\( \displaystyle
  F(x) = \sqrt[3]{x^{14} + 1066} + \sqrt[43]{x^2 + 1776}.
\)
Either prove or disprove: 
%\begin{quote}
  The function \(F\) has a limit toward 1.
%\end{quote}

\question   Define $F =  x \in \mathbf{Z} \mapsto \sqrt[3]{x^{14} + 1066} + \sqrt[43]{x^2 + 1776}.$ Show that
$F$  is not continuous at 1.

\begin{solution}
Since \(1\) isn't a limit point of \(\mbox{dom}(F)\),
it follows that \(F\) doesn't have a limit toward 1.
\end{solution}



\question  Let \(F\) be a convergent sequence and let \(\alpha \in
\reals\).  Show that \(\alpha F\) is a convergent sequence.


\question  Let \(|F| \) be a convergent sequence.  Show that \(|F| \) is a convergent sequence.
\begin{solution}

\end{solution}

\question   Use the inequality
\(
   | \sqrt{a} - \sqrt{b} | \leq \sqrt{|a -b|},\mbox{ for } a,b > 0
\)
to show that the function
\[
   F = x \in \reals \mapsto  \sqrt{1+x},
\]
is continuous at 1.

\end{questions}
\end{document}
Show that \(A\) has at
most one infimum. Do this by assuming \(\alpha\) and \(\beta\)
are both infimums of the set \(A\) and demonstrate that \(\alpha = [1,\infty).\beta\).




\vspace*{2.5in}