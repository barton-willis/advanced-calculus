\documentclass[12pt,fleqn,answers]{exam}
\usepackage{pifont}
\usepackage{dingbat}
\usepackage{amsmath,amssymb}
\usepackage{fleqn}
\usepackage{epsfig}
%\usepackage{mathptm}
%\usepackage{euler}
\usepackage{bbding}

\usepackage{geometry}
\geometry{letterpaper, margin=0.6in}
\addpoints
\boxedpoints
\pointsinmargin
\pointname{pts}

\usepackage[activate={true,nocompatibility},final,tracking=true,kerning=true,factor=1100,stretch=10,shrink=10]{microtype}
\frenchspacing
\usepackage[american]{babel}
%\usepackage[T1]{fontenc}
\usepackage{fourier}
\usepackage{isomath}
\usepackage{upgreek,amsmath}
\usepackage{amssymb}
%\usepackage[euler-digits,euler-hat-accent,T1]{eulervm}

\newcommand{\dotprod}{\, {\scriptzcriptztyle
    \stackrel{\bullet}{{}}}\,}

\newcommand{\reals}{\mathbf{R}}
\newcommand{\complex}{\mathbf{C}}
\newcommand{\dom}{\mbox{dom}}
\newcommand{\cover}{{\mathcal C}}
\newcommand{\integers}{\mathbf{Z}}
\newcommand{\vi}{\, \mathbf{i}}
\newcommand{\vj}{\, \mathbf{j}}
\newcommand{\vk}{\, \mathbf{k}}
\newcommand{\bi}{\, \mathbf{i}}
\newcommand{\bj}{\, \mathbf{j}}
\newcommand{\bk}{\, \mathbf{k}}
\DeclareMathOperator{\Arg}{\mathrm{Arg}}
\DeclareMathOperator{\Ln}{\mathrm{Ln}}
\newcommand{\imag}{\, \mathrm{i}}
\newcommand{\true}{\, \mathrm{True}}
\usepackage{amsthm}
\usepackage{xcolor}
\newtheorem{Rubbish}{Theorem}
\usepackage{graphicx}

\usepackage{framed}

\colorlet{shadecolor}{lightgray!15}
\newenvironment{myproof}
  {\begin{shaded}\begin{proof}}
  {\end{proof}\end{shaded}}

  \newtheorem{prop}{Proposistion}
%\usepackage{tgschola} %to look retro
\newenvironment{mypar}[2]
  {\begin{list}{}%
    {\setlength\leftmargin{#1}
    \setlength\rightmargin{#2}}
    \item[]}
  {\end{list}}

  \usepackage{enumerate}
  \newenvironment{alphalist}{
  %\vspace{-0.4in}
  \begin{enumerate}[(a)]
    \addtolength{\itemsep}{-0.50\itemsep}}
  {\end{enumerate}}
\newcommand{\quiz}{1}
\newcommand{\term}{Fall}

\usepackage{xspace}
\makeatletter
\DeclareRobustCommand{\maybefakesc}[1]{%
  \ifnum\pdfstrcmp{\f@series}{\bfdefault}=\z@
    {\fontsize{\dimexpr0.8\dimexpr\f@size pt\relax}{0}\selectfont\uppercase{#1}}%
  \else
    \textsc{#1}%
  \fi
}
\newcommand\AM{\,\maybefakesc{am}\xspace}
\newcommand\PM{\,\maybefakesc{pm}\xspace}

\usepackage{mdframed}
\title{Let/Choose/Suppose Proof Examples}
\author{Fall 2023 Advanced Calculus Class}
\begin{document}

\maketitle

\section{Introduction}
Many statements in analysis consist of concatenated `for every' and `there exists' 
statements that end with an inequality. The inequality part is either true or 
false--I call it the \emph{predicate} part. We'll see that writing a statement in 
symbolic form gives us a perfect road map to constructing a proof. So, as the 
first step in constructing a proof, I encourage you to express the proposition 
symbolically.

For every qualification of the form $\forall x \in A$, our proof will state 
`Let $x \in A$.' Other than assuming that $x \in A$, we cannot make any other 
assumptions about $x$. A synonym for `let' is `allow' and `permit,'  but we'll 
stick to using `let' for consistency.  This part of a proof is easy.

For every qualification of the form $\exists x \in A$, it's just like fourth 
grade show-and-tell. We must tell the reader \emph{exactly} what member of $A$ 
we choose for $x$.  Sometimes there will be more than one choice that allows 
the proof to continue; if so, we stick to exactly one choice instead of 
\emph{clogging} our logic with extraneous possibilities.  This part of
a proof is hard--make a bad choice, and you might get stuck. 

Moving from left to right, the value $x$ we choose for  $\exists x \in A$ can 
depend on all previously defined variables (all variables to the left), but 
it cannot depend on any variables to the right. So it for a  statement fragment
\begin{equation*}
    \left(\forall x \in \reals \right) \left(\exists a \in \reals\right) \left(\forall b \in \reals\right) 
\end{equation*}
the choice for $a$ can depend on $x$, but it cannot depend on $b$.

Finally, what about `suppose?' To alert the reader that 
the truth of a statement is directly due to a hypothesis we often
introduce the hypothesis with `suppose.'

\section{Examples}

Let's practice this skill with some examples. For each given
statement, do the following: 
\begin{alphalist}
\item Write the statement symbolically.
\item Without explicitly using negation, write the negation of 
    the statement symbolically.
\item Decide if the statement is true or false.
\item  Write a proof of the statement that is true.
\end{alphalist}


\begin{prop}
  For all $x,y \in \reals$, there is $a \in \reals$ such that
    $x < y$ implies $x<a<y$.    
\end{prop}

\noindent \textbf{Solution} Symbolically, the statement is 
    \begin{equation*}
      \left(\forall x,y \in \reals\right)
      \left(\exists a \in \reals \right)
      \left ((x< y) \implies x < a < y\right).
    \end{equation*}
    And its negation is 
     \begin{equation*}
      \left(\exists x,y \in \reals\right)
      \left(\forall a \in \reals \right)
      \left ((x< y) \land (x \geq a \lor a \geq y \right).
    \end{equation*}
    We need to decide if the statement or its negation is true. The statement says that between any two real 
    numbers $x$ and $y$ with $x < y$, there is  a real number $a$ that is between $x$ and $y$. One such number is the arithmetic average. So   let's prove the statement, not its negation.
     A key ingredient to the proof is the fact that subtracting
     the same real number from both sides of an inequation yields 
     a logically equivalent inequation. Specifically  
     $(\forall a,x,y \in \reals)( (x< y) \equiv (x-a < y-a))$.   
    \begin{myproof} (BW)
     Let $x,y \in \reals$. Suppose $x < y$. Choose $a = \frac{x+y}{2}$.
     Then $a \in \reals$ as required. We have
     \begin{align*}
       \left[ x < a < y \right] 
           &\equiv \left[ x < \frac{x+y}{2} < y \right], &\mbox{(substitution)} \\
           & \equiv \left[ x - \frac{x+y}{2} < 0 < y - \frac{x+y}{2}  \right], &\mbox{(subtraction)} \\
           & \equiv \left[ \frac{x-y}{2} < 0 < \frac{y-x}{2}  \right], &\mbox{(algebra)} \\
           & \equiv \mbox{true} &((y-x > 0) \land (x-y < 0)).
     \end{align*} 
        \end{myproof}
      \noindent Notice how the proof uses `suppose $x < y$' to 
      inform the reader that what follows is something from the hypothesis.
      Also, if the fact that the arithmetic average of two numbers is between the numbers is an allowed fact, the proof
     could end after the first displayed line.

     The parenthetical remarks about substitution and subtraction can
     help a reader to understand--I encourage you to use them.
     
     \begin{prop} 
     For all $r \in \reals_{>0}$ there is $x \in [0,1)$ such that $1-r < x$. 
     \end{prop}

    \noindent \textbf{Solution} Symbolically, the statement is 
    \begin{equation*}
      \left(\forall r \in \reals_{>0}\right)\left(\exists x \in [0,1)\right)\left(1-r < x\right).
    \end{equation*}
    And its negation is 
     \begin{equation*}
       \left(\exists r \in \reals_{>0}\right)\left(\forall x \in [0,1)\right)\left(1-r \geq  x\right).
    \end{equation*}
     The statement says that there is a real number $x \in [0,1)$ that is between $1-r$ and $1$. This seems
     to be true. When $1-r > 0$, we can choose $x$ to be the arithmetic average of $1-r$; when $1-r \leq 0$,
     we can choose $x$ to be $\frac{1}{2}$.
     
      \begin{myproof} (BW) Let $r \in \reals_{>0}$. Choose $x = \begin{cases} 1 - \frac{r}{2}  & r < 1 \\ \frac{1}{2} & r \geq 1 \end{cases}$. For $r < 1$, we have
      \begin{align*}
       \left[1-r < x \right] &\equiv \left[1-r < 1 - \frac{r}{2}  \right], & \mbox{(substitution)} \\
                                  &\equiv  \left[0 <  \frac{r}{2}  \right], & \mbox{(algebra)} \\
                                  &\equiv \true.   &(0 < r < 1).
      \end{align*}
      And for $r \geq 1$, we have
       \begin{align*}
       \left[1-r < x \right] &\equiv \left[1-r < \frac{1}{2}  \right], & \mbox{(substitution)} \\
                                  &\equiv  \left[\frac{1}{2}  <  r  \right], & \mbox{(algebra)} \\
                                  &\equiv \true.   &(r \geq 1).
      \end{align*}
      \end{myproof}
     

    \begin{prop}   For all $x \in \reals_{>0}$ there is 
      $y \in \reals_{> 0}$ such that $y < x$. 
    \end{prop}
     
    \noindent \textbf{Solution} Symbolically, the statement is 
      \begin{equation*}
         \left(\forall x \in \reals_{>0}\right) \left(\exists y \in \reals_{>0}\right)(y < x).
      \end{equation*}
      And its negation is
       \begin{equation*}
         \left(\exists x \in \reals_{>0}\right) \left(\forall y \in \reals_{>0}\right)(y \geq  x).
      \end{equation*}
      The statement says that for every positive number $x$, there is a positive number $y$ that is smaller than $x$. 
      Surely this is true--we can choose, for example, the 
      arithmetic average of zero and $x$; that is choose     
      $y=x/2$.
      
      \begin{myproof} (BW) Let $x \in \reals_{>0}$. Choose $y = x/2$. Then $y \in \reals_{>0}$ as required. We have
         \begin{align*}
         \left[y < x \right] &\equiv \left[\frac{x}{2} < x \right], &\mbox{(substitution)} \\
                                   &\equiv \left[0 < \frac{x}{2}  \right], &\mbox{(algebra)} \\
                                   &\equiv \true.  &\mbox{(x > 0)}
      \end{align*}
      \end{myproof}

      
      \begin{prop} There is $y \in \reals_{> 0}$ such that for all 
        $x \in \reals_{>0}$ we have $y < x$.
      \end{prop}
       
      \noindent \textbf{Solution}  Symbolically, the statement is
       \begin{equation*}
         \left(\exists y \in \reals_{>0}\right)\left(\forall x \in \reals_{>0}\right) \left(y < x \right).
       \end{equation*}
       This proposition is the same as the previous proposition, but
       the order of the qualified statements are switched. Usually,
       order matters in mathematics (and in life), so it's possible 
       that this statement is false. Indeed, we'll show that this 
       is the case. We might be tempted to prove this proposition as

       \begin{myproof} Choose $y=\frac{x}{2}$. Let $x \in \reals_{>0}$.
        Then $y \in \reals_{>0}$ as required.
             \end{myproof}
       \noindent Let's not continue this ``proof.''  What's wrong?  Plenty--we 
       violated the strict left to right rule. We allowed the 
       first qualified variable $y$ to depend on the second variable 
       $x$. And that is not allowed.

      Returning to the proposition, it says that there is a 
      positive real number $y$ that is smaller 
       than every positive  real number $x$. Since $y$ is positive, we 
        cannot choose $y = 0$. Surely, the proposition is false;
       let's hope its negation is true; its negation is
       \begin{equation*}
         \left(\forall y \in \reals_{>0}\right)\left(\exists x \in \reals_{>0}\right) \left(y \geq  x \right).
       \end{equation*}
       This is almost surely true--it says for every positive real number $y$ there is a 
       positive real number $x$ such that
       $y \geq x$.    Sure, we can choose $x = y$, for example.

       \begin{prop} For every $y \in \reals_{> 0}$ there is  
        $x \in \reals_{>0}$ such that $y \geq  x$.
      \end{prop}

       \begin{myproof} (BW) Let $y \in \reals_{>0}$. Choose $x = y$. Then $x \in \reals_{>0}$, as 
        required. We have
       \begin{align*}
         \left[y \geq x  \right] &\equiv \left[y \geq  y \right], & \mbox{(substitution)} \\
                                      &\equiv \left[0 \geq 0 \right], &\mbox{(algebra)} \\
                                      &\equiv \true.
       \end{align*}
       
       \end{myproof}

   


       
    \begin{prop} For all $x \in \reals_{>0}$, there is $M \in \reals$ such
     that $\frac{1}{x} +1 > M$. \hfill (SB)
    \end{prop}

    \begin{prop}
      There is $M \in \reals$ such that for all $x \in \reals_{>0}$,
     we have $\frac{1}{x} + 1 > M$. \hfill (DD) 
    \end{prop}

     \begin{prop}
      There is $m \in \reals$ such that for all $x \in \reals$, we 
     have $1 + m(x-1) \leq x^2$. \hfill (TK)
     \end{prop}

     \begin{prop} For every $a \in \reals$, there is $m \in \reals$ such 
     that for all $x \in \reals$, we have $a^2 + m(x-a) \leq x^2$. \hfill (AK)
     \end{prop}

     \begin{prop} For all $x,y \in \reals$, we have $(x^2 = y^2) \implies (x=y)$. 
     \hfill (DM)
     \end{prop} 

    \begin{prop} For all $x,y \in \reals$, we have $(x^3 = y^3) \implies (x=y)$. 
    \hfill (CR) 
    \end{prop}


\end{document}