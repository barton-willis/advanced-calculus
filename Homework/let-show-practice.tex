\documentclass[12pt,fleqn,answers]{exam}
\usepackage{pifont}
\usepackage{dingbat}
\usepackage{amsmath,amssymb}
\usepackage{fleqn}
\usepackage{epsfig}
%\usepackage{mathptm}
%\usepackage{euler}
\usepackage{bbding}

\addpoints
\boxedpoints
\pointsinmargin
\pointname{pts}

\usepackage[activate={true,nocompatibility},final,tracking=true,kerning=true,factor=1100,stretch=10,shrink=10]{microtype}
\usepackage[american]{babel}
%\usepackage[T1]{fontenc}
\usepackage{fourier}
\usepackage{isomath}
\usepackage{upgreek,amsmath}
\usepackage{amssymb}
%\usepackage[euler-digits,euler-hat-accent,T1]{eulervm}

\newcommand{\dotprod}{\, {\scriptzcriptztyle
    \stackrel{\bullet}{{}}}\,}

\newcommand{\reals}{\mathbf{R}}
\newcommand{\complex}{\mathbf{C}}
\newcommand{\dom}{\mbox{dom}}
\newcommand{\cover}{{\mathcal C}}
\newcommand{\integers}{\mathbf{Z}}
\newcommand{\vi}{\, \mathbf{i}}
\newcommand{\vj}{\, \mathbf{j}}
\newcommand{\vk}{\, \mathbf{k}}
\newcommand{\bi}{\, \mathbf{i}}
\newcommand{\bj}{\, \mathbf{j}}
\newcommand{\bk}{\, \mathbf{k}}
\DeclareMathOperator{\Arg}{\mathrm{Arg}}
\DeclareMathOperator{\Ln}{\mathrm{Ln}}
\newcommand{\imag}{\, \mathrm{i}}
\usepackage{amsthm}
\newtheorem{Rubbish}{Theorem}
\usepackage{graphicx}

%\usepackage{tgschola} %to look retro
\newenvironment{mypar}[2]
  {\begin{list}{}%
    {\setlength\leftmargin{#1}
    \setlength\rightmargin{#2}}
    \item[]}
  {\end{list}}
  
\newcommand{\quiz}{1}
\newcommand{\term}{Fall}

\usepackage{xspace}
\makeatletter
\DeclareRobustCommand{\maybefakesc}[1]{%
  \ifnum\pdfstrcmp{\f@series}{\bfdefault}=\z@
    {\fontsize{\dimexpr0.8\dimexpr\f@size pt\relax}{0}\selectfont\uppercase{#1}}%
  \else
    \textsc{#1}%
  \fi
}
\newcommand\AM{\,\maybefakesc{am}\xspace}
\newcommand\PM{\,\maybefakesc{pm}\xspace}

\begin{document}

\begin{questions}
\question For each statement, do the following:   (a) Write the statement in symbolic form
    (b) Without explicitly using negation, write the negation of 
    the statement in symbolic form.
    (c) Decide if the statement is true or false.
    (d) Write a proof of the statement that is true.


\begin{parts}

    \part For all $x \in \reals_{>0}$, there is $M \in \reals$ such
     that $\frac{1}{x} +1 > M$. \hfill (SB)

         \part There is $M \in \reals$ such that for all $x \in \reals_{>0}$,
     we have $\frac{1}{x} + 1 > M$. \hfill (DD) 

     \part  There is $m \in \reals$ such that for all $x \in \reals$, we 
     have $1 + m(x-1) \leq x^2$. \hfill (TK)

     \part  For every $a \in \reals$, there is $m \in \reals$ such 
     that for all $x \in \reals$, we have $a^2 + m(x-a) \leq x^2$. \hfill (AK)

     \part For all $x,y \in \reals$, we have $(x^2 = y^2) \implies (x=y)$. 
     \hfill (DM) 

    \part For all $x,y \in \reals$, we have $(x^3 = y^3) \implies (x=y)$. 
    \hfill (CR) 

\end{parts}
\end{questions}
\end{document}