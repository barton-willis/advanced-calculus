\documentclass[12pt,fleqn,answers]{exam}
\usepackage{pifont}
\usepackage{dingbat}
\usepackage{amsmath,amssymb}
\usepackage{fleqn}
\usepackage{epsfig}
%\usepackage{mathptm}
%\usepackage{euler}
\usepackage{bbding}
\usepackage{url}
\addpoints
\boxedpoints
\pointsinmargin
\pointname{pts}

\usepackage{xcolor}
\usepackage{framed}
\colorlet{shadecolor}{lightgray!15}
\newenvironment{myproof}
  {\begin{shaded}\begin{proof}}
  {\end{proof}\end{shaded}}

\usepackage{amsthm}
\newtheorem{prop}{Proposition}

\usepackage[activate={true,nocompatibility},final,tracking=true,kerning=true,factor=1100,stretch=10,shrink=10]{microtype}
\usepackage[american]{babel}
%\usepackage[T1]{fontenc}
\usepackage{fourier}
\usepackage{isomath}
\usepackage{upgreek,amsmath}
\usepackage{amssymb}
%\usepackage[euler-digits,euler-hat-accent,T1]{eulervm}

\newcommand{\dotprod}{\, {\scriptzcriptztyle
    \stackrel{\bullet}{{}}}\,}

\newcommand{\reals}{\mathbf{R}}
\newcommand{\complex}{\mathbf{C}}
\newcommand{\dom}{\mbox{dom}}
\newcommand{\cover}{{\mathcal C}}
\newcommand{\rat}{\mathbf{Q}}

\newcommand{\curly}[1]{\mathcal #1}
\newcommand{\integers}{\mathbf{Z}}
\newcommand{\vi}{\, \mathbf{i}}
\newcommand{\vj}{\, \mathbf{j}}
\newcommand{\vk}{\, \mathbf{k}}
\newcommand{\bi}{\, \mathbf{i}}
\newcommand{\bj}{\, \mathbf{j}}
\newcommand{\bk}{\, \mathbf{k}}
\DeclareMathOperator{\Arg}{\mathrm{Arg}}
\DeclareMathOperator{\Ln}{\mathrm{Ln}}
\newcommand{\imag}{\, \mathrm{i}}
\newcommand{\range}{\mathrm{range}}
\newcommand{\true}{\mathrm{True}}
\newcommand{\saw}{\mathrm{saw}}
\usepackage{enumerate}
\newenvironment{alphalist}{
  \begin{enumerate}[(a)]
    \addtolength{\itemsep}{-0.5\itemsep}}
  {\end{enumerate}}
 
\usepackage{amsthm}
\newtheorem{Rubbish}{Theorem}
\usepackage{graphicx}

%\usepackage{tgschola} %to look retro
\newenvironment{mypar}[2]
  {\begin{list}{}%
    {\setlength\leftmargin{#1}
    \setlength\rightmargin{#2}}
    \item[]}
  {\end{list}}
  
\newcommand{\quiz}{$\infty$}
\newcommand{\term}{Fall}

\usepackage{xspace}
\makeatletter
\DeclareRobustCommand{\maybefakesc}[1]{%
  \ifnum\pdfstrcmp{\f@series}{\bfdefault}=\z@
    {\fontsize{\dimexpr0.8\dimexpr\f@size pt\relax}{0}\selectfont\uppercase{#1}}%
  \else
    \textsc{#1}%
  \fi
}
\newcommand\AM{\,\maybefakesc{am}\xspace}
\newcommand\PM{\,\maybefakesc{pm}\xspace}

\begin{document}
%\large
\vspace{0.1in}
\noindent\makebox[3.0truein][l]{ \textbf{MATH 460}}
{\bf Name:}  \\
\noindent \makebox[3.0truein][l]{\textbf{Homework \quiz, \term \/ \the\year}}
%{\bf Row:}\hrulefill\
\vspace{0.1in}

\noindent  Homework \quiz\/  has questions 1 through  \numquestions \/ with a total 
of  \numpoints\/  points. 
This work is never due.


\begin{questions}

 \question For first term calculus students, an intimidating function is $F(x) = \sqrt{x + \sqrt{x + \sqrt{x}}}$. Finding the derivative of 
 this function will strike fear many students. Using the chain rule twice and not ``simplifying'' between applications, 
  I claim that a formula for the derivative is
\begin{equation*}
F^\prime(x) = \frac{\frac{\frac{1}{2 \sqrt{x}}+1}{2 \sqrt{x+\sqrt{x}}}+1}{2 \sqrt{\sqrt{x+\sqrt{x}}+x}}.
\end{equation*}
Opinions might vary on a ``proper'' simplification, but likely most would say that the simplified expression is
\begin{equation*}
F^\prime(x) = \frac{4 \sqrt{x} \sqrt{x+\sqrt{x}}+2 \sqrt{x}+1}{8 \sqrt{x} \sqrt{x+\sqrt{x}} \sqrt{\sqrt{x+\sqrt{x}}+x}}.
  \end{equation*}
  Although all this calculation is frightening for many calculus students, it's an algorithmic process and fairly straightforward.  Teachers need to push
  the boundaries of the comfort zone of students, and this is a good problem to do that.  And just for fun the derivative of 
  $F(x) = \sqrt{x+\sqrt{x + \sqrt{x + \sqrt{x}}}}$ is the monstrous   
  \begin{equation*}
 F^\prime(x) =  \frac{\frac{\frac{\frac{1}{2 \sqrt{x}}+1}{2 \sqrt{x+\sqrt{x}}}+1}{2 \sqrt{\sqrt{x+\sqrt{x}}+x}}+1}{2 \sqrt{\sqrt{\sqrt{x+\sqrt{x}}+x}+x}}.
  \end{equation*}
  
  \quad As good as these problem is, let's kick it up an notch.  Say we infinitely repeat the pattern and ``define''  a function $Q$ by
  \begin{equation*}
 Q(x) = \sqrt{x + \sqrt{x + \sqrt{x + \cdots}}}.
\end{equation*}
In mathematics, using an ellipsis is like waving a white flag--use your imagination to continue something in the most natural way, 
but with no clue of how to define ``most natural.''  

\quad Can we give meaning to the putative function $Q$ without using an ellipsis?  Oh, sure.  Let's define a sequence of functions whose limit
is $Q$.  We can do this recursively
\begin{equation*}
   G_{n}(x) = \begin{cases}  \sqrt{x}  & n = 0 \\ \sqrt{x + G_{n-1}(x)}  & n \in \integers_{> 0} \end{cases}.
\end{equation*}
Then, for example $G_1(x) = \sqrt{x+\sqrt{x}}$ and $G_2(x) = \sqrt{x+\sqrt{x + \sqrt{x}}}$.   And it seems like
the sequence $k \in \integers_{\geq 0} \mapsto G_k$ converges to $Q$.

\quad So our question now is does the sequence $k \in \integers_{\geq 0} \mapsto G_k$ converge?  Immediately, we have a problem; we're asking a 
question about the convergence of a function-valued function, not a real-valued function.  And we don't know what that means.  But one 
possible meaning is the question:  For $x \in \reals_{\geq 0}$, does the limit
\begin{equation*}
      \lim_{n \to \infty} G_n(x) 
\end{equation*}
exist?  This notion of convergence of a function-valued sequence is known as \emph{pointwise} convergence.  But there are other definitions
of such convergence (weak convergence and norm convergence, to name a few).

\quad Your challenge is to show that for all $x \in \reals_{\geq 0}$, we have 
\begin{equation*}
      \lim_{n \to \infty} G_n(x)  = \frac{1 + \sqrt{1 + 4 x}}{2}.
\end{equation*}
Amusingly, the natural domain of $\frac{1 + \sqrt{1 + 4 x}}{2}$ is $[-1/4, \infty)$, but the domain of each function in the sequence is $[0,\infty)$.
This doesn't mean that my answer is rubbish, but it does mean that it's amusing. 

\quad If you accept this challenge, don't forget GNAT.  Try graphing $y = G_2(x), y = G_3(x), \dots$, and $y = \frac{1 + \sqrt{1 + 4 x}}{2}$.


  


\end{questions}
\end{document}