\documentclass[12pt,fleqn,answers]{exam}
\usepackage{pifont}
\usepackage{dingbat}
\usepackage{amsmath,amssymb}
\usepackage{epsfig}
\usepackage[]{hyperref}
\usepackage{geometry}
\geometry{letterpaper, margin=0.5in}
\addpoints
\boxedpoints
\pointsinmargin
\pointname{pts}

\usepackage{enumerate}
\newenvironment{alphalist}{
  %\vspace{-0.4in}
  \begin{enumerate}[(a)]
    \addtolength{\itemsep}{0.0\itemsep}}
  {\end{enumerate}}
\frenchspacing
\usepackage[activate={true,nocompatibility},final,tracking=true,kerning=true,factor=1100,stretch=10,shrink=10]{microtype}
\usepackage[american]{babel}
%\usepackage[T1]{fontenc}
\usepackage{fourier}
\usepackage{isomath}
\usepackage{upgreek,amsmath}
\usepackage{amssymb}

\newcommand{\dotprod}{\, {\scriptzcriptztyle
    \stackrel{\bullet}{{}}}\,}

\newcommand{\reals}{\mathbf{R}}
\newcommand{\lub}{\mathrm{lub}} 
\newcommand{\glb}{\mathrm{glb}} 
\newcommand{\complex}{\mathbf{C}}
\newcommand{\dom}{\mbox{dom}}
\newcommand{\cover}{{\mathcal C}}
\newcommand{\integers}{\mathbf{Z}}
\newcommand{\vi}{\, \mathbf{i}}
\newcommand{\vj}{\, \mathbf{j}}
\newcommand{\vk}{\, \mathbf{k}}
\newcommand{\bi}{\, \mathbf{i}}
\newcommand{\bj}{\, \mathbf{j}}
\newcommand{\bk}{\, \mathbf{k}}
\DeclareMathOperator{\Arg}{\mathrm{Arg}}
\DeclareMathOperator{\Ln}{\mathrm{Ln}}
\newcommand{\imag}{\, \mathrm{i}}
\newcommand{\true}{\mbox{true}}
\usepackage{graphicx}
\newcommand\AM{{\sc am}}
\newcommand\PM{{\sc pm}}
     
\usepackage{color}
\shadedsolutions
\definecolor{SolutionColor}{rgb}{0.8,0.9,1}

\newcommand{\quiz}{Review for Exam I}
\newcommand{\term}{Fall}
\newcommand{\due}{Saturday 10 September  at 11:59 \PM}
\begin{document}
\large
\noindent{\textbf \quiz}
%\vspace{0.1in}
%\noindent\makebox[3.0truein][l]{{\bf MATH 460}}
%{\bf Name:}  \\
%\noindent \makebox[3.0truein][l]{\bf Homework   \quiz, \term \/ \the\year}
%{\bf Row:}\hrulefill\
%\vspace{0.1in}


\begin{questions} 




\question Show that 
\[
    \left(\forall k \in \integers_{>1} \right) 
      \left(\frac{1}{k^2} \leq \frac{1}{k-1} - \frac{1}{k} \right).
\]

\begin{solution}
  We'll write our solution as a sequence of logical equivalences. Let $k \in 
  \integers_{>1}$. We have
  \begin{align*}
    \left[\frac{1}{k^2} \leq \frac{1}{k-1} - \frac{1}{k} \right] &\equiv
    \left[\frac{1}{k^2} - \frac{1}{k-1} + \frac{1}{k}  \leq 0 \right], &\mbox{(algebra)} \\
    &\equiv \left[ -\frac{1}{(k-1) k^2}  \leq 0 \right], &\mbox{(factor)} \\
    &\equiv \true. &\mbox{($k - 1 > 0$ and $k^2 > 0$)}
  \end{align*}
  
\end{solution}
\question Show that
\[
    \left(\forall x \in (-\infty, 1) \right)\left( \exists r \in \reals_{>0} \right)
    \left((x-r,x+r) \subset (-\infty, 1) \right).
\]

\begin{solution}
  We need to choose a number $r$ such that $x+r < 1$ and $0 < r$. 
  Thus $0 < r < 1-x$. One choice is $r = \frac{1-x}{2}$. Since $x < 1$,
  this choice does satisfy the condition $r > 0$.

  \textbf{Proof} Let $x \in (-\infty, 1) $. Choose $r = \frac{1-x}{2}$. Since  $x < 1$,
  it follows that $r \in  \reals_{>0}$ as required. Since $ r > 0$, the condition
  \( (x-r,x+r) \subset (-\infty, 1)\) is equivalent to $x + r < 1$.
  We have
  \[
    \left[x + r < 1 \right] \equiv  \left[x +  \frac{1-x}{2} < 1 \right]
    \equiv \left[\frac{1+x}{2} < 1 \right] \equiv \left[\frac{x-1}{2} < 0 \right]
    \equiv \left[ x < \frac{1}{2} \right] \equiv \true.
  \]
\end{solution}

\question Let $A,B$ be subsets of $\reals$ and let $A$ be bounded above.
Show that $A \setminus B$ is bounded above.

\begin{solution}
  Since $A$ is bounded above, there is $M \in \reals$ such that
  $(\forall x \in A)(x \leq M)$.  We will show that
  \[
    (\exists M^\prime \in \reals)(\forall x \in A \setminus B)(x \leq M^\prime).
  \]
  Choose $M^\prime= M$. Let $x \in A \setminus B$. Then $x \in A$; thus we have
  \[
     [x \leq M^\prime]  \equiv [x \leq M] \equiv \true.
  \]


\end{solution}

\question Give an example of subsets $A,B$ be subsets of $\reals$
such that $A \setminus B$ is bounded above, but $A$ is not bounded
above.

\begin{solution}
  One (of many) example is $A = \reals$ and $B = \reals$. Then
  $A$ is not bounded above, but $A \setminus B = \varnothing$,
  so  $A \setminus B$ is bounded above (because the empty set is bounded above).
\end{solution}

\question Define $F = x \in \reals \mapsto x^2$.  Enumerate the members of the set
\begin{equation*}
  F(\{-4,-1,0,1,4 \}).
\end{equation*}
\begin{solution}
  \[
    F(\{-4,-1,0,1,4 \}) = \{F(-4),F(-1),F(0),F(1),F(4) \} =
     \{0,1,16\}.
  \]
\end{solution}
\question Define $F = x \in \reals \mapsto x^2$.  Enumerate the members of the set
\[
      F^{-1}(\{-4,-1,0,1,4 \}).
\]
\begin{solution}
We need to gather the solution sets of each of the
equations $F(x) = -4$, $F(x)=1$, $F(0)=0$, $F(x)=1$, 
and $F(x)=4$.
Thus
\[
  F^{-1}(\{-4,-1,0,1,4 \}) = \{-2,-1,0,1,2\}.
\]
\end{solution}
\question Using the definition from the QRS, show that the sequence 
$k \in \integers_{\geq 0} \mapsto \frac{k-6}{k+28}$ converges.

\begin{solution}
We'll show that
\[
 (\exists L \in \reals)
 (\forall \varepsilon \in \reals_{>0})
 (\exists n \in \integers)
(\forall k \in \integers_{>n}) 
\left ( \left |\frac{k-6}{k+28} - L \right | < \varepsilon \right ).
\]
Choose $L=1$. Let \(\varepsilon \in \reals_{>0} \).
Choose $n = \lceil \frac{34}{\varepsilon} \rceil$.
Let $k \in \integers_{>n}$. We have
\begin{align*}
  \left|\frac{k-6}{k+28} - L \right| &= \left |-\frac{34}{k+28} \right |,  &\mbox{(substitution \& algebra)}\\
                         &= \frac{34}{k+28}, &\mbox{(absolute value properties)} \\
                         &< \frac{34}{n}, &\mbox{($k + 28 > n$)} \\
                         &= \frac{34}{\lceil \frac{34}{\varepsilon} \rceil}, &\mbox{(substitution)} \\
                         &\leq \frac{34}{\frac{34}{\varepsilon}}, &\mbox{(ceiling function property)}\\
                         &= \varepsilon. &\mbox{(algebra)}
\end{align*}
\end{solution}
\question Show that the sequence $n \in \integers_{>0} \mapsto \sum_{k=1}^n \frac{1}{k^2}$ is
bounded above. To do this, use the fact that for all positive integers $k$, we have 
$\frac{1}{k^2} \leq \frac{1}{k-1} - \frac{1}{k}$.
\begin{solution}
  Let $n \in \integers_{>1}$. We have
  \begin{align*}
    \sum_{k=1}^n \frac{1}{k^2} &= 1 + \sum_{k=2}^n \frac{1}{k^2}, &\mbox{(peel off first term of sum)}\\
    &\leq 1 + \sum_{k=2}^n \frac{1}{k-1} - \frac{1}{k}, &\mbox{(given inequality)}\\
    &= 1 + \left(1 - \frac{1}{n} \right), &\mbox{(telescoping sum)} \\
    &= 2 - \frac{1}{n}, &\mbox{(algebra)}\\
    &< 2. &\mbox{(algebra)}
  \end{align*}
  
\end{solution}
\question Either show that the sequence
\[
      k \in \integers \mapsto \sin(\uppi k)
\]
converges or that it diverges. For either case, you're proof will
must use the definition from the QRS.

\begin{solution}
Since \(\sin(\uppi k) = 0\) for all integers $k$, an
alternative formula for the sequence is 
\[
      k \in \integers \mapsto 0.
\]
We'll show that this sequence converges to zero.
Specifically, we'll show that
\[
 (\exists \,\, L \in \reals)
 (\forall \,\, \varepsilon \in \reals_{>0})
 (\exists \,\, n \in \integers)
(\forall \,\, k \in \integers_{>n}) 
\left ( \left |0 - L \right | < \varepsilon \right ).
\]
Choose $L=0$. Let \(\varepsilon \in \reals_{>0} \).
Choose $n = 1$.
Let $k \in \integers_{>n}$. We have
\[
  \left |0 - L \right | = 0 < \varepsilon.
\]


\end{solution}


\question Show that the sequence
\[
    k \in \integers \mapsto \begin{cases} k!  & k < 100\\
        \frac{1}{k}  & k \geq  100 \end{cases}
\]
converges. You must use the definition in the QRS.

\begin{solution}
We'll show that
\[
 (\exists L \in \reals)
 (\forall \varepsilon \in \reals_{>0})
 (\exists n \in \integers)
(\forall k \in \integers_{>n}) 
\left ( \left | L - \begin{cases} k!  & k < 100\\
  \frac{1}{k}  & k \geq  100 \end{cases} \right | < \varepsilon \right ).
\]
Choose $L=0$. Let \(\varepsilon \in \reals_{>0} \).
Choose $n = \max(100, \lceil \frac{1}{\varepsilon} \rceil)$.
Let $k \in \integers_{>n}$. We have
\begin{align*}
  \left | \begin{cases} k!  & k < 100\\
    \frac{1}{k}  & k \geq  100 \end{cases} - L \right |
    &= \frac{1}{k}, &\mbox{($k > 100$)}  \\
    &< \frac{1}{n}, &\mbox{($k > n$)}\\
    &\leq \frac{1}{\lceil \frac{1}{\varepsilon} \rceil},\\
    &\leq \varepsilon.
\end{align*}
\end{solution}

\question Show that
\[
    \left(\forall a \in \reals \right) \left(\exists m \in \reals \right)
    \left(\forall x \in \reals \right) \left(x^2 - a^2 \geq m (x-a) \right).
\]
\begin{solution}
  We will write our proof as a sequence of logical 
  equivalences.   Let $a \in \reals$. Choose $m = 2 a$. Let $x \in \reals$. We have
  \begin{align*}
    \left [x^2 - a^2 \geq m (x-a) \right ] &=
    \left [x^2 - a^2 \geq 2a  (x-a) \right ], &\mbox{(substitution for $m$)} \\
    &=
    \left [x^2 -2 a (x-a) - a^2 \geq 0 \right ],  &\mbox{(algebra)}\\
    &=
    \left [x^2 -2 a  + a^2 \geq 0 \right ], &\mbox{(algebra)} \\
    &=
    \left [(x - a)^2 \geq 0 \right ], &\mbox{(factor)} \\
    &=
    \true.
  \end{align*}
  
 
\end{solution}

\end{questions}
\end{document}