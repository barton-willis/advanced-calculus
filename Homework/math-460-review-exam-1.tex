\documentclass[12pt,fleqn,answers]{exam}
\usepackage{pifont}
\usepackage{dingbat}
\usepackage{amsmath,amssymb}
\usepackage{epsfig}
\usepackage[]{hyperref}
\usepackage{geometry}
\geometry{letterpaper, margin=0.5in}
\addpoints
\boxedpoints
\pointsinmargin
\pointname{pts}

\usepackage{enumerate}
\newenvironment{alphalist}{
  %\vspace{-0.4in}
  \begin{enumerate}[(a)]
    \addtolength{\itemsep}{0.0\itemsep}}
  {\end{enumerate}}

\usepackage[activate={true,nocompatibility},final,tracking=true,kerning=true,factor=1100,stretch=10,shrink=10]{microtype}
\usepackage[american]{babel}
%\usepackage[T1]{fontenc}
\usepackage{fourier}
\usepackage{isomath}
\usepackage{upgreek,amsmath}
\usepackage{amssymb}

\newcommand{\dotprod}{\, {\scriptzcriptztyle
    \stackrel{\bullet}{{}}}\,}

\newcommand{\reals}{\mathbf{R}}
\newcommand{\lub}{\mathrm{lub}} 
\newcommand{\glb}{\mathrm{glb}} 
\newcommand{\complex}{\mathbf{C}}
\newcommand{\dom}{\mbox{dom}}
\newcommand{\cover}{{\mathcal C}}
\newcommand{\integers}{\mathbf{Z}}
\newcommand{\vi}{\, \mathbf{i}}
\newcommand{\vj}{\, \mathbf{j}}
\newcommand{\vk}{\, \mathbf{k}}
\newcommand{\bi}{\, \mathbf{i}}
\newcommand{\bj}{\, \mathbf{j}}
\newcommand{\bk}{\, \mathbf{k}}
\DeclareMathOperator{\Arg}{\mathrm{Arg}}
\DeclareMathOperator{\Ln}{\mathrm{Ln}}
\newcommand{\imag}{\, \mathrm{i}}

\usepackage{graphicx}
\newcommand\AM{{\sc am}}
\newcommand\PM{{\sc pm}}
     
\usepackage{color}
\shadedsolutions
\definecolor{SolutionColor}{rgb}{0.8,0.9,1}

\newcommand{\quiz}{Review for Exam I}
\newcommand{\term}{Fall}
\newcommand{\due}{Saturday 10 September  at 11:59 \PM}
\begin{document}
\large
\quiz
%\vspace{0.1in}
%\noindent\makebox[3.0truein][l]{{\bf MATH 460}}
%{\bf Name:}  \\
%\noindent \makebox[3.0truein][l]{\bf Homework   \quiz, \term \/ \the\year}
%{\bf Row:}\hrulefill\
%\vspace{0.1in}


\begin{questions} 




\question Show that 
\[
    \left(\forall k \in \integers_{>0} \right) 
      \left(\frac{1}{k} \leq \frac{1}{k} - \frac{1}{k+1} \right).
\]

\question Show that
\[
    \left(\forall x \in (-\infty, 1) \right)\left( \exists r \in \reals \right)
    \left((x-r,x+r) \subset (-\infty, 1) \right).
\]

\question Let $A,B$ be subsets of $\reals$ and let $A$ be bounded above.
Show that $A \setminus B$ is bounded above.

\question Give an example of subsets $A,B$ be subsets of $\reals$
such that $A \setminus B$ is bounded above, but $A$ is not bounded
above.

\question Define $F = x \in \reals \mapsto x^2$.  Enumerate the members of the set
\[
      F(\{-4,-1,0,1,4 \}).
    \]

\question Define $F = x \in \reals \mapsto x^2$.  Enumerate the members of the set
\[
      F^{-1}(\{-4,-1,0,1,4 \}).
\]

\question Using the definition from the QRS, show that the sequence 
$k \in \integers_{\geq 0} \mapsto \frac{k-6}{k+28}$ converges.

\question Show that the sequence $n \in \integers_{>0} \mapsto \sum_{k=1}^n \frac{1}{k^2}$ is
bounded above. To do this, use the fact that for all positive integers $k$, we have 
$\frac{1}{k} \leq \frac{1}{k} - \frac{1}{k+1}$.

\question Either show that the sequence
\[
      k \in \integers \mapsto \sin(\uppi k)
\]
converges or that it diverges. For either case, you're proof will
must use the definition from the QRS.

\question Show that the sequence
\[
    k \in \integers \mapsto \begin{cases} k!  & k < 100\\
        \frac{1}{k}  & k \geq  100 \end{cases}
\]
converges. You must use the definition in the QRS.

\question Show that
\[
    \left(\forall a \in \reals \right) \left(\exists m \in \reals \right)
    \left(\forall x \in \reals \right) \left(x^2 - a^2 \leq m (x-a) \right).
\]
\end{questions}
\end{document}