\documentclass[12pt,fleqn,answers]{exam}
\usepackage{pifont}
\usepackage{dingbat}
\usepackage{amsmath,amssymb}
\usepackage{fleqn}
\usepackage{epsfig}
%\usepackage{mathptm}
%\usepackage{euler}
\usepackage{bbding}

\addpoints
\boxedpoints
\pointsinmargin
\pointname{pts}

\usepackage[activate={true,nocompatibility},final,tracking=true,kerning=true,factor=1100,stretch=10,shrink=10]{microtype}
\usepackage[american]{babel}
%\usepackage[T1]{fontenc}
\usepackage{fourier}
\usepackage{isomath}
\usepackage{upgreek,amsmath}
\usepackage{amssymb}
%\usepackage[euler-digits,euler-hat-accent,T1]{eulervm}

\newcommand{\dotprod}{\, {\scriptzcriptztyle
    \stackrel{\bullet}{{}}}\,}

\newcommand{\reals}{\mathbf{R}}
\newcommand{\complex}{\mathbf{C}}
\newcommand{\dom}{\mbox{dom}}
\newcommand{\cover}{{\mathcal C}}
\newcommand{\integers}{\mathbf{Z}}
\newcommand{\vi}{\, \mathbf{i}}
\newcommand{\vj}{\, \mathbf{j}}
\newcommand{\vk}{\, \mathbf{k}}
\newcommand{\bi}{\, \mathbf{i}}
\newcommand{\bj}{\, \mathbf{j}}
\newcommand{\bk}{\, \mathbf{k}}
\DeclareMathOperator{\Arg}{\mathrm{Arg}}
\DeclareMathOperator{\Ln}{\mathrm{Ln}}
\newcommand{\imag}{\, \mathrm{i}}
\usepackage{amsthm}
\newtheorem{Rubbish}{Theorem}
\usepackage{graphicx}

%\usepackage{tgschola} %to look retro
\newenvironment{mypar}[2]
  {\begin{list}{}%
    {\setlength\leftmargin{#1}
    \setlength\rightmargin{#2}}
    \item[]}
  {\end{list}}
  
\newcommand{\quiz}{1}
\newcommand{\term}{Fall}

\usepackage{xspace}
\makeatletter
\DeclareRobustCommand{\maybefakesc}[1]{%
  \ifnum\pdfstrcmp{\f@series}{\bfdefault}=\z@
    {\fontsize{\dimexpr0.8\dimexpr\f@size pt\relax}{0}\selectfont\uppercase{#1}}%
  \else
    \textsc{#1}%
  \fi
}
\newcommand\AM{\,\maybefakesc{am}\xspace}
\newcommand\PM{\,\maybefakesc{pm}\xspace}

\begin{document}
\large
\vspace{0.1in}
\noindent\makebox[3.0truein][l]{ \textbf{MATH 460}}
{\bf Name:}  \\

\noindent This past week, I reread a book of my youth (the actual copy): \emph{Introduction
to Mathematical Logic}, by Flora Dinkines. Here are 
some fun questions that are adapted from this textbook. This is 
\emph{not} an assignment, bonus or otherwise. But if you would like 
a mid-July logic exercise that is more educational than 
solving another  Sudoku puzzle, give these problems a try. And if you have 
questions about them, please let me know.

\begin{questions}
\question Define the predicate 
\begin{equation*}
   Q = (x,y) \in \reals^2 \mapsto [x + 2 y = 4].
\end{equation*}
For example, we have $Q(1,1) = [3 = 4] = \mbox{False}$
and $Q(2,1) = [ 4= 4] = \mbox{True}$. Decide on the truth 
value of each the following statements; if the statement 
is true, prove it; if the statement is false, prove that 
its negation is true.

\begin{parts}

    \part $\left(\forall x \in \reals \right)
           \left(\exists y \in \reals \right)
           \left(Q(x,y)\right)$.

    \part $\left(\exists x \in \reals \right)
           \left(\forall y \in \reals \right)
           \left(Q(x,y)\right)$.

    \part $\left(\forall x \in \reals \right)
           \left(\exists y \in \reals \right)
           \left(\lnot Q(x,y)\right)$.  

    \part $\left(\forall x  \in \reals \right)
           \lnot \left(\forall y \in \reals \right)
           \left(Q(x,y)\right)$.  

\end{parts}

\question Find examples of predicates $P$ and $Q$ such
that the statement
\begin{equation*}
  \left(\forall x \right) \left(P(x) \lor Q(x) \right) \equiv
    \left(\forall x \right) (P(x)) \lor 
    \left(\forall x \right) (Q(x))
\end{equation*}
is false. This shows that the existential qualifier does not 
distribute over the disjunction.

\question Does the existential qualifier distribute over the 
conjunction?

\question For any predicate $P$, show that
\begin{equation*}
  \left(\exists x\right)
  \left(\exists y \right)
  \left(P(x,y)\right) \equiv
  \left(\exists y \right)
  \left(\exists x \right)
  \left(P(x,y)\right).
\end{equation*}

\question For any predicate $P$, show that
\begin{equation*}
  \left(\forall x\right)
  \left(\forall y \right)
  \left(P(x,y)\right) \equiv
  \left(\forall y \right)
  \left(\forall x \right)
  \left(P(x,y)\right).
\end{equation*}


\question For any predicate $P$, show that
\begin{equation*}
    \left(\exists x\right)
    \left(\forall y \right)
    \left(P(x,y)\right) \implies 
    \left(\forall y \right)
    \left(\exists x \right)
    \left(P(x,y)\right).
  \end{equation*}

  \question Show there is a predicate $P$ such that 
  \begin{equation*}
    \left(\forall y \right)
    \left(\exists x \right)
    \left(P(x,y)\right) \implies 
      \left(\exists x\right)
      \left(\forall y \right)
      \left(P(x,y)\right).
\end{equation*}
is false.
  


\end{questions}
\end{document}