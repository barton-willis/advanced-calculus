\documentclass[12pt,fleqn,answers]{exam}
\usepackage{pifont}
\usepackage{dingbat}
\usepackage{amsmath}
\usepackage{epsfig}
%\usepackage{amsym}
\usepackage[]{hyperref}
\usepackage{geometry}
\geometry{letterpaper, margin=0.5in}
\addpoints
\boxedpoints
\pointsinmargin
\pointname{pts}

\usepackage[activate={true,nocompatibility},final,tracking=true,kerning=true,factor=1100,stretch=10,shrink=10]{microtype}
\usepackage[american]{babel}
%\usepackage[T1]{fontenc}
\usepackage{fourier}
\usepackage{isomath}
\usepackage{upgreek,amsmath}
\usepackage{amssymb}

\newcommand{\dotprod}{\, {\scriptzcriptztyle
    \stackrel{\bullet}{{}}}\,}

\newcommand{\reals}{\mathbf{R}}
\newcommand{\lub}{\mathrm{lub}} 
\newcommand{\glb}{\mathrm{glb}} 
\newcommand{\complex}{\mathbf{C}}
\newcommand{\dom}{\mbox{dom}}
\newcommand{\cover}{{\mathcal C}}
\newcommand{\integers}{\mathbf{Z}}
\newcommand{\vi}{\, \mathbf{i}}
\newcommand{\vj}{\, \mathbf{j}}
\newcommand{\vk}{\, \mathbf{k}}
\newcommand{\bi}{\, \mathbf{i}}
\newcommand{\bj}{\, \mathbf{j}}
\newcommand{\bk}{\, \mathbf{k}}
\DeclareMathOperator{\Arg}{\mathrm{Arg}}
\DeclareMathOperator{\Ln}{\mathrm{Ln}}
\newcommand{\imag}{\, \mathrm{i}}

\newcommand{\union}{\cup}
\newcommand{\intersection}{\cap}
\newcommand{\True}{\mbox{True}}
\usepackage{graphicx}
\usepackage{color}
\shadedsolutions
\definecolor{SolutionColor}{rgb}{0.8,0.9,1}
\newcommand\AM{{\sc am}}
\newcommand\PM{{\sc pm}}

\begin{document}
\large

\begin{questions} 

\question Define $F = x \in \reals \mapsto x^2$. Enumerate
the members of $F(\{-2,-1,0,1,2 \})$.
\begin{solution}
\[
    F(\{-2,-1,0,1,2 \} = \{F(-2),F(-1),F(0),F(1), F(2) \}
                       = \{4,1,0,1,4 \} =  \{0,1,4 \}.
\]
\end{solution}

\question  Define $F = x \in \reals \mapsto x^2$. Enumerate
the members of $F^{(-1)} (\{0,1,4 \})$.

\begin{solution}
The solution set to $F(x) = 4$ is $\{-2,2\}$; the solution set to $F(x) = 1$ is $\{-1,1\}$;
and the solution set to $F(x) = 0$ is $\{0\}$. So
\[
    F^{(-1)} (\{0,1,4 \}) = \{-2,-1,0,1,2 \}.
\]
\end{solution}

\question Show that
\begin{equation*}
 \left(\forall a \in \reals_{>0}\right) \left  (\exists \,\, m \in \reals \right )  \left (\forall \,\,
 x \in \reals_{\geq 0} \right ) \left (\sqrt{x} \leq \sqrt{a} + m (x-a) \right).
\end{equation*}
\textbf{Hints:} You might like to use the facts:
\begin{align*}
    \left[ \sqrt{x} \leq \sqrt{a} + m (x-a) \right] &\equiv
    \left[ \sqrt{x} - \sqrt{a} - m (x-a) \leq 0 \right], \\
    &\equiv  \left[ \sqrt{x} - \sqrt{a} - m  (\sqrt{x} - \sqrt{a}) 
          (\sqrt{x} + \sqrt{a}) \leq 0 \right], \\
    &\equiv \left[ (\sqrt{x} - \sqrt{a}) (1 - m  (\sqrt{x} + \sqrt{a}) \leq 0 \right].
\end{align*}

\begin{solution}
Let $a \in \reals_{>0}$. Choose $m = \frac{1}{2 \sqrt{a}}$. We have
\begin{align*}
    \left[ \sqrt{x} \leq \sqrt{a} + m (x-a) \right] &\equiv
    \left[ \sqrt{x} - \sqrt{a} - m (x-a) \leq 0 \right], &\mbox{(algebra)}\\
    &\equiv  \left[ \sqrt{x} - \sqrt{a} - m  (\sqrt{x} - \sqrt{a}) 
          (\sqrt{x} + \sqrt{a}) \leq 0 \right], &\mbox{(algebra)} \\
    &\equiv \left[ (\sqrt{x} - \sqrt{a}) (1 - m  (\sqrt{x} + \sqrt{a}) \leq 0 \right], &\mbox{(factor)} \\
    &\equiv \left[ (\sqrt{x} - \sqrt{a}) \frac{\sqrt{a} - \sqrt{x}}{2 \sqrt{a}}\leq 0 \right], &\mbox{(substitution)} \\
    &\equiv \left[ -\frac{(\sqrt{x} - \sqrt{a})^2} {2 \sqrt{a}}\leq 0 \right], &\mbox{(factor)} \\
    &= \True.
\end{align*}

\end{solution}

\question  Show that for all sets $A$ and $B$ that $\left(B \setminus  A = B \right)  \implies \left( A \intersection B = \varnothing \right)$.
\textbf{Hint:} Try proving the contrapositive.

\begin{solution}
We'll show that $A \intersection B \neq \varnothing \implies B \setminus  A \neq B$.
Since $A \intersection B \neq \varnothing$, there is $x$ such that $x \in A$ add
$x \in B$; thus $x \notin B \setminus  A$ and $x \in B$. Since there is a member 
of $B$ that isn't a member of $B \setminus  A$, we've shown that $B \setminus  A \neq B$.


\end{solution}






\end{questions}



\end{document}