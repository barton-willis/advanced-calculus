\documentclass[12pt,fleqn,answers]{exam}
\usepackage{pifont}
\usepackage{dingbat}
\usepackage{amsmath}
\usepackage{epsfig}
%\usepackage{amsym}
\usepackage[]{hyperref}
\usepackage{geometry}
\geometry{letterpaper, margin=0.5in}
\addpoints
\boxedpoints
\pointsinmargin
\pointname{pts}

\usepackage[activate={true,nocompatibility},final,tracking=true,kerning=true,factor=1100,stretch=10,shrink=10]{microtype}
\usepackage[american]{babel}
%\usepackage[T1]{fontenc}
\usepackage{fourier}
\usepackage{isomath}
\usepackage{upgreek,amsmath}
\usepackage{amssymb}

\newcommand{\dotprod}{\, {\scriptzcriptztyle
    \stackrel{\bullet}{{}}}\,}

\newcommand{\reals}{\mathbf{R}}
\newcommand{\lub}{\mathrm{lub}} 
\newcommand{\glb}{\mathrm{glb}} 
\newcommand{\complex}{\mathbf{C}}
\newcommand{\dom}{\mbox{dom}}
\newcommand{\cover}{{\mathcal C}}
\newcommand{\integers}{\mathbf{Z}}
\newcommand{\vi}{\, \mathbf{i}}
\newcommand{\vj}{\, \mathbf{j}}
\newcommand{\vk}{\, \mathbf{k}}
\newcommand{\bi}{\, \mathbf{i}}
\newcommand{\bj}{\, \mathbf{j}}
\newcommand{\bk}{\, \mathbf{k}}
\DeclareMathOperator{\Arg}{\mathrm{Arg}}
\DeclareMathOperator{\Ln}{\mathrm{Ln}}
\newcommand{\imag}{\, \mathrm{i}}

\newcommand{\union}{\cup}
\newcommand{\intersection}{\cap}
\usepackage{graphicx}
\newcommand\AM{{\sc am}}
\newcommand\PM{{\sc pm}}
     
\newcommand{\quiz}{2}
\newcommand{\term}{Fall}
\newcommand{\due}{Saturday 3 September at 11:59 \PM}
\begin{document}
\large
\vspace{0.1in}
\noindent\makebox[3.0truein][l]{{\bf MATH 460}}
{\bf Name:} Larry Kuúruks  \\
\noindent \makebox[3.0truein][l]{\bf Homework \quiz, \term \/ \the\year}
%{\bf Row:}\hrulefill\
\vspace{0.1in}

\begin{quote}
    \fbox{I have neither given nor received unauthorized assistance on this assignment.}
    \end{quote}
\noindent  Homework    \quiz\/  has questions 1 through  \numquestions \/ with a total of  \numpoints\/  points.   Edit this file and append you answers using La\TeX. Be sure to fill in your name. Upload the converted pdf of your work to Canvas.   This assignment is due \emph{\due}.

\vspace{0.1in}

\noindent{\textbf{Link to your Overleaf work: }}\url{XXX}

\begin{questions} 

\question [5] Define $F = x \in \reals \mapsto x^2$. Enumerate
the members of $F(\{-2,-1,0,1,2 \})$.
\begin{solution}
\end{solution}

\question [5] Define $F = x \in \reals \mapsto x^2$. Enumerate
the members of $F^{(-1)} (\{0,1,4 \})$.

\begin{solution}
\end{solution}

\question[5] Show that
\begin{equation*}
 \left(\forall a \in \reals_{>0}\right) \left  (\exists \,\, m \in \reals \right )  \left (\forall \,\,
 x \in \reals_{\geq 0} \right ) \left (\sqrt{x} \leq \sqrt{a} + m (x-a) \right).
\end{equation*}
\textbf{Hints:} You might like to use the facts:
\begin{align*}
    \left[ \sqrt{x} \leq \sqrt{a} + m (x-a) \right] &\equiv
    \left[ \sqrt{x} - \sqrt{a} - m (x-a) \leq 0 \right], \\
    &\equiv  \left[ \sqrt{x} - \sqrt{a} - m  (\sqrt{x} - \sqrt{a}) 
          (\sqrt{x} + \sqrt{a}) \leq 0 \right], \\
    &\equiv \left[ (\sqrt{x} - \sqrt{a}) (1 - m  (\sqrt{x} + \sqrt{a}) \leq 0 \right]
\end{align*}

\begin{solution}
\end{solution}

\question [5] Show that for all sets $A$ and $B$ that $\left(B \setminus  A = B \right)  \implies \left( A \intersection B = \varnothing \right)$.
\textbf{Hint:} Try proving the contrapositive.

\begin{solution}
\end{solution}






\end{questions}



\end{document}