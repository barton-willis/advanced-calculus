\documentclass[12pt,fleqn,answers]{exam}
\usepackage{pifont}
\usepackage{dingbat}
\usepackage{amsmath,amssymb}
\usepackage{fleqn}
\usepackage{epsfig}
%\usepackage{mathptm}
%\usepackage{euler}
\usepackage{bbding}
\usepackage{url}
\addpoints
\boxedpoints
\pointsinmargin
\pointname{pts}

\usepackage{xcolor}
\usepackage{framed}
\colorlet{shadecolor}{lightgray!15}
\newenvironment{myproof}
  {\begin{shaded}\begin{proof}}
  {\end{proof}\end{shaded}}

\usepackage{amsthm}
\newtheorem{prop}{Proposition}

\usepackage[activate={true,nocompatibility},final,tracking=true,kerning=true,factor=1100,stretch=10,shrink=10]{microtype}
\usepackage[american]{babel}
%\usepackage[T1]{fontenc}
\usepackage{fourier}
\usepackage{isomath}
\usepackage{upgreek,amsmath}
\usepackage{amssymb}
%\usepackage[euler-digits,euler-hat-accent,T1]{eulervm}

\newcommand{\dotprod}{\, {\scriptzcriptztyle
    \stackrel{\bullet}{{}}}\,}

\newcommand{\reals}{\mathbf{R}}
\newcommand{\complex}{\mathbf{C}}
\newcommand{\dom}{\mbox{dom}}
\newcommand{\cover}{{\mathcal C}}
\newcommand{\rat}{\mathbf{Q}}

\newcommand{\curly}[1]{\mathcal #1}
\newcommand{\integers}{\mathbf{Z}}
\newcommand{\vi}{\, \mathbf{i}}
\newcommand{\vj}{\, \mathbf{j}}
\newcommand{\vk}{\, \mathbf{k}}
\newcommand{\bi}{\, \mathbf{i}}
\newcommand{\bj}{\, \mathbf{j}}
\newcommand{\bk}{\, \mathbf{k}}
\DeclareMathOperator{\Arg}{\mathrm{Arg}}
\DeclareMathOperator{\Ln}{\mathrm{Ln}}
\newcommand{\imag}{\, \mathrm{i}}
\newcommand{\range}{\mathrm{range}}
\newcommand{\true}{\mathrm{True}}
\newcommand{\saw}{\mathrm{saw}}
\usepackage{enumerate}
\newenvironment{alphalist}{
  \begin{enumerate}[(a)]
    \addtolength{\itemsep}{-0.5\itemsep}}
  {\end{enumerate}}
 
\usepackage{amsthm}
\newtheorem{Rubbish}{Theorem}
\usepackage{graphicx}

%\usepackage{tgschola} %to look retro
\newenvironment{mypar}[2]
  {\begin{list}{}%
    {\setlength\leftmargin{#1}
    \setlength\rightmargin{#2}}
    \item[]}
  {\end{list}}
  
\newcommand{\quiz}{12}
\newcommand{\term}{Fall}
\usepackage{units}
\usepackage{xspace}
\makeatletter
\DeclareRobustCommand{\maybefakesc}[1]{%
  \ifnum\pdfstrcmp{\f@series}{\bfdefault}=\z@
    {\fontsize{\dimexpr0.8\dimexpr\f@size pt\relax}{0}\selectfont\uppercase{#1}}%
  \else
    \textsc{#1}%
  \fi
}
\newcommand\AM{\,\maybefakesc{am}\xspace}
\newcommand\PM{\,\maybefakesc{pm}\xspace}

\begin{document}
\large
\vspace{0.1in}
\noindent\makebox[3.0truein][l]{ \textbf{MATH 460}}
{\bf Name:}  \\
\noindent \makebox[3.0truein][l]{Riemann integral practice for final}
%{\bf Row:}\hrulefill\
\vspace{0.1in}

\noindent \emph{“I'm killing time while I wait for life to shower me 
with meaning and happiness.”} \\ $\phantom{xxx}$ \hfill  {\sc Calvin (Bill Watterson)}




\vspace{0.1in}


\begin{questions}

 \question Define $f = x \in [0,2] \mapsto \begin{cases} 
    1 & x \neq 1 \\ 99 & x = 1 \end{cases}$. Given a positive
    number $\varepsilon$, explicitly find a partition $P$ of $[0,2]$ such
    that $\overline{S}(f,P)-\underline{S}(f,P)< \varepsilon$

    \question Define $f = x \in [0,2] \mapsto \begin{cases} 
        x & x \in Q  \\ 0 & x \not \in Q \end{cases}$. Show 
        that $f$ is not Riemann integrable on $[0,2]$.

    \question Define $f = x \in[0,3] \mapsto \begin{cases} 3 &  
        0 \leq x < 1 \\ 4 & 1 \leq x < 2 \\ x+2 & 2 \leq x \leq 3   
    \end{cases}$.  Define $ g = x \in [0,3] \mapsto \int_0^x f(t) \, 
    \mathrm{d} t$. 

    \begin{parts}
       \part Find a formula for $g$.
       \part Sketch a graph of $g$.
       \part Show that $g$ is not differentiable at 1.
       \part Show that $g$ is  differentiable at 2.
    \end{parts}

   \question Define $Q = [0,3] \mapsto \int_0^x 
   \lfloor t \rfloor \, \mathrm{d} t$. Sketch 
   a pretty good graph of $Q$.


\end{questions}
\end{document}