\documentclass[12pt,fleqn,answers]{exam}
\usepackage{pifont}
\usepackage{dingbat}
\usepackage{amsmath,amssymb}
\usepackage{fleqn}
\usepackage{epsfig}
%\usepackage{mathptm}
%\usepackage{euler}
\usepackage{bbding}
\usepackage{url}
\addpoints
\boxedpoints
\pointsinmargin
\pointname{pts}

\usepackage[activate={true,nocompatibility},final,tracking=true,kerning=true,factor=1100,stretch=10,shrink=10]{microtype}
\usepackage[american]{babel}
%\usepackage[T1]{fontenc}
\usepackage{fourier}
\usepackage{isomath}
\usepackage{upgreek,amsmath}
\usepackage{amssymb}
%\usepackage[euler-digits,euler-hat-accent,T1]{eulervm}

\newcommand{\dotprod}{\, {\scriptzcriptztyle
    \stackrel{\bullet}{{}}}\,}

\newcommand{\reals}{\mathbf{R}}
\newcommand{\complex}{\mathbf{C}}
\newcommand{\dom}{\mbox{dom}}
\newcommand{\cover}{{\mathcal C}}
\newcommand{\rat}{\mathbf{Q}}

\newcommand{\curly}[1]{\mathcal #1}
\newcommand{\integers}{\mathbf{Z}}
\newcommand{\vi}{\, \mathbf{i}}
\newcommand{\vj}{\, \mathbf{j}}
\newcommand{\vk}{\, \mathbf{k}}
\newcommand{\bi}{\, \mathbf{i}}
\newcommand{\bj}{\, \mathbf{j}}
\newcommand{\bk}{\, \mathbf{k}}
\DeclareMathOperator{\Arg}{\mathrm{Arg}}
\DeclareMathOperator{\Ln}{\mathrm{Ln}}
\newcommand{\imag}{\, \mathrm{i}}
\usepackage{amsthm}
\newtheorem{Rubbish}{Theorem}
\usepackage{graphicx}

%\usepackage{tgschola} %to look retro
\newenvironment{mypar}[2]
  {\begin{list}{}%
    {\setlength\leftmargin{#1}
    \setlength\rightmargin{#2}}
    \item[]}
  {\end{list}}
  
\newcommand{\quiz}{2}
\newcommand{\term}{Fall}

\usepackage{xspace}
\makeatletter
\DeclareRobustCommand{\maybefakesc}[1]{%
  \ifnum\pdfstrcmp{\f@series}{\bfdefault}=\z@
    {\fontsize{\dimexpr0.8\dimexpr\f@size pt\relax}{0}\selectfont\uppercase{#1}}%
  \else
    \textsc{#1}%
  \fi
}
\newcommand\AM{\,\maybefakesc{am}\xspace}
\newcommand\PM{\,\maybefakesc{pm}\xspace}

\begin{document}
\large
\vspace{0.1in}
\noindent\makebox[3.0truein][l]{ \textbf{MATH 460}}
{\bf Name:}  \\
\noindent \makebox[3.0truein][l]{\textbf{Homework \quiz, \term \/ \the\year}}
%{\bf Row:}\hrulefill\
\vspace{0.1in}

\noindent  Homework \quiz\/  has questions 1 through  \numquestions \/ with a total 
of  \numpoints\/  points. When I record your grade, I will scale it to twenty points. 
For details of the grading scheme for this assignment, please see the section 
`Grading rubric' of our syllabus.

Revise, proofread, revise again (and again), typeset your work using Overleaf, and 
upload the converted pdf of your compiled file work to Canvas.  
This work is due \textbf{Saturday 9 September} at 11:59 \PM.

\vspace{0.1in}

\begin{questions}

\question [10] Let $(\mathcal{F},+, \times)$ be a field with $\theta$ as its additive identity. And let $a \in \mathcal{F}_{\neq \theta}$ and $b \in \mathcal{F}$. Show that there is a unique $x \in \mathcal{F}$ such that $a \times x = b$.


\question Define $\rat$ to be the set of rational numbers. Thus $\frac{2}{3}$ and $\sqrt{2} \notin \rat$, for example.

For $(a,b), (a^\prime, b^\prime) \in \rat \times \rat$, define the binary operators $+$ and $\times$
on $\rat \times \rat$ by
\begin{align*}
  (a,b) +  (a^\prime, b^\prime) &= (a + a^\prime,b + b^\prime), \\
  (a,b) \times  (a^\prime, b^\prime) &= (a a^\prime + 2 b b^\prime, a b^\prime + b a^\prime).
\end{align*}
These definitions make $(\rat \times \rat, +, \times)$ into a field. If you don't believe me, I invite you to check. But
you can take my word.

\begin{parts}

\part [10] Find the additive identity for this field. You don't need to show the algebra you used to find the
additive identity, but you do need to prove that your putative answer is correct.

\part [10] Find the multiplicative identity for this field. You don't need to show  the algebra you used to find the
multiplication identity, but you do need to prove that
your putative answer is correct.

\part [10] Find the multiplicative inverse of $(1,2)$. You don't need to show the algebra you used to find the answer, but you do need to prove that your putative answer is correct.

\part [10] For $a,b \in     \rat$, a calculation gives
\begin{equation}
   (a,b)^{-1} = \left(- \frac{a}{2 b^2 - a^2}, \frac{b}{2 b^2 - a^2}\right)
\end{equation}
But wait! What's the story if $2 b^2 - a^2 = 0$?  Either, I flubbed and $(\rat \times \rat, +, \times)$
isn't a field, or the above formula for the inverse is rubbish, or something else. What is the something else?



\end{parts}

\end{questions}
\end{document}