\documentclass[12pt,fleqn]{article}
\usepackage{amssymb,upgreek,isomath,xcolor,amsthm,nicefrac}
\usepackage[letterpaper, margin=0.5in]{geometry}
\usepackage[final]{microtype}
\usepackage[american]{babel}
\usepackage{fourier}
\usepackage{fleqn}
\usepackage[intlimits]{amsmath}
\newcommand{\reals}{\mathbf{R}}
\newcommand{\complex}{\mathbf{C}}
\newcommand{\dom}{\mbox{dom}}
\newcommand{\cover}{{\mathcal C}}
\newcommand{\integers}{\mathbf{Z}}
\newcommand{\true}{\mathrm{True}}
\newcommand{\false}{\mathrm{False}}

\usepackage{centernot}
\newcommand{\notimplies}{\centernot\implies}
% Let's put a proof inside a shaded box
\usepackage{framed}

\colorlet{shadecolor}{lightgray!15}
\newenvironment{myproof}
  {\begin{shaded}\begin{proof}}
  {\end{proof}\end{shaded}}

\newtheorem{prop}{Proposition}

  \usepackage{enumerate}
  \newenvironment{alphalist}{
  \begin{enumerate}[(a)]
    \addtolength{\itemsep}{-0.50\itemsep}}
  {\end{enumerate}}

\usepackage{calc}
\newcounter{ex}\setcounter{ex}{0}
\newcommand{\ex}{%\
\setcounter{ex}{\value{ex}+1}
\paragraph{Example \theex}}

\newmuskip\pFqmuskip

\newcommand*\pFq[6][8]{%
  \begingroup % only local assignments
  \pFqmuskip=#1mu\relax
  \mathchardef\normalcomma=\mathcode`,
  % make the comma math active
  \mathcode`\,=\string"8000
  % and define it to be \pFqcomma
  \begingroup\lccode`\~=`\,
  \lowercase{\endgroup\let~}\pFqcomma
  % typeset the formula
  {}_{#2}\mathrm{F}_{#3}{\left[\genfrac..{0pt}{}{#4}{#5};#6\right]}%
  \endgroup
}
\newcommand{\pFqcomma}{{\normalcomma}\mskip\pFqmuskip}

\title{What your calculus textbook doesn't tell you about trigonometric substitution}

\author{Barton Willis}
\begin{document}
\maketitle

\begin{shaded}
  \noindent \textbf{Abstract} The method of trigonometric substitution for indefinite
  integration can give impressive looking results, but these results 
  are sometimes poorly suited for numerical evaluation.
\end{shaded}

After making multiple bad starts, wasting ten sheets of engineering paper, 
 and erasing more  than you would like to admit, finally you are confident 
 that you have correctly used integration by trigonometric substitution and have
 deduced an impressive looking answer to your first homework question. Your answer
 is
 \begin{equation*}
    \int {\left. \frac{{{\left( {{x}^{2}}-1\right) }^{\nicefrac{7}{2}}}}{{{x}^{3}}} \, \mathrm{d}x\right.}=\frac{{{\left( {{x}^{2}}-1\right) }^{\nicefrac{9}{2}}}}{2 {{x}^{2}}}-\frac{{{\left( {{x}^{2}}-1\right) }^{\nicefrac{7}{2}}}}{2}+\frac{7 {{\left( {{x}^{2}}-1\right) }^{\nicefrac{5}{2}}}}{10}-\frac{7 {{\left( {{x}^{2}}-1\right) }^{\nicefrac{3}{2}}}}{6}+\frac{7 (x^2-1)^{\nicefrac{1}{2}}}{2}+\frac{7 \operatorname{arcsin}\left( \frac{1}{x}\right) }{2}.
\end{equation*}
But wait! There's more. To add to the punishment, your teacher asks for the value of  
the definite integral $\int_1^{10^9} {\left. \frac{{{\left( {{x}^{2}}-1\right) }^{\nicefrac{7}{2}}}}{{{x}^{3}}} \, \mathrm{d}x\right.}$. Easy-peasy, you think.
Surely after the tortuous trigonometric substitution problem, this should be easy. 
All you need to do is evaluate
\begin{equation*}
  \frac{{{\left( {{10}^{18}}-1\right) }^{\nicefrac{9}{2}}}}{2 {{10}^{18}}}-\frac{{{\left( {{10}^{18}}-1\right) }^{\nicefrac{7}{2}}}}{2}+\frac{7 {{\left( {{10}^{18}}-1\right) }^{\nicefrac{5}{2}}}}{10}-\frac{7 {{\left( {{10}^{18}}-1\right) }^{\nicefrac{3}{2}}}}{6}+\frac{7 (10^{18}-1)^{\nicefrac{1}{2}}}{2}+\frac{7 \operatorname{arcsin}\left( 10^{-18}\right) }{2}
    - \frac{7 \operatorname{arcsin}\left( 1 \right) }{2}
\end{equation*}
You tap this into your nearest computing device, and out  pops the answer to an 
impressive sixteen decimal places. Your final answer is
\begin{equation*}
    \int_1^{10^9} {\left. \frac{{{\left( {{x}^{2}}-1\right) }^{\frac{7}{2}}}}{{{x}^{3}}} \, \mathrm{d}x\right.}
       \approx 9.134385233318144 \times {{10}^{46}}.
\end{equation*}
So you circle this answer and turn it in. Good job, you think---let's very quickly move on to the 
next question.

Not so fast. Your ten sheets of wasted engineering paper rewarded
you with full credit for your antiderivative, but your
decimal approximation earns you a score of zero. Although your teacher says that 
your numerical value isn't \emph{manifestly wrong}, its 
just shy of being \emph{obviously wrong}. Some thought 
shows that $0 < \frac{{{\left( {{x}^{2}}-1\right) }^{\frac{7}{2}}}}{{{x}^{3}}}
< \frac{(x^2)^{\nicefrac{7}{2}}} {x^3} = x^4$. So it must be true that
\begin{equation*}
   0 <  \int_1^{10^9} {\left. \frac{{{\left( {{x}^{2}}-1\right) }^{\nicefrac{7}{2}}}}{{{x}^{3}}} \, \mathrm{d}x\right.}
       < \int_1^{10^9} x^4 \, \mathrm{d}x =  \frac{10^{45} - 1 }{5} < 2  \times 10^{44}
\end{equation*}
Yikes! Using an exact calculation and some arithmetic, your value 
for the definite integral is nearly one thousand times larger than
an easily found upper bound.

What's the story?  Pasting in the upper limit of $10^9$ into our 
painstakingly determined antiderivative, we need to sum
\begin{align*}
&5.000000000000003 \times  {{10}^{62}} - 5.000000000000002  \times 
{{10}^{62}} + 7.000000000000003 \times {{10}^{44}}  \\
&-1.1666666666666669 \times {{10}^{27}} + 
3.5000000000000005 \times {{10}^{9}} + 3.5000000000000003 \times {{10}^{-9}}.
\end{align*}
This is an example of what is known as an \emph{ill conditioned sum}, and 
it is the first two terms of the sum that are the 
troublemakers. Each term in this sum is properly rounded to sixteen 
decimal digits. This means that the relative difference between 
each term in the sum and its true value are no more than about 
$10^{-16}$. Specifically, the true value of the first term 
can be as small as $5.000000000000003 \times  {{10}^{62}} \times
(1 - 10^{-16})$ or as large as $5.000000000000003 \times  {{10}^{62}} 
\times (1 + 10^{-16})$. And similarly for the second term.

Putting this together, the sum of the first two terms might be as small
as 
\begin{equation*}
  5.000000000000003 \times 10^{62} \times (1 -  10^{-16}) - 5
   .000000000000002 \times 10^{62} \times  (1  + 10^{-16})
   \approx -1.1182158029987521 \times 10^{46},
\end{equation*}
and as large as 
\begin{equation*}
  5.000000000000003 \times 10^{62} \times (1 + 10^{-16}) - 5
   .000000000000002 \times 10^{62} \times  (1  - 10^{-16})
   \approx 1.888178419700126 \times {{10}^{47}}
\end{equation*}
We don't even know if the sum of the first two terms is negative 
or positive.  Using sixteen decimal digits, we cannot accurately
evaluate this expression.

Is there a cure?  Well, yes there is. It is tedious to show, but 
an algebraically equivalent form for the antiderivative is
\begin{equation*}
  \int {\left. \frac{{{\left( {{x}^{2}}-1\right) }^{\nicefrac{7}{2}}}}{{{x}^{3}}} \, \mathrm{d}x\right.}
  = \frac{\sqrt{{{x}^{2}}-1}\, \left( 6 {{x}^{6}}-32 {{x}^{4}}+116 {{x}^{2}}+15\right) +105 \operatorname{asin}\left( \frac{1}{x}\right)  {{x}^{2}}}{30 {{x}^{2}}}.
\end{equation*}
Numerically evaluating this expression at $10^9$ does not involve
subtracting two numbers that are nearly equal.  Unlike, our first
expression, this form of the answer is well-conditioned. Typing this 
into the nearest calculator yields
\begin{equation*}
\int {\left. \frac{{{\left( {{x}^{2}}-1\right) }^{\nicefrac{7}{2}}}}{{{x}^{3}}} \, \mathrm{d}x\right.} 
\approx 1.9999999999999998 \times 10^{44}.
\end{equation*}
This approximation is accurate and it is smaller than the 
upper bound of $2 \times 10^{44}$.


If you are willing to dip into hypergeometric functions, there is 
another cure. In terms of the Gauss hypergeometric function, we have
\begin{equation*}
  \int {\left. \frac{{{\left( {{x}^{2}}-1\right) }^{\nicefrac{7}{2}}}}{{{x}^{3}}} \, \mathrm{d}x\right.}=
 \pFq{2}{1}{2,9/2}{11/2}{1-x^2} \frac{(x^2-1)^{9/2}}{9}.
\end{equation*}
Any worthy computer algebra system will give
\begin{equation*}
  \pFq{2}{1}{2,9/2}{11/2}{1-10^{18}} \frac{(10^{18}-1)^{9/2}}{9}
  = 1.999999999999999988 \dots \times 10^{44}
\end{equation*}

\end{document}