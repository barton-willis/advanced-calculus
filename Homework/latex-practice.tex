\documentclass[12pt,fleqn]{article}
\usepackage{amsmath,amssymb,upgreek,isomath,xcolor,amsthm,nicefrac}
\usepackage[letterpaper, margin=0.7in]{geometry}
\usepackage[final]{microtype}
\usepackage[american]{babel}
\usepackage{fourier}

\newcommand{\reals}{\mathbf{R}}
\newcommand{\complex}{\mathbf{C}}
\newcommand{\dom}{\mbox{dom}}
\newcommand{\cover}{{\mathcal C}}
\newcommand{\integers}{\mathbf{Z}}
\newcommand{\true}{\mathrm{True}}
\newcommand{\false}{\mathrm{False}}

\usepackage{centernot}
\newcommand{\notimplies}{\centernot\implies}
% Let's put a proof inside a shaded box
\usepackage{framed}

\colorlet{shadecolor}{lightgray!15}
\newenvironment{myproof}
  {\begin{shaded}\begin{proof}}
  {\end{proof}\end{shaded}}

\newtheorem{prop}{Proposition}

  \usepackage{enumerate}
  \newenvironment{alphalist}{
  \begin{enumerate}[(a)]
    \addtolength{\itemsep}{-0.50\itemsep}}
  {\end{enumerate}}

\usepackage{calc}
\newcounter{ex}\setcounter{ex}{0}
\newcommand{\ex}{%\
\setcounter{ex}{\value{ex}+1}
\paragraph{Example \theex}}

\title{Let/show/suppose proof examples}
\author{Fall 2023, Advanced Calculus Class}
\begin{document}
\maketitle

\section{Introduction}


Many propositions in analysis consist of concatenated `for all' and `there exist'  qualifiers followed by something that has a Boolean value.  
The part with the Boolean value is called the \emph{predicate}. We'll see that expressing the proposition in symbolic 
form gives us a road map for constructing  a proof. To follow the road map, we to 
adhere to some rules, and it's best to follow some traditions.  Following the rules is 
needed for correctness,  and using the traditions makes 
 our work easier for the reader  to understand our work.

Our first rule tells us that for every qualification of the form $\forall x \in A$, our proof includes the sentence `Let $x \in A$.'  Synonyms  for  `let' 
are `permit' and `allow,' but in mathematics the tradition is to use `let.' In the context of mathematics, `Let $x \in \reals$' means that we can only assume that $x$ is real number, but we cannot assume anything more about $x$. It would be wrong, for example, to write in a proof  
`Let $x \in \reals$. Assume $x = 0$.'

And our second rule is that for every qualification of the form $\exists x \in A$, it's just like fourth grade show and tell. We must choose a \emph{specific} value for the  variable $x$ and \emph{show} the reader our choice. Choices are, of course, sometimes difficult, so we need to 
be careful in making them. When we make a bad choice, we might not be able to complete our proof.  And when we flub, we need to backtrack 
and re-do our work from where we made the bad choice.

For proofs in analysis, often the choice isn't unique, but when there is more than one possible choice,
we still give the reader exactly one choice that allows the proof to proceed; we don't clog the logic by giving more than one.
Additionally, when there are multiple choices, the tradition is that we use the most natural choice we can think of. For example, if 
we need to choose a number that is between given numbers, a natural choice is to use the arithmetic average. Do not make your
 work harder for the reader to understand  by making a correct but obscure choice.

The third and final rule is that when we  make a choice for a variable, our choice can depend on all previously
introduced variables (that is, variables to the left), but it cannot  depend on variables that are yet to be introduced (variables to the right). 
For example, for the statement
\begin{equation*}
   \left(\forall x \in \reals\right)
   \left(\exists M \in \reals\right)
   \left(\forall y \in \reals \right)
 \end{equation*}
 what we choose for $M$ can depend on $x$, but it \emph{cannot}  depend on $y$. To explicitly show this dependence, some writers
 use functional notion, for example `there is $M(x) \in \reals$.'  If this functional notation helps you, use it. But arguably  
 $M(x) \in \reals$ conflates a real number (in this case $M$) with a function.

We've covered the rules and traditions for `let` and `choose.' What about `suppose'?  When a proof involves `suppose,' generally 
it means that a hypothesis follows the `suppose.' For example, if we are proving a conditional `if $x > 0$, then $x > -1$,' our proof will start with 
`Suppose $x > 0$.' This language alerts the reader that what follows `suppose' is a hypothesis.  When a proof involves special cases, 
a proof might also use `suppose' to introduce each case.

Now we give examples of what we will call a `let/show/suppose' proof.

\section{Examples}
            
\ex We start with a fairly simple example that says that there is a 
real number that is between any two distinct numbers. Specifically
the proposition is
\begin{prop}
  For all $x,y \in \reals$, there is $a \in \reals$ such that
    $x < y$ implies $x<a<y$. \label{p1}   
\end{prop}
In symbolic form, Proposition \ref{p1} is
$\left(\forall x,y \in \reals \right) 
 \left( \exists a \in \reals \right)
\left(  x < y \implies \left(x < a < y \right) \right)$.  The symbolic form 
of the proposition is our road map. Since $\left(\exists a \in \reals \right)$ follows 
$\left(\forall x,y \in \reals \right)$, what we choose for $a$ is allowed to 
depend on both $x$ and $y$. The most natural choice for $a$ is the
arithmetic average. With this, our proof is:
\begin{myproof}
    Let $x,y \in \reals$. Suppose $x < y$. Choose $a = \frac{x+y}{2}$.
    Then $a \in \reals$, as required. We have
    \begin{align*}
      \left[ x < a < y \right] 
          &\equiv \left[ x < \frac{x+y}{2} < y \right], &&\text{(substitution for $a$)} \\
          & \equiv \left[ x - \frac{x+y}{2} < 0 < y - \frac{x+y}{2}  \right], &&\text{(subtract $\nicefrac{(x+y)}{2}$)} \\
          & \equiv \left[ \frac{x-y}{2} < 0 < \frac{y-x}{2}  \right], &&\text{(simplification)} \\
          & \equiv \true. && \text{(hypothesis $x < y$)} \qedhere
    \end{align*} 
\end{myproof}
In the first line of our proof, the `Suppose' in  `Suppose $x < y$' alerts the reader that a hypothesis follows.
Each step in the proof has a parenthetical remark clarifying the reason for the equivalence. Such remarks make 
a proof easier to follow.  You are encouraged to include such remarks in your proofs.

If the fact that the arithmetic average is known to be ``in the middle,'' our proof could end at 
the second line. It's sometimes a judgment call on whether a fact can be used in a proof. When in doubt, include, don't assume.
But it is, of course, always wrong to use a result in a proof  that is way outside the context of our class.

Notice that starting with the conclusion of Proposition \ref{p1}, we expressed our proof as a series of logical \emph{equivalences.}  Logical
equivalence has the transitive property, so indeed our proof shows that $x < a < y$ is true.  
When we start with the conclusion as we did in this proof, it's vitally important that we use a series of  logical equivalences, \emph{not}
logical implications. If we expressed this work as
 \begin{align*}
      \left[ x < a < y \right] 
          &\implies \left[ x < \frac{x+y}{2} < y \right], &&\text{(substitution)} \\
          &\implies  \left[ x - \frac{x+y}{2} < 0 < y - \frac{x+y}{2}  \right], &&\text{(subtract $\nicefrac{(x+y)}{2}$)} \\
          &\implies  \left[ \frac{x-y}{2} < 0 < \frac{y-x}{2}  \right], &&\text{(simplification)} \\
          &\implies \true. && \text{(hypothesis $x < y$)} \qedhere
    \end{align*} 
we have shown that  $(x < a < y) \implies \true$. But $P \implies \true$ is
true \emph{regardless} of the truth value of $P$ (that is $P \implies \true$ is a 
tautology). So the above work proves \emph{nothing}. 

But we know that in the above work, each implication can be replaced by logical
equivalence and our work would still be correct. In general, this isn't true 
because we know that $P \implies Q$ is not logically equivalent to $Q \implies P$.
But for the above work, we can reverse the order of the 
statements and generate a proof of what was intended:
 \begin{align*}
    x < y  &\implies  \left[ \frac{x-y}{2} < 0 < \frac{y-x}{2}  \right], && \left((y-x > 0) \land  (x-y < 0) \right) \\
             &\implies   \left[ x - \frac{x+y}{2} < 0 < y - \frac{x+y}{2}  \right], &&\text{(algebraic equivalence)} \\
             &\implies \left[ x < \frac{x+y}{2} < y  \right], &&\text{(add $\nicefrac{(x+y)}{2}$)} \\
             &\implies \left[ x < a < y  \right].  &&\text{(substitution for $a$)}.
 \end{align*} 
 Using the transitive property of implication, we have proved that $x < y \implies x < a < y  $. And since $x < y$ is a hypothesis, we've indeed shown that $ x < a < y $ is true,
as we intended to do. Compared to our proof that used a series of logical equivalences,  the first logical implication in this proof 
(that is $x < y \implies 
 \frac{x-y}{2} < 0 < \frac{y-x}{2} $) seems a bit like a rabbit pulled out of a magician's hat, making the proof somewhat mysterious.
 When it's possible to write a proof as a series of logical equivalences that starts with the conclusion and ends with true, it's often 
 the clearest way to write the proof. 


 \ex In life order usually matters, and it's certainly true in logic.  Let's reverse the order of the qualified statements in Proposition 
 (\ref{p1}) and examine its truth value. We have 

 \begin{prop}
  There is $a \in \reals$ such that for all $x,y \in \reals$ we have
    $x < y$ implies $x<a<y$. \label{p2}   
\end{prop}
This is almost surely false. There is no number that is between 
any two given distinct numbers. So no, this is false. To show that its false,
let's prove that its negation is true. The negation is

\begin{prop}
  For all $a \in \reals$ there are  $x,y \in \reals$ such that
    $x < y$ and  $x \geq a$ or $a \geq y$. \label{p3} 
\end{prop}
To write the negation of the predicate, remember that
$P \notimplies Q \equiv P \land \lnot Q$. And negating $x<a<y$ is 
a bit tricky too, but remember that 
$x<a<y$ is logically equivalent to $(x \leq a) \lor (a \leq y)$. After that 
we can find the negation by applying the De Morgan law.

The conclusion of the Proposition (\ref{p3}) is a disjunction.
To write a proof, we can either choose $a$ such that $x \leq a$
or $a \leq y$. Let's choose $x$ to make  $x \leq a$. The most
natural choice is $x = a$. We still need to choose a value for $y$.
Our choice for $y$ needs to be consistent with $x < y$. Our proof is

\begin{myproof} Let $a \in \reals$.  Choose $x=a$ and $y=a+1$.
  Then $x < y$ as required. We have
  \begin{equation*}
    \left[x \geq a \lor  a \geq y \right] 
    \equiv \left[a \geq a \lor  a \geq a+1 \right]
    \equiv \left[\true  \lor  \false  \right]
    \equiv \true. \qedhere
  \end{equation*}
\end{myproof}

\ex Our next example looks a great deal like Proposition  (\ref{p1}) where we used the arithmetic average to choose
a number that was between two given numbers,  but we'll see that this proposition has an extra twist.
\begin{prop} 
     For all $r \in \reals_{>0}$ there is $x \in [0,1)$ such that $1-r < x$.  \label{p4}
\end{prop}

In symbolic form, Proposition \ref{p4} is $\left(\forall r  \in \reals_{>0} \right) \left(\exists x \in [0,1)\right) \left (1-r  < x \right)$.
And since $x \in [0,1)$, additionally we know that $x < 1$.  So the predicate $1-r  < x$ is equivalent to $1-r  < x  < 1$. Now Proposition \ref{p4}
looks a great deal like  Proposition \ref{p1}. Maybe we can simply 
choose $x$ to be the arithmetic average of $1-r$ and $1$. That is,
$x = 1 - \frac{r}{2}$.  But the requirement that $0 < x$ spoils this choice. If $r = 4$, for example, we are choosing $x = -1$. And that's 
not allowed.    

One way to fix this is to choose $x$ to be the arithmetic average of $1-r$ and $1$ when $r < 1$ and choose $x = \frac{1}{2}$ when $r \geq 1$. 
Actually, when $r \geq 1$, we could choose $x$ to be any member of $[0,1)$, but choosing the midpoint is arguably the most natural.
Our proof breaks into two cases:
     
      \begin{myproof} Let $r \in \reals_{>0}$. Choose $x = \begin{cases} 1 - \frac{r}{2}  & r < 1 \\ \frac{1}{2} & r \geq 1 \end{cases}$. For $r < 1$, we have
      \begin{align*}
       \left[1-r < x \right] &\equiv \left[1-r < 1 - \frac{r}{2}  \right], && \text{(substitution for $x$)} \\
                                  &\equiv  \left[0 <  \frac{r}{2}  \right], && \text{(add $r-1$)} \\
                                  &\equiv \true.   &&(0 < r < 1)
      \end{align*}
      And for $r \geq 1$, we have
       \begin{align*}
       \left[1-r < x \right] &\equiv \left[1-r < \frac{1}{2}  \right], && \text{(substitution for $x$)} \\
                                  &\equiv  \left[\frac{1}{2}  <  r  \right], && \text{(add $r-1$)} \\
                                  &\equiv \true.   &&(r \geq 1) \qedhere
      \end{align*}
      \end{myproof}
 If you prefer to condense this proof into one case, we can choose  $x = \max(\frac{1}{2}, 1-\frac{r}{2})$. 
 
     \begin{myproof} Let $r \in \reals_{>0}$. Choose $x = \max(\frac{1}{2}, 1-\frac{r}{2})$.  Then
     $\frac{1}{2} \leq x$ and  $x  \leq 1-\frac{r}{2}$. But $ r > 0$, so $x \leq 1$. Thus, $x \in [0,1)$ as required. We have     
      \begin{align*}
       \left[1-r < x \right] &\equiv \left[1-r <  \max \left(\frac{1}{2}, 1-\frac{r}{2} \right)  \right], && \text{(substitution for $x$)} \\
                                 &\equiv  \left[0 <  \max \left (r - \frac{1}{2}, \frac{r}{2} \right)  \right], && \text{(add $r-1$)} \\
                                  &\equiv \true.   &&(0 < \nicefrac{r}{2}) \qedhere
      \end{align*}
 \end{myproof}
 \noindent This proof uses the true, but possibly obscure, fact that for all $x,y,z \in \reals$, the equation 
 \begin{equation*}
    \max(x,y) - z = \max(x-z, y-z)
\end{equation*}
 is an identity.  Additionally, it uses the facts that $\max(x,y) \leq x$ and $\max(x,y) \leq y$.
 

 \ex   Generically, our next proposition looks like another choose 
 the arithmetic average proof. 
      
    \begin{prop}   For all $x \in \reals_{>0}$ there is 
      $y \in \reals_{> 0}$ such that $y < x$.  \label{p5}
    \end{prop}
   
    For this proof,  we need to choose a  number between  zero and $y$. Our proof is
      \begin{myproof}  Let $x \in \reals_{>0}$. Choose $y = x/2$. Then $y \in \reals_{>0}$ as required. We have
         \begin{align*}
         \left[y < x \right] &\equiv \left[\frac{x}{2} < x \right], &&\text{(substitution for $y$)} \\
                                   &\equiv \left[0 < \frac{x}{2}  \right], &&\text{(subtract $\nicefrac{x}{2}$)} \\
                                   &\equiv \true.  &&\text{(x > 0)} \qedhere
      \end{align*}
      \end{myproof}
Abstracted, Proposition \ref{p5} says that there is no smallest positive number. That is, given any positive number, there is a 
positive number that is smaller.  This is an important property of the field of real numbers.
       
\ex Our next proposition is the well known fact that there is
no largest positive number. Specifically
       \begin{prop} For every $y \in \reals_{> 0}$ there is  
        $x \in \reals_{>0}$ such that $y \geq  x$. \label{p6}
      \end{prop}
      Paraphrased, Proposition \ref{p6} says that given any positive number $x$, there is a positive number $y$ that is greater or equal to $x$. Alternatively, we
      might say that there is no largest positive number. 
      
      Adding one is a good strategy for constructing a larger number, but this proposition
      allows equality, not strict inequality.  When possible, choose equality. Our proof
      
       \begin{myproof} Let $y \in \reals_{>0}$. Choose $x = y$. Then $x \in \reals_{>0}$, as 
        required. We have
       \begin{align*}
         \left[y \geq x  \right] &\equiv \left[y \geq  y \right], && \text{(substitution for $x$)} \\
                                      &\equiv \left[0 \geq 0 \right], &&\text{(subtract $y$)} \\
                                      &\equiv \true. && \qedhere
       \end{align*}
       
       \end{myproof}

   


       \ex The next example possibly encourages us to carefully 
       consider the left to right rule and to make our choices 
       as naturally as possible.

    \begin{prop} For all $x \in \reals_{>0}$ there is $M \in \reals$ such
     that $\frac{1}{x} +1 > M$. 
    \end{prop}
    Remember our road map? Since $M$ follows $x$, we can choose $M$ to depend on $x$. We could, for example, 
    choose $M = \frac{1}{x}$. That gives our first proof:
         \begin{myproof} 
      Let $x \in \reals_{>0}$. Choose $M = \frac{1}{x}$.  Then $x  \in \reals_{>0}$ as required. We have
          \begin{align*}
          \left[\frac{1}{x}+1>M \right] &\equiv \left[\frac{1}{x}+1> \frac{1}{x} \right], &&\text{(substitution for $M$)} \\
                                  &\equiv \left [1 >0 \right], && \text{(subtract $x$)} \\
                                  &\equiv \true.  &&  \qedhere
      \end{align*}
\end{myproof}
And if you like, we can choose $M$ that is independent of $x$; for example    
    \begin{myproof} 
      Let $x \in \reals_{>0}$. Choose $M = 1$. We have
          \begin{align*}
          \left[\frac{1}{x}+1>M \right] &\equiv \left[\frac{1}{x}+1>1\right], &&\text{(substitution for $M$)} \\
                                  &\equiv \left [\frac{1}{x}>0 \right],
                                  &&\text {(subtract 1)} \\
                                  &\equiv \true. &&\text {(Ordered Field Axioms)} \qedhere
      \end{align*}
\end{myproof}

\ex The next proposition has to do with a function being bounded.
    \begin{prop}
      There is $M \in \reals$ such that for all $x \in \reals_{>0}$,
     we have $\frac{1}{x} + 1 > M$. 
    \end{prop}
    In function language, this proposition  says that the function $x \in \reals_{>0} \mapsto 1+\frac{1}{x}$ is bounded below.
        A quick graph of $ y = \frac{1}{x} + 1 $ for  $x \in \reals_{>0}$ shows that its graph is above the horizontal line $y = 1$. For $M$, 
    we can choose any number that is one or less.
    
    This proposition  is just like the previous one, but with a switched order of the variables.  Order almost always matters, but in this case, it 
    doesn't. Our proof
    
    \begin{myproof} Choose $M=1$. Let $x  \in \reals_{>0}$, we have
    \begin{align*}
      \left [ \frac{1}{x} + 1 > M \right] &\equiv    \left [ \frac{1}{x} + 1 > 1 \right], &&(\text{substitution}) \\
                                                               &\equiv    \left [ \frac{1}{x}  > 0 \right], &&(\text{subtract one}) \\
                                                               &\equiv \true.    &&  (x > 0) \qedhere
  \end{align*} 
    
    
    \end{myproof}


    \ex Here is a bit of a mysterious proposition. A key to coming 
    up with a proof is to think graphically (and that's G from GNAT).
        \begin{prop}
      There is $m \in \reals$ such that for all $x \in \reals$, we 
     have $1 + m(x-1) \leq x^2$. 
     \end{prop}
     
      A good way of thinking about this proposition geometrically is to view it as two graphs. In the $xy$ plane, the graph of  $y = 1 + m(x-1)$, is a line  and the
      graph of $y = x^2$  is a parabola.   Both the line  and the parabola contain the point $(x=1,y=1)$. And the 
      condition $1 + m(x-1) \leq x^2$ tells
      us that the line $y = 1 + m(x-1)$ must be below or touching the parabola $y=x^2$.  How can this be? A few quick sketches of lines that
      contain the point $(x=1,y=1)$  shows that the line $y = 1 + m(x-1)$ must be a tangent line to the parabola at the point $(x=1,y=1)$.  If 
      the line isn't tangent to the parabola, the line will be below the parabola on one side of $x=1$ and above on the other.   We need the
      line to be on or below the parabola.  So   elementary calculus tells us that the slope of the line must be two. With that choice
      our proof is      
       \begin{myproof} Choose $m$ = 2. Let $x\in \reals$. We have
       \begin{align*}
         \left[1 + m(x-1) \leq x^2  \right] &\equiv \left[1+2(x-1) \leq  x^2 \right], && \text{(substitution for $m$)} \\
                                      &\equiv \left[2x-1 \leq x^2 \right], 
                                       &&\text{(Algebra)} \\
                                    &\equiv \left[0 \leq x^2 -2x+1 \right], 
                                       &&\text{(Algebra)} \\
                                    &\equiv \left[0 \leq (x-1)^2 \right], 
                                       &&\text{(Factor)} \\
                                      &\equiv \true. &&\text{(Ordered field axioms)} \qedhere
       \end{align*}
       \end{myproof}
       \noindent If you enjoy crabbed proofs, you can rearrange this as a sequence of implications starting from $0 \leq (x-1)^2$. 
   
       \ex And just a tiny generalization of the previous proposition.

     \begin{prop} For every $a \in \reals$, there is $m \in \reals$ such 
     that for all $x \in \reals$, we have $a^2 + m(x-a) \leq x^2$.
     \end{prop}
     This proposition generalizes the previous proposition to a point of tangency $(x=a,y=a^2)$. Given that we choose $m=2a$. 

      \begin{myproof} 
      Let $a \in \reals$. Choose $m=2a$. We have
      \begin{align*}
       \left[{a^2+m(x-a) \le x^2} \right]   
       &\equiv \left[a^2+2a(x-a)\le x^2\right], &&\text{(substitution)}\\
      &\equiv \left[0 \le x^2-2ax+a^2\right], &&\text{(algebra)}\\
      &\equiv \left[0 \le (x-a)^2\right]. &&\text{factor}  \qedhere
      \end{align*}
  \end{myproof}

  \ex One-to-oneness is a familiar concept. The next proposition 
  involves this concept.

     \begin{prop} For all $x,y \in \reals$, we have $(x^2 = y^2) \implies (x=y)$. 
     \end{prop} 
    \noindent This says that the function that squares is one-to-one. We know this is false. It's certainly possible for outputs of 
    the squaring function to be equal and the inputs distinct. An example is $(-1)^2 = 1^2$ and $-1 \neq 1$. Although this example
    is a counterexample to the proposition, for  practice negating qualified statements and conditionals,  we'll explicitly negate the proposition and write a let/show/suppose  proof that its negation is true.
    To do this we need to remember the tautology
    \begin{equation*}
       \left(P \notimplies Q\right) \equiv \left(P \land \lnot Q\right)
 \end{equation*}
   This we will prove
    \begin{prop} There are $x,y \in \reals$ such that  $(x^2 = y^2)$ and $x\neq y$. 
     \end{prop} 
     
     \begin{myproof} Choose $x=-1$ and $y=1$. We have $(-1)^2 = 1^2$ and $-1 \neq 1$.      \end{myproof}

     \ex The function that cubes is a well known example of a nonlinear 
     function that is one-to-one. Let's prove that it is.

    \begin{prop} For all $x,y \in \reals$, we have $(x^3 = y^3) \implies (x=y)$. 
      \end{prop}
\noindent Distilled, this proposition says that the real-valued cubing function is one-to-one. Graphically this is easy to verify, but algebraically, it
requires the tricky factorization
\begin{equation*}
  x^3 - y^3 = (x-y) (x^2+x y+y^2) =  (x-y) \left(x + y/2)^2 + 3 y^2 /4\right).
\end{equation*}
The factor $\left(x + y/2\right)^2 + 3 y^2 /4$ is a sum of squares (SOS) that only vanishes when $x=0$ and $y=0$.  This fact is the core of
our proof.
\begin{myproof}
Let $x,y \in \reals$. Suppose $x^3 = y^3$; we'll show that $x=y$. We have
\begin{align*}
 \left[x^3 = y^3 \right] &\equiv \left [x^3 - y^2 = 0 \right],   &&(\mbox{algebra}) \\
                                        &\equiv \left [(x-y)  \left((x + y/2)^2 + 3 y^2 /4 \right)  = 0 \right],   &&(\mbox{tricky algebra}) \\
                                        &\equiv \left [(x-y) = 0   \lor  (x + y/2)^2 + 3 y^2 /4)  = 0 \right],   &&(\text{ordered field properties}) \\
                                         &\equiv \left [x = y    \lor  \left(x=0 \land y = 0 \right) \right],   &&(\text{SOS fact}) \\
                                         &\equiv \left[x=y\right].  &&(\text{unify disjunction}) \qedhere
\end{align*}
\end{myproof}
    \begin{prop} For all $r \in \reals_{>0}$ there is $x \in \reals$
      such that $1 < x < 1+r$. 
    \end{prop} 

To prove this, we only need to choose a number that is between 1 and $1+r$. Again, we can choose the arithmetic average.
\begin{myproof} Let $r \in \reals_{>0}$.  Choose $x= 1+r/2$. Then indeed $x \in \reals_{> 0}$ as required. We have

\begin{align*}
  \left[1 < x < 1+r \right] &\equiv \left[1 < 1+r/2 < 1+r \right],   && (\text{substitution for } x )\\
                                          & \equiv \left[-r/2 < 0  < r/2 \right],  && (\text{subtraction})  \\
                                          &\equiv \true.                                      && (r > 0). \qedhere
\end{align*}

\end{myproof}

\ex Our next example  shows that when we violate the left-to-right rule, we can generate an  argument that might look OK, but 
is logically faulty.  Here is an example

\begin{prop} There is $M \in \reals$ such that for all $x \in \reals_{>0}$ we have  $\frac{1}{x} \leq M$.  
\end{prop}

Almost surely, the truth value of this proposition is false. By choosing $x$ to be positive and close to zero,  we can make $\frac{1}{x}$ 
arbitrarily large, but the proposition requires that  $\frac{1}{x}$   be bounded above by $M$. If we violate the left-to-right rule
we can construct a faulty proof of this false proposition:

\begin{myproof} (rubbish) Choose $M = \frac{1}{x}$. Let  $x \in \reals_{\geq 0}$. We have
\begin{align*}
 \left[\frac{1}{x}  \leq M  \right] &\equiv
 \left[\frac{1}{x}  \leq  \frac{1}{x}  \right],  &&\mbox{(substitution for $M$)} \\
 &\equiv \mbox{True}. &&\mbox{(syntactic equality)}.   \qedhere
\end{align*}
\end{myproof}
\noindent Again, what's wrong with this work?   The proof is rubbish because it violates  the left-to-right rule.  Since $M$ is qualified before $x$,
  the value we choose for $M$ is not allowed to depend on $x$. Not only is the proof faulty, but proposition is false.  To show that the 
  truth value of this proposition is false, let's write a correct proof of its negation.  The negation  is 

\begin{prop} For all $M \in \reals$  there is $x \in \reals_{>0}$ such that $\frac{1}{x} >  M$.  \label{px14} \end{prop}

Let's write a proof of  Proposition (\ref{px14}).

\begin{myproof} Let $M \in \reals$.  Choose $x  = \frac{1}{|M| + 1 }$. Then  $x \in \reals_{>0}$ as required.  We have
\begin{align*}
  \left [ \frac{1}{x} > M \right] &\equiv \left[  \frac{1}{\frac{1}{|M| + 1|} } > M   \right], &&\text{(substitution)} \\
                                                      &\equiv \left[|M| + 1 > M \right], &&\text{(algebra)} \\
                                                      &\equiv \true.  &&( M \leq |M|) \qedhere
\end{align*}
\end{myproof}                                   

\end{document}