\documentclass[12pt,fleqn,answers]{exam}
\usepackage{pifont}
\usepackage{dingbat}
\usepackage{amsmath,amssymb}
\usepackage{fleqn}
\usepackage{epsfig}
%\usepackage{mathptm}
%\usepackage{euler}
\usepackage{bbding}

\usepackage{geometry}
\geometry{letterpaper, margin=0.6in}
\addpoints
\boxedpoints
\pointsinmargin
\pointname{pts}

\usepackage[activate={true,nocompatibility},final,tracking=true,kerning=true,factor=1100,stretch=10,shrink=10]{microtype}
\frenchspacing
\usepackage[american]{babel}
%\usepackage[T1]{fontenc}
\usepackage{fourier}
\usepackage{isomath}
\usepackage{upgreek,amsmath}
\usepackage{amssymb}
%\usepackage[euler-digits,euler-hat-accent,T1]{eulervm}

\newcommand{\dotprod}{\, {\scriptzcriptztyle
    \stackrel{\bullet}{{}}}\,}

\newcommand{\reals}{\mathbf{R}}
\newcommand{\complex}{\mathbf{C}}
\newcommand{\dom}{\mbox{dom}}
\newcommand{\cover}{{\mathcal C}}
\newcommand{\integers}{\mathbf{Z}}
\newcommand{\vi}{\, \mathbf{i}}
\newcommand{\vj}{\, \mathbf{j}}
\newcommand{\vk}{\, \mathbf{k}}
\newcommand{\bi}{\, \mathbf{i}}
\newcommand{\bj}{\, \mathbf{j}}
\newcommand{\bk}{\, \mathbf{k}}
\DeclareMathOperator{\Arg}{\mathrm{Arg}}
\DeclareMathOperator{\Ln}{\mathrm{Ln}}
\newcommand{\imag}{\, \mathrm{i}}
\newcommand{\true}{\, \mathrm{True}}
\newcommand{\false}{\, \mathrm{False}}
\usepackage{amsthm}
\usepackage{xcolor}
\newtheorem{Rubbish}{Theorem}
\usepackage{graphicx}

\usepackage{framed}

\colorlet{shadecolor}{lightgray!15}
\newenvironment{myproof}
  {\begin{shaded}\begin{proof}}
  {\end{proof}\end{shaded}}

  \newtheorem{prop}{Proposistion}

\newenvironment{mypar}[2]
  {\begin{list}{}%
    {\setlength\leftmargin{#1}
    \setlength\rightmargin{#2}}
    \item[]}
  {\end{list}}

  \usepackage{enumerate}
  \newenvironment{alphalist}{
  %\vspace{-0.4in}
  \begin{enumerate}[(a)]
    \addtolength{\itemsep}{-0.50\itemsep}}
  {\end{enumerate}}
\newcommand{\quiz}{1}
\newcommand{\term}{Fall}

\usepackage{xspace}
\makeatletter
\DeclareRobustCommand{\maybefakesc}[1]{%
  \ifnum\pdfstrcmp{\f@series}{\bfdefault}=\z@
    {\fontsize{\dimexpr0.8\dimexpr\f@size pt\relax}{0}\selectfont\uppercase{#1}}%
  \else
    \textsc{#1}%
  \fi
}
\newcommand\AM{\,\maybefakesc{am}\xspace}
\newcommand\PM{\,\maybefakesc{pm}\xspace}

\begin{document}

\begin{prop}
  For all $x,y \in \reals$, there is $a \in \reals$ such that
    $x < y$ implies $x<a<y$.    
\end{prop}

\begin{myproof} (BW)
    Let $x,y \in \reals$. Suppose $x < y$. Choose $a = \frac{x+y}{2}$.
    Then $a \in \reals$ as required. We have
    \begin{align*}
      \left[ x < a < y \right] 
          &\equiv \left[ x < \frac{x+y}{2} < y \right], &\mbox{(substitution)} \\
          & \equiv \left[ x - \frac{x+y}{2} < 0 < y - \frac{x+y}{2}  \right], &\mbox{(subtract $\frac{x+y}{2}$)} \\
          & \equiv \left[ \frac{x-y}{2} < 0 < \frac{y-x}{2}  \right], &\mbox{(simplification)} \\
          & \equiv \true. &((y-x > 0) \land (x-y < 0))
    \end{align*} 
       \end{myproof}


     \begin{prop} 
     For all $r \in \reals_{>0}$ there is $x \in [0,1)$ such that $1-r < x$. 
     \end{prop}

    
     
      \begin{myproof} (BW) Let $r \in \reals_{>0}$. Choose $x = \begin{cases} 1 - \frac{r}{2}  & r < 1 \\ \frac{1}{2} & r \geq 1 \end{cases}$. For $r < 1$, we have
      \begin{align*}
       \left[1-r < x \right] &\equiv \left[1-r < 1 - \frac{r}{2}  \right], & \mbox{(substitution)} \\
                                  &\equiv  \left[0 <  \frac{r}{2}  \right], & \mbox{(algebra)} \\
                                  &\equiv \true.   &(0 < r < 1).
      \end{align*}
      And for $r \geq 1$, we have
       \begin{align*}
       \left[1-r < x \right] &\equiv \left[1-r < \frac{1}{2}  \right], & \mbox{(substitution)} \\
                                  &\equiv  \left[\frac{1}{2}  <  r  \right], & \mbox{(algebra)} \\
                                  &\equiv \true.   &(r \geq 1).
      \end{align*}
      \end{myproof}
     

    \begin{prop}   For all $x \in \reals_{>0}$ there is 
      $y \in \reals_{> 0}$ such that $y < x$. 
    \end{prop}
     
    
      
      \begin{myproof} (BW) Let $x \in \reals_{>0}$. Choose $y = x/2$. Then $y \in \reals_{>0}$ as required. We have
         \begin{align*}
         \left[y < x \right] &\equiv \left[\frac{x}{2} < x \right], &\mbox{(substitution)} \\
                                   &\equiv \left[0 < \frac{x}{2}  \right], &\mbox{(algebra)} \\
                                   &\equiv \true.  &\mbox{(x > 0)}
      \end{align*}
      \end{myproof}

       

       \begin{prop} For every $y \in \reals_{> 0}$ there is  
        $x \in \reals_{>0}$ such that $y \geq  x$.
      \end{prop}

       \begin{myproof} (BW) Let $y \in \reals_{>0}$. Choose $x = y$. Then $x \in \reals_{>0}$, as 
        required. We have
       \begin{align*}
         \left[y \geq x  \right] &\equiv \left[y \geq  y \right], & \mbox{(substitution)} \\
                                      &\equiv \left[0 \geq 0 \right], &\mbox{(algebra)} \\
                                      &\equiv \true.
       \end{align*}
       
       \end{myproof}

   


       
    \begin{prop} For all $x \in \reals_{>0}$, there is $M \in \reals$ such
     that $\frac{1}{x} +1 > M$. \hfill (SB)
    \end{prop}

    \begin{prop}
      There is $M \in \reals$ such that for all $x \in \reals_{>0}$,
     we have $\frac{1}{x} + 1 > M$. \hfill (DD) 
    \end{prop}

     \begin{prop}
      There is $m \in \reals$ such that for all $x \in \reals$, we 
     have $1 + m(x-1) \leq x^2$. \hfill (TK)
     \end{prop}

     \begin{prop} For every $a \in \reals$, there is $m \in \reals$ such 
     that for all $x \in \reals$, we have $a^2 + m(x-a) \leq x^2$. \hfill (AK)
     \end{prop}

     \begin{prop} For all $x,y \in \reals$, we have $(x^2 = y^2) \implies (x=y)$. 
     \hfill (DM)
     \end{prop} 

    \begin{prop} For all $x,y \in \reals$, we have $(x^3 = y^3) \implies (x=y)$. 
    \hfill (CR) 
    \end{prop}


\end{document}