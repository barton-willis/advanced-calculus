\documentclass[12pt,fleqn,answers]{exam}
\usepackage{pifont}
\usepackage{dingbat}
\usepackage{amsmath,amssymb}
\usepackage{fleqn}
\usepackage{epsfig}
%\usepackage{mathptm}
%\usepackage{euler}
\usepackage{bbding}
\usepackage{url}
\addpoints
\boxedpoints
\pointsinmargin
\pointname{pts}

\usepackage{xcolor}
\usepackage{framed}
\colorlet{shadecolor}{lightgray!15}
\newenvironment{myproof}
  {\begin{shaded}\begin{proof}}
  {\end{proof}\end{shaded}}

\usepackage{amsthm}
\newtheorem{prop}{Proposition}

\usepackage[activate={true,nocompatibility},final,tracking=true,kerning=true,factor=1100,stretch=10,shrink=10]{microtype}
\usepackage[american]{babel}
%\usepackage[T1]{fontenc}
\usepackage{fourier}
\usepackage{isomath}
\usepackage{upgreek,amsmath}
\usepackage{amssymb}
%\usepackage[euler-digits,euler-hat-accent,T1]{eulervm}

\newcommand{\dotprod}{\, {\scriptzcriptztyle
    \stackrel{\bullet}{{}}}\,}

\newcommand{\reals}{\mathbf{R}}
\newcommand{\complex}{\mathbf{C}}
\newcommand{\dom}{\mbox{dom}}
\newcommand{\cover}{{\mathcal C}}
\newcommand{\rat}{\mathbf{Q}}

\newcommand{\curly}[1]{\mathcal #1}
\newcommand{\integers}{\mathbf{Z}}
\newcommand{\vi}{\, \mathbf{i}}
\newcommand{\vj}{\, \mathbf{j}}
\newcommand{\vk}{\, \mathbf{k}}
\newcommand{\bi}{\, \mathbf{i}}
\newcommand{\bj}{\, \mathbf{j}}
\newcommand{\bk}{\, \mathbf{k}}
\DeclareMathOperator{\Arg}{\mathrm{Arg}}
\DeclareMathOperator{\Ln}{\mathrm{Ln}}
\newcommand{\imag}{\, \mathrm{i}}
\newcommand{\range}{\mathrm{range}}
\newcommand{\true}{\mathrm{True}}
\newcommand{\saw}{\mathrm{saw}}
\usepackage{enumerate}
\newenvironment{alphalist}{
  \begin{enumerate}[(a)]
    \addtolength{\itemsep}{-0.5\itemsep}}
  {\end{enumerate}}
 
\usepackage{amsthm}
\newtheorem{Rubbish}{Theorem}
\usepackage{graphicx}

%\usepackage{tgschola} %to look retro
\newenvironment{mypar}[2]
  {\begin{list}{}%
    {\setlength\leftmargin{#1}
    \setlength\rightmargin{#2}}
    \item[]}
  {\end{list}}
  
\newcommand{\quiz}{11}
\newcommand{\term}{Fall}
\usepackage{units}
\usepackage{xspace}
\makeatletter
\DeclareRobustCommand{\maybefakesc}[1]{%
  \ifnum\pdfstrcmp{\f@series}{\bfdefault}=\z@
    {\fontsize{\dimexpr0.8\dimexpr\f@size pt\relax}{0}\selectfont\uppercase{#1}}%
  \else
    \textsc{#1}%
  \fi
}
\newcommand\AM{\,\maybefakesc{am}\xspace}
\newcommand\PM{\,\maybefakesc{pm}\xspace}

\begin{document}
\large
\vspace{0.1in}
\noindent\makebox[3.0truein][l]{ \textbf{MATH 460}}
{\bf Name:}  \\
\noindent \makebox[3.0truein][l]{\textbf{Homework \quiz, \term \/ \the\year}}
%{\bf Row:}\hrulefill\
\vspace{0.1in}

\noindent  Homework \quiz\/  has questions 1 through  \numquestions \/ with a total 
of  \numpoints\/  points. 
This work is due \textbf{Saturday 25 November} at 11:59 \PM.
\textbf{For this assignment, neatly handwrite your solutions and submit
a digitized version to Canvas.}

\vspace{0.1in}


\begin{questions}

 \question Let $F = \in [0,1] \mapsto x^2$. For any $n \in \integers_{>0}$,
 define a partition $P_n$ of $[0,1]$, by 
 $ \left \{\nicefrac{k}{n} \, | \, k \in \integers_{\geq 0, \leq n} \right \}$.
 For this partition, the length of each subinterval is $1/n$.
 So a general Riemann sum has the form
 \begin{equation*}
     \frac{1}{n }\sum_{k=0}^{n-1}  F(c_k),
 \end{equation*}
where $c_k \in [\nicefrac{k}{n},\nicefrac{(k+1)}{n} ]$.
Choosing $c_k = \nicefrac{k}{n}$ gives the left point Riemann sum 
$L_n$. Specifically
 \begin{equation*}
    L_n = \frac{1}{n} \sum_{k=0}^{n-1} F(k/n)
 \end{equation*}
 And choosing $c_k = \nicefrac{(k+1)}{n}$ gives the right point Riemann sum 
 $L_n$. Specifically
  \begin{equation*}
     R_n = \frac{1}{n} \sum_{k=0}^{n-1} F(k+1/n)
  \end{equation*}
Changing the sum index $k$ to $k-1$ gives an alternative
formula for $R_n$. It is
\begin{equation*}
    R_n = \frac{1}{n} \sum_{k=1}^{n} F(k/n)
 \end{equation*}


 \begin{parts}

    \part [10] Find a simple, and I mean simple representation for 
    $R_n - L_n$.  \textbf{Hint:} When I'm overwhelmed, I go to my 
    happy place. For sums, my happy place is to list a few terms:
    \begin{align*}
        L_n &= \frac{1}{n} \left(F(0/n) + F(1/n) + F(2/n) + \cdots + F((n-1)/n) \right), \\
        R_n &= \frac{1}{n} \left(F(1/n) + F(2/n) + F(3/n) + \cdots + F(1) \right).
    \end{align*}

    \part [10] Let $\varepsilon \in \reals_{>0}$.  Show that there is 
    $n \in \integers_{>0}$ such that $|R_n - L_n| < \varepsilon$.

 \end{parts}





\end{questions}
\end{document}