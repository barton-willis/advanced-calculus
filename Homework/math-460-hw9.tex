\documentclass[12pt,fleqn,answers]{exam}
\usepackage{pifont}
\usepackage{dingbat}
\usepackage{amsmath,amssymb}
\usepackage{fleqn}
\usepackage{epsfig}
%\usepackage{mathptm}
%\usepackage{euler}
\usepackage{bbding}
\usepackage{url}
\addpoints
\boxedpoints
\pointsinmargin
\pointname{pts}

\usepackage{xcolor}
\usepackage{framed}
\colorlet{shadecolor}{lightgray!15}
\newenvironment{myproof}
  {\begin{shaded}\begin{proof}}
  {\end{proof}\end{shaded}}

\usepackage{amsthm}
\newtheorem{prop}{Proposition}

\usepackage[activate={true,nocompatibility},final,tracking=true,kerning=true,factor=1100,stretch=10,shrink=10]{microtype}
\usepackage[american]{babel}
%\usepackage[T1]{fontenc}
\usepackage{fourier}
\usepackage{isomath}
\usepackage{upgreek,amsmath}
\usepackage{amssymb}
%\usepackage[euler-digits,euler-hat-accent,T1]{eulervm}

\newcommand{\dotprod}{\, {\scriptzcriptztyle
    \stackrel{\bullet}{{}}}\,}

\newcommand{\reals}{\mathbf{R}}
\newcommand{\complex}{\mathbf{C}}
\newcommand{\dom}{\mbox{dom}}
\newcommand{\cover}{{\mathcal C}}
\newcommand{\rat}{\mathbf{Q}}

\newcommand{\curly}[1]{\mathcal #1}
\newcommand{\integers}{\mathbf{Z}}
\newcommand{\vi}{\, \mathbf{i}}
\newcommand{\vj}{\, \mathbf{j}}
\newcommand{\vk}{\, \mathbf{k}}
\newcommand{\bi}{\, \mathbf{i}}
\newcommand{\bj}{\, \mathbf{j}}
\newcommand{\bk}{\, \mathbf{k}}
\DeclareMathOperator{\Arg}{\mathrm{Arg}}
\DeclareMathOperator{\Ln}{\mathrm{Ln}}
\newcommand{\imag}{\, \mathrm{i}}
\newcommand{\range}{\mathrm{range}}
\newcommand{\true}{\mathrm{True}}
\newcommand{\saw}{\mathrm{saw}}
\usepackage{enumerate}
\newenvironment{alphalist}{
  \begin{enumerate}[(a)]
    \addtolength{\itemsep}{-0.5\itemsep}}
  {\end{enumerate}}
 
\usepackage{amsthm}
\newtheorem{Rubbish}{Theorem}
\usepackage{graphicx}

%\usepackage{tgschola} %to look retro
\newenvironment{mypar}[2]
  {\begin{list}{}%
    {\setlength\leftmargin{#1}
    \setlength\rightmargin{#2}}
    \item[]}
  {\end{list}}
  
\newcommand{\quiz}{9}
\newcommand{\term}{Fall}

\usepackage{xspace}
\makeatletter
\DeclareRobustCommand{\maybefakesc}[1]{%
  \ifnum\pdfstrcmp{\f@series}{\bfdefault}=\z@
    {\fontsize{\dimexpr0.8\dimexpr\f@size pt\relax}{0}\selectfont\uppercase{#1}}%
  \else
    \textsc{#1}%
  \fi
}
\newcommand\AM{\,\maybefakesc{am}\xspace}
\newcommand\PM{\,\maybefakesc{pm}\xspace}

\begin{document}
\large
\vspace{0.1in}
\noindent\makebox[3.0truein][l]{ \textbf{MATH 460}}
{\bf Name:}  \\
\noindent \makebox[3.0truein][l]{\textbf{Homework \quiz, \term \/ \the\year}}
%{\bf Row:}\hrulefill\
\vspace{0.1in}

\noindent  Homework \quiz\/  has questions 1 through  \numquestions \/ with a total 
of  \numpoints\/  points. 
This work is due \textbf{Saturday 4 November} at 11:59 \PM.

\vspace{0.1in}


\begin{questions}

 \question [10] Let $F \in \reals \to \reals_{\geq 0}$ have a limit  toward $5$. Use the QRS definition of a limit to 
 show that $\sqrt{F}$ has a limit toward $5$.  Almost surely, you will want to use the fact that the 
 square root function is subadditive; that is
 \begin{equation*}
 \left(\forall x, y \in \reals_{>0} \right)(|\sqrt{x} - \sqrt{y}| \leq \sqrt{|x - y|}).
 \end{equation*}
 \begin{solution}

\begin{proof}

\end{proof}
\end{solution}
 
\question [10] Use the fact that the absolute value function is continuous
on $\reals$ to show that the function $F = x \in \reals \mapsto x |x|$ 
is differentiable at zero.   (Actually, you'll only use the fact that the absolute value function is continuous
at zero. But it's true that it is continuous on $\reals$.)

\begin{solution}

\begin{proof}

\end{proof}
\end{solution}

\question [10] Show that the function $F = x \in \reals \mapsto \begin{cases}
    x \sin(\frac{1}{x}) & x \neq 0 \\
    0 & x = 0
\end{cases}$ has a limit toward zero.  To do this, use the 
fact that for all $x \in \reals_{\neq 0}$, we have $|x \sin(\frac{1}{x})| \leq |x|$.
From this work, show that $F$ is continuous at zero.
\begin{solution}

\begin{proof}

\end{proof}
\end{solution}

\question [10] Show that the function 
$G = x \in \reals \mapsto \begin{cases}
    x^2 \sin(\frac{1}{x}) & x \neq 0 \\ 0 & x = 0
\end{cases}$ is differentiable at zero. One way to proceed is to use the result of the previous question.
\begin{solution}

\begin{proof}

\end{proof}
\end{solution}
\question [10] Again, define $G = x \in \reals \mapsto \begin{cases}
    x^2 \sin(\frac{1}{x}) & x \neq 0 \\ 0 & x = 0 \end{cases}$. Away from zero, the product rule, the 
    chain rule, and the rule for the derivative of sine, gives 
    $G^\prime(x) = 2 x \sin{\left( \frac{1}{x}\right) }  - \cos{\left( \frac{1}{x}\right) }\); and
    at zero, we know that $G^\prime(0) = 0$. Thus $G$ is differentiable on $\reals$ and 
    \begin{equation*}
     G^\prime = x \in \reals \mapsto \begin{cases} 
      2 x  \sin{\left( \frac{1}{x}\right) } - \cos{\left( \frac{1}{x}\right) } & x \neq 0 \\
      0 & x = 0
      \end{cases}.
     \end{equation*}
    Show that $G^\prime$ does not have a limit toward 0.  Consequently, $\displaystyle G^\prime (0) \neq \lim_{0} G^\prime$ 
     
     \textbf{Hints}  Let $\delta \in \reals_{>0}$.
     Archimedes tells us that there is a positive integer $n$ such that $ \frac{1}{n} < \delta$. Also
     for all positive integers $n$, we have
    $G^\prime(\frac{1}{2 \uppi n}) = -1$ and $G^\prime(\frac{1}{ \uppi (2 n + 1)}) = 1$. 
     
     \begin{solution}

\begin{proof}

\end{proof}
\end{solution}

\end{questions}
\end{document}

    \question [10] Let $f$ and $g$ be smooth functions.  That means that these 
    functions have as many derivatives as we like. Then repeated use of
    the product rule gives the identities
    \begin{align*}
       (f g)^{(1)} &= f g^{(1)} + f^{(1)} g, \\
       (f g)^{(2)} &= f g^{(2)} + 2 f^{(1)} g^{(1)} + f^{(2)} g, \\
       (f g)^{(3)} &= f g^{(3)} + 3 f^{(1)} g^{(2)} +  3 f^{(2)} g^{(1)} + f^{(3)} g.        
    \end{align*}
    It's not too much of a stretch to guess that for any positive integer $n$ that
    \begin{equation*}
        (f g)^{(n)} = \sum_{k=0}^n \binom{n}{k} f^{(k)} g^{(n-k)}.
    \end{equation*}
    The pattern of the first three derivatives is pretty compelling, but we 
    need a proof. Your task is it proof it. To do this, you'll need a few
    facts about the binomial coefficients. These facts are 
    \begin{itemize}
        \item For all $n \in \integers_{\geq 0}$, we have 
              $ \binom{n}{-1} = 0 $.

        \item For all $n \in \integers_{\geq 0}$, we have
                $\binom{n+1}{n} = 0$.

        \item For all integers $n$ and $k$, we have 
        $\binom{n+1}{k} = \binom{n}{k-1} + \binom{n}{k}$.

    \end{itemize}
 Finally, you'll likely need to use the underappreciated
 facts that $\displaystyle \sum_{k=0}^n  a_{k+1} =  \sum_{k=1}^{n+1}  a_k$
 and that $\displaystyle \sum_{k=0}^n  a_{k-1} =  \sum_{k=-1}^{n-1}  a_k$



\end{questions}
\end{document}