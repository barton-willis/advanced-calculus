\documentclass[12pt,fleqn,answers]{exam}
\usepackage{pifont}
\usepackage{dingbat}
\usepackage{amsmath,amssymb}
\usepackage{fleqn}
\usepackage{epsfig}
%\usepackage{mathptm}
%\usepackage{euler}
\usepackage{bbding}

\addpoints
\boxedpoints
\pointsinmargin
\pointname{pts}

\usepackage[activate={true,nocompatibility},final,tracking=true,kerning=true,factor=1100,stretch=10,shrink=10]{microtype}
\usepackage[american]{babel}
%\usepackage[T1]{fontenc}
\usepackage{fourier}
\usepackage{isomath}
\usepackage{upgreek,amsmath}
\usepackage{amssymb}
%\usepackage[euler-digits,euler-hat-accent,T1]{eulervm}

\newcommand{\dotprod}{\, {\scriptzcriptztyle
    \stackrel{\bullet}{{}}}\,}

\newcommand{\reals}{\mathbf{R}}
\newcommand{\complex}{\mathbf{C}}
\newcommand{\dom}{\mbox{dom}}
\newcommand{\cover}{{\mathcal C}}
\newcommand{\integers}{\mathbf{Z}}
\newcommand{\vi}{\, \mathbf{i}}
\newcommand{\vj}{\, \mathbf{j}}
\newcommand{\vk}{\, \mathbf{k}}
\newcommand{\bi}{\, \mathbf{i}}
\newcommand{\bj}{\, \mathbf{j}}
\newcommand{\bk}{\, \mathbf{k}}
\DeclareMathOperator{\Arg}{\mathrm{Arg}}
\DeclareMathOperator{\Ln}{\mathrm{Ln}}
\newcommand{\imag}{\, \mathrm{i}}
\usepackage{amsthm}
\newtheorem{Rubbish}{Theorem}
\usepackage{graphicx}

%\usepackage{tgschola} %to look retro
\newenvironment{mypar}[2]
  {\begin{list}{}%
    {\setlength\leftmargin{#1}
    \setlength\rightmargin{#2}}
    \item[]}
  {\end{list}}
  
\newcommand{\quiz}{1}
\newcommand{\term}{Fall}

\usepackage{xspace}
\makeatletter
\DeclareRobustCommand{\maybefakesc}[1]{%
  \ifnum\pdfstrcmp{\f@series}{\bfdefault}=\z@
    {\fontsize{\dimexpr0.8\dimexpr\f@size pt\relax}{0}\selectfont\uppercase{#1}}%
  \else
    \textsc{#1}%
  \fi
}
\newcommand\AM{\,\maybefakesc{am}\xspace}
\newcommand\PM{\,\maybefakesc{pm}\xspace}

\begin{document}
\large
\vspace{0.1in}
\noindent\makebox[3.0truein][l]{ \textbf{MATH 460}}
{\bf Name:}  \\
\noindent \makebox[3.0truein][l]{\textbf{Homework \quiz, \term \/ \the\year}}
%{\bf Row:}\hrulefill\
\vspace{0.1in}

\noindent  Homework \quiz\/  has questions 1 through  \numquestions \/ with a total 
of  \numpoints\/  points. When I record your grade, I will scale it to twenty points. 
For details of the grading scheme for this assignment, please see the section 
`Grading rubric' of our syllabus.

Revise, proofread, revise again (and again), \emph{neatly} 
hand write your solutions, digitize your work, and 
up load the converted pdf of your work to Canvas.  
This work is due \textbf{Saturday 26 August} at 11:59 \PM.

\vspace{0.1in}




\begin{questions}

\question[10] For the statement  \(\left(\exists M \in \reals  \right)
\left(\forall x \in \reals_{\geq 0} \right) \left(\frac{5 x}{x+1} \leq M \right)\), explain why

\begin{proof} Choose $M = \frac{5 x}{x+1}$. Let  $x \in \reals_{\geq 0}$. We have
\begin{align*}
 \left[\frac{5 x}{x+1} \leq M  \right] &\equiv
 \left[\frac{5 x}{x+1} \leq  \frac{5 x}{x+1}  \right],  &\mbox{(substitution for $M$)} \\
 &\equiv \mbox{True}. &\mbox{(syntactic equality)}. 
\end{align*}
\end{proof}
is \emph{abject rubbish}.

\question[10] Write a correct proof of \(\left(\exists M \in \reals\right)
\left(\forall x \in \reals_{\geq 0} \right) \left(\frac{5 x}{x+1} \leq  M \right)\).

\question [10]  Without explicitly using negation (either $\lnot$ or anything equivalent to negation), write the negation of the statement
\begin{equation*}
\left(\exists M \in  \reals_{< 5} \right)
\left(\forall x \in  \reals_{\geq 0} \right) \left(\frac{5 x}{x+1} <  M \right).
\end{equation*}
Unlike the previous questions, the number $M$ in this question must be \emph{less} than five. Also,
the final inequality is now a strict inequality (equality is not allowed). These differences are \emph{not} typos.

\question [10]  Show that the statement 
\begin{equation*}
\left(\exists M \in  \reals_{< 5} \right)
\left(\forall x \in  \reals_{\geq 0} \right) \left(\frac{5 x}{x+1}  < M \right).
\end{equation*}
is \emph{false} by showing that its negation is true.


 
\end{questions}
\end{document}
\
\question [10] Show that for every real number $b$ there is a real number $m$ such that
\begin{equation*}
 \{x \in \reals \mid x^2-b^2 = m (x-b) \} = \{b\}.
 \end{equation*}
 \textbf{Hint} Solving (either by factoring or by quadratic formula) $ x^2-b^2 = m (x-b)$ for $x$ gives the
 solution set $\{b,m-b\}$.
 
\question [10] Without explicitly using negation ($\lnot$), write the negation of the statement
\begin{equation*}
  \left(\exists M \in \reals\right)
  \left(\forall x \in \reals_{\neq 0} \right)
  \left(\frac{1}{x^2} < M \right).
\end{equation*}

\question[10] Show that the statement 
\begin{equation*}
  \left(\exists M \in \reals\right)
  \left(\forall x \in \reals_{\neq 0} \right)
  \left(\frac{1}{x^2} < M \right).
\end{equation*} 
is false. To do this, show that its negation is true. This shows that the function $x \in \reals_{\neq 0} \mapsto \frac{1}{x^2}$ is
not bounded above.


 
 This shows that there is exactly one line of the form $y-b^2 = m (x-b)$ that intersects the curve $y=x^2$ in
 exactly one point.  The line  $y-b^2 = m (x-b)$ is the tangent line. But in general``touching at one point'' is \emph{not}
 a characteristic of a tangent line.
 
\question [10] Find functions $P,Q \in \integers \mapsto \{\mbox{false}, \mbox{true}\}$ that makes
the statement
\begin{equation*}
  \left(\forall k \in \integers\right) \left (P(k) \lor Q(k) \right) \equiv 
  \left(\forall k \in \integers\right) \left (P(k) \right)  \lor 
  \left(\forall k \in \integers\right) \left (Q(k) \right)  
\end{equation*}
false.  This shows that the universal qualifier does not distribute over the disjunction. 
\textbf{Hint:} Choose $P = k \in \integers \mapsto \left[x \mbox{is even}\right]$
and $Q = k \in \integers \mapsto \left[x \mbox{ is odd}\right]$. Explain why this choice
shows that the given statement is false.



\end{questions}
\end{document}