\documentclass[12pt,fleqn,answers]{exam}
\usepackage{pifont}
\usepackage{dingbat}
\usepackage{amsmath,amssymb}
\usepackage{fleqn}
\usepackage{epsfig}
%\usepackage{mathptm}
%\usepackage{euler}
\usepackage{bbding}

\addpoints
\boxedpoints
\pointsinmargin
\pointname{pts}

\usepackage[activate={true,nocompatibility},final,tracking=true,kerning=true,factor=1100,stretch=10,shrink=10]{microtype}
\usepackage[american]{babel}
%\usepackage[T1]{fontenc}
\usepackage{fourier}
\usepackage{isomath}
\usepackage{upgreek,amsmath}
\usepackage{amssymb}
%\usepackage[euler-digits,euler-hat-accent,T1]{eulervm}

\newcommand{\dotprod}{\, {\scriptzcriptztyle
    \stackrel{\bullet}{{}}}\,}

\newcommand{\reals}{\mathbf{R}}
\newcommand{\complex}{\mathbf{C}}
\newcommand{\dom}{\mbox{dom}}
\newcommand{\cover}{{\mathcal C}}
\newcommand{\true}{\mathrm{True}}
\newcommand{\false}{\mathrm{False}}
\newcommand{\integers}{\mathbf{Z}}
\newcommand{\vi}{\, \mathbf{i}}
\newcommand{\vj}{\, \mathbf{j}}
\newcommand{\vk}{\, \mathbf{k}}
\newcommand{\bi}{\, \mathbf{i}}
\newcommand{\bj}{\, \mathbf{j}}
\newcommand{\bk}{\, \mathbf{k}}
\DeclareMathOperator{\Arg}{\mathrm{Arg}}
\DeclareMathOperator{\Ln}{\mathrm{Ln}}
\newcommand{\imag}{\, \mathrm{i}}
\usepackage{amsthm}
\newtheorem{Rubbish}{Theorem}
\usepackage{graphicx}

%\usepackage{tgschola} %to look retro
\newenvironment{mypar}[2]
  {\begin{list}{}%
    {\setlength\leftmargin{#1}
    \setlength\rightmargin{#2}}
    \item[]}
  {\end{list}}
  
\newcommand{\quiz}{1}
\newcommand{\term}{Fall}

\usepackage{xspace}
\makeatletter
\DeclareRobustCommand{\maybefakesc}[1]{%
  \ifnum\pdfstrcmp{\f@series}{\bfdefault}=\z@
    {\fontsize{\dimexpr0.8\dimexpr\f@size pt\relax}{0}\selectfont\uppercase{#1}}%
  \else
    \textsc{#1}%
  \fi
}
\newcommand\AM{\,\maybefakesc{am}\xspace}
\newcommand\PM{\,\maybefakesc{pm}\xspace}

\usepackage{framed}
\usepackage{xcolor}

% Define light grey shading for solutions
\colorlet{shadecolor}{gray!20} % Set the light grey color
\renewenvironment{solution}
  {\begin{shaded*}} % Start a shaded box
  {\end{shaded*}}   % End the shaded box

\begin{document}
\large
\vspace{0.1in}
\noindent\makebox[3.0truein][l]{ \textbf{MATH 460}}
{\bf Name:}  \\
\noindent \makebox[3.0truein][l]{\textbf{Homework \quiz, \term \/ \the\year}}
%{\bf Row:}\hrulefill\
\vspace{0.1in}

\noindent  Homework \quiz\/  has questions 1 through  \numquestions \/ with a total 
of  \numpoints\/  points. When I record your grade, I will scale it to twenty points. 
For details of the grading scheme for this assignment, please see the section 
`Grading rubric' of our syllabus.

Revise, proofread, revise again (and again), \emph{neatly} 
hand write your solutions, digitize your work, and 
upload the converted PDF of your work to Canvas.  
This work is due \textbf{Saturday 26 August} at 11:59 \PM.

\vspace{0.1in}




\begin{questions}

\question[10] For the statement  \(\left(\exists M \in \reals  \right)
\left(\forall x \in \reals_{\geq 0} \right) \left(\frac{5 x}{x+1} \leq M \right)\), explain why
the following proof is abject rubbish:
\begin{proof} Choose $M = \frac{5 x}{x+1}$. Let  $x \in \reals_{\geq 0}$. We have
\begin{align*}
 \left[\frac{5 x}{x+1} \leq M  \right] &\equiv
 \left[\frac{5 x}{x+1} \leq  \frac{5 x}{x+1}  \right],  &\mbox{(substitution for $M$)} \\
 &\equiv \mbox{True}. &\mbox{(syntactic equality)}. 
\end{align*}  \qedhere
\end{proof}


\begin{solution} The proof is abject rubbish because it violates
  what I call the \emph{left-to-right rule for quantified statements}. 
  Specifically, since $M$ is qualified before $x$,
  the value we choose for $M$ is not allowed to depend on $x$. From the first 
  sentence, ``Choose $M = \frac{5 x}{x+1}$,''   the proof is wrong, and the only
  cure is to start from scratch.
 

  \quad Our work is, however, a \emph{correct} proof of the statement 
  \begin{equation}
  \left(\forall x \in \reals_{\geq 0} \right) \left(\exists M \in \reals  \right)
 \left(\frac{5 x}{x+1} \leq M \right),
\end{equation}
but these two statements are \emph{not} logically equivalent.
 
\end{solution}



\question[10] Write a correct proof of \(\left(\exists M \in \reals\right)
\left(\forall x \in \reals_{\geq 0} \right) \left(\frac{5 x}{x+1} \leq  M \right)\).

\begin{solution} 

\begin{proof} Choose $M = 5$. Let $x \in \reals_{\geq 0}$. 
  We have
  \begin{align*}
     \left[\frac{5 x}{x+1} \leq  M \right]
       &\equiv \left[\frac{5 x}{x+1} \leq  5 \right], &&\text{(substitution for $M$)} \\
       &\equiv \left[5 x \leq  5 (x+1) \right], &&\text{(multiply by $x+1$)} \\
       &\equiv \left[ 0 \leq 5 \right], &&\text{(subtract $5 x$)} \\
       &\equiv \true. \qedhere 
  \end{align*}
\end{proof}
\textbf{Notes:}
\begin{itemize}
\item Since $M$ is qualified before $x$, what we choose for $M$ cannot depend
on $x$. Our choice of $M = 5$ satisfies this requirement.

\item How did I know to choose $M = 5$? One approach is to sketch a graph (try Desmos)
of the equation $y = \frac{5 x}{x+1}$ for $x > 0$. You will see that the maximum of 
$y$ coordinate on the graph is $5$. That's how.

\item Instead of making the choice $M = 5$, we could choose $M$ to 
be any real number that is greater than $5$.

\item We've expressed the proof as a sequence of logical equivalences.
 A more traditional way to write the proof is a sequence of inequalities 
that starts with $\frac{5 x}{x+1}$ and that ends with $5$. Each
inequality must be either $<, \leq$, or $=$. You might like to write the
proof this way. Which is more natural? Which proof is easier to understand? 
Which proof is easier to invent?
Why?

\item In the third line, we multiplied an inequality by a variable 
(specifically $x+1$). We need to be careful with such operations, but
it is justified because $x \in \reals_{\geq 0}$.
\end{itemize}
\end{solution}

\question [10]  Without explicitly using negation (either $\lnot$ or anything equivalent to negation), write the negation of the statement
\begin{equation*}
\left(\exists M \in  \reals_{< 5} \right)
\left(\forall x \in  \reals_{\geq 0} \right) \left(\frac{5 x}{x+1} <  M \right).
\end{equation*}
Unlike the previous questions, the number $M$ in this question must be \emph{less} than five. Also,
the final inequality is now a strict inequality (equality is not allowed). These differences are \emph{not} typos.

\begin{solution} For a refresher course on how to negate qualified
  statements, please see our course quick reference sheet. Using these
  rules, we have:
  \begin{equation*}
    \left(\forall M \in  \reals_{< 5} \right)
    \left(\exists x \in  \reals_{\geq 0} \right) 
    \left(\frac{5 x}{x+1} \geq   M \right).
    \end{equation*}  
\end{solution}

\question [10]  Show that the statement 
\begin{equation*}
\left(\exists M \in  \reals_{< 5} \right)
\left(\forall x \in  \reals_{\geq 0} \right) \left(\frac{5 x}{x+1}  < M \right).
\end{equation*}
is \emph{false} by showing that its negation is true.

\begin{solution}

\begin{proof} We need to show that 
  \begin{equation*}
    \left(\forall M \in  \reals_{< 5} \right)
    \left(\exists x \in  \reals_{\geq 0} \right) 
    \left(\frac{5 x}{x+1} \geq   M \right).
    \end{equation*}
    Let $M \in \reals_{< 5}$. Choose 
  $x = \begin{cases}  0 & M < 0 \\
                      \frac{M}{5-M} & M \geq 0 
  \end{cases}$. Then $x \in \reals_{\geq 0}$ as required.
  We will consider the cases $M < 0$ and $M \geq 0$ separately.
  First, for $M < 0$, we have
  \begin{align*}
    \left[ \frac{5 x}{x+1}  \geq  M \right] &\equiv \left[0 \geq  M \right], \\
                                            &\equiv \true.
  \end{align*}
  And second for $M \geq 0$, we have
  \begin{align*}
    \left[ \frac{5 x}{x+1}  \geq  M \right] 
       &\equiv \left[M \geq  M \right], &&\text{(algebra)} \\
      &\equiv \true. \qedhere
  \end{align*}


\end{proof}
  
\end{solution}
 
\end{questions}

\subsection*{Unassigned questions}

\begin{questions}
\question [10] Show that for every real number $b$ there is a real number $m$ such that
\begin{equation*}
 \{x \in \reals \mid x^2-b^2 = m (x-b) \} = \{b\}.
 \end{equation*}
 This shows that there is exactly one line of the form $y-b^2 = m (x-b)$ that intersects the curve $y=x^2$ in
 exactly one point. Apparently, the line  $y-b^2 = m (x-b)$ is the tangent line. But ``touching at one point'' is \emph{not}
 a requirement of a tangent line.

 \textbf{Hint} Solving (either by factoring or by quadratic formula) 
 $ x^2-b^2 = m (x-b)$ for $x$ gives the
 solution set $\{b,m-b\}$.

 \begin{solution} Our proof can either include or exclude the way we
  discover a correct value for $m$. Either way is correct.  If you can find the value of $m$
  by divine intervention and then write a proof that chooses the value
  of $m$ from thin air, that's fine--a proof doesn't need to explain
  the logic of how we make choices.

  \quad For this solution, let's exclude the method we use for discovering
  the value of $m$ in the body of the proof, but instead include this logic outside the 
  proof. To find $m$, we solve the equation $ x^2-b^2 = m (x-b)$ for $x$.
  The solution(s) are $x =b$ and $x = m-b$.  But the statement says 
  there is only one solution, so we must choose $m = 2 b$, making each solution $x = b$.
  With this choice, our proof is:

  \begin{proof} Let $b \in \reals$. Choose $m = 2 b$. We have
    \begin{align*}
      \{x \in \reals \mid x^2-b^2 = m (x-b) \} &= 
      \{x \in \reals \mid x^2-b^2 = 2 b (x-b) \}, \\
      &= \{x \in \reals \mid x^2 - 2 b x + b^2 = 0\},\\
      &= \{x \in \reals \mid (x-b)^2 = 0\},\\
      &= \{b\}. \qedhere
      \end{align*}
    \end{proof}
\noindent \textbf{Notes:}

\begin{itemize}
\item The problem statement allows $m$ to depend on $b$. That is
because the qualification on $m$ follows the qualification on $b$.

\item Our initial work in determining the value of $m$ shows that the
choice is actually unique. So if you flub and choose $m$ to not equal 
$2 b$, you will not be able to write a proof.

\item Tangent lines are often described as ``touching in one spot,''
but that's not true. The tangent line to $y=\cos(x)$ at the point 
$(x=0,y=1)$ is $y = 1$. But the curves $y=\cos(x)$ and $y=1$ touch in
infinitely many locations.

\end{itemize}
 \end{solution}
 
\question [10] Without explicitly using negation ($\lnot$), write the negation of the statement
\begin{equation*}
  \left(\exists M \in \reals\right)
  \left(\forall x \in \reals_{\neq 0} \right)
  \left(\frac{1}{x^2} < M \right).
\end{equation*}

\begin{solution} We have
  \begin{equation*}
    \left(\forall M \in \reals\right)
    \left(\exists x \in \reals_{\neq 0} \right)
    \left(\frac{1}{x^2} \geq  M \right).
  \end{equation*}
\end{solution}

\question[10] Show that the statement 
\begin{equation*}
  \left(\exists M \in \reals\right)
  \left(\forall x \in \reals_{\neq 0} \right)
  \left(\frac{1}{x^2} < M \right).
\end{equation*} 
is false. To do this, show that its negation is true. This shows that the function $x \in \reals_{\neq 0} \mapsto \frac{1}{x^2}$ is
not bounded above.

\begin{solution}
  \begin{proof} We will show that
    \begin{equation*}
      \left(\forall M \in \reals\right)
      \left(\exists x \in \reals_{\neq 0} \right)
      \left(\frac{1}{x^2} \geq  M \right).
    \end{equation*}
  Let $M \in \reals$. Choose $x = 
    \begin{cases} 1 & M \leq 0 \\ \sqrt{\frac{1}{M}} & M > 0 
    \end{cases}.$ Then $x  \in \reals_{\neq 0}$ as required.
    We consider the cases $M \leq 0$ and $M > 0$ separately.
    For $M \leq 0$, we have
    \begin{equation*}
      \left[\frac{1}{x^2} \geq  M \right] \equiv \left[1 \geq M \right]
      \equiv \text{True}.
    \end{equation*}
    And for $M > 0$, we have
    \begin{equation*}
      \left[\frac{1}{x^2} \geq  M \right] \equiv 
      \left[M \geq M \right]
      \equiv \text{True}.
    \end{equation*}
  \end{proof}
\end{solution}


 
\question [10] Find functions $P,Q \in \integers \mapsto \{\mbox{false}, \mbox{true}\}$ that makes
the statement
\begin{equation*}
  \left(\forall k \in \integers\right) \left (P(k) \lor Q(k) \right) \equiv 
  \left(\forall k \in \integers\right) \left (P(k) \right)  \lor 
  \left(\forall k \in \integers\right) \left (Q(k) \right)  
\end{equation*}
false.  This shows that the universal qualifier does not distribute over 
the disjunction. One notational subtly is the fact that the various
variables $k$ on the left and right sides of this equivalence are
actually different.  This means that the given statement is logically
equivalent to 
\begin{equation*}
  \left(\forall k \in \integers\right) \left (P(k) \lor Q(k) \right) \equiv 
  \left(\forall \ell \in \integers\right) \left (P(\ell) \right)  \lor 
  \left(\forall m \in \integers\right) \left (Q(m) \right).  
\end{equation*}


\textbf{Hint:} Choose $P = k \in \integers \mapsto \left[x \mbox{ is even}\right]$
and $Q = k \in \integers \mapsto \left[x \mbox{ is odd}\right]$. Explain why this choice
shows that the given statement is false.

\begin{solution} 
  \begin{proof} 
    Choose $P = k \in \integers \mapsto \left[x \mbox{ is even}\right]$
    and $Q = k \in \integers \mapsto \left[x \mbox{ is odd}\right]$.
    We will choose integers $k,\ell$, and $m$ such that
    \begin{equation*}
      P(k) \lor Q(k) \neq P(\ell) \lor Q(m).
    \end{equation*}
    Choose $k = 1, \ell = 1$ and $m = 2$. We have
    \begin{equation*}
      \left[P(k) \lor Q(k) \right] \equiv \left[P(1) \lor Q(1)\right]
      \equiv \true.
    \end{equation*}
    But
    \begin{equation*}
      \left[P(\ell) \lor Q(m) \right] \equiv \left[P(1) \lor Q(2)\right]
      \equiv \false.
    \end{equation*}
     \end{proof}
  
 
\end{solution}

\end{questions}
\end{document}