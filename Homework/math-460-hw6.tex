\documentclass[12pt,fleqn,answers]{exam}
\usepackage{pifont}
\usepackage{dingbat}
\usepackage{amsmath,amssymb}
\usepackage{fleqn}
\usepackage{epsfig}
%\usepackage{mathptm}
%\usepackage{euler}
\usepackage{bbding}
\usepackage{url}
\addpoints
\boxedpoints
\pointsinmargin
\pointname{pts}

\usepackage{xcolor}
\usepackage{framed}
\colorlet{shadecolor}{lightgray!15}
\newenvironment{myproof}
  {\begin{shaded}\begin{proof}}
  {\end{proof}\end{shaded}}

\usepackage{amsthm}
\newtheorem{prop}{Proposition}

\usepackage[activate={true,nocompatibility},final,tracking=true,kerning=true,factor=1100,stretch=10,shrink=10]{microtype}
\usepackage[american]{babel}
%\usepackage[T1]{fontenc}
\usepackage{fourier}
\usepackage{isomath}
\usepackage{upgreek,amsmath}
\usepackage{amssymb}
%\usepackage[euler-digits,euler-hat-accent,T1]{eulervm}

\newcommand{\dotprod}{\, {\scriptzcriptztyle
    \stackrel{\bullet}{{}}}\,}

\newcommand{\reals}{\mathbf{R}}
\newcommand{\complex}{\mathbf{C}}
\newcommand{\dom}{\mbox{dom}}
\newcommand{\cover}{{\mathcal C}}
\newcommand{\rat}{\mathbf{Q}}

\newcommand{\curly}[1]{\mathcal #1}
\newcommand{\integers}{\mathbf{Z}}
\newcommand{\vi}{\, \mathbf{i}}
\newcommand{\vj}{\, \mathbf{j}}
\newcommand{\vk}{\, \mathbf{k}}
\newcommand{\bi}{\, \mathbf{i}}
\newcommand{\bj}{\, \mathbf{j}}
\newcommand{\bk}{\, \mathbf{k}}
\DeclareMathOperator{\Arg}{\mathrm{Arg}}
\DeclareMathOperator{\Ln}{\mathrm{Ln}}
\newcommand{\imag}{\, \mathrm{i}}
\newcommand{\range}{\mathrm{range}}
\newcommand{\true}{\mathrm{True}}

\usepackage{enumerate}
\newenvironment{alphalist}{
  \begin{enumerate}[(a)]
    \addtolength{\itemsep}{-0.5\itemsep}}
  {\end{enumerate}}
 
\usepackage{amsthm}
\newtheorem{Rubbish}{Theorem}
\usepackage{graphicx}

%\usepackage{tgschola} %to look retro
\newenvironment{mypar}[2]
  {\begin{list}{}%
    {\setlength\leftmargin{#1}
    \setlength\rightmargin{#2}}
    \item[]}
  {\end{list}}
  
\newcommand{\quiz}{6}
\newcommand{\term}{Fall}

\usepackage{xspace}
\makeatletter
\DeclareRobustCommand{\maybefakesc}[1]{%
  \ifnum\pdfstrcmp{\f@series}{\bfdefault}=\z@
    {\fontsize{\dimexpr0.8\dimexpr\f@size pt\relax}{0}\selectfont\uppercase{#1}}%
  \else
    \textsc{#1}%
  \fi
}
\newcommand\AM{\,\maybefakesc{am}\xspace}
\newcommand\PM{\,\maybefakesc{pm}\xspace}

\begin{document}
\large
\vspace{0.1in}
\noindent\makebox[3.0truein][l]{ \textbf{MATH 460}}
{\bf Name:}  \\
\noindent \makebox[3.0truein][l]{\textbf{Homework \quiz, \term \/ \the\year}}
%{\bf Row:}\hrulefill\
\vspace{0.1in}

\noindent  Homework \quiz\/  has questions 1 through  \numquestions \/ with a total 
of  \numpoints\/  points. When I record your grade, I will scale it to twenty points. 
For details of the grading scheme for this assignment, please see the section 
`Grading rubric' of our syllabus.

Revise, proofread, revise again (and again), typeset your work using Overleaf, and 
upload the converted pdf of your compiled file work to Canvas.  
This work is due \textbf{Saturday 7 October} at 11:59 \PM.

\vspace{0.1in}

\noindent Let $A \subset \reals$ and $F \in A \to \reals$.  We say $F$ is subadditive provided
\begin{alphalist}
\item $\left(\forall x,y \in a \right) \left(x+y \in A\right)$
\item $\left(\forall x,y \in a \right) \left(F(x+y) \leq F(x) + F(y)\right)$
\end{alphalist}
\begin{questions}

\question [10] Either prove that the function $x \in \reals \mapsto x^2$ is subadditive
or prove that it is not subadditive.

\question [10] The square root function is subadditive. A variant of this fact is that
\begin{equation*}
    \left(\forall x,y \in \reals_{\geq 0} \right)
    \left( | \sqrt{x} - \sqrt{y} | \leq \sqrt{|x-y|} \right).
\end{equation*}
For $y=1$, graphically verify that for all $x \in \reals_{\geq 0}$, we have
$ | \sqrt{x} - \sqrt{1} | \leq \sqrt{|x-1|}$. Include your graph in your work.


\question [10] Let $F$ be a sequence and suppose that 
\begin{alphalist}
\item $F$ converges,
\item $\range(F) \subset [0,\infty)$.
\end{alphalist}
Show that $\sqrt{F}$ converges. To do this, use the inequality from the previous question.

\question[10] Is the set of rational numbers open? Justify your answer.

\question[10] Is the set rational numbers closed? Justify your answer.

\question [10] Is the set $(0,1) \cup \{\uppi\}$ open? Justify your answer.

\question [10] Is the set $\reals \setminus \{\uppi\}$ open? Justify your answer.

\end{questions}
\end{document}