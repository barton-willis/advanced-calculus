\documentclass[12pt,fleqn,answers]{exam}
\usepackage{pifont}
\usepackage{dingbat}
\usepackage{amsmath,amssymb}
\usepackage{fleqn}
\usepackage{epsfig}
%\usepackage{mathptm}
%\usepackage{euler}
\usepackage{bbding}
\usepackage{url}
\addpoints
\boxedpoints
\pointsinmargin
\pointname{pts}

\usepackage{xcolor}
\usepackage{framed}
\colorlet{shadecolor}{lightgray!15}
\newenvironment{myproof}
  {\begin{shaded}\begin{proof}}
  {\end{proof}\end{shaded}}

\usepackage{amsthm}
\newtheorem{prop}{Proposition}

\usepackage[activate={true,nocompatibility},final,tracking=true,kerning=true,factor=1100,stretch=10,shrink=10]{microtype}
\usepackage[american]{babel}
%\usepackage[T1]{fontenc}
\usepackage{fourier}
\usepackage{isomath}
\usepackage{upgreek,amsmath}
\usepackage{amssymb}
%\usepackage[euler-digits,euler-hat-accent,T1]{eulervm}

\newcommand{\dotprod}{\, {\scriptzcriptztyle
    \stackrel{\bullet}{{}}}\,}

\newcommand{\reals}{\mathbf{R}}
\newcommand{\complex}{\mathbf{C}}
\newcommand{\dom}{\mbox{dom}}
\newcommand{\cover}{{\mathcal C}}
\newcommand{\rat}{\mathbf{Q}}

\newcommand{\curly}[1]{\mathcal #1}
\newcommand{\integers}{\mathbf{Z}}
\newcommand{\vi}{\, \mathbf{i}}
\newcommand{\vj}{\, \mathbf{j}}
\newcommand{\vk}{\, \mathbf{k}}
\newcommand{\bi}{\, \mathbf{i}}
\newcommand{\bj}{\, \mathbf{j}}
\newcommand{\bk}{\, \mathbf{k}}
\DeclareMathOperator{\Arg}{\mathrm{Arg}}
\DeclareMathOperator{\Ln}{\mathrm{Ln}}
\newcommand{\imag}{\, \mathrm{i}}

\newcommand{\true}{\mathrm{True}}

\usepackage{amsthm}
\newtheorem{Rubbish}{Theorem}
\usepackage{graphicx}

%\usepackage{tgschola} %to look retro
\newenvironment{mypar}[2]
  {\begin{list}{}%
    {\setlength\leftmargin{#1}
    \setlength\rightmargin{#2}}
    \item[]}
  {\end{list}}
  
\newcommand{\quiz}{4}
\newcommand{\term}{Fall}

\usepackage{xspace}
\makeatletter
\DeclareRobustCommand{\maybefakesc}[1]{%
  \ifnum\pdfstrcmp{\f@series}{\bfdefault}=\z@
    {\fontsize{\dimexpr0.8\dimexpr\f@size pt\relax}{0}\selectfont\uppercase{#1}}%
  \else
    \textsc{#1}%
  \fi
}
\newcommand\AM{\,\maybefakesc{am}\xspace}
\newcommand\PM{\,\maybefakesc{pm}\xspace}

\begin{document}
\large
\vspace{0.1in}
\noindent\makebox[3.0truein][l]{ \textbf{MATH 460}}
{\bf Name:}  \\
\noindent \makebox[3.0truein][l]{\textbf{Homework \quiz, \term \/ \the\year}}
%{\bf Row:}\hrulefill\
\vspace{0.1in}

\noindent  Homework \quiz\/  has questions 1 through  \numquestions \/ with a total 
of  \numpoints\/  points. When I record your grade, I will scale it to twenty points. 
For details of the grading scheme for this assignment, please see the section 
`Grading rubric' of our syllabus.

Revise, proofread, revise again (and again), typeset your work using Overleaf, and 
upload the converted pdf of your compiled file work to Canvas.  
This work is due \textbf{Saturday 23 September} at 11:59 \PM.

\vspace{0.1in}

\begin{questions}

\question [10] Show that $\left(\forall a,b \in \reals \right)
\left(\max(a,b) = \frac{a+b+|a-b|}{2}\right)$. One approach is 
to separately consider the cases $a<b$ and $a \geq b$.

\begin{solution}

\end{solution}

\question [10] Consider the proposition

\begin{prop} For all $a,b \in \reals$, we have $ a < b \implies a^2 < b^2 $. \end{prop}

If the proposition is true, prove it; if not prove that its negation is true.


\begin{solution}

\end{solution}



\question [10] Let $\phi \in \integers_{\geq 0} \mapsto \integers_{\geq 0} $ be strictly increasing. This means that for all
$k \in \integers_{\geq 0}$, we have $\phi_{k+1} >  \phi_k$.  Show that for all $k \in \integers_{\geq 0}$, we have $\phi_k \geq k$.

\quad To do this, you will need to use induction.  The predicate is
\begin{equation*}
   P = k \in \integers_{\geq 0} \mapsto \left[\phi_k \geq k \right].
\end{equation*}
The base case is $P(0)$, or  equivalently $\phi_0 \geq 0$.   Once you prove the base case, you need to prove that for all $k \in \integers_{\geq 0}$
that $P(k) \implies P(k+1)$; equivalently $\left(\phi_k \geq k \right) \implies \left(\phi_{k+1} \geq k+1 \right)$.  
You might like to use the fact that for integers $k$ and $\ell$, that $ k > \ell \implies k \geq \ell+1$.

\begin{solution}

\end{solution}

\question [10] Show that the sequence 
    $k \in \integers_{\geq 0} \mapsto \frac{1}{k+1}$.
converges.

\begin{solution}

\end{solution}
    

\question [10] Show that the sequence 
$k \in \integers_{\geq 0} \mapsto \begin{cases}
k! & k <  10^9 \\ \frac{1}{k+1} & k \geq   10^9 \end{cases}$ converges.

\begin{solution}

\end{solution}



\question [10] Let $f$ and $g$ be convergent sequences. Show that the sequence
\begin{equation*}
   k \in \integers_{\geq 0} \mapsto \max(f_k, g_k)
\end{equation*}
converges.  You may use the facts that (a) a sum or difference of convergent
sequences converges (b) the absolute value of a convergent sequence converges.


\end{questions}
\end{document}