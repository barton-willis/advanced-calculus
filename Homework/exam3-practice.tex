\documentclass[12pt,answers,fleqn]{exam}
\usepackage{pifont}
\usepackage{dingbat}
\usepackage{amsmath,amsthm}
\usepackage{fleqn}
\usepackage{epsfig}
%\usepackage{mathptm}
\usepackage{amssymb,enumerate}
\usepackage{fourier}
\newcommand{\reals}{\mathbf{R}}
\newcommand{\dom}{\mbox{dom}}
\newcommand{\cover}{{\mathcal C}}
\newcommand{\integers}{\mathbf{Z}}
\newcommand{\lp}{\mathrm{LP}}
\newcommand{\ball}{\mathrm{ball}} 
\newenvironment{numlist}{
  \begin{enumerate}[(1)]
    \addtolength{\itemsep}{-1.0\itemsep}}
  {\end{enumerate}}

\newenvironment{alphalist}{
  \begin{enumerate}[(a)]
    \addtolength{\itemsep}{-1.0\itemsep}}
  {\end{enumerate}}

\newenvironment{handlist}{
  \begin{enumerate}[\leftthumbsup]
    \addtolength{\itemsep}{-1.0\itemsep}}
  {\end{enumerate}}

\addpoints
\boxedpoints
\pointsinmargin
\pointname{pts}

\newcommand{\dotprod}{\, {\scriptzcriptztyle \stackrel{\bullet}{{}}}\,}
\begin{document}
\large
\vspace{0.1in}
\noindent\makebox[3.0truein][l]{{\bf Advanced Calculus, Fall \the\year}}
{\bf Name:}\hrulefill\
\noindent \makebox[3.0truein][l]{\bf Practice Exam 3}
{\bf Row and Seat:}\hrulefill\
%\normalsize


\begin{questions}

\question [10] \label{q1} Define the \emph{derivative} as the limit of a \emph{Newton
quotient}.

\begin{solution}
Let \(F  \in  \reals \to \reals\).  We say a function $F$ is \emph{differentiable at} $a$ provided (i) \(a \in
\dom(F)\), (ii) $a \in \lp(\dom(F))$, (iii)   and \(\displaystyle \lim_{x \to a} \frac{F(x) -
  F(a)}{x-a}\) is a real number.
\end{solution}

\question [10] Show that the function $x \in (-\infty, 0) \cup {5} \mapsto x$ is not differentiable at $5$.
\begin{solution}
Since $5 \notin \lp((-\infty, 0) \cup {5})$, the given function isn't differentiable at 5.
\end{solution}

\question [10] Use the definition you gave in question \ref{q1} to find
the derivative of \mbox{\(x \in \reals \mapsto x^2 + x\)} at 2.

\begin{solution}
 We have
\begin{align*}
  \lim_{x \to 2} \frac{x^2 + x - 6}{x-2} &= \lim_{x \to 2}
  \frac{(x-2)(x+3)}{x-2}, \\
   &= \lim_{x \to 2} (x+3), \\
   &= 5.
\end{align*}
\end{solution}

\question [10] Use the definition you gave in question \ref{q1} to find
the derivative of \mbox{\(x \in \reals \mapsto x^2 + x\)} at \(a\), where
\(a\) is any real number.

\question [10] Use the definition you gave in question \ref{q1} to find
the derivative of \mbox{\(x \in \reals \mapsto \sqrt{x}\)} at \(3\).


\question [10] Use the QRS definition of uniformly continuity to show that  that \(x \in [-1,1] \mapsto x^2 \)
is {\em uniformly continuous} on its domain.

\begin{solution}
\begin{proof}
Let \(\varepsilon\) be a positive number, and let \(x,y \in
[-1,1]\). If \(|x - y| < \varepsilon/2\), we have
\begin{align*}
  |x^2 - y^2| &= |x - y| |x + y|, \\
              &< \frac{\varepsilon}{2} \left(|x| + |y|\right), \\
              &= \frac{\varepsilon}{2} \left(1 + 1 \right),\\
              &= \varepsilon.
\end{align*}
\end{proof}
\end{solution}

\question [10]  Use the undefinition of uniformly continuity to
show that the function \(x \in \reals \mapsto 8 x^2\) is
not uniformly continuous on its domain.

\begin{solution}
\begin{proof} For every positive number \(\delta\) we'll find \(x,y
\in \reals\) such that \(|x - y| < \delta\) yet 
\(|8 x^2  -  8 y^2 | > 1/2\).


Choose \(x = \frac{1}{8 \delta} - \frac{\delta}{4}\) and
  \(y = \frac{1}{8 \delta} + \frac{\delta}{4}\).  Then
\(\displaystyle
   |x - y| = \frac{\delta}{2} < \delta
\). Further
\begin{align*}
  |8 x^2 - 8 y^2| &= 8 |x - y||x + y|, \\
                  &< 8  \times  \frac{\delta}{2} \times  \frac{1}{4 \delta}, \\
                  &= 1, \\
                  &> \frac{1}{2}.
\end{align*}
\end{proof}
\end{solution}


\question [10] Show that the function \(x \in \reals \mapsto \begin{cases} x
  \cos(1/x) & x \neq 0 \\ 0 & x = 0 \end{cases}\) is {\em continuous\/}
at 0. You may use the fact that \(|\cos(x)| \leq 1\) for all real
\(x\) without proving it.


\begin{solution}
  \begin{proof}
Let \(F\) be the given function, and let 
\(\varepsilon\) be a positive number.  For \mbox{\(x \in \ball^\prime(0, \varepsilon)\)} we have
\[ 
  |F(x) - F(0)| = |x \cos(1/x) - 0| = |x| |\cos(1/x)| \leq |x| < \varepsilon.
\] 
Also,
\(
  |F(0) - F(0)| = 0 < \varepsilon.
\)
So for all  \(x \in \ball(0, \varepsilon)\) we have
\(|F(x) - F(0)| < \varepsilon\).
\end{proof}
\end{solution}


\question [10] Show that the function \(x \in \reals \mapsto \begin{cases} x^2
  \cos(1/x) & x \neq 0 \\ 0 & x = 0 \end{cases}\) is differentiable
at 0. Depending on your method, the result of the previous question
might be useful.

\begin{solution}
\begin{proof} Define functions \(F\) and \(G\) by
\(\displaystyle
   F(x) = \begin{cases} x^2
  \cos(1/x) & x \neq 0 \\ 0 & x = 0 \end{cases}
\)
and\\
\mbox{\(\displaystyle
   G(x) = \begin{cases} x
  \cos(1/x) & x \neq 0 \\ 0 & x = 0 \end{cases}
\)}.  The function \(G\) is continuous at 0; further we have
\(F(x) = F(0) + (x-0) G(x)\).   This shows that \(F\) is
differentiable at 0.
   \end{proof}
\end{solution}
%\vspace{2.5in}




%\vfill


\question [10] Let \(F : \reals \to \reals\) be continuous at \(a\).
If \(F(a) > 0\), show that there is a positive number \(\delta\)
such that \(F(x) > 0\) for all \(x \in \ball(a, \delta) \cap \dom(F)\).

\question [10] Show that the function \(x \in \reals_{> 0} \mapsto
\frac{1}{x}\) is not uniformly continuous on its domain.

\question [10]  Let \(F : \reals \to \reals\) be differentiable at
\(a\) and suppose \(F^\prime(a) > 0\). Is it true that \(F\) is
increasing on a neighborhood of \(a\)?  If so, prove it.


\question [10] Give an example of a function \(F : [-1,1] \to \reals\)
such that \mbox{\(\mbox{sup} \left(\mbox{range}(F)\right) \not \in
\mbox{range}(F) \)}.

\begin{solution}
$F = x \in [-1,1] \mapsto \begin{cases} x & x \neq 1 \\ 0 & x = 1 \end{cases}$.
\end{solution}
\question [10] Give an example of a function \(F : (-1,1) \to \reals\)
such that \mbox{\(\mbox{sup} \left(\mbox{range}(F)\right) \not \in
\mbox{range}(F) \)} and \(F\) is continuous on \((-1,1)\).
\begin{solution}
$F = x \in (-1,1) \mapsto x$.
\end{solution}


\question Let \(F : \mathbf{R} \to \mathbf{R}\) satisfy
the inequality
\(
   | F(x) - F(y) | \leq |x - y|
\)
for all \(x, y \in \mathbf{R} \).

\begin{parts}

\part [10] Show that \(F\) is {\em continuous at zero}.

\part [10] Show that \(F\) is {\em uniformly continuous} on \(\reals\).




\end{parts}


\question [10] Show that \(x \in \reals \mapsto x^3\) is continuous at 10.

\begin{solution}
  \begin{proof}
Let \(\varepsilon\) be a positive number. Define
\(\delta = \min \left(1, \frac{\varepsilon}{331} \right) \).  For
\(x \in \ball(10; \delta)\), we have \(9 \leq x \leq 11\). Further, we have
\begin{align*}
  |x^3 - 10^3| &= |x - 10| |x^2 + 10 x + 100|, \\
               &< |x - 10| \left(|x|^2 + 10 |x| + 100 \right), \\
               &\leq |x-10| \left(11^2 + 110 + 100 \right), \\
               &= 331 \, |x-10|, \\
               &\leq \varepsilon.
\end{align*}
\end{proof}
\end{solution}

\question [10] \label{q0} Show that the function with signature  \(F : \reals \to \reals\) 
and formula \(F(x) = \begin{cases} \sin \left(\frac{1}{x} \right) & x \neq 0 \\
                                   0  & x = 0 \end{cases} \)
is not continuous at 0.

\begin{solution}
  \begin{proof}
 We'll show that for every positive number \(\delta\)
there is \(x \in \ball(0; \delta)\) such that \( |F(x) - F(0)| >
\frac{1}{2}\).
Let \(\delta\) be a positive real number.  There is an integer \(n\)
such that \(\displaystyle 0 < \frac{2}{(n+1) \pi} < \delta\). Further
we have
\[
  \left|F \left (\frac{2}{(n+1) \pi} \right) - F(0) \right | = |1 - 0| > \frac{1}{2}.
\]
  \end{proof}
\end{solution}

\question [10] Either prove or disprove:  Let \(F, G : \reals \to
\reals\), and let \(a \in \mbox{dom}(F G)\).  
If \(F G\) is continuous at \(a\), then both \(F\) and \(G\) are
continuous at \(a\).

\begin{solution}
The statement is false. Let \(F(x) = \begin{cases} \sin \left(\frac{1}{x} \right) & x \neq 0 \\
                                   0  & x = 0 \end{cases} \) and let \(G(x) = x\).
In question \ref{q1}, we showed that \(F G\) is continuous at 0, but in question \ref{q0}, we
showed that \(F\) is \emph{not} continuous at 0.


\end{solution}

\question [10] Let  \(F : \reals  \to \reals\) be continuous at
\(a\). Show that \(|F|\) is continuous at \(a\). 

\begin{solution}
  \begin{proof}
 Let \(\varepsilon\) be a positive number.  Since
\(F\) is continuous at \(a\), there is a positive number \(\delta\)
such that for all \(x \in \ball(a,  \delta) \cap \mbox{dom}(F)\), we have
\[
  \left|F(x) - F(a) \right| < \varepsilon.
\]
For all \(x \in \ball(a, \delta) \cap \mbox{dom}(F)\), we have
\[
 \big \vert |F(x)| - |F(a)|  \big \vert \leq  \left|F(x) - F(a) \right| < \varepsilon.
\] 
Therefore \(|F|\) is continuous at \(a\).
  \end{proof}
\end{solution}

\question [10] Use the inequality
\(
   | \sqrt{a} - \sqrt{b} | \leq \sqrt{|a -b|},\mbox{ for } a,b > 0
\)
to show that the square root function is uniformly continuous on
\([0,\infty)\).

\begin{solution}
  \begin{proof}
 Let \(\varepsilon\) be a positive number.  For
\(x,y \in  [0,\infty)\) with \(|x - y| < \varepsilon^2\), we have
\(
  \big \vert \sqrt{x} - \sqrt{y} \big \vert \leq \sqrt{|x - y|} < \varepsilon.
\)
  \end{proof}
\end{solution}

\question [10] Show that the function \(x \in \reals \mapsto x^2\) is not
uniformly continuous on \(\reals\).

\begin{solution}
 Let \(\delta\) be a positive number.  Define
\(x = \frac{1}{\delta} + \frac{\delta}{4}\) and \(y = \frac{1}{\delta} -
\frac{\delta}{4}\). Then \(|x - y| < \delta\); however,
\[
  |x^2 - y^2| = |x - y| |x + y| = \frac{\delta}{2} \,\, \frac{2}{\delta} = 1.
\]

The definition of uniform continuity: Let \(F : E \to \reals\) . We say
\(F\) is \emph{uniformly} continuous on \(E\) provided

\begin{quote}
\begin{alphalist}
\item For \emph{every} \(\varepsilon > 0\)
\item \emph{there is} a \(\delta > 0\)
\item for \emph{all} \(x,y \in \dom(F)\) such that \(|x - y| < \delta\)
\item we have \(|F(x) - F(y) | < \varepsilon\). 
\end{alphalist}
\end{quote}

And the ``undefinition'' of uniform continuity:
\begin{quote}
\begin{alphalist}
\item For \emph{some} \(\varepsilon > 0\)
\item and \emph{every} \(\delta > 0\)
\item \emph{there are}  \(x,y \in \dom(F)\) and \(|x - y| < \delta\)
\item such that \(|F(x) - F(y) | > \varepsilon\). 
\end{alphalist}
\end{quote}


\end{solution}

\question [10] Show that \(x \in \reals  \mapsto  x^2 |x|\) is differentiable at 0. (The
absolute value function isn't differentiable at 0, so the product rule \emph{isn't} an
option!)


\question [10] Use the MVT to show that for all \(x,y \in \reals\), we
have
\(
  | \cos(x) - \cos(y) | \leq |x - y|.
\)
You may use the facts (i) \(\cos^\prime = \sin\) and (ii) \(|\sin(x)|
\leq 1\) for all real \(x\).


\end{questions}
\end{document}

\question  Write the correct spelling of the word I usually write as
``sim--scribble-scribble-scribble'' (one point bonus).

\textbf{Solution} A dictionary spelling is ``similarly.''  Creative
folks sometimes spell it with a double `m.'  Really bad spellers
can make a true mess: ``simiarly'' or worse.

\question [10] Let \(F : \reals \to \reals\) be the \(F(x)  = |x|\).
Either find an open set \(\mathcal{G}\) such that \(F({\mathcal G})\) is 
not open, or prove that \(F({\mathcal G})\) is always open
when  \(\mathcal{G}\) is open. If you choose to give an example of
an open set \(\mathcal{G}\) such that  \(F({\mathcal G})\) is 
not open, you needn't prove all your assertions, but you must 
clearly define \(\mathcal{G}\) and you must give a correct
value for  \(F({\mathcal G})\). \textbf{Notice:} The question does
\textbf{not} say open relative to the range of \(F\).

\textbf{Solution} There are lots of answers--I think any open interval
such that 0 is an interior point will work; for example
\(
  F(\reals) = [0,\infty)
\).  A more popular answer was the interval \((-1,1)\); we have
\(F(-1,1) = [0,1)\).


\question [10] Show that the function with signature \(F : \reals \to \reals\) 
and formula \(F(x) = x^3\) is continuous at 10.

\question [10] Use the inequality
\(
   | \sqrt{a} - \sqrt{b} | \leq \sqrt{|a -b|},\mbox{ for } a,b > 0
\)
to show that the function
\(\displaystyle
   F(x) = \sqrt{1+x}, \,\, \mbox{dom}(F) = \mathbf{R}
\)
is continuous at 1.


\question [10] Show that 
\(
   | \sqrt{a} - \sqrt{b} | \leq \sqrt{|a -b|},
\)
for all positive numbers \(a\) and \(b\).

\question [10] Let \(F : \mathbf{R} \to \mathbf{R}\) satisfy
the inequality
\(
   | F(x) - F(y) | < |x - y|
\)
for all \(x, y \in \mathbf{R} \).

\begin{parts}

\part [10] Show that \(F\) is {\em continuous at zero}.

\part [10] Show that \(F\) is {\em uniformly continuous} on \(\reals\).

\vspace*{1.5in}



\end{parts}\(
   F(x) = x^2, \,\, \mbox{dom}(F) = (-1,1).
\)



\question [10] Show that the function with signature \(F : \reals \to \reals\) 
and formula \(F(x) = x (x-1)\) is continuous at 2.

\vspace*{2.0in}



\question [10] Define the {\em derivative\/} as the limit of a {\em Newton
quotient}.\(
   F(x) = x^2, \,\, \mbox{dom}(F) = (-1,1).
\)



\question [10] Show that the function with signature \(F : \reals \to \reals\) 
and formula \(F(x) = x (x-1)\) is continuous at 2.

\vspace*{2.0in}



%\vspace*{1.5in}


\question [10] Use the Newton quotient to find the derivative
of the function
\mbox{\(
  x  \in \reals \mapsto (x-1) \, |x-1|
\)}
at 1.


%\vspace*{2.5in}

%\newpage

\question [10] Use a Newton quotient to show that the function
\(
   x \in \reals \mapsto x^2 + 5 x + 12
\)
is {\em differentiable at 1}.

%\vspace*{2.5in}

\question [10] Prove the {\em product rule\/} for derivatives.  Use
either of the two definitions for the derivative.

\question [10] Use the MVT to show that for all \(x,y \in \reals\), we
have
\(
  | \cos(x) - \cos(y) | \leq |x - y|.
\)
You may use the facts (i) \(\cos^\prime = \sin\) and (ii) \(|\sin(x)|
\leq 1\) for all real \(x\).


\question [10] Let \(F : \reals \to \reals\) be continuous at \(a\).
If \(F(a) > 0\), show that there is a positive number \(\delta\)
such that \(F(x) > 0\) for all \(x \in \ball(a, \delta) \cap \dom(F)\).

\question [10] Show that the function \(x \in \reals_{> 0} \mapsto
\frac{1}{x}\) is not uniformly continuous on its domain.

\question [10]  Let \(F : \reals \to \reals\) be differentiable at
\(x_o\) and suppose \(F^\prime(x_o) > 0\). Is it true that \(F\) is
increasing on a neighborhood of \(x_o\)?  If so, prove it.

\question [10] Let \(F(x) = \begin{cases} 1 & \,\mbox{ if } x \in
  \mathbf{Q} \\  0 & \,\mbox{ if } x \notin \mathbf{Q} \end{cases}\),
where \(\mathbf{Q}\) is the set of rational numbers.
Show that \(F\) is discontinuous on \(\reals\).  You may use the fact
that for any two real numbers \(a\) and \(b\) with \(a < b\), we have
\mbox{\((a,b) \cap \mathbf{Q} \neq \varnothing\)} and 
\mbox{\((a,b) \cap \left(\reals - \mathbf{Q} \right) \neq \varnothing\)}.
    

\question [10] Let \(F : \reals \to \reals\) be continuous on
\(\reals\) and let \(\mathcal{G}\) be an open subset of \(\reals\).
Show that \(F(\mathcal{G}\) needn't be open relative to 
\(\mbox{range}(F)\).

\question [10] Give an example of a function \(F : [-1,1] \to \reals\)
such that \mbox{\(\mbox{sup} \left(\mbox{range}(F)\right) \not \in
\mbox{range}(F) \)}.

\question [10] Give an example of a function \(F : (-1,1) \to \reals\)
such that \mbox{\(\mbox{sup} \left(\mbox{range}(F)\right) \not \in
\mbox{range}(F) \)} and \(F\) is continuous on \((-1,1)\).

\question [10] Show that the function \(x \mapsto \begin{cases} x^2
  \cos(1/x) & x \neq 0 \\ 0 & x = 0 \end{cases}\) is differentiable
at 0. You may use the fact that \(|cos(x)| \leq 1\) for all real
\(x\) without proving it.

\question [10] Let \(F : \reals \to \reals\) be continuous on
\(\reals\) and let \(\mathcal{G}\) be an open subset of \(\reals\).
Show that \(F(\mathcal{G}\) needn't be open relative to 
\(\mbox{range}(F)\).
\end{questions}
\end{document}