\documentclass[12pt,fleqn,answers]{exam}
\usepackage{pifont}
\usepackage{dingbat}
\usepackage{amsmath,amssymb}
\usepackage{fleqn}
\usepackage{epsfig}
%\usepackage{mathptm}
%\usepackage{euler}
\usepackage{bbding}
\usepackage{url}
\addpoints
\boxedpoints
\pointsinmargin
\pointname{pts}

\usepackage[activate={true,nocompatibility},final,tracking=true,kerning=true,factor=1100,stretch=10,shrink=10]{microtype}
\usepackage[american]{babel}
%\usepackage[T1]{fontenc}
\usepackage{fourier}
\usepackage{isomath}
\usepackage{upgreek,amsmath}
\usepackage{amssymb}
%\usepackage[euler-digits,euler-hat-accent,T1]{eulervm}

\newcommand{\dotprod}{\, {\scriptzcriptztyle
    \stackrel{\bullet}{{}}}\,}

\newcommand{\reals}{\mathbf{R}}
\newcommand{\complex}{\mathbf{C}}
\newcommand{\dom}{\mbox{dom}}
\newcommand{\cover}{{\mathcal C}}
\newcommand{\curly}[1]{\mathcal #1}
\newcommand{\integers}{\mathbf{Z}}
\newcommand{\vi}{\, \mathbf{i}}
\newcommand{\vj}{\, \mathbf{j}}
\newcommand{\vk}{\, \mathbf{k}}
\newcommand{\bi}{\, \mathbf{i}}
\newcommand{\bj}{\, \mathbf{j}}
\newcommand{\bk}{\, \mathbf{k}}
\DeclareMathOperator{\Arg}{\mathrm{Arg}}
\DeclareMathOperator{\Ln}{\mathrm{Ln}}
\newcommand{\imag}{\, \mathrm{i}}
\usepackage{amsthm}
\newtheorem{Rubbish}{Theorem}
\usepackage{graphicx}

%\usepackage{tgschola} %to look retro
\newenvironment{mypar}[2]
  {\begin{list}{}%
    {\setlength\leftmargin{#1}
    \setlength\rightmargin{#2}}
    \item[]}
  {\end{list}}
  
\newcommand{\quiz}{2}
\newcommand{\term}{Fall}

\usepackage{xspace}
\makeatletter
\DeclareRobustCommand{\maybefakesc}[1]{%
  \ifnum\pdfstrcmp{\f@series}{\bfdefault}=\z@
    {\fontsize{\dimexpr0.8\dimexpr\f@size pt\relax}{0}\selectfont\uppercase{#1}}%
  \else
    \textsc{#1}%
  \fi
}
\newcommand\AM{\,\maybefakesc{am}\xspace}
\newcommand\PM{\,\maybefakesc{pm}\xspace}

\begin{document}
\large
\vspace{0.1in}
\noindent\makebox[3.0truein][l]{ \textbf{MATH 460}}
{\bf Name:}  \\
\noindent \makebox[3.0truein][l]{\textbf{Homework \quiz, \term \/ \the\year}}
%{\bf Row:}\hrulefill\
\vspace{0.1in}

\noindent  Homework \quiz\/  has questions 1 through  \numquestions \/ with a total 
of  \numpoints\/  points. When I record your grade, I will scale it to twenty points. 
For details of the grading scheme for this assignment, please see the section 
`Grading rubric' of our syllabus.

Revise, proofread, revise again (and again), typeset your work using Overleaf, and 
upload the converted pdf of your compiled file work to Canvas.  
This work is due \textbf{Saturday 2 September} at 11:59 \PM.

\vspace{0.1in}

For Question 1, I will compile the class work into a single document. To allow me to do this without retyping your work,
copy a link to your Overleaf file (either read only or read and write) here: \url{  } (insert a url for your Overleaf work).


\begin{questions}

%\question For the Week 1 problem that was assigned to you, typeset your proof.

\question Define a set of sets $\curly{C}$ by $\curly{C} = \{\{\uppi\}, \{\uppi, \infty\},
 \{\uppi, \infty, \sqrt{3} \} \}$ Enumerate the members of each of the following sets:

 \begin{parts}

 \part [10] $ \underset{x \in \curly{C}}{\cap} x $

\begin{solution} Set intersection is associative, so intersections of
  two or more sets can be expressed unambiguously without using parenthesis.
  We have
  \begin{align*}
    \underset{x \in \curly{C}}{\cap} x 
     &= \{\uppi\} \cap \{\uppi, \infty\} \cap \{\uppi, \infty, \sqrt{3} \}, \\
     &= \{\uppi\}.
  \end{align*}

\end{solution}
 \part [10] $ \underset{x \in \curly{C}}{\cup} x $
\begin{solution} Set union is associative, so intersections of
  two or more sets can be expressed unambiguously without using parenthesis.
  We have
  \begin{align*}
    \underset{x \in \curly{C}}{\cup} x 
     &= \{\uppi\} \cup \{\uppi, \infty\} \cup \{\uppi, \infty, \sqrt{3} \}, \\
     &= \{\uppi, \infty, \sqrt{3}\}.
  \end{align*}
  The order of the set members doesn't matter, so $ \{\uppi,  \sqrt{3} , \infty\}$,
  for example, is also an answer.

\end{solution}
 \part [10] $ \underset{x \in \curly{C}} {\cup} x  \setminus \underset{x \in \curly{C}}{\cap} x$
\begin{solution} Using the previous two parts, we have
  \begin{equation}
    \underset{x \in \curly{C}} {\cup} x  \setminus \underset{x \in \curly{C}}{\cap} x
       =  \{\uppi, \infty, \sqrt{3}\} \setminus  \{\uppi\} =
       \{\infty, \sqrt{3}\}
  \end{equation}

\end{solution}

    
 \end{parts}

\question [10] Let $X$ and $Y$ be sets and let $F \in X \to Y$. For all subsets $A$ and $B$
of $X$, show that $F(A \cap B) \subset F(A) \cap F(B)$.
\begin{solution}  The conclusion of this proposition is set inclusion,
  so we need to use pick-and-show.

  \begin{proof}  Suppose $y \in F(A \cap B)$. We'll show that
    $y \in  F(A) \cap F(B)$. Since $y \in F(A \cap B)$, there 
    is $x \in A \cap B$ such that $y = F(x)$. But $x \in A \cap B$,
    so $x \in A$ and $x \in  \cap B$. So $y \in F(A)$ and 
    $y \in F(B)$; therefore $y \in F(A) \cap F(B)$.\qedhere   
  \end{proof}

\end{solution}

\question Define $F = x \in \reals \mapsto x^2$.  For these problems, freely assume all College Algebra facts
about real numbers and square roots.
\begin{parts}

    \part [10] Show that $F((-\infty,0)) = (0,\infty)$.
\begin{solution} The conclusion of this proposition is set 
  equality, so we'll proof two set inclusions. Each proof 
  of a set inclusion requires a pick-and-show proof.
  First, we'll show that  $F((-\infty,0)) \subset (0,\infty)$ 

  \begin{proof} Suppose $y \in F((-\infty,0))$. We'll show that 
    $y \in (0,\infty)$.  Since $y \in F((-\infty,0))$, there 
    is $x \in (-\infty,0)$ such that $y = x^2$. But $-\infty < x < 0$
    implies $0 < x^2 < \infty$. So $y \in (0,\infty)$.  \end{proof}

  Second, we'll show that $(0,\infty) \subset F((-\infty,0))$.
  To do this, we'll assume that all nonnegative numbers have a 
  square root.
  \begin{proof} Suppose $y \in (0,\infty)$. We'll show that 
    $y \in F((-\infty,0))$.  We have $-\sqrt{y} \in (-\infty,0)$
    and $F(-\sqrt{y}) = y$, so indeed $y \in F((-\infty,0))$.
  
  \end{proof}
  
\end{solution}
    \part [10] Show that $F \left( \left (0,\infty \right) \right ) = (0,\infty)$.
\begin{solution}

\end{solution}
    \part [10] Find an example of subsets $A$ and $B$ of $\reals$ such that 
    \mbox{$F(A) \cap F(B) \not \subset F(A \cap B)$}
    \begin{solution} Our example is hiding in plain sight. Choose
      $A = (-\infty,0)$ and $B = (0,\infty)$. We have
      \begin{equation*}
        F(A \cap B) = F(\varnothing) = \varnothing.
      \end{equation*}
      And we have
      \begin{equation*}
        F(A) \cap F(B) = (0,\infty) \cap  (0,\infty) =  (0,\infty).
      \end{equation*}
      Certainly, $ (0,\infty) \not \subset \varnothing$.


\end{solution}
\end{parts}
\end{questions}
\end{document}