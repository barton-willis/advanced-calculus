\documentclass[12pt,fleqn]{exam}
\usepackage{pifont}
\usepackage{dingbat}
\usepackage{amsmath,amssymb}
\usepackage{fleqn}
\usepackage{epsfig}
%\usepackage{mathptm}
%\usepackage{euler}
\usepackage{bbding}
\usepackage{url}
\addpoints
\boxedpoints
\pointsinmargin
\pointname{pts}

\usepackage{xcolor}
\usepackage{framed}
\colorlet{shadecolor}{lightgray!15}
\newenvironment{myproof}
  {\begin{shaded}\begin{proof}}
  {\end{proof}\end{shaded}}

\usepackage{amsthm}
\newtheorem{prop}{Proposition}

\usepackage[activate={true,nocompatibility},final,tracking=true,kerning=true,factor=1100,stretch=10,shrink=10]{microtype}
\usepackage[american]{babel}
%\usepackage[T1]{fontenc}
\usepackage{fourier}
\usepackage{isomath}
\usepackage{upgreek,amsmath}
\usepackage{amssymb}
%\usepackage[euler-digits,euler-hat-accent,T1]{eulervm}

\newcommand{\dotprod}{\, {\scriptzcriptztyle
    \stackrel{\bullet}{{}}}\,}

\newcommand{\reals}{\mathbf{R}}
\newcommand{\complex}{\mathbf{C}}
\newcommand{\dom}{\mbox{dom}}
\newcommand{\cover}{{\mathcal C}}
\newcommand{\rat}{\mathbf{Q}}
\newcommand{\range}{\mathrm{range}}
\newcommand{\curly}[1]{\mathcal #1}
\newcommand{\integers}{\mathbf{Z}}
\newcommand{\vi}{\, \mathbf{i}}
\newcommand{\vj}{\, \mathbf{j}}
\newcommand{\vk}{\, \mathbf{k}}
\newcommand{\bi}{\, \mathbf{i}}
\newcommand{\bj}{\, \mathbf{j}}
\newcommand{\bk}{\, \mathbf{k}}
\DeclareMathOperator{\Arg}{\mathrm{Arg}}
\DeclareMathOperator{\Ln}{\mathrm{Ln}}
\newcommand{\imag}{\, \mathrm{i}}

\newcommand{\true}{\mathrm{True}}

\usepackage{amsthm}
\newtheorem{Rubbish}{Theorem}
\usepackage{graphicx}

%\usepackage{tgschola} %to look retro
\newenvironment{mypar}[2]
  {\begin{list}{}%
    {\setlength\leftmargin{#1}
    \setlength\rightmargin{#2}}
    \item[]}
  {\end{list}}
  
\newcommand{\quiz}{5}
\newcommand{\term}{Fall}

\usepackage{xspace}
\makeatletter
\DeclareRobustCommand{\maybefakesc}[1]{%
  \ifnum\pdfstrcmp{\f@series}{\bfdefault}=\z@
    {\fontsize{\dimexpr0.8\dimexpr\f@size pt\relax}{0}\selectfont\uppercase{#1}}%
  \else
    \textsc{#1}%
  \fi
}
\newcommand\AM{\,\maybefakesc{am}\xspace}
\newcommand\PM{\,\maybefakesc{pm}\xspace}

\begin{document}
\large
\vspace{0.1in}
\noindent\makebox[3.0truein][l]{ \textbf{MATH 460}}
{\bf Name:}  \\
\noindent \makebox[3.0truein][l]{\textbf{Homework \quiz, \term \/ \the\year}}
%{\bf Row:}\hrulefill\
\vspace{0.1in}

\noindent  Homework \quiz\/  has questions 1 through  \numquestions \/ with a total 
of  \numpoints\/  points. When I record your grade, I will scale it to twenty points. 
For details of the grading scheme for this assignment, please see the section 
`Grading rubric' of our syllabus.

Revise, proofread, revise again (and again), typeset your work using Overleaf, and 
upload the converted pdf of your compiled file work to Canvas.  
This work is due \textbf{Saturday 30 September} at 11:59 \PM.

\vspace{0.1in}

\begin{questions}


\question [10] Let $F$ and $G$ be sequences each with
domain $\integers_{\geq 0}$ and suppose that 
\begin{equation*}
     \left(\forall k \in \integers_{\geq 0}\right) \left(F_k < G_k\right)
\end{equation*}
Show that if $G$ is bounded above, then $F$ is bounded above.

\begin{solution}

    
\end{solution}

\question [10] Let $F$ and $G$ be sequences each with
domain $\integers_{\geq 0}$ and suppose that 
\begin{equation*}
     \left(\forall k \in \integers_{\geq 0}\right) \left(F_k < G_k\right)
\end{equation*}
Show that if $F$ is not bounded above, then $G$ is not bounded above.

\begin{solution}
\end{solution}
 \question [10] Show that the sequence $n \in \integers_{\geq 1} \mapsto 
 \sum_{k=1}^n \left(\ln(k+1) - \ln(k)\right)$ is not bounded above.
You may freely use the fact that the sequence $k \in \integers_{\geq 1} \mapsto \ln(k+1)$
is not bounded above.
\begin{solution}
\end{solution}
\question[10] Use Desmos to draw  graphs of both $y=1/x$ and $y = \ln(x+1) - \ln(x)$ for 
$1 \leq x \leq 5$. Use your graph to conjecture whether $1/x > \ln(x+1) - \ln(x)$
or $1/x < \ln(x+1) - \ln(x)$ for all $x \in [1, \infty]$. Include
your graph along with your conjecture.

Notice: One way to prove this is to use a MVT (mean value theorem). 
Since we're not there yet, for now, we'll use graphical evidence as 
a fairly convincing argument.

\begin{solution}
% Use  Desmos to create and export a graph, save as a png file 
% (its name will % be something like desmos-graph(1).png. Upload 
% the graph to overleaf, and change the name of "imagefile" to 
% the name of  the desmos graph. You might need to adjust the
% scale from 0.2 to something else.
%\includegraphics[scale=0.2]{imagefile}
\end{solution}

\question[10] Show that the sequence 
$n \in \integers_{\geq 1} \mapsto 
\sum_{k=1}^n \frac{1}{k}$ is not bounded above. Use the result from 
the previous question.

You may freely use the fact: Let $F$ and $G$ be sequences and suppose 
that for all $k \in \integers_{\geq 0}$ we have $F_k < G_k$, then
for all $n \in \integers_{\geq 0}$ we have 
$\sum_{k=1}^n F_k < \sum_{k=1}^n G_k$. 

\begin{solution}

    
\end{solution}
\question[10] Show that the sequence $n \in \integers_{\geq 1} \mapsto 
\sum_{k=1}^n \frac{1}{k}$ does not converge.
\begin{solution}
\end{solution}

\question[10] Let $F$ be a sequence that converges to zero; 
further suppose that $\range(F) \subset [0,\infty]$. Show that $\sqrt{F}$
converges to zero. We define $\sqrt{F} = k \in \integers_{\geq 0} 
\mapsto \sqrt{F_k}$.
\begin{solution}
\end{solution}

\end{questions}
\end{document}