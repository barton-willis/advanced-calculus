\documentclass[fleqn]{beamer}
%\usetheme[height=7mm]{Rochester}
\usetheme{Boadilla} %{Rochester}

\setbeamertemplate{footline}[text line]{%
\parbox{\linewidth}{\vspace*{8pt}\hfill\insertshortauthor\hfill\insertpagenumber}}
\setbeamertemplate{navigation symbols}{}
%\author[BW]{Barton Willis}
\usepackage{amsmath}\usepackage{amsthm}
\usepackage{isomath}
\usepackage{upgreek}
\usepackage{comment,enumerate,xcolor}

\usepackage[english]{babel}
\usepackage[final,babel]{microtype}%\usepackage[dvipsnames]{color}
%\usefonttheme{professionalfonts}
%\usefonttheme{serif}

\newcommand{\reals}{\mathbf{R}}
\newcommand{\complex}{\mathbf{C}}
\newcommand{\integers}{\mathbf{Z}}
\DeclareMathOperator{\range}{range}
\DeclareMathOperator{\domain}{dom}
\DeclareMathOperator{\dom}{dom}
\DeclareMathOperator{\codomain}{codomain}
\DeclareMathOperator{\sspan}{span}
\DeclareMathOperator{\F}{F}
\DeclareMathOperator{\G}{G}
\DeclareMathOperator{\B}{B}
\DeclareMathOperator{\D}{D}
\DeclareMathOperator{\id}{id}
\DeclareMathOperator{\ball}{ball}

\usepackage{graphicx}
\usepackage{color}
\usepackage{amsmath}
\DeclareMathOperator{\nullspace}{nullity}
\theoremstyle{definition}
\newtheorem{mydef}{Definition}
\newtheorem{myqdef}{Quasi-definition}
\newtheorem{myex}{Example}
\newtheorem{myth}{Theorem}
\newtheorem{myaxiom}{Axiom}
\newtheorem{myfact}{Fact}
\newtheorem{metathm}{Meta Theorem}
\newtheorem{Question}{Question}
\newtheorem{Answer}{Answer}
\newtheorem{myproof}{Proof}

\newtheorem{myfakeproof}{Fake Proof}
\newtheorem{mybadformproof}{Bad Form Proof}

\newtheorem{hurestic}{Hurestic}

\newenvironment{alphalist}{
  \vspace{-0.4in}
  \begin{enumerate}[(a)]
    \addtolength{\itemsep}{1.0\itemsep}}
  {\end{enumerate}}

\newenvironment{snowflakelist}{
  \vspace{-0.4in}
  \begin{enumerate}[\textleaf]
    \addtolength{\itemsep}{-1.2\itemsep}}
  {\end{enumerate}}

\newenvironment{checklist}{
  \begin{enumerate}[\ding{52}]
    \addtolength{\itemsep}{-1.0\itemsep}}
  {\end{enumerate}}

\newenvironment{numberlist}
   {\begin{enumerate}[(1)]
       \addtolength{\itemsep}{-0.5\itemsep}}
     {\end{enumerate}}
\usepackage{amsfonts}
\makeatletter
\def\amsbb{\use@mathgroup \M@U \symAMSb}
\makeatother
\usepackage{bbold}

\usepackage{array}
\newcolumntype{C}{>$c<$}

\newcommand{\llnot}{\lnot \,} % is accepted
\newcommand{\mydash}{\text{--}}


%------------------


\title{\textbf{Well ordering}}
%\author[Barton Willis] % (optional, for multiple authors)
%{Barton~Willis}%
%\institute[UNK] % (optional)
\subtitle{Lesson 10   \\ \vspace{0.5in}
  ``If you can't prove what you want to prove, demonstrate something else and pretend that they are the same.''  \\   \vspace{0.15in}{Darrell Huff} \\ \vspace{0.15in} \emph{How to Lie with Statistics} \\
 \vspace{1.0in}
  \tiny Barton Willis, Attribution 4.0 International (CC BY 4.0), 2020 \normalsize
   }
  \date{}

\begin{document}



\frame{\titlepage}


\begin{frame}

\begin{mydef} Let \(A\) be a subset of \(reals\). We say that \(A\) is \emph{bounded above} provided
\[
   \left (\exists M \in \reals \right) \left( \forall x \in A \right)(x \leq M).
\]
We say that the number \(M\) is \emph{an upper bound for  the set } \(A\).
The set  \(A\) is \emph{bounded below} provided
\[
   \left (\exists M \in \reals \right) \left( \forall x \in A \right)(M \leq x).
\]
We say that the number \(M\) is \emph{a lower bound for  the set } \(A\). If \(A\) is bounded below and bounded above, we say \(A\) is \emph{bounded}.
\end{mydef}

\end{frame}
\begin{frame}{Bounded details}

\begin{numberlist}

\item Notice that we do \emph{not} require that a upper bound for a set \(A\) to be a member of \(A\).

\item Same for a lower bound.

\item If \(M\) is an upper bound for a set \(A\), any number that is greater than \(M\) is also an upper bound for \(A\).

\item Since we require that an upper bound be a real number, we disallow infinity from being an upper bound.  If we did, every set would be bounded.

\end{numberlist}

\end{frame}

\begin{frame}{Bounded examples}

\begin{myex}
  \begin{numberlist}
    \item The empty set is bounded above by 0.
      \item Actually every real number is an upper bound for the empty set.
      \item Every real number is a lower bound for the empty set.
   \item The interval \([0,1] \) is bounded above by 1.
   \item The interval \([0,1] \) is bounded above by 107.
   \item The interval \([0,\infty) \) is bounded below by 0.
   \item The interval \([0,\infty) \) is not bounded above.
\end{numberlist}
\end{myex}

\end{frame}




\begin{frame}{Being least}
\begin{mydef} Let \(A\) be a subset of \(\reals\). The set \(A\) has a least
  member provided
  \[
     \left (\exists a^\star \in A \right) \left( \forall a \in A \right)
     (a^\star \leq a).
  \]
  We say that \(a^\star\) is a least member.  The set \(A\) has a greatest 
  member provided
  \[
     \left (\exists a^\star \in A \right) \left( \forall a \in A \right)
     (a \leq a^\star).
  \]
  \end{mydef} 


\begin{numberlist}
 \item We require that a least member of a set \(A\) be a member of the set.
\end{numberlist}

\end{frame}



\begin{frame}{Well ordering principle}

\begin{myaxiom} Let \(A \subset \integers\) be (i) \emph{nonempty} and (ii) \emph{bounded below}. Then \(A\) has a least member.
\end{myaxiom}

\begin{numberlist}

\item This is an axiom--we'll take it on faith.

\item Again, a least member of a set \(A\) \emph{must} be a member of \(A\).

\item Thus the empty set does \emph{not} have a least member.

\item The qualification that the set be nonempty for it to have a least member is crucial.

\end{numberlist}


\begin{myth} Let \(A \subset \integers\) be (i) \emph{nonempty} and (ii) \emph{bounded above}. Then \(A\) has a greatest  member.
\end{myth}

\end{frame}
\begin{frame}{Well ordering principle for the reals?}

\textbf{Question} Are the real numbers well ordered?  That is, does every nonempty subset of \(\reals\) that is bounded below have a least member?

\vspace{0.5in}
\textbf{Answer} No. The interval \((0,1)\) is nonempty and  bounded below, but it doesn't have a least member. Although zero is less than
every member of \((0,1)\), since zero isn't a member of \((0,1)\), it is not a least member.

\end{frame}

\begin{frame}{Uniqueness of the least}
\begin{myth} If a subset of the reals has a least member, it is unique.

\end{myth}

\begin{myproof} Let \(A \subset \reals\). Suppose \(x\) and  \(x^\prime\) are least members of \(A\). Since \(x\) is a least member of \(A\) we have
  \(x \in A\). But \(x^\prime\) is a least member, so \(x^\prime \leq x\).
Interchanging the roles of \(x\) and \(x^\prime\), we have \(x^\prime \leq x\); therefore \(x = x^\prime\).

\end{myproof}

\begin{numberlist}
\item Equality is hard, inequality is easier.
\item We proved equality by proving two inequalities.
\end{numberlist}
\end{frame}

\begin{frame}{Existence of the floor}

Let \(x \in \reals\). Define the set \(M\) by
\[
     M = \{k \in \reals \mid k \geq x \}
\]

\end{frame}

\end{document}
