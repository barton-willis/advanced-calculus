\documentclass[fleqn]{beamer}
%\usetheme[height=7mm]{Rochester}
\usetheme{Boadilla} %{Rochester}

\setbeamertemplate{footline}[text line]{%
\parbox{\linewidth}{\vspace*{8pt}\hfill\insertshortauthor\hfill\insertpagenumber}}
\setbeamertemplate{navigation symbols}{}
%\author[BW]{Barton Willis}
\usepackage{amsmath}\usepackage{amsthm}
\usepackage{isomath}
\usepackage{upgreek}
\usepackage{comment,enumerate,xcolor}

\usepackage[english]{babel}
\usepackage[final,babel]{microtype}%\usepackage[dvipsnames]{color}
%\usefonttheme{professionalfonts}
%\usefonttheme{serif}

\newcommand{\reals}{\mathbf{R}}
\newcommand{\complex}{\mathbf{C}}
\newcommand{\integers}{\mathbf{Z}}
\DeclareMathOperator{\range}{range}
\DeclareMathOperator{\domain}{dom}
\DeclareMathOperator{\dom}{dom}
\DeclareMathOperator{\codomain}{codomain}
\DeclareMathOperator{\sspan}{span}
\DeclareMathOperator{\F}{F}
\DeclareMathOperator{\G}{G}
\DeclareMathOperator{\B}{B}
\DeclareMathOperator{\D}{D}
\DeclareMathOperator{\id}{id}
\DeclareMathOperator{\ball}{ball}

\usepackage{graphicx}
\usepackage{color}
\usepackage{amsmath}
\DeclareMathOperator{\nullspace}{nullity}
\theoremstyle{definition}
\newtheorem{mydef}{Definition}
\newtheorem{myqdef}{Quasi-definition}
\newtheorem{myex}{Example}
\newtheorem{myth}{Theorem}
\newtheorem{myaxiom}{Axiom}
\newtheorem{myfact}{Fact}
\newtheorem{metathm}{Meta Theorem}
\newtheorem{Question}{Question}
\newtheorem{Answer}{Answer}
\newtheorem{myproof}{Proof}

\newtheorem{myfakeproof}{Fake Proof}
\newtheorem{mybadformproof}{Bad Form Proof}

\newtheorem{hurestic}{Hurestic}

\newenvironment{alphalist}{
  \vspace{-0.4in}
  \begin{enumerate}[(a)]
    \addtolength{\itemsep}{1.0\itemsep}}
  {\end{enumerate}}

\newenvironment{snowflakelist}{
  \vspace{-0.4in}
  \begin{enumerate}[\textleaf]
    \addtolength{\itemsep}{-1.2\itemsep}}
  {\end{enumerate}}

\newenvironment{checklist}{
  \begin{enumerate}[\ding{52}]
    \addtolength{\itemsep}{-1.0\itemsep}}
  {\end{enumerate}}

\newenvironment{numberlist}
   {\begin{enumerate}[(1)]
       \addtolength{\itemsep}{-0.5\itemsep}}
     {\end{enumerate}}
\usepackage{amsfonts}
\makeatletter
\def\amsbb{\use@mathgroup \M@U \symAMSb}
\makeatother
\usepackage{bbold}

\usepackage{array}
\newcolumntype{C}{>$c<$}

\newcommand{\llnot}{\lnot \,} % is accepted
\newcommand{\mydash}{\text{--}}


%------------------


\title{\textbf{Well ordering}}
%\author[Barton Willis] % (optional, for multiple authors)
%{Barton~Willis}%
%\institute[UNK] % (optional)
\subtitle{Lesson 11   \\ \vspace{0.5in}
  ``The only way to learn mathematics is to do mathematics.''  \\   \vspace{0.15in}{Paul Halmos} \\
 \vspace{1.0in}
  \tiny Barton Willis, Creative Commons CC0 1.0 Universal, \the\year \normalsize
   }


  \date{}

\begin{document}



\frame{\titlepage}


\begin{frame}

\begin{mydef} Let \(A\) be a subset of \(\reals\). We say that \(A\) is \emph{bounded above} provided
\[
   \left (\exists M \in \reals \right) \left( \forall x \in A \right)(x \leq M).
\]
The number \(M\) is \emph{an upper bound for  the set } \(A\).
We say that \begin{tabular}{column alignment}

\end{tabular}he set  \(A\) is \emph{bounded below} provided
\[
   \left (\exists M \in \reals \right) \left( \forall x \in A \right)(M \leq x).
\]
The number \(M\) is \emph{a lower bound for  the set } \(A\). If \(A\) is bounded below and bounded above, we say \(A\) is \emph{bounded}.
\end{mydef}

\begin{numberlist}
  \item If \(M\) is an upper bound for a set \(A\), and \(M \leq M^\prime\),
  then \(M^\prime\) is an upper bound for \(A\).

   \item So we need to say \emph{a (not the) upper bound}.

   \item Similarly, lower bounds are not unique.
\end{numberlist}
\end{frame}
\begin{frame}{Bounded details}

\begin{numberlist}

\item Notice that we do \emph{not} require that a upper bound for a set \(A\) to be a member of \(A\).

\vspace{0.15in}
\item Same for a lower bound.

\vspace{0.15in}

\item Since we require that an upper bound be a \emph{real number}, we disallow infinity from being an upper bound.  If we did, every set would be bounded.

\vspace{0.15in}
\item Although infinity is a number, it isn't a \emph{real} number.
\end{numberlist}

\end{frame}

\begin{frame}{Bounded and unbounded examples}

\begin{myex}
  \begin{numberlist}
    \item The empty set is bounded above by 0.
    \vspace{0.15in}
      \item Actually every real number is an upper bound for the empty set.
      \vspace{0.15in}
      \item Every real number is a lower bound for the empty set.
      \vspace{0.15in}
   \item The interval \([0,1] \) is bounded above by 1.
   \vspace{0.15in}
   \item The interval \([0,1] \) is bounded above by 107.
   \vspace{0.15in}
   \item The interval \([0,\infty) \) is bounded below by 0.
   \vspace{0.15in}
   \item The interval \([0,\infty) \) is not bounded above.
\end{numberlist}
\end{myex}

\end{frame}




\begin{frame}{Being least}
\begin{mydef} Let \(A\) be a subset of \(\reals\). The set \(A\) has a \emph{least member} provided
  \[
     \left (\exists a^\star \in A \right) \left( \forall a \in A \right)
     (a^\star \leq a).
  \]
  We say that \(a^\star\) is a least member.  The set \(A\) has a \emph{greatest
  member} provided
  \[
     \left (\exists a^\star \in A \right) \left( \forall a \in A \right)
     (a \leq a^\star).
  \]
  \end{mydef}


\begin{numberlist}
 \item Unlike a lower bound, we require that a least member of a set \(A\) be a member of the set.

 \vspace{0.15in}

 \item The same for a greatest member.
\end{numberlist}

\end{frame}

\begin{frame}{Uniqueness of being least}

\begin{myth} If a subset of the reals has a least member, it is unique.

\end{myth}

\begin{myproof} Let \(A \subset \reals\). Suppose \(x\) and  \(x^\prime\) are least members of \(A\). Since \(x\) is a least member of \(A\) we have
  \(x \in A\). But \(x^\prime\) is a least member, so \(x^\prime \leq x\).
Interchanging the roles of \(x\) and \(x^\prime\), we have
\(x \leq x^\prime\); therefore \(x = x^\prime\).

\end{myproof}

\begin{numberlist}
\item Equality is hard, inequality is easier.
\item We proved equality by proving two inequalities.
\end{numberlist}
\end{frame}

\begin{frame}{Well ordering principle}

\begin{myaxiom} Let \(A\) be a nonempty subset of \(\integers\) that is \emph{bounded below}. Then \(A\) has a least member.
\end{myaxiom}

\begin{numberlist}

\item This is an axiom--we'll take it on faith.

\item Again, a least member of a set \(A\) \emph{must} be a member of \(A\).

\item Thus the empty set does \emph{not} have a least member.

\item The qualification that the set be nonempty for it to have a least member is crucial.

\end{numberlist}


\begin{myth} Let \(A \subset \integers\) be (i) \emph{nonempty} and (ii) \emph{bounded above}. Then \(A\) has a greatest  member.
\end{myth}

\end{frame}
\begin{frame}{Well ordering principle for the reals?}

\textbf{Question} Are the real numbers well ordered?  That is, does every nonempty subset of \(\reals\) that is bounded below have a least member?

\vspace{0.5in}
\textbf{Answer} No. The interval \((0,1)\) is nonempty and  bounded below, but it doesn't have a least member. Although zero is less than
every member of \((0,1)\), since zero isn't a member of \((0,1)\), it is not a least member.

\end{frame}



\begin{frame}{Existence of the floor}

Let \(x \in \reals_{\geq 0}\). Define the set \(M\) by
\(
     M = \{k \in \reals \mid k \leq x \}
\).

\begin{numberlist}

 \item Since \(x \geq 0\), it follows that \(0 \in M\).

\vspace{0.15in}
 \item So, the set \(M\) is nonempty.
\vspace{0.15in}
 \item Further the set \(M\) is bounded above by \(x\).
\vspace{0.15in}
  \item The well ordering principle tells us that \(M\) as
  a least member.
\vspace{0.15in}
  \item Actaully, the least member is unique.
\vspace{0.15in}
  \item Of course the least member depends on \(x\).
\vspace{0.15in}
  \item Something that (i) depends on \(x\) and (ii) is unique defines a function!
\vspace{0.15in}
  \item We've used the well ordering principle to define the \emph{floor function} for nonnegative inputs.

%  \item Similarly, we can define the ceiling function for nonnegative %inputs.

%  \item The putative identity \(\lfoor x \rfloor = - \lceil -x \rceil %\) extends the floor and ceiling functions from the nonnegative %numbers to the reals.



\end{numberlist}
\end{frame}

\end{document}
