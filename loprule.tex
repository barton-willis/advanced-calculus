\documentclass[fleqn]{beamer}
%\usetheme[height=7mm]{Rochester}
\usetheme{Boadilla} %{Rochester}

\setbeamertemplate{footline}[text line]{%
\parbox{\linewidth}{\vspace*{8pt}\hfill\insertshortauthor\hfill\insertpagenumber}}
\setbeamertemplate{navigation symbols}{}
%\author[BW]{Barton Willis}
\usepackage{amsmath}\usepackage{amsthm}
\usepackage{isomath}
\usepackage{upgreek}
\usepackage{comment,enumerate,xcolor}

\usepackage[english]{babel}
\usepackage[final,babel]{microtype}%\usepackage[dvipsnames]{color}
\usefonttheme{professionalfonts}
%\usefonttheme{serif}

\newcommand{\reals}{\mathbf{R}}
\newcommand{\complex}{\mathbf{C}}
\newcommand{\integers}{\mathbf{Z}}
\newcommand{\cover}{\mathcal{C}}

\DeclareMathOperator{\range}{range}
\DeclareMathOperator{\domain}{dom}
\DeclareMathOperator{\dom}{dom}
\DeclareMathOperator{\codomain}{codomain}
\DeclareMathOperator{\sspan}{span}
\DeclareMathOperator{\F}{F}
\DeclareMathOperator{\G}{G}
\DeclareMathOperator{\B}{B}
\DeclareMathOperator{\D}{D}
\DeclareMathOperator{\id}{id}
\DeclareMathOperator{\ball}{B}

\usepackage{graphicx}
\usepackage{color}
\usepackage{amsmath}
\DeclareMathOperator{\nullspace}{nullity}
\theoremstyle{definition}
\newtheorem{mydef}{Definition}
\newtheorem{myqdef}{Quasi-definition}
\newtheorem{myex}{Example}
\newtheorem{myth}{Theorem}
\newtheorem{myfact}{Fact}
\newtheorem{metathm}{Meta Theorem}
\newtheorem{Question}{Question}
\newtheorem{Answer}{Answer}
\newtheorem{myproof}{Proof}

\newtheorem{myfakeproof}{Fake Proof}
\newtheorem{mybadformproof}{Bad Form Proof}

\newtheorem{hurestic}{Hurestic}

\newenvironment{alphalist}{
  \vspace{-0.4in}
  \begin{enumerate}[(a)]
    \addtolength{\itemsep}{1.0\itemsep}}
  {\end{enumerate}}

\newenvironment{snowflakelist}{
  %\vspace{-0.4in}
  \begin{enumerate}[\textleaf]
    %\addtolength{\itemsep}{1.2\itemsep}
 }
  {\end{enumerate}}

\newenvironment{checklist}{
  \begin{enumerate}[\ding{52}]
    \addtolength{\itemsep}{-1.0\itemsep}}
  {\end{enumerate}}

\newenvironment{numberlist}
   {\begin{enumerate}[(1)]
       \addtolength{\itemsep}{-0.5\itemsep}}
     {\end{enumerate}}
\usepackage{amsfonts}
\makeatletter
\def\amsbb{\use@mathgroup \M@U \symAMSb}
\makeatother
\usepackage{bbold}

\usepackage{array}
\newcolumntype{C}{>$c<$}

\newcommand{\llnot}{\lnot \,} % is accepted
\newcommand{\mydash}{\text{--}}


%------------------


\title{\textbf{Trouble Makers}}
%\author[Barton Willis] % (optional, for multiple authors)
%{Barton~Willis}%
%\institute[UNK] % (optional)

\begin{document}

\frame{\titlepage}

\begin{frame}{The most important theorem that isn't in our book}

\begin{myth}   If a function \(F\) is differentiable at  \(a\) there is a function \(\Phi\) that is continuous at \(a\) and
\[
   F(x) = F(a) + (x-a) \Phi(x).
\]
Further \(F^\prime(a) = \Phi(a) \).

\end{myth}

\begin{enumerate}

\item A formula for \(\Phi\) is
\[
     \Phi(x) = \begin{cases} \frac{F(x) - F(a)}{x - a}  & x \neq a \\  F^\prime(a) & x = a \end{cases}.
\]

\item Although \(F^\prime(a) = \Phi(a)\), generally   \(F^\prime(x) \neq  \Phi(x)\). 
\end{enumerate}

\end{frame}

\begin{frame}{The L' H\^opital Rule}

Suppose \(F\) and \(G\) are differentiable at \(a\) and that \(F(a) = 0\) and \(G(a) =  0\), but    \(F^\prime(a) \neq  0\) and \(G^\prime(a) \neq  0\) There are functions \(\Phi\) and \(\Psi\) that are continuous at \(a\) 
such that
\begin{align*}
  F(x) &= (x-a) \Phi(x), \\
  G(x) &= (x-a) \Psi(x).
\end{align*}
Thus
\[
   \lim_{x \to a} \frac{F(x)}{G(x)} =    \lim_{x \to a} \frac{(x-a) \Phi(x)}{(x-a) \Psi(x)} =   \lim_{x \to a} \frac{ \Phi(x)}{\Psi(x)} =  \frac{\Phi(a)}{\Psi(a)} = \frac{F^\prime(a)}{G^\prime(a)}.
\]

\begin{enumerate}

\item The L'H\^opital rule is more general than this--see page 247.

\item Specifically if \(F\) and \(G\) are differentiable in a neighborhood of \(a\), the derivative of \(G\) doesn't vanish in this neighborhood, and \(\lim_{x \to a} \frac{F^\prime(x)}{G^\prime(x)} \) exists, then
\[
  \lim_{x \to a} \frac{F(x)}{G(x)} =  \lim_{x \to a}  \frac{F^\prime(x)}{G^\prime(x)}.
\] 
\end{enumerate}
\end{frame}
\end{document}