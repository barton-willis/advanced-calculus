\documentclass[12pt,fleqn, answers]{exam}
\usepackage{pifont}
\usepackage{dingbat}
\usepackage{amsmath,amssymb}
\usepackage{epsfig}
\usepackage[]{hyperref}
\usepackage{geometry}
\geometry{letterpaper, margin=0.5in}
\addpoints
\boxedpoints
\pointsinmargin
\pointname{pts}

\usepackage[activate={true,nocompatibility},final,tracking=true,kerning=true,factor=1100,stretch=10,shrink=10]{microtype}
\usepackage[american]{babel}
%\usepackage[T1]{fontenc}
\usepackage{fourier}
\usepackage{isomath}
\usepackage{upgreek,amsmath}
\usepackage{amssymb}

\newcommand{\dotprod}{\, {\scriptzcriptztyle
    \stackrel{\bullet}{{}}}\,}

\newcommand{\reals}{\mathbf{R}}
\newcommand{\lub}{\mathrm{lub}} 
\newcommand{\glb}{\mathrm{glb}} 
\newcommand{\complex}{\mathbf{C}}
\newcommand{\dom}{\mbox{dom}}
\newcommand{\cover}{{\mathcal C}}
\newcommand{\integers}{\mathbf{Z}}
\newcommand{\vi}{\, \mathbf{i}}
\newcommand{\vj}{\, \mathbf{j}}
\newcommand{\vk}{\, \mathbf{k}}
\newcommand{\bi}{\, \mathbf{i}}
\newcommand{\bj}{\, \mathbf{j}}
\newcommand{\bk}{\, \mathbf{k}}
\DeclareMathOperator{\Arg}{\mathrm{Arg}}
\DeclareMathOperator{\Ln}{\mathrm{Ln}}
\newcommand{\imag}{\, \mathrm{i}}

\usepackage{graphicx}
\newcommand\AM{{\sc am}}
\newcommand\PM{{\sc pm}}
     
\newcommand{\quiz}{5}
\newcommand{\term}{Fall}
\newcommand{\due}{Saturday 1 October  at 11:59 \PM}
\begin{document}
\large
\vspace{0.1in}
\noindent\makebox[3.0truein][l]{{\bf MATH 460}}
{\bf Name:}  \\
\noindent \makebox[3.0truein][l]{\bf Homework   \quiz, \term \/ \the\year}
%{\bf Row:}\hrulefill\
\vspace{0.1in}

\begin{quote}
    \fbox{I have neither given nor received unauthorized assistance on this assignment.}
    \end{quote}
\noindent  Homework    \quiz\/  has questions 1 through  \numquestions \/ with a total of  \numpoints\/  points.   
Neatly \textbf{handwrite your solutions}, digitize your work, and turn it into Canvas. You do \textbf{not} need to 
use  La\TeX\,  for this assignment. This work is due \emph{\due}.

\vspace{0.1in}

%\noindent{\textbf{Link to your Overleaf work: }}\url{XXX}

\begin{questions} 

\question[5] Show that \(\left( \forall x \in \reals \right) \left(\max(0,x) = \frac{x+|x|}{2} \right) \).
Likely you will want to consider the cases $x < 0$ and $x \geq 0$ separately.

\begin{solution}  
\end{solution}

\question[5] Let $F$ be a convergent sequence. Show that the sequence 
\(k \in \integers_{\geq 0} \mapsto \max(F_k,0)\) converges.
\textbf{Hint:} Use the result of Question 1. 

\begin{solution}  
\end{solution}


\question Define 
\(\displaystyle
    H = n \in \integers_{> 0} \mapsto \sum_{k=1}^n \frac{1}{k}.
\)

\begin{parts}

    \part[5] Use the fact that 
    \(\displaystyle
        \left(\forall x \in \reals_{\geq 1}\right)
         \left(\ln(x+1) - \ln(x) \leq \frac{1}{x} \right)
    \)
    to show that the sequence $H$ is not bounded above (and consequently does 
    not converge). To do this, you will use some standard calculus facts
    about telescoping sums and about the natural logarithm. One fact that you might use is the fact
    that the natural logarithm is not bounded above.
    \begin{solution}  
    \end{solution}

    \part[5] Show that 
    \(
        \left( \forall \, \varepsilon \in \reals_{>0} \right)
        \left(\exists \, n \in \integers \right)
        \left(\forall \, k \in \integers_{> n}\right)
        \left(|H_{k+1} - H_k| < \varepsilon\right)
    \).
\begin{solution}  
\end{solution}
    \part[5] Show that the sequence $H$ is not Cauchy.
    \begin{solution}  
    \end{solution}

\part [5] Draw a graph that shows that
the fact
\(\displaystyle
\left(\forall x \in \reals_{\geq 1}\right)
 \left(\ln(x+1) - \ln(x) \leq \frac{1}{x} \right)
\)
is due to the fact that the natural logarithm function is
concave down. \textbf{Hint:} For the graph \mbox{$y = \ln(x)$}
and a positive number $k$, show the tangent line at the point 
\mbox{$(x=k, y=\ln(k))$} and the 
secant line through the points \mbox{$(x=k, y=\ln(k))$} and 
$(x=k+1, y=\ln(k+1))$.

\end{parts}



\end{questions}



\end{document}