\documentclass[12pt,fleqn,answers]{exam}
\usepackage{pifont}
\usepackage{dingbat}
\usepackage{amsmath,amssymb}
\usepackage{epsfig}
\usepackage[]{hyperref}
\usepackage{geometry}
\geometry{letterpaper, margin=0.5in}
\addpoints
\boxedpoints
\pointsinmargin
\pointname{pts}

\usepackage{enumerate}
\newenvironment{alphalist}{
  %\vspace{-0.4in}
  \begin{enumerate}[(a)]
    \addtolength{\itemsep}{0.0\itemsep}}
  {\end{enumerate}}

\usepackage[activate={true,nocompatibility},final,tracking=true,kerning=true,factor=1100,stretch=10,shrink=10]{microtype}
\usepackage[american]{babel}
%\usepackage[T1]{fontenc}
\usepackage{fourier}
\usepackage{isomath}
\usepackage{upgreek,amsmath}
\usepackage{amssymb}

\newcommand{\dotprod}{\, {\scriptzcriptztyle
    \stackrel{\bullet}{{}}}\,}

\newcommand{\reals}{\mathbf{R}}
\newcommand{\lub}{\mathrm{lub}} 
\newcommand{\glb}{\mathrm{glb}} 
\newcommand{\complex}{\mathbf{C}}
\newcommand{\dom}{\mbox{dom}}
\newcommand{\cover}{{\mathcal C}}
\newcommand{\integers}{\mathbf{Z}}
\newcommand{\vi}{\, \mathbf{i}}
\newcommand{\vj}{\, \mathbf{j}}
\newcommand{\vk}{\, \mathbf{k}}
\newcommand{\bi}{\, \mathbf{i}}
\newcommand{\bj}{\, \mathbf{j}}
\newcommand{\bk}{\, \mathbf{k}}
\newcommand{\union}{\cup}
\newcommand{\intersection}{\cap}
\DeclareMathOperator{\Arg}{\mathrm{Arg}}
\DeclareMathOperator{\Ln}{\mathrm{Ln}}
\newcommand{\imag}{\, \mathrm{i}}

\usepackage{graphicx}
\newcommand\AM{{\sc am}}
\newcommand\PM{{\sc pm}}
     
\usepackage{color}
\shadedsolutions
\definecolor{SolutionColor}{rgb}{0.8,0.9,1}

\newcommand{\quiz}{4}
\newcommand{\term}{Fall}
\newcommand{\due}{Saturday 17 September  at 11:59 \PM}
\begin{document}
\large
\vspace{0.1in}
\noindent\makebox[3.0truein][l]{{\bf MATH 460}}
{\bf Name:}  \\
\noindent \makebox[3.0truein][l]{\bf Homework \quiz, \term \/ \the\year}
%{\bf Row:}\hrulefill\
\vspace{0.1in}

\begin{quote}
    \fbox{I have neither given nor received unauthorized assistance on this assignment.}
    \end{quote}
\noindent  Homework    \quiz\/  has questions 1 through  \numquestions \/ with a total of  \numpoints\/  points.   Edit this file and append you answers using La\TeX. Be sure to fill in your name. Upload the converted pdf of your work to Canvas.   This assignment is due \emph{\due}.

\vspace{0.1in}

\noindent{\textbf{Link to your Overleaf work: }}\url{XXX}

\begin{questions} 


\question [5] Show that the union of sets that are 
bounded above is bounded above. Specifically, show that if $A$ and $B$
are subsets of $\reals\) and  both $A$ and $B$ are 
bounded above, $A \union B$ is bounded above.

\question [5] Using induction, we can show that for any a finite
union of sets that are bounded above is bounded above. Specifically,
if for every positive integer $n$ the sets $A_1, A_2, \dots A_n$ are 
bounded above, $\union_{k=1}^n A_k$ is bounded above. 

\quad Your task is to find an example of sets $A_1, A_2, A_3, \dots$ such that 
$\displaystyle \union_{k=1}^\infty A_k$ is \emph{not} bounded above, but each set  $A_1, A_2, A_3, \dots$
is bounded above.

\question [5] Let $A$ and $B$ be subsets of \(\reals\) and let $A$ and $B$ be bounded above.
Show that 
\[
    \lub(A \union B) = \max(\lub(A), \lub(B)).
\]

\question [5] Show that
\(
     \left(\exists x \in [0,1] \right)
     \left(\forall r \in \reals_{<0}\right)
     \left (\left(x-r, x+r \right) \not \subset [0,1] \right)
\).  We'll learn in a bit that this is a proof that the
set $[0,1]$ is not open. 
\end{questions}



\end{document}