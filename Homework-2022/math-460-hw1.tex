\documentclass[fleqn, 12pt,answers]{exam}
\usepackage{pifont}
\usepackage{dingbat}
\usepackage{amsmath,amssymb}
\usepackage{fleqn}
\usepackage{epsfig}
\usepackage{comment}
\usepackage{geometry}
\geometry{letterpaper, margin=0.75in}
\usepackage{color}
\shadedsolutions
\definecolor{SolutionColor}{rgb}{0.8,0.9,1}
\addpoints
\boxedpoints
\pointsinmargin
\pointname{pts}

\usepackage[activate={true,nocompatibility},final,tracking=true,kerning=true,factor=1100,stretch=10,shrink=10]{microtype}
\usepackage[american]{babel}
%\usepackage[T1]{fontenc}
\usepackage{fourier}
\usepackage{isomath}
\usepackage{upgreek,amsmath}
\usepackage{amssymb}
\usepackage{centernot}

\newcommand{\dotprod}{\, {\scriptzcriptztyle
    \stackrel{\bullet}{{}}}\,}

\newcommand{\reals}{\mathbf{R}}
\newcommand{\lub}{\mathrm{lub}} 
\newcommand{\glb}{\mathrm{glb}} 
\newcommand{\complex}{\mathbf{C}}
\newcommand{\dom}{\mbox{dom}}
\newcommand{\cover}{{\mathcal C}}
\newcommand{\integers}{\mathbf{Z}}
\newcommand{\vi}{\, \mathbf{i}}
\newcommand{\vj}{\, \mathbf{j}}
\newcommand{\vk}{\, \mathbf{k}}
\newcommand{\bi}{\, \mathbf{i}}
\newcommand{\bj}{\, \mathbf{j}}
\newcommand{\bk}{\, \mathbf{k}}
\newcommand{\true}{\mathrm{True}}
\newcommand{\false}{\mathrm{False}}
\DeclareMathOperator{\Arg}{\mathrm{Arg}}
\DeclareMathOperator{\Ln}{\mathrm{Ln}}
\newcommand{\imag}{\, \mathrm{i}}

\usepackage{graphicx}
\newcommand\AM{{\sc am}}
\newcommand\PM{{\sc pm}}
     
\newcommand{\quiz}{1}
\newcommand{\term}{Fall}
\newcommand{\due}{Saturday 27 August  at 11:59 \PM}
\begin{document}
\begin{comment}
\large
\vspace{0.1in}
\noindent\makebox[3.0truein][l]{{\bf MATH 460}}
{\bf Name:}  \\
\noindent \makebox[3.0truein][l]{\bf Homework   \quiz, \term \/ \the\year}
%{\bf Row:}\hrulefill\
\vspace{0.1in}

\noindent  Homework    \quiz\/  has questions 1 through  \numquestions \/ with a total of  \numpoints\/  points.   

\textbf{Very neatly} hand write your solutions, digitize them 
(pdf works the best--please no *.HEIC files. Canvas cannot display 
them), and submit the digitized copy to Canvas.  This assignment is 
due \emph{\due}.

\begin{quote}
  \fbox{I have neither given nor received unauthorized assistance on this assignment.}
  \end{quote}
\vspace{0.1in}


\end{comment}

\begin{questions} 

\question A function $F$ is increasing on its domain provided
\begin{equation*}
  \left(\forall x,y \in \dom(F) \right) \left(x < y \implies F(x) \leq F(y) \right).
  \end{equation*}
  
  \begin{parts}
  
  \part [5] Without using negation (the symbol \(\lnot\)), write the negation of 
\begin{equation*} 
  \left(\forall x,y \in \dom(F) \right) \left(x < y \implies F(x) \leq F(y) \right).
  \end{equation*}
   in symbolic form. For assistance with the logic, see the 
   section ``Tautologies'' in our class QRS.
  
   \begin{solution}
    \begin{equation*} 
      \left(\exists x,y \in \dom(F) \right) \left((x < y) \land  (F(x) > F(y)) \right).
      \end{equation*}
  Alternatively, a solution is
  \begin{equation*} 
    \left(\exists x,y \in \dom(F) \right) \left(x < y \centernot \implies F(x) \leq F(y) \right).
    \end{equation*} 
  But for most people, the meaning of $(x < y) \land  (F(x) > F(y))$ is clear and the
  meaning of $(x < y) \centernot \implies (F(x) \leq F(y)) $ is less clear.
   \end{solution}
    
  \part[5] Show that the function $x \in [-1,1] \mapsto |x|$ is not increasing on its domain.
  \begin{solution}
    We'll show that
    \begin{equation*} 
      \left(\exists x,y \in [-1,1] \right) \left((x < y) \land  (|x| > |y|) \right).
      \end{equation*}
    Choose $x = -1$ and $y=0$.  We have
    \[
      \left[(x < y) \land  (|x| > |y|)\right] \equiv
      \left[(-1 < 0) \land  (|-1| > |0|)\right] \equiv \true.
    \]

    There are infinitely many choices for $x$ and $y$ that yield
    a proof. We only need one--don't burden the reader with more 
    than one choice.
   \end{solution}
   
  \end{parts} 
  
  \question A function $F$ is subadditive on its domain provided
  \begin{equation*}
  \left(\forall x,y \in \dom(F) \right) \left(  F(x+y) \leq F(x) + F(y) \right).
  \end{equation*}
  
  \begin{parts}
  
  \part [5] Without using negation (the symbol \(\lnot\)), write the negation of 
  \begin{equation*}
  \left(\forall x,y \in \dom(F) \right) \left(  F(x+y) \leq F(x) + F(y) \right)
  \end{equation*}
   in symbolic form.
   \begin{solution}
    \begin{equation*}
    \left(\exists x,y \in \dom(F) \right) \left(  F(x+y) > F(x) + F(y) \right).
    \end{equation*}
  \end{solution}
    
  
  \part[5] Show that the function $x \in \reals \mapsto x^2 $ is not subadditive on its domain.

  \begin{solution} We'll show that
         \begin{equation*}
      \left(\exists x,y \in \reals \right) \left(  (x+y)^2 > x^2 + y^2 \right).
      \end{equation*}
  Choose $x=1$ and $y=1$. We have
  \begin{equation*}
    \left[(x+y)^2 > x^2 + y^2 \right] \equiv 
      \left[4 > 1 + 1 \right] \equiv \true.
  \end{equation*}
  Again, there are many choices for $x$ and $y$, but we need only
  once choice.
   \end{solution}
  
  \part[5]  Show that the function  $x \in \reals \mapsto |x| $ is  subadditive on its domain.
  To do this, you may use the triangle inequality without proving it.
  \end{parts}
  

\begin{solution} We'll show that
  \begin{equation*}
    \left(\forall x,y \in \reals \right) \left( |x+y|  \leq  |x| + |y|  \right).
    \end{equation*}
This is the triangle inequality, a fact we were allowed to use without proof.

The familiar triangle inequality expresses the fact that the absolute value function is subadditive.
\end{solution}




\end{questions}



\end{document}