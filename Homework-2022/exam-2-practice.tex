\documentclass[12pt,fleqn,answers]{exam}
\usepackage{pifont}
\usepackage{dingbat}
\usepackage{amsmath}
\usepackage{fleqn}
\usepackage{epsfig}
\usepackage{fourier}
\usepackage{amssymb}
\usepackage[margin=0.75in]{geometry}

\usepackage[activate={true,nocompatibility},final,tracking=true,kerning=true,spacing=true,factor=1100,stretch=10,shrink=10]{microtype}
\addpoints
\boxedpoints
\pointsinmargin
\pointname{pts}

\newcommand{\reals}{\mathbf{R}}
\newcommand{\ball}{\mathrm{ball}}
\newcommand{\integers}{\mathbf{Z}}
\newcommand{\LP}{\mathrm{LP}}
\newcommand{\True}{\mathrm{True}}
\newcommand{\dotprod}{\, {\scriptzcriptztyle \stackrel{\bullet}{{}}}\,}

\let\oldforall\forall
\renewcommand{\forall}{\oldforall \, }

\let\oldexist\exists
\renewcommand{\exists}{\oldexist \: }
\begin{document}
\large
\begin{flushleft}
  \textbf{Advanced Calculus, Fall \the\year}\\
  \emph{Review for Exam II}
\end{flushleft}

\begin{questions}

\question Show that the sequence $k \in \integers_{>0} \mapsto \frac{k+1}{k+5}$ converges.
\begin{solution} We'll show that
\begin{equation*}
  (\exists L \in \reals)
  (\forall \varepsilon \in \reals_{>0})
  (\exists n \in \integers) 
  \left(\forall k \in \integers_{>n}\right )
  \left (\left |  \frac{k+1}{k+5} - L \right | < \varepsilon \right).
\end{equation*}
Choose $L=1$.  Let $\varepsilon \in \reals_{>0}$.  Choose 
$n = \left \lceil \frac{4}{\varepsilon} \right \rceil$.  Let $k \in \integers_{>n}$. We have
\begin{align*}
\left|  \frac{k+1}{k+5} - L \right | &=  \left | \frac{k+1}{k+5} - 1 \right |, & (\mbox{substitution}) \\
                                        &=   \frac{4}{k+5},            &(\mbox{algebra}) \\
                                    %    &<    \frac{4}{n},                 &(k+5 > n) \\
                                     %  &=   \frac{4}{\left \lceil \frac{4}{\varepsilon} \right \rceil},  & (\mbox{substitution}) \\ 
                                    %    &\leq  \frac{4}{ \frac{4}{\varepsilon}},  & (\mbox{ceiling property}) \\ 
                                        &= \varepsilon.  &(\mbox{algebra})
\end{align*}

\end{solution}

\end{questions}
\end{document}

\question Give an example of a convergent subsequence of $F = k \in \integers_{>0} \mapsto (-1)^k$.
\begin{solution}
Define $\phi = n \in \integers \mapsto 2 n$.  
Then $F \circ \phi = k \in \integers \mapsto (-1)^{2 k}$.
This is a constant sequence; it converges.

\end{solution}
\question Show that sequence $k \in \integers_{>0} \mapsto \begin{cases}  k!  & k < 1000 \\
\frac{k+1}{k+5} & k \geq 1000 \end{cases} $ converges.

\begin{solution} We'll show that
\[
  (\exists L \in \reals)
  (\forall \varepsilon \in \reals_{>0})
  (\exists n \in \integers)
  (\forall k \in \integers_{>n})
  \left ( \left |  \frac{k+1}{k+5} - L \right | < \varepsilon \right ).
\]
Choose $L=1$.  Let $\varepsilon \in \reals_{>0}$.  Choose $n = \max(1000, \lceil \frac{4}{\varepsilon} \rceil)$.  Let $k \in \integers_{>n}$. We have
\begin{align*}
\left|  \frac{k+1}{k+5} - L \right | &=   \left |\frac{k+1}{k+5} - L \right |,   &(k > 1000) \\
                                         &= \left | \frac{k+1}{k+5} - 1 \right|,      & (\mbox{substitution}) \\
                                        &=   \frac{4}{k+5},             &(\mbox{algebra}) \\
                                        &<    \frac{4}{n},                &(k+5 > n) \\
                                        &=   \frac{4}{\lceil \frac{4}{\varepsilon} \rceil},  & (\mbox{substitution}) \\ 
                                        &\leq  \frac{4}{ \frac{4}{\varepsilon}},  & (\mbox{ceiling property})\\ 
                                        &= \varepsilon.  &(\mbox{algebra})
\end{align*}
\end{solution}

\question  Use the QRS definition of an open set to show that interval \((0,1)\) is open.
\begin{solution}  We'll show that
\[
   (\forall x \in (0,1))(\exists r \in \reals_{>0})(\ball(x,r) \subset (0,1)).
\]
Let $x \in (0,1)$.  Choose $r = \frac{1}{2} \min(1-x,x)$. Since
$x > 0$ and $x < 1$, we have $r \in \reals_{>0}$ as required.
We need to show that $0 < x - r$ and $x+r < 1$. We have
\[
    [0 < x-r ] \equiv [r < x] \equiv \True
\]
And
\[
    [x+r < 1 ] \equiv [x < 1-r] \equiv \True
\]
\end{solution}

\question  Use the QRS definition of a closed set to show that interval \([0,1]\) is closed.

\begin{solution} We need to show that the complement of  \([0,1]\) is open. We have
$ [0,1]^{\mathrm{C}} = (-\infty,0) \cup (1,\infty)$.  But both  $(-\infty,0)$  and $(1,\infty)$ are open. And we know that the union of
open sets is open, so  $[0,1]^{\mathrm{C}}$ is open; therefore \([0,1]\) is closed.

\quad Arguably we should use the QRS definition to show that both  $(-\infty,0)$  and $(1,\infty)$ are open. But the fact that these sets are
open is a book theorem.

\end{solution}

\question   Use the QRS definitions of open and closed to show that the set \(\reals\) is open and closed.

\begin{solution}
Let \(x\) be a real number.  We have \(\ball(x,1) \subset \reals\); therefore, \(\reals\) is open.  To show that \(\reals\) is
closed, we'll show that \(\reals^{\mathrm{C}} \) is open; since  \(\reals^{\mathrm{C}} = \varnothing\), we'll show that 
\(\varnothing\) is open. To show that \(\varnothing\) is open, we need to show that
\begin{equation*}
  \left(\forall x \in \varnothing \right)
  \left(\exists r \in \reals_{>0} \right)
  \left(\ball(x,r) \subset \varnothing \right).
\end{equation*}
This is vacuously true.
 \end{solution}
 
\question   Use the QRS definition of a boundary point to show that \(\partial(0,1] = \{0,1\}\).  Use this
result to explain why \((0,1]\) is not closed.
\begin{solution}
 First, we'll show that \( 0 \in \partial(0,1]
\). Let \(\delta\) be a positive real number.  We have \(-\delta/2
\notin (0,1]\). Further define
\( x^\star = \displaystyle \begin{cases} \frac{\delta}{2} & \mbox{ if } \delta <
  1 \\ \frac{1}{2} & \mbox { if } \delta \geq 1 
   \end{cases} \).  Then \(x^\star \in (0,1]\) and \(x^\star \in
   \ball(0,\delta)\). I'll leave it to you to show that  \(1 \in (0,1]\). 


\quad The set  \((0,1]\) is not closed because \(0 \in
 \partial(0,1]\) and \(0 \notin (0,1] \). (A closed set
must contain all of its boundary points.)
\end{solution}

\question  Use the QRS definition to show that \(0 \not \in  \LP(\integers)\).

\begin{solution}
  We need to show that $(\exists r \in \reals_{>0})
                (\ball^\prime(0,r) \cap \integers = \varnothing)$.
Choose $r=1/2$. Then $\ball^\prime(0,r) \cap \integers
=  \varnothing$.
\end{solution}

\question  Show that the function
\(\displaystyle
  F(x) = \begin{cases} -1 & \mbox{if } x < 5 \\
                        1 & \mbox{if } x \geq 5
         \end{cases}
\)
does not have a limit \mbox{toward 5.}

\begin{solution} We'll show that
\begin{equation*}
  \left(\forall L \in \reals \right)
  \left(\exists \varepsilon \in \reals_{>0} \right)
  \left(\forall \delta \in \reals_{>0} \right)
  \left(\exists x \in \ball^\prime(5,\delta) \right)
  \left( \left|F(x) - L \right| \geq \varepsilon \right).
\end{equation*}
Let $L \in \reals$. Choose $\varepsilon = 1$. Let 
$\delta \in \reals_{>0}$. Choose $x = \begin{cases}
  5+\delta/2 & L \leq 0 \\ 5 -\delta/2 & L \geq 0 \end{cases}$.
When $L \leq 0$, we have
\begin{equation*}
  |F(x) - L| = |F(5+\delta/2) - L| = |1 - L| \geq 1.
\end{equation*}
The case $L > 0$ is similar.

\end{solution}

\question   Show that the function \(F(x) = x^2\) has a limit
toward 2.
\begin{solution}
Let \(\varepsilon\) be a positive real number. Choose
\(\delta = \mbox{min} \{1, \frac{\delta}{5} \}\).  Let  \(x \in
\ball(2,\delta)\). To start, we notice that since $|x-2| \leq 1$, we have $ 1 \leq x \leq 3$;
thus $3 \leq x+2 \leq 5$. So $|x+2| \leq 5$. We have
\begin{align*}
  | x^2 - 4 | &= |x - 2| |x + 2|, & (\mbox{algebra}) \\
              &< |x + 2| \delta, & (x \in \ball(2,\delta))\\
              &\leq 5 \delta, &(|x+2| \leq 5) \\
              &\leq \varepsilon.  &(\delta \leq \varepsilon/5)
\end{align*}
\end{solution}

\question   Show that the set \((0,\infty)\) is not compact by 
showing that there is an open cover of \((0,\infty)\) that has no
finite subcover.

\begin{solution}
 For \(k \in \integers_{>0}\), define \(I_k = (-k,k)\).
The set \(\mathcal{C} = \{I_k | k \in \integers_{>0}\}\) is a cover of 
\((0,\infty)\).  The union of every finite subset of \(\mathcal{C}\)
is bounded. Thus no finite subset of \(\mathcal{C}\) is a cover of
\((0,\infty)\); therefore, \((0,\infty)\) is not compact.
\end{solution}

\question   Show that if a subset of \(\reals\) is not bounded, it
is not compact. Do this using the definition of compact that involves
open covers.

\begin{solution}
\textbf{Proof} See your class notes.
\end{solution}






\question   Show that the union of two compact sets is compact.  Do
this using  the definition of compact that involves
open covers.

\begin{solution}
 Let \(F_1\) and \(F_2\) be compact, and let
\(\mathcal{C}\) be an open cover of \(F_1 \cup  F_2\). Then
\(\mathcal{C}\) is an open cover of \(F_1\) and \(\mathcal{C}\) is an
open cover of \(F_2\). Since \(F_1\) and \(F_2\) are compact,
there are finite sets \(\mathcal{C}_1 \subset \mathcal{C}\) and
\(\mathcal{C}_2 \subset \mathcal{C}\) such that 
\(
  F_1 \subset \underset{x \in \mathcal{C}_1}{\cup} x\) and
\(
  F_2 \subset \underset{x \in \mathcal{C}_2}{\cup} x.
\)
Thus
\(\displaystyle
   F_1 \cup  F_2  \subset \underset{x \in \mathcal{C}_1 \cup \mathcal{C}_2 }{\cup} x
\).
Since \(\mathcal{C}_1 \cup \mathcal{C}_2\) is finite, it follows that 
\( F_1 \cup  F_2 \) is compact.

\end{solution}

\question Show that if sets $A$ and $B$ are closed, so is $A \cup B$.

\begin{solution}
We'll show that $(A \cup B)^{\mathrm{C}}$ is open. We have   $(A \cup B)^{\mathrm{C}} = A^{\mathrm{C}} \cap B^{\mathrm{C}}$. But both $A^{\mathrm{C}}$ and $B^{\mathrm{C}}$ are open;
since a finite intersection of open sets is open, we have $A^{\mathrm{C}} \cap B^{\mathrm{C}}$ is open.

\end{solution}

\question   Give an example of open sets \(G_1, G_2, G_3, \dots \) such that the intersection
\( \displaystyle   \underset{k \in \integers_{> 0}}{\cap} G_k
\) is not open.

\begin{solution}
For \(k \in \integers_{>0}\), define \(I_k = (1-1/k,
\infty)\). We have
\(\displaystyle
  \underset{k \in \integers_{>0}}{\cap} I_k = [1,\infty).
\)
The set \([1,\infty)\) isn't open.

\end{solution}

\question   Define  \(F =  x \in \integers \mapsto \sqrt[3]{x^{14} + 1066} + \sqrt[43]{x^2 + 1776}\).
Either prove or disprove: 
%\begin{quote}
  The function \(F\) has a limit toward 1.
%\end{quote}

\begin{solution}
Since $1 \notin \LP(\integers)$, the sequence $F$ doesn't have a limit toward 1.

\end{solution}

\question   Define $F =  x \in \integers \mapsto \sqrt[3]{x^{14} + 1066} + \sqrt[43]{x^2 + 1776}.$ Show that
$F$  is continuous at 1.
\begin{solution}
We'll show that
\[
  (\forall \varepsilon \in \reals_{>0})(\exists \delta \in \reals_{>0})(\forall x \in \ball(1,\delta) \cap \integers)(
  (|F(x) - F(1)| < \varepsilon).
\]
Let $ \varepsilon \in \reals_{>0}$. Choose $\delta = \frac{1}{2}$.  Let $x \in \ball(1,\delta) \cap \integers$. We have
$x = 1$. Thus  $$|F(x) - F(1)|  = |F(1) - F(1)| = 0 < \varepsilon.$$
\end{solution}



\question  Let \(F\) be a convergent sequence and let \(\alpha \in \reals\).  Show that \(\alpha F\) is a convergent sequence.

\begin{solution} We'll prove this from scratch.  
  Suppose $F$ converges to $L$. We'll show that $\alpha F$ converges 
  to $\alpha L$.
Let $\varepsilon \in \reals_{>0}$.  
We have $\frac{\varepsilon}{1+|\alpha|} \in \reals_{>0}$. 
Choose $n \in \integers$ such that 
$(\forall k \in \integers_{>n})(|\alpha F(k) - L | <  \frac{\varepsilon}{1+|\alpha|}$.  
Again for all $k \in \integers_{>n}$, we have
\[
  |\alpha F(k) - \alpha L| = |\alpha| |F(k) - L| \leq \frac{\varepsilon |\alpha|}{1+|\alpha|} < \varepsilon.
\]



\end{solution}
\question  Let \(|F| \) be a convergent sequence.  Show that \(|F| \) is a convergent sequence.
\begin{solution}
We'll prove this from scratch.  Suppose $F$ converges to $L$. We'll show that $|F|$ converges to $|L|$.
Let $\varepsilon \in \reals_{>0})$. Choose $n \in \integers$ such that 
$$(\forall k \in \integers_{>n})(|F(k) - L | <  \varepsilon.$$  
Again for all $k \in \integers_{>n}$, we have
\[
  ||F(k)| - |L||  \leq |F(k) - L|  < \varepsilon.
\]
\end{solution}

\question   Use the inequality
\(
   | \sqrt{a} - \sqrt{b} | \leq \sqrt{|a -b|},\mbox{ for } a,b > 0
\)
to show that the function
\(
   F = x \in [-1,\infty) \mapsto  \sqrt{1+x},
\)
is continuous at 1.

\begin{solution} Let $\varepsilon \in \reals$. Choose $\delta  = \varepsilon^2$. Let $x \in \ball(1,\delta) \cap [-1,\infty)$ We have
\[
   | \sqrt{1+x} - \sqrt{2} | \leq \sqrt{|x-1|} < \sqrt{\delta} = \sqrt{\varepsilon^2} = \varepsilon.
\]

\end{solution}
\end{questions}
\end{document}
