\documentclass[12pt,fleqn]{exam}
\usepackage{pifont}
\usepackage{dingbat}
\usepackage{amsmath,amssymb}
\usepackage{epsfig}
\usepackage[]{hyperref}
\usepackage{geometry}
\geometry{letterpaper, margin=0.75in}
\addpoints
\boxedpoints
\pointsinmargin
\pointname{pts}

\usepackage[activate={true,nocompatibility},final,tracking=true,kerning=true,factor=1100,stretch=10,shrink=10]{microtype}
\usepackage[american]{babel}
%\usepackage[T1]{fontenc}
\usepackage{fourier}
\usepackage{isomath}
\usepackage{upgreek,amsmath}
\usepackage{amssymb}

\newcommand{\dotprod}{\, {\scriptzcriptztyle
    \stackrel{\bullet}{{}}}\,}

\newcommand{\reals}{\mathbf{R}}
\newcommand{\lub}{\mathrm{lub}} 
\newcommand{\glb}{\mathrm{glb}} 
\newcommand{\complex}{\mathbf{C}}
\newcommand{\dom}{\mbox{dom}}
\newcommand{\cover}{{\mathcal C}}
\newcommand{\ball}{\mathrm{ball}}
\newcommand{\integers}{\mathbf{Z}}
\newcommand{\vi}{\, \mathbf{i}}
\newcommand{\vj}{\, \mathbf{j}}
\newcommand{\vk}{\, \mathbf{k}}
\newcommand{\bi}{\, \mathbf{i}}
\newcommand{\bj}{\, \mathbf{j}}
\newcommand{\bk}{\, \mathbf{k}}
\DeclareMathOperator{\Arg}{\mathrm{Arg}}
\DeclareMathOperator{\Ln}{\mathrm{Ln}}
\newcommand{\imag}{\, \mathrm{i}}

\usepackage{graphicx}
\newcommand\AM{{\sc am}}
\newcommand\PM{{\sc pm}}
     
\newcommand{\quiz}{III}
\newcommand{\term}{Fall}
\newcommand{\due}{Saturday 27 August  at 11:59 \PM}
\begin{document}
\large
\vspace{0.1in}
\noindent\makebox[3.0truein][l]{{\bf MATH 460}}
{\bf Name:}  \\
\noindent \makebox[3.0truein][l]{\bf Exam   \quiz, \term \/ \the\year}
%{\bf Row:}\hrulefill\
\vspace{0.1in}

\noindent  Exam   \quiz\/  has questions 1 through  \numquestions \/ 
with a total of  \numpoints\/  points.   

\vspace{0.1in}


\begin{questions} 

\question[10] Using a limit of \emph{Newton quotient}, show that the function
$x \in \reals \mapsto x^2+ 5 x$ is differentiable at 2. \emph{Justify} each step of
your calculation with a \emph{word or a phrase}.

\begin{solution}%[3.5in]
\end{solution}

\newpage

\question[10] Using a limit of \emph{Newton quotient}, show that the function
$x \in \reals \mapsto |5-x| + x$ is differentiable at 2. \emph{Justify} each step of
your calculation with a \emph{word or a phrase}.

\begin{solution}%[3.5in]
\end{solution}
\newpage
\question [10] Given the fact that the function $x \in \reals \mapsto \max(1,x)$ is 
\emph{continuous} on $\reals$, show that the function 
$F = x \in \reals \mapsto x \max(1,x)$ is \emph{differentiable} at zero.


\begin{solution}%[3.5in]
\end{solution}
\newpage

\question [10] Show that the function
\begin{equation*}
    x \in \reals \mapsto x \begin{cases} 
             x \cos \left(\frac{1}{x} \right) & x \neq 0 \\
             0   & x = 0
    \end{cases}  
\end{equation*}
is \emph{differentiable} at zero.
\begin{solution}%[3.5in]
\end{solution}

\newpage
\question [10] Let $F$ be a function and let $a \in \dom(F)$.  Show that if
$F$ is continuous at $a$ and $F(a) > 0$,  then
\begin{equation*}
    \left(\exists r \in \reals_{>0} \right)
    \left(\forall x \in \ball(a,r) \cap \dom(F) \right)
    \left (F(x) > 0 \right)
\end{equation*}
(We used this fact to prove the reciprocal rule for derivatives.)


\newpage

\question [10] Give an example of a function $F$ that is not differentiable at zero, but 
the function $x \in \dom(F) \mapsto F(x)^2$ is differentiable at zero.

\end{questions}



\end{document}