\documentclass[12pt,fleqn]{exam}
\usepackage{pifont}
\usepackage{dingbat}
\usepackage{amsmath}
\usepackage{fleqn}
\usepackage{epsfig}
\usepackage{fourier}
\usepackage{amssymb}
\usepackage[margin=0.75in]{geometry}

\usepackage[activate={true,nocompatibility},final,tracking=true,kerning=true,spacing=true,factor=1100,stretch=10,shrink=10]{microtype}
\addpoints
\boxedpoints
\pointsinmargin
\pointname{pts}

\newcommand{\reals}{\mathbf{R}}
\newcommand{\ball}{\mathrm{ball}}
\newcommand{\integers}{\mathbf{Z}}
\newcommand{\LP}{\mathrm{LP}}
\newcommand{\True}{\mathrm{True}}
\newcommand{\dotprod}{\, {\scriptzcriptztyle \stackrel{\bullet}{{}}}\,}

\let\oldforall\forall
\renewcommand{\forall}{\oldforall \, }

\let\oldexist\exists
\renewcommand{\exists}{\oldexist \: }


\newcommand{\quiz}{II}
\newcommand{\term}{Fall}
\newcommand{\due}{Wednesday 14 September 13:15 \PM}
\newcommand{\class}{Advanced Calculus I}
\begin{document}
\large
\vspace{0.1in}
\noindent\makebox[3.0truein][l]{\textbf{\class}}
\textbf{Name:} \hrulefill \\
\noindent \makebox[3.0truein][l]{\textbf{Exam  \quiz, \term \/ \the\year}}
\textbf{Row and Seat}:\hrulefill\\
\vspace{0.1in}

\noindent  This exam  has questions 1 through  \numquestions \/ with a total of  \numpoints\/  points.  
\begin{questions}

\question[10] Use the QRS definition to show that the sequence $k \in \integers_{\geq 0} \mapsto \frac{k+3}{k+2}$ converges.
\begin{solution} We'll show that
\begin{equation*}
  (\exists L \in \reals)
  (\forall \varepsilon \in \reals_{>0})
  (\exists n \in \integers) 
  \left(\forall k \in \integers_{>n}\right )
  \left (\left |  \frac{k+1}{k+5} - L \right | < \varepsilon \right).
\end{equation*}
Choose $L=1$.  Let $\varepsilon \in \reals_{>0}$.  Choose 
$n = \left \lceil \frac{4}{\varepsilon} \right \rceil$.  Let $k \in \integers_{>n}$. We have
\begin{align*}
\left|  \frac{k+1}{k+5} - L \right | &=  \left | \frac{k+1}{k+5} - 1 \right |, & (\mbox{substitution}) \\
                                        &=   \frac{4}{k+5},            &(\mbox{algebra}) \\
                                        &<    \frac{4}{n},                 &(k+5 > n) \\
                                       &=   \frac{4}{\left \lceil \frac{4}{\varepsilon} \right \rceil},  & (\mbox{substitution}) \\ 
                                        &\leq  \frac{4}{ \frac{4}{\varepsilon}},  & (\mbox{ceiling property}) \\ 
                                        &= \varepsilon.  &(\mbox{algebra})
\end{align*}

\end{solution}

\newpage 

\question [10] Use the QRS definition of an \emph{open set} to show that the interval \((-\infty,1)\) is open.
\begin{solution}[4.5in]
  We'll show that
\[
   (\forall x \in (0,1))(\exists r \in \reals_{>0})(\ball(x,r) \subset (0,1)).
\]
Let $x \in (0,1)$.  Choose $r = \frac{1}{2} \min(1-x,x)$. Since
$x > 0$ and $x < 1$, we have $r \in \reals_{>0}$ as required.
We need to show that $0 < x - r$ and $x+r < 1$. We have
\[
    [0 < x-r ] \equiv [r < x] \equiv \True
\]
And
\[
    [x+r < 1 ] \equiv [x < 1-r] \equiv \True
\]
\end{solution}

\newpage

\question  [10] Use the QRS definition of a \emph{closed set} to show that interval \([-1,1]\) is closed.

\begin{solution} We need to show that the complement of  \([0,1]\) is open. We have
$ [0,1]^{\mathrm{C}} = (-\infty,0) \cup (1,\infty)$.  But both  $(-\infty,0)$  and $(1,\infty)$ are open. And we know that the union of
open sets is open, so  $[0,1]^{\mathrm{C}}$ is open; therefore \([0,1]\) is closed.

\quad Arguably we should use the QRS definition to show that both  $(-\infty,0)$  and $(1,\infty)$ are open. But the fact that these sets are
open is a book theorem.

\end{solution}

\newpage


 







\question   [10] Show that the set \(\reals \) is not compact by 
showing that there is an open cover of \(\reals \) that has no
finite subcover.

\begin{solution}
 For \(k \in \integers_{>0}\), define \(I_k = (-k,k)\).
The set \(\mathcal{C} = \{I_k | k \in \integers_{>0}\}\) is a cover of 
\((0,\infty)\).  The union of every finite subset of \(\mathcal{C}\)
is bounded. Thus no finite subset of \(\mathcal{C}\) is a cover of
\((0,\infty)\); therefore, \((0,\infty)\) is not compact.
\end{solution}

\newpage









\question  [10]  Use the inequality
\(
   | \sqrt{a} - \sqrt{b} | \leq \sqrt{|a -b|},\mbox{ for } a,b > 0
\)
to show that the function
\(
   F = x \in [-1,\infty) \mapsto  \sqrt{1+x},
\)
is continuous at 1.

\begin{solution} Let $\varepsilon \in \reals$. Choose $\delta  = \varepsilon^2$. Let $x \in \ball(1,\delta) \cap [-1,\infty)$ We have
\[
   | \sqrt{1+x} - \sqrt{2} | \leq \sqrt{|x-1|} < \sqrt{\delta} = \sqrt{\varepsilon^2} = \varepsilon.
\]

\end{solution}
\end{questions}
\end{document}
