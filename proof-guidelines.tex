\documentclass[12pt,fleqn]{article}
\usepackage{amsmath,pifont,moreverb,enumerate}
\usepackage{amssymb,url}
\usepackage{upgreek}
\usepackage[letterpaper,left=0.75in, right=0.75in,top=0.75in,bottom=0.75in]{geometry}
\newcommand{\reals}{\mathbf{R}}

\usepackage[final]{microtype}

\newenvironment{alphalist}{
  \begin{enumerate}[(a)]
    \addtolength{\itemsep}{-1.0\itemsep}}
  {\end{enumerate}}

\newenvironment{numlist}{
  \begin{enumerate}[(1)]
    \addtolength{\itemsep}{-1.0\itemsep}}
  {\end{enumerate}}

\newenvironment{handlist}{
  \begin{enumerate}[\ding{43}]
    \addtolength{\itemsep}{-1.0\itemsep}}
  {\end{enumerate}}

\usepackage{fourier,calc}

\newcounter{ex}\setcounter{ex}{0}
\newcommand{\ex}{%\
\hspace{-0.2in} \setcounter{ex}{\value{ex}+1}
\theex \,\,}

\newcounter{id}\setcounter{id}{0}
\newcommand{\id}{%\
\hspace{-0.2in} \setcounter{id}{\value{id}+1}
\theid \,\,}

\newcounter{se}\setcounter{se}{0}
\newcommand{\se}{%\
\hspace{-0.2in} \setcounter{se}{\value{se}+1}
\these \,\,}

\usepackage{isomath}
\renewcommand\thesubsubsection{\arabic{subsubsection}}
\title{ Guidelines for Writing Proofs}

\author{Barton Willis}

\usepackage{titlesec}
%\date{\today}                   
\begin{document}

\maketitle



\begin{quote}
  \emph{ “Bad thinking never produces good writing.”} \hfill   {\sc Leslie Lamport}
\end{quote}

\normalsize
The proofs you write for this class must be logical and they should be written using standard mathematical notation and expressed as English sentences. Other than logical correctness and using English sentences, you are free to write your proofs
in a way that makes sense and pleases you.  I like to keep the focus on logical correctness, not on adherence to a style template. If you have a question that is mostly about style, not content, don't be surprised if my answer is ``It doesn't 
matter--either way is OK.'' 

Our textbook and class notes provides us with plenty of examples of well-written proofs. You may, but are not required to, imitate the proof style of these resources. As you become more comfortable writing proofs, it's good to develop your own style. Although you are free to develop your own proof style, there are constraints, and this document will give a guide
to proper style.  Many examples that follow have been gleaned from student papers, so  unless you want to be  immortalized, write carefully.


\subsubsection{Write Sentences}

Write proofs using sentences. Do not write proofs as sentence fragments connected by 
arrows.
\begin{quote}
\textbf{Replace}
\begin{quote}
\(n,m\) integers \(\sqrt{2} = n / m  \longrightarrow 2 m^2 = n^2\). 
\end{quote}
\textbf{with}
\begin{quote}
Let \(n\) and \(m\) be integers. If \(\sqrt{2} = n / m\), then \(2 \, m^2 = n^2\).
\end{quote}
\end{quote}


%%----Style rules---------------------------------
\subsubsection{Format proofs as regular text}

Poets frequently format their work  with centered lines and wide
margins. Mathematics isn't poetry; don't format mathematics as if it were poetry.


\begin{quote}
\textbf{Replace}
\begin{quote}
\begin{centering}
Let \(\varepsilon > 0\). \\
Choose \(\delta = \varepsilon / 3 \). \\
For \(|x - 1| < \delta \) we have \(x < 1 + \delta\). \\
\end{centering}
\end{quote}
\textbf{with}
\begin{quote}
Let \(\varepsilon > 0\).  Choose \(\delta = \varepsilon / 3 \).
For \(|x - 1| < \delta \) we have \mbox{\(x<1 + \delta\)}.
\end{quote}
\end{quote}
When typesetting mathematics using \TeX, use the end of line command (a double backslash) and other
such manual formatting commands infrequently.

\subsubsection{Do not scribble}

Don’t turn in homework that is scribbled or barely readable.


\subsubsection{Do not use the first person} 

In a proof, do not  use the first person.  

\begin{quote}
\textbf{Replace}
\begin{quote}
First, I consider the case \(x < 0\).
\end{quote}
\textbf{with}
\begin{quote}
 First, consider \(x < 0\).
\end{quote}
\end{quote}

Outside a proof, use the first person when it is natural.

 \subsubsection{  Use ``we''}

In mathematical text, ``we'' means the author and the reader.
Use ``we'' instead of the passive voice when it makes the 
text shorter.

\begin{quote}
\textbf{Replace} (passive voice)
\begin{quote}
  It has been shown that \(x\) is prime.
\end{quote}
\textbf{with} (active voice)
\begin{quote}
  We've shown that \(x\) is prime.
\end{quote}
Alternatively, write this as
\begin{quote}
  The number \(x\) is prime.
\end{quote}
\end{quote}

 \subsubsection{  Refer to theorems and axioms by name}  

It is usually better to write
\begin{quote}
The completeness of the real numbers implies that \dots
\end{quote}
then it is to write
\begin{quote}
 It follows from Axiom  (A9) that \dots
\end{quote}
The meaning of the first is clear, but the second won't be clear
to most readers until they look up Axiom (A9).

 \subsubsection{  Use consistent terminology}

Mathematics is full of synonyms; for example, \emph{set} and 
\emph{collection}. In a proof, it's best to stay with the
same word to express a concept.

\begin{quote}
\textbf{Replace}
\begin{quote}
Let \(A\) be a set and let \(B\) be a nonempty collection.
\end{quote}
\textbf{with}
\begin{quote}
Let \(A\) be a set and let \(B\) be a nonempty set.
\end{quote}
\end{quote}

%%----end style rules---------------------------------
%%----clarity rules---------------------------------

 \subsubsection{  Use words with clear meanings}

A \mbox{well-written} proof can be hard to follow. Do not make it more
difficult to understand by using imprecise language.  Informally 
we might say that a quantity is ``tiny,'' but don't use such language in a proof.



 \subsubsection{ Mathematical writing should be readable}

Some mathematical writing looks OK in print, but it is nonsense when read
aloud.  Read your writing out loud to find such mistakes.

\begin{quote}
\textbf{Replace}
\begin{quote}
  Let \(k > 0\) be an integer.
\end{quote}
\textbf{with}
\begin{quote}
  Let \(k\) be a positive integer.
\end{quote}
Alternatively,  this can be expressed as
\begin{quote}
  Let \(k \in \mathbf{Z}_ {\geq 0}.\)
\end{quote}
\end{quote}
Spoken,``Let \(k > 0\) be an integer'' is equivalent to
``Let \(k\) is greater than zero  be an integer.''



 \subsubsection{  Write succinctly}  

Extraneous facts make it difficult for the reader to follow a logical
argument.  Re-read your work and make certain that each thread of
reasoning is needed.

 \subsubsection{  Include sufficient detail}  

The proofs you write should contain enough detail so that a 
classmate could  follow it, but you should omit most purely algebraic 
steps. 

\begin{quote}
\textbf{Replace}
\begin{quote}
Subtracting 8 from both sides of the inequality (1)  and dividing by 42,
we deduce that \(x < 1\).
\end{quote}
\textbf{with}
\begin{quote}
Inequality (1) implies that \(x < 1\).
\end{quote}
\end{quote}

 \subsubsection{ Be careful with ``it''}

Make sure the meaning of every pronoun is unambiguous.

\begin{quote}
\textbf{\textbf{Replace}}
\begin{quote}
   Let \(\ a = 1\). To find \(b\), add 5 to it and then multiply it by 8.
\end{quote}
\textbf{with} 
\begin{quote}
   Let \(a = 1\).  Define \(b = 8 (a + 5)\).
\end{quote}
\end{quote}

 \subsubsection{ Define symbols before you use them}

With few exceptions (\(\uppi\), for example), every symbol you use must be defined
before, or near, the place you first use it.

\begin{quote}
\textbf{\textbf{Replace}}
\begin{quote}
The volume is \(\uppi r^2 h\).
\end{quote}
\textbf{with}
\begin{quote}
The volume of a right circular cylindrical with height \(h\) and
radius \(r\) is \(\uppi r^2 h\).

\end{quote}
\end{quote}
 \subsubsection{  Qualify identifiers} 

Some symbols have more than one meaning; for example, $(0,1)$ can
represent an open interval in $\mathbf{R}$, an element of
\(\mathbf{R}^2\), or a complex number.  Do not leave it up to the reader
to use conventions or context (\(n\) is an integer, \(z\) is complex
number, \(F\) is a function, etc.) to guess the type of each
identifier; rather, tell the reader the meaning of symbols.  If $F$ is
a function, for example, immediately let the reader know:
\begin{quote}
   Let $F$ be a function \dots
\end{quote}
Unless a proof is very long, the reader needn't be reminded that
$F$ is a function. 
\begin{quote}
\textbf{\textbf{Replace}}
\begin{quote}
  Let $F$ be a function defined on $\mathbf{R}$.  If the function $F$ is continuous on
  the open interval $(0,1)$,  \dots
\end{quote}
\textbf{with}
\begin{quote}
  Let \(F\) be a function defined on \(\mathbf{R}\).  If \(F\) is continuous on
  the open interval \((0,1)\), \dots
\end{quote}
\end{quote}

 \subsubsection{ Mathematics is case-sensitive}

Lower and upper case identifiers are distinct.  This means, for
example that \(a\) and \(A\) are different symbols.

 \subsubsection{  Avoid unnecessary notation}

Introduce notation and identifiers only when needed. 
\begin{quote}
\textbf{Replace}
\begin{quote}
 \textbf {Proposition} The number \(x = \sqrt{2}\) is irrational.
\end{quote}
\textbf{with}
\begin{quote}
\textbf{Proposition} The number \(\sqrt{2}\) is irrational.
\end{quote}
\end{quote}
In addition to being verbose, the first statement of the 
proposition has several other problems. First, \(x = \sqrt{2}\) isn't 
a number, it's an  {\em equation}. Second, only a number can be 
irrational, but the first proposition asserts that an {\em equation\/}
can be irrational. 

 \subsubsection{  Don't confuse a function with its formula}

If \(F\) is a function, don't write \(F(x)\) when you mean \(F\).

\begin{quote}
\textbf{\textbf{Replace}}
\begin{quote}
If \(F(x) \) and \(G(x)\) are continuous, \(F(x) + G(x)\) is 
continuous.
\end{quote}
\textbf{with}
\begin{quote}
If \(F\) and \(G\) are continuous, \(F + G\) is 
continuous.
\end{quote}
\end{quote}

 \subsubsection{  Use function signature notation}

When you introduce a new function, immediately tell the reader its 
name, domain, and possibly its co-domain. This is the \emph{signature}
of a function.
\begin{quote}
\textbf{\textbf{Replace}}
\begin{quote}
Let \(F(x,y)\) be a real-valued function.
\end{quote}
\textbf{with}
\begin{quote}
Let \(F : \mathbf{R}^2 \to \mathbf{R}\).
\end{quote}
\end{quote}
Writing \(F(x,y)\) to \emph{imply} that \(F\) is defined
on \( \mathbf{R}^2\) is a poor substitute for a clear statement of 
the function signature. If the exact domain of a function isn't 
important, use the notation
\begin{quote}
Let \(F :\,\, \subset \mathbf{R}^2 \to \mathbf{R}\).
\end{quote}
This tells the reader that the domain of \(F\) is a subset
of \(\mathbf{R}^2\), but it doesn't give its actual domain.


%%---end clarity-----------------------------------------
%%---start word selection--------------------------------
 \subsubsection{ Don't confuse ``let'' with ``therefore''}

To introduce a new object, use ``let.''  Use ``therefore'' to
state a logical consequence of the facts that precede it.


\begin{quote}
\textbf{\textbf{Replace}}
\begin{quote}
  Let \(n\) be an integer. Let \(n (n+1)\) be even.
\end{quote}
\textbf{with}
\begin{quote}
Let \(n\) be an integer.  Therefore \(n (n+1)\) is even.
\end{quote}
\end{quote}
Alternatives to ``therefore'' are ``so'' and ``thus.''


 \subsubsection{  Omit empty words and phrases} 

It doesn't help the reader to say that something is 
``easy to show'' or ``obvious.'' Omit or replace
the empty words and phrases listed in the following table.

\begin{center}
\begin{tabular}{| l l |}
\hline  \textbf {Word or Phrase} &  \textbf{Replacement} \\ 
\hline
It is trivial to show    & (omit) \\
It is easy to show     & (omit) \\   
It can be seen that    & (omit) \\
We are able to show that & (omit) \\
Clearly   & (omit) \\
Obviously  & (omit) \\
It immediately follows & It follows \\
For the purpose of contradiction &  Suppose \\
By definition                    & Thus \\
\hline
\end{tabular}
\end{center}
Also, expunge weasel words from your text. For a list of weasel words,  see, for example,  
\url{http://en.wikipedia.org/wiki/Weasel_word}.




 \subsubsection{  Do not be wishy-washy}  

A proof is no place to be indefinite.  Do not use  phrases similar 
to ``I believe,'' ``should be'' or ``I think that.''



%---end word selection
%-----start structure----------


 \subsubsection{  Use related theorems}  

A short proof that uses several powerful theorems is more likely to be
correct than is a long proof that starts from scratch.  Whenever
possible, base your proofs on propositions that have been proved.  Of
course, your work should only reference theorems we have covered in class.

 \subsubsection{  Let the reader know where the argument is heading} 

It often helps the reader to know in advance where a proof is heading.
Depending on the audience, it may be better to write

\begin{quote}
  Let $x \in A$.  To show that $ A \subset B$, we'll show that
  $x \in B$. Since, \dots. Thus $x \in B$; therefore $A \subset B$.
\end{quote}
then to write
\begin{quote}
  Let $x \in A$.  Since, \dots. Thus $x \in B$; therefore $A \subset B$.
\end{quote}

 \subsubsection{  Use idioms}

Some patterns enter into proofs so frequently that we express them
using an idiom.\footnote{Idiom (noun): A style or manner of expression
peculiar to a given people. (From dictionary.com)} Think of an idiom
as a template for a proof.  When a proof conforms to a template, the
reader has a good idea of what will follow.  This can make a proof easier to
read. We'll learn a few templates for proofs. 


%%---end structure ----------------
%%---start grammar ----------------

 \subsubsection{  Use correct spelling, punctuation, and grammar}

Misspelled words, incorrect punctuation, and poor grammar distract the reader.  
Always carefully check your work for these errors. Additionally, 
displayed formulas should be punctuated as regular text.  For example,
\begin{quote}
Repeated use of the  double angle formula gives
\[
   \sin^4(x) = \frac {\cos \left(4\,x\right)}{8}-\frac {\cos \left(2\,x\right)}{2%
 }+\frac {3}{8}. 
\]
\end{quote}
The displayed formula ends the sentence; thus it terminates with a period.

 \subsubsection{  Begin each sentence with a word}  


A sentence that starts with a symbol is difficult to 
read; every sentence must begin with a word, not a symbol.

\begin{quote}
\textbf{\textbf{Replace}}
\begin{quote}
$A$ is continuous on $\mathbf{R}$; we have \dots
\end{quote}
\textbf{with}
\begin{quote}
Since $A$ is continuous on $\mathbf{R}$, we have \dots
\end{quote}
\end{quote}

 \subsubsection{ Place a word between adjacent formulas}

When one formula follows another, place a word in between the formulas.

\begin{quote}
\textbf{\textbf{Replace}}
\begin{quote}
   When \(u > n\)  \(k < 0\).
\end{quote}
\textbf{with}
\begin{quote}
  When \(u > n\), we also have \(k < 0\).
\end{quote}
\end{quote}

 \subsubsection{ Don't over use the colon}

Do not use a colon immediately before a formula when the formula 
completes the sentence.

\begin{quote}
\textbf{\textbf{Replace}}
\begin{quote}
  For all integers \(n\), the number: \(n (n + 1) \) is even.
\end{quote}
\textbf{with}
\begin{quote}
   For all integers \(n\), the number \(n (n + 1) \) is even.
\end{quote}
\end{quote}
If the phrase that follows a colon is a complete sentence, 
capitalize the first word following a colon; otherwise, do
not capitalize the first word following a colon.
 
 \subsubsection{ Don't use unnecessary commas}

Do not surround an appositive with commas.

\begin{quote}
\textbf{\textbf{Replace}}
\begin{quote}
  The revenue, \(R\), is an increasing function of price, \(p\).
\end{quote}
\textbf{with}
\begin{quote}
  The revenue \(R\) is an increasing function of price \(p\).
\end{quote}
\end{quote}
When a sentence starts with ``further,'' ``therefore,'' or ``thus,''
a do not put a comma after the first word.


 \subsubsection{ If there is a ``first,'' there must be a ``second''}

If you enumerate ideas using starting with ``first,'' you must identify
the remaining ideas using second, third, etc.

\begin{quote}
\textbf{\textbf{Replace}}
\begin{quote}
First, we'll show that \(x \leq 0\). Next, we'll show that
\(x \geq 0\).
\end{quote}
\textbf{with}
\begin{quote}
First, we'll show that \(x \leq 0\).  Second, we'll show that
\mbox{\(x \geq 0\)}.
\end{quote}
\end{quote}



 \subsubsection{  Capitalize proper names and theorems}

Capitalize the words {\em axiom, definition, proposition\/}, and {\em theorem\/}
when they refer to something specific; otherwise, do not
capitalize them.

\begin{quote}
\textbf{\textbf{Replace}}
\begin{quote}
Today, Mary proved theorem 1. Yesterday, she proved two Theorems.
\end{quote}
\textbf{with}
\begin{quote}
Today, Mary proved Theorem 1. Yesterday, she proved two theorems.
\end{quote}
\end{quote}
Also capitalize proper names; for example, \emph{the Euler
constant}. 


 \subsubsection{  Don't confuse it's and its}

The word ``it's'' always means it is. Don't confuse it's with its.\footnote{I suppose you
could name your dog ``It.'' Then ``It's doghouse has two stories.'' is correct,
and ``It's'' does not mean ``It is.''} 

\begin{quote}
\textbf{\textbf{Replace}}
\begin{quote}
Memorize Definition 1-1; its the most important thing you will learn
this term.
\end{quote}
\textbf{with}
\begin{quote}
Memorize Definition 1-1; it's the most important thing you will learn
this term.
\end{quote}
\end{quote}


%%---start appearance ---------------------


 \subsubsection{ Do scratch work}

By all means, if it helps you construct a proof, draw pictures and
diagrams filled with lines and arrows.  But do not include your scratch
work in the final copy.

 \subsubsection{ Proofread your work}

When you have finished a proof, put it aside for a while.  After that,
proofread it several more times. Imagine that it is someone else's
work.  Does it still make sense?  Read your proof one line at a time
(cover up all the other lines). Finally, try reading your work out
loud. Mistakes that are easy to skip often manifest when spoken.


 \subsubsection{Check that you used every hypothesis}

After your first draft, check that you have used every hypothesis.
If you did not, it doesn't mean that your proof is wrong.  But it
does mean that it is  suspect.  Generally mathematicians
state theorems without unnecessary hypotheses.

%%---end appearance--------------------------------------
%%---start Logic-----------------------------------------
 \subsubsection{ Prove, show, and demonstrate all mean prove}

If you are asked to show that something is true, you need to
supply a proof.  The words prove, show, and demonstrate all mean the
same thing. 

 \subsubsection{ Prove what is given--not a special case}

It's not fair to append additional conditions to a proposition and 
then use these additional conditions to prove the proposition.  Put
differently, proving a {\em special case\/} of a proposition does not
prove the proposition. 

\begin{quote}
\textbf{Replace}
\begin{quote}
 \textbf {Proposition} Let \(A\) and \(B\) be sets.  Show that 
\(A \subset A \cup B\).

\vspace{0.1in}

\textbf{Proof} Let \(A = \{2,3,4\}\) and \(B = \{3,4,5\}\).  So
\mbox{\(A \cup B = \{2,3,4,5\}\)}.  Since \(\{2,3,4\} \subset \{2,3,4,5\}\),
we've shown that \(A \subset A \cup B\)
\end{quote}
\textbf{with}
\begin{quote}
 \textbf {Proof}  If \(x \in A\), we have \(x \in A \cup B\); therefore
\(A \subset A \cup B\).
\end{quote}
\end{quote}

 \subsubsection{ Prove the conclusion, not the hypothesis}

Make certain that the logic of your proof flows from the 
hypothesis to the conclusion. I've seen proofs that
start with the conclusion and end with either the same statement
or with one or more of the hypothesis.  



 \subsubsection{Don't abuse equality}

In this class, we mostly work with {\em functions, sets}, or {\em
numbers\/}.  A set can never equal a number and a number can never
equal a function.  Carefully check your work and make sure that you
have equality written only between objects with the same type.
\begin{quote}
The \emph{range} of a function is a \emph{set}; it is not a
number. If your proof contains
\begin{quote}
 \(\mbox{range}(F) = 1\),
\end{quote}
possibly you need to replace it with
\begin{quote}
 \(\mbox{range}(F) = \{1\}\)
\end{quote}
or with
\begin{quote}
 \(1 \in \mbox{range}(F)\)
\end{quote}
\end{quote}
Of course, equality shouldn't be written between things with the
same type that aren't equal; however, this isn't the point that 
is being made here.


 \subsubsection{ The symbol ``='' means equal, not the next step is}

Use ``='' between things that are equal. Do not use ``='' to  mean ``the next step is'' or ``implies.''

\begin{quote}
\textbf{Replace}
\begin{quote}
  \(\displaystyle F(x) = x^2 = \frac{\mathrm{d}}{\mathrm{d} x} (x^2) = 2x = F^\prime(x)\)
\end{quote}
\textbf{with}
\begin{quote}
  \(F^\prime(x) = (x^2)^\prime = 2 x. \)
\end{quote}
\end{quote}

\noindent Also, do not practice what I call non-committal math. This is, do not
express mathematics as a stream of disconnected nouns.

\begin{quote}
\textbf{Replace}
\begin{quote}
  \( x (x + 1) \quad (x (x + 1))^\prime \quad (x)^\prime (x + 1) + x (x+1)^\prime
 \quad x + 1 + x \quad 2x + 1 \)
\end{quote}
\textbf{with}
\begin{quote}
   \((x (x + 1))^\prime = (x)^\prime (x + 1) + x (x+1)^\prime =  x + 1 + x = 2x + 1 \)
\end{quote}
\end{quote}

Finally, do not use an arrow when you mean equality.

\begin{quote}
\textbf{Replace}
\begin{quote}
  \( (x+1)^2 \rightarrow x^2 + 2 x + 1 \)
\end{quote}
\textbf{with}
\begin{quote}
   \( (x+1)^2 = x^2 + 2 x + 1 \)
\end{quote}
\end{quote}

 \subsubsection{ Avoid arrows}
The double arrow \(\Rightarrow\) means \emph{therefore}, not
equal. When you want to express the fact that expressions are
\emph{equal}, use the equal sign, not the double arrow. Although we do
not generally use symbols for logical connectives, a correct usage of
the double arrow is
\begin{equation*}
  \mbox{Chocolate comes from the bean of the
cacao tree} \Rightarrow  \mbox{Chocolate contains caffeine.} 
\end{equation*}
 An \emph{incorrect} usage is
 \begin{equation*}
  \left(1 + x^2\right)^\prime \Rightarrow 2 x. 
 \end{equation*}
A single arrow (either left or right pointing) generally means
``Replace by.''  A correct usage of the single arrow is:
\emph{Substituting \(x \to 5\) in \(y = 6 + x\) yields \(y = 11\).} An
\emph{incorrect} usage is: \(x (1-x) = 0 \to x = 0,1\).



 \subsubsection{ Check for dangling silent variables}

Nothing should depend in any significant way on a  silent variable. If your
work has a  silent variable, make sure it is qualified in some way.


\begin{quote}
In the following \(i\) is a  silent variable. In the last sentence, the
 silent variable is unqualified:
\begin{quote}
If \(x \in \left(\underset{i \in I}{\bigcap} \,\, A_i \right)^C\), then
\(x \in \mathcal{U}\) and \(x \notin \underset{i \in I}{\bigcap} \,\, A_i\).
So \(x \notin A_i\).
\end{quote}
\textbf{Replace} this with
\begin{quote}
If \(x \in \left(\underset{i \in I}{\bigcap} \,\, A_i \right)^C\), then
\(x \in \mathcal{U}\) and \(x \notin \underset{i \in I}{\bigcap} \,\, A_i\).
So \(x \notin A_i\) for some \(i \in I\).
\end{quote}
\end{quote}

 \subsubsection{ A statement must follow ``such that''}

What follows ``such that'' must be a statement (something that must either
be true or false).  In 

\begin{quote}
Let \(A\) and \(B\) be sets such that \(A \cap B\).
\end{quote}
A noun \(A \cap B\) follows the ``such that''; it's not clear what the writer intended.
One possibility is 
\begin{quote}
Let \(A\) and \(B\) be sets such that \(A \cap B\) is nonempty.
\end{quote}

 \subsubsection{ Say what you mean}

Consider
\begin{quote}
If \(x \in A \cap B\), then \(x \in A\) and \(x \in B\). So
\(x \in A \cap B \subset A\).
\end{quote}
Spoken the last sentence is 
 \emph{So \(x\) is a member of \(A\) intersect \(B\) is a subset of \(A\).} To fix this,
end the proof with \emph{So \(x \in A\).}


 \subsubsection{Use correct italics}
 
 Most mathematical text is typeset in italic type, but there are some exceptions. Function names that are two or more characters long are  typeset in upright text; for example \(\sin(x+y)\) is correct, but \(sin(x+y)\) is wrong.  Other
 exceptions include the circular constant, the imaginary unit,  and the Euler number, each  of these symbols should be in upright text, 
 for example \(\uppi\) and  \(\mathrm{e}\), not \(\pi\) and \(e\).  Another exception is  that differentials should be in 
 upright text, not italics.  So \(\int x^2 \, \mathrm{d} x\) is correct and \(\int x^2 \, d x\) is nonstandard. But this rule is nearly universally ignored.

 \subsubsection{Don't use note taker shorthand}

 Notetaker shorthand, for example $\because$ for `because,' is  clear and it saves time.  But outside displayed or inline mathematics,  do not use such shorthand.

 \begin{center}
  \begin{tabular}{| l l |}
  \hline  \textbf {symbol} &  \textbf{Replacement} \\ 
  \hline
  $\because$  & because \\
  $\therefore$  &  therefore \\
  s.t.          & such that \\
  $\forall$     & for all \\
  $\exists$      & there exists \\
  iff      & if and only if  \\ \hline 
 \end{tabular}
\end{center}

\begin{quote}
  \textbf{Replace} $f(x) = f(y)$  $\therefore$  $x = y$

  \textbf{With} $f(x) = f(y)$ therefore  $x = y$

\end{quote}

\subsubsection{Use Iverson notation}

If $e$ is a mathematical expression that has a boolean value, enclosing
$e$ in square brackets means its boolean value; for example, for any
real number, we have 
\begin{align*}
   \left[x^2 \geq 0 \right] &\equiv \mbox{True}, \\
   \left[x^2 + 2 x + 1 = (x+1)^2 \right] &\equiv \mbox{True}, \\
   \left[x^2 < 0 \right] &\equiv \mbox{False}.
\end{align*}

\subsubsection{Be careful with compound inequations}

The expression $a=b=c$ means $a=b$ and $b=c$. And 
$a=b<c$ means $a=b$ and $b<c$. But following this
standard $a^2 \geq 0 = \mbox{True}$ would mean
$a^2 \geq 0$ and $0 = \mbox{True}$. But that is rubbish.
A good way to fix this is to use Iverson notation; thus
\begin{equation*}
 \left[a^2 \geq 0 \right] \equiv \mbox{True}.
\end{equation*}
Finally, according to this rule $a>b<c$ means $a>b$ and $b< c$. But
 it's far to easy to misread $a>b<c$ and incorrectly conclude that
 $a > c$.  So be careful.


\section*{Further Reading}

\begin{numlist}

\item Leslie Lamport, ``How to Write a Proof,'' American Mathematical Monthly, \textbf{102} 
(1993) 600-608.  

\item Donald Knuth, Tracy Larrabee, and Paul Roberts,  \emph{Mathematical   Writing} MAA Notes Series, 1989.

\item Leonard Gillman, \emph{Writing Mathematics Well}, Mathematical
Association of America, 1987.

\item Ashley Reiter, ``Writing a Research Paper in Mathematics,''  \url{http://web.mit.edu/jrickert/www/mathadvice.html}.

\item \url{https://sites.math.washington.edu/~lee/Writing/writing-proofs.pdf}


\item ``How to write Mathematics,'' Paul Halmos, \url{https://entropiesschool.sciencesconf.org/data/How_to_Write_Mathematics.pdf}

\item ``A Guide to Writing Mathematics,'' Kevin P. Lee,  \url{https://web.cs.ucdavis.edu/~amenta/w10/writingman.pdf}


\end{numlist}

\end{document}
\subsection*{\se Proof idioms}

\subsubsection*{\id The let-choose idiom}

To show that the quantified statement such as \(\left(\exists x_o \in \reals \right) \left(\forall x > x_o\right) 
\left(\exists M \in \reals \right)  \left(|7 + 5 x| \leq M x^2\right)\),
use the let-choose idiom. For each \(\forall\), use the word `let,' and for each \(\exists\) use the word `choose.' For example

\textbf{Proof} Choose \(x_o = 1\) and let \(x > 1\). Choose \(M = 12\). For all \(x > 1\), we have
\begin{align*}
  |7 + 5 x| &\leq |7| + 5 |x|, &\mbox{(triangle inequality)} \\
            &\leq 7 x + 5 x,   &\mbox{(using } x > 1) \\
            &= 12 x,        &\mbox{(arithmetic)} \\
            &\leq 12 x^2,   &\mbox{(using } x^2 > x) \\
            &= M x^2.       &\mbox{(substitution)} \\
\end{align*}



\subsubsection*{\id The one--bad--apple idiom}

You can show that a proposition is {\em false\/} by displaying just
one example that shows that it is false. You don't need two examples
or infinitely many examples; just one ``bad apple'' is enough.  





\subsubsection*{\id The pick-and-show idiom}

Anytime you need to show one set is a subset of another, you should use the
``pick-and-show'' idiom; it looks like this

\begin{quote}

\textbf{Proposition} Let \(A\) and \(B\) be sets and suppose \(H_1, H_2 , \dots
,\mbox{ and } H_n\). Then  \(A \subset B\).

\vspace{0.1in}

\textbf{Proof} If \(x \in A\), we have (deductions made using the 
facts \(H_1\) through \(H_n\)); therefore \(x \in B\).

\end{quote}
Here, the statements \(H_1\) through \(H_n\) are the hypothesis of the
proposition. To demonstrate set equality, use the pick-and-show idiom twice. Here
is and example of using pick-and-show.

\begin{quote}
\textbf{Proposition} Let \(A\) and \(B\) be nonempty sets and suppose \mbox{\(A \times B
= B \times A\)}.  Then \(A = B\).
\end{quote}
The conclusion of the proposition is \(A = B\); we need to use the 
pick-and-show idiom twice. The proof starts with
\begin{quote}
 \textbf{Proof}  First we show that \(A \subset B\). If \(a \in A\), we have \dots. 
\end{quote}
We need a consequence of \(a \in A\) that somehow involves the
hypothesis \mbox{\(A \times B = B \times A\)}.  Since \(B\) is
nonempty, it has an element \(b\). Thus we have \((a,b) \in A \times
B\).  It's downhill from here. For our proof, it might be best to
explain that \(B\) has an element and give it a name before we start
the pick-and-show idiom.  Here's a proof.



\begin{quote}
 \textbf{Proof} First we show that \(A \subset B\). Since \(B\) is
 nonempty, it has at least one element, call it \(b\). If \(x \in A\), we
 have \((x,b) \in A \times B\).  But \mbox{\(A \times B = B \times A\)};
 thus \((x,b) \in B \times A\).  Therefore \(x \in B\); consequently
\(A \subset B\).

Second we show that \(B \subset C\).  Since \(A\) is
 nonempty, it has at least one element, call it \(a\). If \(x \in B\), we
 have \((a,x) \in A \times B\).  But \(A \times B = B \times A\);
 thus \((a,x) \in B \times A\).  Therefore \(x \in A\); consequently
\(A \subset B\).

\end{quote}


It's tempting to write the proof as

\begin{quote}
 \textbf{Proof} If \((a,b) \in A \times B\), we have \((a,b) \in B \times
 A\). Thus \(a \in B\) and \(b \in A\); therefore \(A \subset B\) and
\(B \subset A\).
\end{quote}

What's wrong with this? Plenty: First it uses pick-and-show on \(A
\times B\) and \(B \times A\); it should use pick-and-show on \(A\)
and \(B\). Second, the proof never makes any use of the hypothesis
that \(A\) and \(B\) are nonempty; this doesn't mean that the proof is
wrong, just highly suspect. Third, and most importantly, the proof is
illogical. To see the flaw, look at it carefully and decide
what has really been proved. Since we've used pick-and-show on \(A
\times B\), we're aiming toward a proof that  \(A \times B = B
\times A\) implies \(B \times A \subset A \times B\).  We don't need
to do any work to conclude that! (For any set \(A\), we have \(A
\times \varnothing = \varnothing\). Think about that.)


\setcounter{id}{0}

\subsection* {\se How to get started}

Students often tell me that ``I know how to prove it, but I can't get started.''  Here are a few suggestions on how to start.


\subsubsection*{\id Know the definitions}

Have you ever tried to translate an article written in a foreign language by looking up the definition of each word?  If you have, you'll agree that the method doesn't work.  If you don't thoroughly understand the definition of every word in the statement of a theorem, you won't be able to prove it.  When you thoroughly understand a mathematical definition, you should be able to:


\begin{quote}
\begin{alphalist}

\item write a definition without peeking at the textbook,

\item give examples (and supply proofs) of things that satisfy the definition, 

\item give examples (and supply proofs) of things  that don't satisfy the definition.

\end{alphalist}
\end{quote}

\subsubsection*{\id List the hypothesis and the conclusion}

List each hypothesis and the conclusion. The conclusion often tells us what kind of proof to use. If, for example,
the conclusion is set inclusion, you will likely use a pick-and-show proof.

\subsubsection*{\id Express the proposition in symbolic form}

Write the proposition in symbolic form--expressed this way, it can be a road map for constructing the proof.

\subsubsection*{\id Use an idiom}

Does the proof follow a standard pattern?  If it does, look at a proof that
follows the same pattern and try to use the template.


\subsubsection*{\id Look at related theorems}

Look at all theorems that have been previously proved that involve one or
more of the hypothesis of the theorem you are trying to prove.  These
related theorems might help you prove your theorem.

\subsubsection*{\id Checked that you have used every hypothesis}

If you are stuck part way through a proof, check that you have
used {\em every\/} hypothesis.  If you have ignored one fact,  it's likely
that it holds the key to finishing the proof.

\subsubsection*{\id Try to disprove the proposition}

Sometimes if you try to disprove a proposition, you'll discover why
it is true.

\subsubsection*{\id Take a walk}

It's easy to get flustered or stuck on a bad idea.  When that happens,
put the work away and do some else for a while.  When you return, you may
be unstuck.






\end{document}




%{ \textbf Tricks for Writing Proofs} \\
%\vspace{0.1in}
%\normalsize
%Dr. Barton  Willis\\
%MATH 460 \\
%16 Sept 2003 \\
%\end{flushleft}

\noindent Analysis proofs often start with something like the following:
\begin{quote}
  Let \(\varepsilon > 0\).  Choose \(\delta = \mbox{min} \{1, \varepsilon\}\).
\end{quote}
We need to become comfortable with such statements. Understanding hinges on the fact:
\begin{quote}
{ \textbf Fact} If \(x = \mbox{min} \{a,b\}\), we have \(x \leq  a\) and  \(x \leq b\).
\end{quote}
Let's illustrate this with an example. Suppose
\begin{equation*}
  x = \mbox{min} \{5,6\}. \tag{$\star$}
\end{equation*}
Since \( \mbox{min} \{5,6\} = 5\), we could have simply written
\(\star\) as  \(x = 5 \) instead; \((\star)\) implies \emph{two}
facts
\[
  x < 5 \mbox{ and } x < 6.
\]
Of course if \(x < 5\) then \(x < 6\) is redundant; nevertheless,
\((\star)\) is a tricky way of writing two inequalities as one.
The full power of this idea is more clear when we have something
like
\begin{equation*}
  x < \mbox{min}\{1, \varepsilon\},  \tag{$\star \star$}
\end{equation*}
where \(\varepsilon\) is some real number. Now $(\star \star)$
implies
\[
  x < 1 \mbox{ and } x < \varepsilon.
\]
Given a \emph{specific} value for \(\varepsilon\), we could reduce
these two inequalities into just one; for example, if \(\varepsilon =
10\), we'd simply have \(x < 1\). And if \(\varepsilon =
\frac{1}{2} \), we'd simply have \(x < \frac{1}{2}\).

\ding{43} \textbf{Fact} If \(x < \mbox{min} \{a,b\}\), we have
\(x < a\) and \(x < b\).

OK, let's try something more challenging. Given that
\[
   | x - 1 | < \mbox{min} \{1, \frac{\varepsilon}{3}\},
\]
show that
\[
  |x^2 - 1| < \varepsilon.
\]
Expressing our given as two inequalities, yields
\[
  |x - 1| < 1 \mbox{ and } |x - 1| < \frac{\varepsilon}{3}.
\]
What we're to show looks like a difference of squares; let's factor it
\[
  |x^2 - 1| = |x - 1| \, |x + 1|
\]
The first factor \(|x - 1|\) is bounded above by
\( \frac{\varepsilon}{3}\).  The second factor \(|x+1|\) is more
of a puzzle. Using our other given, we have
\[
  | x - 1| < 1 \Rightarrow -1 < x -1 < 1 \Rightarrow 1 < x + 1 < 3
\Rightarrow |x+1| < 3.  
\]
Using \(|x+1| < 3 \) and \(|x-1| < \frac{\varepsilon}{3} \), we get
what we were after; thus
\[
    |x^2 - 1| = |x - 1| \, |x + 1| < 3  \frac{\varepsilon}{3} = \varepsilon.
\]


\end{document} 

https://sites.math.washington.edu/~lee/Writing/writing-proofs.pdf


