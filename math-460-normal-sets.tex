\documentclass[fleqn]{beamer}
%\usetheme[height=7mm]{Rochester}
\usetheme{Boadilla} %{Rochester}

\setbeamertemplate{footline}[text line]{%
\parbox{\linewidth}{\vspace*{-8pt}\hfill\insertshortauthor\hfill\insertpagenumber}}
\setbeamertemplate{navigation symbols}{}
%\author[BW]{Dr.\ Barton Willis}
\usepackage{amsmath}\usepackage{amsthm}
\usepackage{isomath}
\usepackage{upgreek}
\usepackage{comment,enumerate,xcolor}

\usepackage[english]{babel}
\usepackage[final,babel]{microtype}%\usepackage[dvipsnames]{color}
\usefonttheme{professionalfonts}
%\usefonttheme{serif}

\newcommand{\reals}{\mathbf{R}}
\newcommand{\complex}{\mathbf{C}}
\newcommand{\integers}{\mathbf{Z}}
\DeclareMathOperator{\range}{range}
\DeclareMathOperator{\domain}{dom}
\DeclareMathOperator{\dom}{dom}
\DeclareMathOperator{\codomain}{codomain}
\DeclareMathOperator{\sspan}{span}
\DeclareMathOperator{\F}{F}
\DeclareMathOperator{\G}{G}
\DeclareMathOperator{\B}{B}
\DeclareMathOperator{\D}{D}
\DeclareMathOperator{\id}{id}
\DeclareMathOperator{\ball}{ball}

\newcommand{\true}{\mathrm{true}}
\newcommand{\false}{\mathrm{false}}

\usepackage{graphicx}
\usepackage{color}
\usepackage{amsmath}
\DeclareMathOperator{\nullspace}{nullity}
\theoremstyle{definition}
\newtheorem{mydef}{Definition}
\newtheorem{myqdef}{Quasi-definition}
\newtheorem{myex}{Example}
\newtheorem{myth}{Theorem} 
\newtheorem{myfact}{Fact}
\newtheorem{metathm}{Meta Theorem}
\newtheorem{Question}{Question}
\newtheorem{Answer}{Answer}
\newtheorem{myproof}{Proof}
\newtheorem{hurestic}{Hurestic}

%\usepackage{array}   % for \newcolumntype macro
%\newcolumntype{L}{>{$}l<{$}} % math-mode version of "l" column type

\newenvironment{alphalist}{
  \vspace{-0.4in}
  \begin{enumerate}[(a)]
    \addtolength{\itemsep}{1.0\itemsep}}
  {\end{enumerate}}



\usepackage{pifont}

\newenvironment{checklist}{
  \begin{enumerate}[\ding{51}]
    \addtolength{\itemsep}{-0.0\itemsep}}
  {\end{enumerate}}

\newenvironment{numberlist}
   {\begin{enumerate}[(1)]
       \addtolength{\itemsep}{-0.5\itemsep}}
     {\end{enumerate}}
\usepackage{amsfonts}
\makeatletter
\def\amsbb{\use@mathgroup \M@U \symAMSb}
\makeatother
\usepackage{bbold}



\newcommand{\llnot}{\lnot \,} % is accepted


\subtitle{Lesson 5 \\ \vspace{1.0in} The whole problem with the world is that fools and fanatics are always so certain of themselves, and wiser people so full of doubts. \\ \vspace{0.25in} Bertrand Russel}
\title{\textbf{Being normal}}
%\author[Barton Willis] % (optional, for multiple authors)
%{Barton~Willis}%
%\institute[UNK] % (optional)

%{
 % \inst{1}%
%  ``The secret of getting ahead is getting started.'' Mark Twain
%   }
  \date{}
 

%\usepackage{courier}
%\lstset{basicstyle=\ttfamily\footnotesize,breaklines=true}
%\lstset{framextopmargin=50pt,frame=bottomline}


%\begin{document}



%--------
%usepackage[usenames,dvipsnames,svgnames,table]{color}



\begin{document}

\frame{\titlepage}

\begin{frame}{A not so innocent predicate}
Define  a predicate \(P\) by
\[
       P = x \mapsto \begin{cases}   \mbox{true}  & \mbox{if } x \mbox{ is a set} \\
        \mbox{false}  & \mbox{if } x \mbox{ is not a  set} \end{cases}.
\]
For example
\[
    P(\varnothing) = \true, \quad    P(107) = \false, \quad P(\{107\}) = \true.
\]
Define the set of all sets \(S\) by
\[
    S = \{x \mid  P(x) \}.
\]
Since \(S\) is a set, we have \(S \in S\).  After all, \(S\) is the set of all sets.

\vspace{0.2in}

\textbf{Fact} It's unusual for a set to be a member of itself; that is, it's unusual for \(S \in S\).

\end{frame}

\begin{frame}{Normal sets}


Define a set \(N\) by  (recall \(S\) is the set of all sets)
\[
     N = \{x \in S \mid x \notin x \}.
\]
Since, for example \(\varnothing \notin \varnothing,  \{107\} \notin \{107\}\), we have
\[
   \varnothing \in N  \mbox{ and } \{107\}  \in N.
\] 

\textbf{Question:}  Is \(N \in N\)?

\vspace{0.2in}

\textbf{Answer:}   For any set \(A\), we have \(A \in N \equiv A \notin A\).  Also  \(A \notin N \equiv A \in A\).

Yikes!  Either \(   N \in N\) or   \(N \not \in N\).  But we have
\[
    N \in N  \equiv  N \notin N.
\]
This cannot be--a statement cannot be logically equivalent to its negation.
 And this is a problem. It's called \emph{Russell's paradox} (Bertrand Russell,  1872 -- 1970). 

\end{frame}

\begin{frame}{Is mathematics doomed?}

\begin{alphalist}

\item No.  We've been informal (= nonlogical) about what is and isn't a set.  To sort this learn, for example, Zermelo set theory. 

\item Part of the resolution is that we need to specify  \emph{universal set} and to assume that all sets are subsets of the universal set.

\item Depending on context, our universal set might be the set of all real numbers, or the set of all open sets of real numbers, or \dots.

\end{alphalist}

\end{frame}

\end{document}